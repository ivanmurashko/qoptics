%% -*- coding:utf-8 -*-
The necessity of this section is primarily caused by the specifics
of teaching quantum mechanics in technical universities. The main emphasis
is placed on practical applications, often overlooking
the theoretical and philosophical foundations of quantum mechanics.

For example, practical techniques for processing
the results of numerous similar measurements are described in detail, i.e., the rules
for calculating the average values of physical quantities in
a given quantum state are explained. At the same time, answers to questions
arising from the analysis of single experiments are omitted, such as the relation
between the instrument readings and the change of the wave function upon measurement, what
decoherence \rindex{Decoherence}
is, and many others. The relevance of these questions
has increased recently due to the necessity to analyze
the results of single experiments, which has led to the ability
to design new devices that rely on these properties of pure
quantum states. In particular, the 2012 Nobel Prize in Physics
was awarded to Serge Haroche and
David J. Wineland for ``for ground-breaking experimental methods that enable measuring and manipulation of individual quantum systems.''
\footnote{
``ground-breaking
experimental methods that made it possible to measure
and control individual quantum systems.''}