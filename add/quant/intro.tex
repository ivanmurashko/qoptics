%% -*- coding:utf-8 -*-
The necessity of this section is primarily due to the specifics of teaching quantum mechanics in technical universities. The main emphasis is placed on practical application, and in the process, the theoretical and philosophical foundations of quantum mechanics are often overlooked.

For example, practical techniques for processing the results of many similar measurements are described in detail, i.e., the rules for calculating the average values of physical quantities in a given quantum state are described. However, answers to questions arising from the analysis of single experiments, such as the relationship between the instrument's readings and the change in the wave function during measurement, what decoherence \rindex{Decoherence} is, and many others, are omitted. The relevance of these issues has increased recently due to the necessity to analyze the results of single experiments, leading to the possibility of designing new instruments that rely on these properties of pure quantum states. In particular, the Nobel Prize in Physics in 2012 was awarded to Serge Haroche and David J. Wineland for "ground-breaking experimental methods that enable measuring and manipulation of individual quantum systems."
\footnote{
"ground-breaking experimental methods that enable measuring and manipulation of individual quantum systems."
}