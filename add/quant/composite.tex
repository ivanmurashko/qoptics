\section{Composite Systems}
\label{sec:add:quantum:composite}
Systems consisting of multiple parts behave fundamentally differently for the case of classical and quantum systems.

As an example, consider a system of two particles. The position of the first one is described by 3 coordinates, which represent a point in some linear space $L_1$: $(x_1, y_1, z_1) \in L_1$. For the second, we assume we have 2 coordinates in space $L_2$ that fully determine the location: $(x_2, y_2) \in L_2$. Obviously, to fully describe the position of the two particles we need 5 coordinates: $(x_1, y_1, z_1, x_2, y_2) \in L_1 \times L_2$. Thus, composite systems, in the classical case, are described by points in space $L^{classical}$, which is the Cartesian product of the originals: $L = L_1 \times L_2$. A distinguishing feature of the Cartesian product is that the dimensions of the corresponding spaces add:
\[
\dim{L^{classical}} = \dim{L_1} + \dim{L_2}.
\]

In the quantum case, a composite system is described by vectors (points) in space that is the tensor product of the originals:
\[
L^{quantum} = L_1 \otimes L_2.
\]
In this space, there are 6 basis vectors, and accordingly, to describe the state of the system, 6 numbers are required: $(x_1 \cdot x_2, y_1 \cdot x_2, z_1 \cdot x_2, x_1 \cdot y_2, y_1 \cdot y_2, z_1 \cdot y_2) \in L^{quantum}$. Accordingly, the dimensions multiply:
\[
\dim{L^{quantum}} = \dim{L_1} \cdot \dim{L_2}.
\]

Thus, if we have two independent quantum systems with wave functions $\ket{\psi_1}$ and $\ket{\psi_2}$, then the composite system will have the following wave function
\[
\ket{\psi_{12}} = \ket{\psi_1} \otimes \ket{\psi_2},
\]
and it is obvious that the following equalities hold
\begin{eqnarray}
\ket{\psi_1} = Sp_2 \ket{\psi_{12}}, \\
\nonumber
\ket{\psi_2} = Sp_1 \ket{\psi_{12}}
\nonumber
\end{eqnarray}
i.e., to obtain the state of subsystem 1, one must take the trace over the states of system 2 from the overall wave function.

Composite systems have an interesting property where, despite the parts of this system being mixed states, the entire system as a whole is pure. As an example, consider the so-called entangled states \index{entangled states}:
\[
\ket{\psi} = \frac{\ket{0_1}\ket{1_2} - \ket{1_1}\ket{0_2}}{\sqrt{2}}
\]
which is pure. At the same time, for the first particle, we have
\begin{eqnarray}
\rho_1 = Sp_2 \ket{\psi} \otimes \bra{\psi} =
\nonumber \\
=
\bra{0_2}\ket{\psi}\bra{\psi}\ket{0_2} +
\bra{1_2}\ket{\psi}\bra{\psi}\ket{1_2} =
\nonumber \\
= \frac{\ket{1_1}\bra{1_1} +
\ket{0_1}\bra{0_1}}{2}
\nonumber
\end{eqnarray}
i.e., the state of the first particle is mixed.