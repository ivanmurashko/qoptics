%% -*- coding:utf-8 -*- 
\section{Dynamics of wave function change}
\label{AddWaveFunc}
The wave function $\left| \phi \right>$ can change through
two mechanisms:
\begin{itemize}
\item Collapse of the wave function during measurement
\item Schrödinger equation in the intervals between two consecutive
  measurements 
\end{itemize}

\subsection{Schrödinger Equation}
The change of the state of a pure quantum system between two
consecutive measurements is described by the following equation (Schrödinger)
\begin{equation}
i \hbar \frac{\partial \left| \phi \right>}{\partial t} = \hat{\mathcal{H}}
\left| \phi \right>.
\label{eqAddWaveFunc_Shredinger}
\end{equation}

Equation \eqref{eqAddWaveFunc_Shredinger} is reversible and,
accordingly, not applicable to describing the change of the wave function at
the moment of measurement.

It is worth noting the connection of the Schrödinger equation with 
\myref{thm:stone}{Stone's theorem} TBD.

\subsubsection{Schrödinger equation in the interaction picture}
\label{AddWaveFuncInter}
Suppose that the Hamiltonian can be separated into two parts:
\begin{equation}
\hat{\mathcal{H}} = \hat{\mathcal{H}}_0 + \hat{\mathcal{V}}.
\nonumber
\end{equation}

We introduce the following transformation of the wave function:
\[
\left| \phi \right>_I = 
\exp{\left(\frac{i \hat{\mathcal{H}}_0 t}{\hbar}\right)}
\left| \phi \right>
\]
and examine the value of the following expression:
\begin{eqnarray}
i \hbar \frac{\partial \left| \phi \right>_I}{\partial t} = 
i \hbar \frac{i \hat{\mathcal{H}}_0}{\hbar} 
\exp{\left(\frac{i \hat{\mathcal{H}}_0 t}{\hbar}\right)}
\left| \phi \right> +
\exp{\left(\frac{i \hat{\mathcal{H}}_0 t}{\hbar}\right)}
i \hbar \frac{\partial \left| \phi \right>}{\partial t} = 
\nonumber \\
= - \hat{\mathcal{H}}_0 
\exp{\left(\frac{i \hat{\mathcal{H}}_0 t}{\hbar}\right)}
\left| \phi \right> +
\exp{\left(\frac{i \hat{\mathcal{H}}_0 t}{\hbar}\right)}
\left(
\hat{\mathcal{H}}_0 + \hat{\mathcal{V}}
\right)
\left| \phi \right> =
\nonumber \\ 
- \hat{\mathcal{H}}_0 
\exp{\left(\frac{i \hat{\mathcal{H}}_0 t}{\hbar}\right)}
\left| \phi \right> +
\hat{\mathcal{H}}_0 
\exp{\left(\frac{i \hat{\mathcal{H}}_0 t}{\hbar}\right)}
\left| \phi \right>
+
\exp{\left(\frac{i \hat{\mathcal{H}}_0 t}{\hbar}\right)}
 \hat{\mathcal{V}}
\left| \phi \right> =
\nonumber \\
= 
\exp{\left(\frac{i \hat{\mathcal{H}}_0 t}{\hbar}\right)}
 \hat{\mathcal{V}}
\left| \phi \right> = 
\nonumber \\
= 
\exp{\left(\frac{i \hat{\mathcal{H}}_0 t}{\hbar}\right)}
 \hat{\mathcal{V}}
\exp{\left( - \frac{i \hat{\mathcal{H}}_0 t}{\hbar}\right)}
\exp{\left(\frac{i \hat{\mathcal{H}}_0 t}{\hbar}\right)}
\left| \phi \right> = 
\nonumber \\
= 
 \hat{\mathcal{V}}_I \left| \phi \right>_I,
\nonumber
\end{eqnarray}
where 
\begin{equation}
\hat{\mathcal{V}}_I = 
\exp{\left(\frac{i \hat{\mathcal{H}}_0 t}{\hbar}\right)}
 \hat{\mathcal{V}}
\exp{\left( - \frac{i \hat{\mathcal{H}}_0 t}{\hbar}\right)}
\label{eqAddWaveFunc_VInter}
\end{equation} 
is the interaction Hamiltonian in the interaction picture.

Thus, we obtain the Schrödinger equation in the interaction picture:
\begin{equation}
i \hbar \frac{\partial \left| \phi \right>_I}{\partial t} = \hat{\mathcal{V}}_I
\left| \phi \right>_I.
\label{eqAddWaveFunc_ShredingerInter}
\end{equation}


\subsubsection{Equation of motion for the density matrix}
From relation \eqref{eqAddWaveFunc_Shredinger} we have
\begin{eqnarray}
i \hbar \frac{\partial \left| \phi \right>}{\partial t} = \hat{\mathcal{H}}
\left| \phi \right>,
\nonumber \\
- i \hbar \frac{\partial \left< \phi \right|}{\partial t} = \hat{\mathcal{H}}
\left< \phi \right|,
\nonumber
\end{eqnarray}
thus, for the density matrix 
$\hat{\rho} = \left| \phi \right>\left< \phi \right|$ we get
\begin{eqnarray}
i \hbar \frac{\partial \hat{\rho} }{\partial t} = 
i \hbar \frac{\partial  \left| \phi \right>\left< \phi \right|
}{\partial t} = 
i \hbar \left( \frac{\partial \left| \phi \right>}{\partial t}\left< \phi
\right| +
\left| \phi \right> \frac{\partial \left< \phi \right|}{\partial t}
\right) =
\nonumber \\
=  \hat{\mathcal{H}} \left| \phi \right>\left< \phi \right| -
\left| \phi \right>\left< \phi \right|\hat{\mathcal{H}} = 
\left[ \hat{\mathcal{H}}, \hat{\rho} \right]
\label{eqAddWaveFunc_Pho}
\end{eqnarray}
Equation \eqref{eqAddWaveFunc_Pho} is often called the quantum
Liouville equation and the von Neumann equation.

\subsubsection{Evolution operator. Heisenberg and Schrödinger pictures}

The change of the wave function according to the law \eqref{eqAddWaveFunc_Shredinger}
can also be described using a certain operator (evolution) $\hat{U}\left(t,t_0\right)$:
\begin{equation}
\left| \phi\left(t\right) \right> = 
\hat{U}\left(t,t_0\right)\left| \phi\left(t_0\right) \right>.
\label{eqAddWaveFunc_ShredingerU}
\end{equation}

Equation \eqref{eqAddWaveFunc_Shredinger} can be rewritten as
\begin{equation}
\left| \phi\left(t\right) \right> = 
\exp\left( -\frac{i}{\hbar} \hat{\mathcal{H}} \left( t - t_0 \right)  \right)
\left| \phi\left(t_0\right) \right>,
\nonumber
\end{equation}
from which for the evolution operator we have
\begin{equation}
\hat{U}\left(t,t_0\right) = 
\exp\left( -\frac{i}{\hbar} \hat{\mathcal{H}} \left( t - t_0 \right)  \right)
\label{eqAddDiracEvolutionOper}
\end{equation}

The evolution operator is unitary. Indeed:
\begin{eqnarray}
\hat{U}\left(t,t_0\right)\hat{U}^\dag\left(t,t_0\right) = 
\nonumber \\
= \exp\left( -\frac{i}{\hbar} \hat{\mathcal{H}} \left( t - t_0 \right)
\right)
\exp\left( +\frac{i}{\hbar} \hat{\mathcal{H}} \left( t - t_0 \right)
\right)
= \hat{I}
\nonumber
\end{eqnarray}

Alongside the Schrödinger picture where operators do not depend on time
and wave functions evolve, there exists the Heisenberg picture where
operators evolve with time.

Obviously, the expectation values of operators should not depend on
the picture:
\begin{eqnarray}
\left< \phi_H\left(t_0\right) \right|\hat{A}_H\left(t\right)\left| 
\phi_H\left(t_0\right) \right> = 
\left< \phi_S\left(t\right) \right|\hat{A}_S\left| 
\phi_S\left(t\right) \right> = 
\nonumber \\
=
\left<
\phi_H\left(t_0\right)\right|\hat{U}^\dag\left(t,t_0\right)\hat{A}_S\hat{U}\left(t,t_0\right)\left|
\phi_H\left(t_0\right) \right>,
\nonumber
\end{eqnarray}
from which, given $\hat{A}_H\left(t_0\right) = \hat{A}_S\left(t_0\right)$, we obtain the evolution law for operators in the Heisenberg picture:
\begin{equation}
\hat{A}_H\left(t\right) = \hat{U}^\dag\left(t,t_0\right)\hat{A}_H\left(t_0\right)\hat{U}\left(t,t_0\right)
\label{eqAddWaveFunc_HeizenbergU}
\end{equation}

At the same time, the equation for the operator $\hat{A}_H$ looks like:
\begin{eqnarray}
  \frac{\partial \hat{A}_H}{\partial t} =
  \frac{i}{\hbar} \hat{\mathcal{H}}
  \hat{U}^\dag\left(t,t_0\right)\hat{A}_H\left(t_0\right)\hat{U}\left(t,t_0\right)
  -
  \nonumber \\
  - \frac{i}{\hbar}
  \hat{U}^\dag\left(t,t_0\right)\hat{A}_H\left(t_0\right)\hat{U}\left(t,t_0\right)
  \hat{\mathcal{H}} =
  \frac{i}{\hbar} \left[\hat{\mathcal{H}}, \hat{A}_H \right]
  \label{eqAddWaveFunc_HeizenbergT}
\end{eqnarray}

\subsection{Differences between pure and mixed states. Decoherence}
\begin{definition}[Pure state]
If the state of a system is described by a density matrix $\hat{\rho}$
which can be written as 
\begin{equation}
\hat{\rho} = \ket{\psi}\bra{\psi}
\label{eq:add:quant:purestate}
\end{equation}
then this state is called pure.
\end{definition}

\begin{definition}[Mixed state]
If the state of a system is described by a density matrix $\hat{\rho}$
which \textbf{cannot} be written in the form
\eqref{eq:add:quant:purestate}, i.e. 
\[
\hat{\rho} \ne \ket{\psi}\bra{\psi}
\]
then this state is called mixed.
\end{definition}


Of particular interest is the difference between pure and mixed
states, especially how the transition from pure
to mixed states occurs.

Consider a two-level state
(see \autoref{figAddDecoherenceModel}). In a pure state it is described 
by the following wave function:
\begin{equation}
\left|\phi\right> = c_a \ket{a } + c_b \ket{b},
\nonumber
\end{equation}
the corresponding density matrix is
\rindex{Density matrix}
\begin{eqnarray}
\hat{\rho} = \left|\phi\right>\left<\phi\right| =
\nonumber \\
= 
\left|c_a\right|^2 \ket{a}\bra{a} + 
\left|c_b\right|^2 \ket{b}\bra{b} +
\nonumber \\
+
c_a c_b^{\ast}\ket{a}\bra{b} +
c_b c_a^{\ast}\ket{b}\bra{a},
\label{eqAddDecoherencePure}
\end{eqnarray}
or in matrix form
\begin{eqnarray}
\hat{\rho} = 
\begin{pmatrix}
\left|c_a\right|^2 & c_a c_b^{\ast} \\
c_b c_a^{\ast} & \left|c_b\right|^2 \\
\end{pmatrix}.
\nonumber
\end{eqnarray}

The density matrix 
\rindex{Density matrix}
for a mixed state has only diagonal
elements:
\begin{eqnarray}
\hat{\rho} = 
\begin{pmatrix}
\left|c_a\right|^2 & 0 \\
0 & \left|c_b\right|^2 \\
\end{pmatrix} = 
\nonumber \\
=
\left|c_a\right|^2 \ket{a}\bra{a} + 
\left|c_b\right|^2 \ket{b}\bra{b}.
\label{eqAddDecoherenceMix}
\end{eqnarray}

\input add/quant/figdecoherence.tex

The transition from \eqref{eqAddDecoherencePure} to \eqref{eqAddDecoherenceMix}
is called decoherence.\rindex{Decoherence}
In describing the decoherence process we will follow
\cite{bMensky2001}. 

The difference of mixed states from pure states manifests in the influence of the environment
$\mathcal{E}$. In the case of pure states the considered system and its
environment are independent, i.e.
\begin{equation}
\left|\phi\right>_{pure} = \left|\phi\right>_{at} \otimes
\left|\mathcal{E}\right>.
\label{eqAddDecoherencePhiPure}
\end{equation}

In the case of mixed states the atom and its environment form what is called an entangled state where the states
$\ket{a}$ and $\ket{b}$ correspond to distinguishable 
environment states $\left|\mathcal{E}_a\right>$ and
$\left|\mathcal{E}_b\right>$.
\begin{equation}
\left|\phi\right>_{mix} = c_a\ket{a} \left|\mathcal{E}_a\right>
+ c_b\ket{b} \left|\mathcal{E}_b\right>.
\label{eqAddDecoherencePhiMix}
\end{equation}

The density matrix 
\rindex{Density matrix}
corresponding to \eqref{eqAddDecoherencePhiMix} is

\begin{eqnarray}
\hat{\rho}_{mix} = \left|\phi\right>_{mix}\left<\phi\right|_{mix} = 
\nonumber \\
= 
\left|c_a\right|^2 \ket{a}\bra{a} \otimes
\left|\mathcal{E}_a\right>\left<\mathcal{E}_a\right| + 
\left|c_b\right|^2 \ket{b}\bra{b} \otimes
\left|\mathcal{E}_b\right>\left<\mathcal{E}_b\right| +
\nonumber \\
+
c_a c_b^{\ast}\ket{a}\bra{b} \otimes
\left|\mathcal{E}_a\right>\left<\mathcal{E}_b\right| +
c_b c_a^{\ast}\ket{b}\bra{a} \otimes
\left|\mathcal{E}_b\right>\left<\mathcal{E}_a\right|.
\label{eqAddDecoherenceRhoMix}
\end{eqnarray}
If we now apply averaging over the environment variables to expression \eqref{eqAddDecoherenceRhoMix},
we obtain
\begin{eqnarray}
\left<\hat{\rho}_{mix}\right>_{\mathcal{E}} = 
Sp_{\mathcal{E}}\left(\hat{\rho}\right) = 
\nonumber \\
=
\left<\mathcal{E}_a\right|\hat{\rho}_{mix}\left|\mathcal{E}_a\right> +
\left<\mathcal{E}_b\right|\hat{\rho}_{mix}\left|\mathcal{E}_b\right>
= 
\nonumber \\
= \left|c_a\right|^2 \ket{a}\bra{a} + 
\left|c_b\right|^2 \ket{b}\bra{b}.
\label{eqAddDecoherenceRhoMixFin}
\end{eqnarray}
Expression  \eqref{eqAddDecoherenceRhoMixFin} is obtained under the assumption
of an orthonormal basis $\left\{\left|\mathcal{E}_a\right>,
\left|\mathcal{E}_b\right>\right\}$: 
\begin{eqnarray}
\left<\mathcal{E}_a\right.\left|\mathcal{E}_a\right> = 
\left<\mathcal{E}_b\right.\left|\mathcal{E}_b\right> = 1,
\nonumber \\
\left<\mathcal{E}_a\right.\left|\mathcal{E}_b\right> = 
\left<\mathcal{E}_b\right.\left|\mathcal{E}_a\right> = 0.
\label{eqAddDecoherenceMixECond}
\end{eqnarray}

Conditions \eqref{eqAddDecoherenceMixECond} are key to understanding why the considered basis of the atomic system is
preferred and why, for instance, in the case of mixed states one does not
consider other bases such as the basis obtained by the Hadamard transform relative to the original \rindex{Hadamard transform}:
\begin{eqnarray}
\left|\mathcal{A}\right> = \frac{\ket{a} + \ket{b}}
              {\sqrt{2}},
\nonumber \\
\left|\mathcal{B}\right> = \frac{\ket{a} - \ket{b}}
              {\sqrt{2}}.
\label{eqAddDecoherenceBaseWrong}
\end{eqnarray}
The environment states corresponding to the basis
\eqref{eqAddDecoherenceBaseWrong} are not orthogonal, hence it is impossible to use \eqref{eqAddDecoherenceBaseWrong}
as basis vectors for mixed states. 

The decoherence process, i.e., the transition from
\eqref{eqAddDecoherencePhiPure} to \eqref{eqAddDecoherencePhiMix}, can
be described using the Schrödinger equation and therefore
theoretically reversible. The only requirement is the orthogonality of distinguishable environment states: 
$\left<\mathcal{E}_a\right.\left|\mathcal{E}_b\right> = 0$. This
requirement is always fulfilled for macroscopic systems, where
the state depends on a very large number of variables. Moreover,
in the case of macroscopic systems there exist many possible final states
$\left|\mathcal{E}_{a,b}\right>$,
which makes the reverse process practically unrealizable
because it is necessary to control a large number of possible
variables describing the environment state. In this sense,
the decoherence process has the same nature as the second law of thermodynamics 
(increasing entropy), which describes
irreversible processes.
\footnote{One should be somewhat careful here
  since the second law of thermodynamics applies to closed systems, while
  the decoherence processes happen in open systems.}

The decoherence process is very fast; in particular,
\cite{bZurek02} provides the following estimate: for a system of mass 1
g at a separation $\Delta x = 1 \mbox{ cm}$ and temperature $T=300
\mbox{ K}$, with relaxation time equal to the lifetime of the Universe
$\tau_R = 10^{17} \mbox{ s}$, the decoherence process takes $10^{-23}
\mbox{ s}$ \rindex{Decoherence!rate}

%% As an example, consider the so-called Schrödinger cat, which can be in two states
%% $\ket{L}$ - the cat is alive, and $\ket{D}$ - the cat is dead.
%% Along with these two states, according to the principle of superposition,
%% the state $\left|\psi\right> = C_L \ket{L} + C_D
%% \ket{D}$ - a superposition of alive and dead cat - is also permitted. This example
%% challenges common sense, which is based on experience among
%% classical systems where such states of both alive and dead simultaneously
%% are impossible. We observe in the classical world so-called mixed
%% states for which the superposition principle does not apply. In the case of
%% a "classical cat," one can say that with probability
%% $\left|C_L\right|^2$ it is in state $\ket{L}$ and with
%% probability $\left|C_D\right|^2$ it is in state
%% $\ket{D}$. Such systems are called mixed. Obviously,
%% there must be a fundamental difference between pure and mixed
%% systems. This difference is the effect of the environment.

\subsection{Collapse of the wave function. Measurement in quantum mechanics}
\label{sec:add:reduction}

The process of selection (measurement result) is one of the most complex in
quantum mechanics. Unlike deterministic change of the wave function described by the Schrödinger
equation \eqref{eqAddWaveFunc_Shredinger}, the measurement process has a random
character and different equations must be used to describe it. 

\input ./add/quant/figmeasur.tex

Let us first consider pure states \rindex{Pure state}
and suppose that a measurement is performed
of a physical observable described by the operator $\hat{L}$. The eigenvalues and eigenfunctions of this operator are $\left\{ l_k \right\}$ and 
$\left\{ \ket{l_k} \right\}$ respectively. At the moment
of measurement, the instrument readout can take values corresponding to
the eigenvalues of the measured operator
(see \autoref{figAddMeasur}). Suppose that the readout 
is $l_n$, in which case the wave function must be 
$\ket{l_n}$, i.e. the following
change of the wave function occurred:
\[
\left| \phi \right> \rightarrow \ket{l_n},
\] 
which can be described by the action of the projection operator 
$\hat{P}_n = \ket{l_n} \bra{l_n}$ \eqref{eqDiracProektor}:
\[
\hat{P}_n \left| \phi \right> = c_n\ket{l_n}.
\]

\begin{example}[Measurement of energy of a two-level atom]
Consider a two-level atom in a pure state
(see \autoref{fig:add:mesure_ex}) 
\(
\left|\psi\right> = \frac{1}{\sqrt{2}}\ket{a}
+ \frac{1}{\sqrt{2}}\ket{b}
\).

\input ./add/quant/figmeasurex.tex

Our instrument measures the energy of this atom and the Hamiltonian operator has 2 eigenfunctions $\ket{a,b}$, corresponding to eigenvalues $E_a, E_b$. Thus, possible
readings of the instrument belong to the set $\{E_a, E_b\}$. 

\input ./add/quant/figmeasurex_a.tex

When the instrument pointer shows $E_a$, the following collapse occurs  (see \autoref{fig:add:mesure_ex_a})
\[
\left|\psi\right> \to \ket{a}.
\]

\input ./add/quant/figmeasurex_b.tex

Similarly, when the reading is $E_b$, the following collapse occurs  (see \autoref{fig:add:mesure_ex_b})
\[
\left|\psi\right> \to \ket{b}.
\]
\end{example}

There is no way to predict the result that will be obtained from a single measurement. However,
one can specify the probability of each possible result.

Indeed, in the case of a mixed state
\begin{equation}
\hat{\rho} = 
\sum_n \left|c_n\right|^2 \ket{l_n}\bra{l_n}
\nonumber
\end{equation}
the coefficients $P_n = \left|c_n\right|^2$
give the probabilities to find the system in state $\ket{l_n}$. 

For a pure state
\begin{equation}
\left| \phi \right> = 
\sum_n c_n \ket{l_n}
\nonumber
\end{equation}
we also have that the probability to find the system in state
$\ket{l_n}$ is given by the number $P_n = \left|c_n\right|^2$. 

The main difference between pure and mixed states from the point of view of measurement
is that in the former (pure states) the wave function and thus the state itself changes during
measurement. In particular,
if a certain final state $\ket{l_i}$ was obtained during measurement, one cannot say that it was the same before
measurement. Mixed states \rindex{Mixed state}
behave like classical objects,
i.e. if a state $\ket{l_i}$ was obtained during measurement, it can be claimed that it was the same before
measurement and the measurement itself represents choosing
one state from many possible ones.

\begin{example}
\emph{Choosing from an urn with balls of two colors}
Suppose we have an urn with 4 balls. With probability $\frac{1}{2}$
either a white or a black ball will be drawn. Suppose that as a result
of the experiment a black ball was obtained. If the system under consideration
is quantum and in a mixed state 
(see \autoref{figAddMixStateExample}), then the state
of the drawn ball (color) did not change as a result of the experiment. 

\input add/quant/figmixstateexample.tex
\input add/quant/figpurestateexample.tex

If the system under consideration is
pure (see \autoref{figAddPureStateExample}), then the state of each
ball is described by a superposition of two colors - black and white. Thus,
as a result of the experiment this superposition collapses and the ball
acquires a definite color (black in our case), i.e., one can say that the color
of the ball changes.
\end{example}