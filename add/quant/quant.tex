%% -*- coding:utf-8 -*- 
\section{Quantization Procedure}
As we have already found out, physical quantities in quantum mechanics
correspond to self-adjoint operators. If we know the wave function
(for example, by means of the Schrödinger equation), then knowing the expression
for the operator of the physical quantity of interest, we can perform
practical calculations, such as expectation values.

The procedure of deriving the relations that operators obey is called
quantization.

Below is information about the connection between classical mechanics and the quantization procedure,
as well as an example of quantizing the angular momentum operator.

\subsection{Generalized Coordinates and Momenta}

To fully describe a mechanical system, it is necessary to specify its
coordinates and momenta. The coordinate is not always a classical coordinate, for example, positions in Cartesian coordinate system $(x,y,z)$.
Sometimes it is convenient to use special coordinates to describe the system under consideration,
such as an angle in polar coordinates. In this case, the question arises of what should be considered
the momentum corresponding to such a coordinate. To answer this question,
one should consider the Lagrangian, which is defined as the difference
between kinetic and potential energies:
\[
\mathcal{L} = T - U.
\]
If we have a set of generalized coordinates ${q_n}$, then the Lagrangian
$\mathcal{L}$ can be presented as a function of these variables and their
time derivatives:
\[
\mathcal{L} = \mathcal{L}\left(q_1, q_2, \dots, \dot{q}_1, \dot{q}_2,
\dots \right).
\]

We can define the generalized momentum $p_i$ corresponding to the generalized
coordinate $q_i$ as
\[
p_i = \frac{\partial \mathcal{L}}{\partial \dot{q}_i}.
\]

In quantum mechanics, the equations of motion in Hamiltonian form play a special role:
\begin{eqnarray}
\dot{q}_i = \frac{\partial \mathcal{H}}{\partial p_i},
\nonumber \\
\dot{p}_i = - \frac{\partial \mathcal{H}}{\partial q_i},
\label{eq:add:quantel:hamilton}
\end{eqnarray}
where $\mathcal{H}$ is the Hamiltonian (total energy) of the system as a function
of generalized coordinates and momenta.


\begin{example}[Rectilinear Motion of a Material Point]
Suppose we have a material point moving along the
$x$-axis under the action of a force $F = - \frac{d U}{d x}$, where $U$ is
the potential energy. In this case:
\[
\mathcal{L} = T - U = \frac{m \dot{x}^2}{2} - U(x).
\] 
The generalized momentum $p$ corresponding to coordinate $x$ is
\[
p = \frac{\partial \mathcal{L}}{\partial \dot{x}} = m \dot{x},
\]
which coincides with the classical definition of momentum.

The Hamiltonian is
\[
\mathcal{H}\left(x, p\right) = T + U(x) = \frac{p^2}{2 m} + U(x).
\]

Equations of motion:
\begin{eqnarray}
\dot{x} = \frac{\partial \mathcal{H}}{\partial p} = \frac{p}{m},
\nonumber \\
\dot{p} = - \frac{\partial \mathcal{H}}{\partial x} = - \frac{d U}{d
  x} = F,
\nonumber
\end{eqnarray}
where the first equation matches the definition of momentum, and the second
represents the well-known Newton's law
\[
F = m \ddot{x}.
\]
\end{example}

\subsection{Quantization of Angular Momentum}

\input ./add/quant/figquant.tex
 
\subsubsection{Classical Mechanics}
The system under consideration consists of a material particle moving on a
circle. The generalized coordinates describing the particle are the angle
$\theta$ and the radius of the circle $r$, which is considered constant: $r =
\left. const \right|_t$ (see \autoref{figAddQuantAngleMoment}).

\rindex{Hamiltonian}
The Hamiltonian of the system is given by
\begin{eqnarray}
\mathcal{H} = T + U = \frac{m v^2}{2} + U\left( r \right) = 
\nonumber \\
= \frac{m r^2 \dot{\theta}^2 }{2} + U\left( r \right),
\nonumber
\end{eqnarray}
where $T = \frac{m v^2}{2}$ is the kinetic energy of the particle,
and $U$ is the potential energy, which, due to the symmetry of the problem,
does not depend on the angle $\theta$ and depends only on the distance $r$.

The Lagrangian of the system is
\[
\mathcal{L} = T - U = \frac{m r^2 \dot{\theta}^2 }{2} - U\left( r \right).
\]
The angular momentum (generalized momentum corresponding to the generalized
coordinate $\theta$) is defined as
\begin{equation}
l = \frac{d \mathcal{L}}{d \dot{\theta}} = 
m r^2 \dot{\theta} = I \dot{\theta},
\label{eqAngualrMomentumClass}
\end{equation}
where $I$ denotes $I = m r^2$ — the moment of inertia.

Equations of motion for the coordinate $r$:
\[
\frac{ \partial \mathcal{L} }{\partial r} = 
\frac{d}{d t} \frac{\partial \mathcal{L}}{\partial \dot{r}},
\]
from which
\begin{eqnarray}
\frac{ \partial \mathcal{L} }{\partial r} = 0,
\nonumber \\
\frac{ \partial T }{\partial r} - \frac{ \partial U }{\partial r} = 0,
\nonumber \\
\frac{\partial U}{\partial r} = m r \dot{\theta}^2,
\nonumber
\end{eqnarray}
or equivalently
\[
 U = \frac{m r^2 \dot{\theta}^2}{2} = \frac{l^2}{2 I}.
\]
 
Thus, the Hamiltonian \rindex{Hamiltonian} of the system under consideration
is
\begin{eqnarray}
\mathcal{H} = \frac{m r^2 \dot{\theta}^2 }{2} + \frac{l^2}{2 I} = 
\nonumber \\
= \frac{l^2}{2 I} + \frac{l^2}{2 I} = \frac{l^2}{I}
\label{eqHClassical}
\end{eqnarray}

\subsubsection{Quantization}

Let $\left|\psi\right>$ be an eigenfunction of the angular momentum operator $\hat{L}$
corresponding to the eigenvalue $l$:
\begin{equation}
\hat{L} \left|\psi\right> = l \left|\psi\right>.
\label{eqLPsi}
\end{equation}
This same wave function must satisfy the Schrödinger equation
\[
i \hbar \frac{\partial \left|\psi\right>}{ \partial t} = 
\hat { \mathcal{H} } \left|\psi\right>.
\]
From \eqref{eqHClassical}, we have
\[
\hat { \mathcal{H} } = \frac{\hat{L}^2}{I},
\]
and taking into account \eqref{eqLPsi},
\[
i \hbar \frac{\partial \left|\psi\right>}{ \partial t} = 
\frac{1}{I} \hat{L} \hat{L} \left|\psi\right> = 
\frac{l^2}{I} \left|\psi\right>,
\]
or equivalently
\[
\frac{\partial \left|\psi\right>}{ \partial t} = 
\frac{-i l^2}{\hbar I} \left|\psi\right>.
\]
Thus,
\[
\left|\psi\left(t \right)\right> = C \cdot \exp\left\{\frac{-i l^2}{\hbar
    I} t
\right\}.
\]
The wave function must satisfy the periodicity condition
\[
\left|\psi\left(t \right) \right>= \left|\psi\left(t + T \right)\right>,
\]
where $T$ is the period of oscillations. It can be found from
\eqref{eqAngualrMomentumClass}:
\begin{eqnarray}
\dot{\theta} = \frac{l}{I},
\nonumber \\
\theta = \theta_0 + \frac{l \cdot t}{I},
\nonumber \\
2 \pi = \frac{l \cdot T}{I},
\nonumber \\
T = \frac{2 \pi I}{l}.
\nonumber 
\end{eqnarray}
Thus,
\[
2 i \pi n = \frac{-i l^2}{\hbar I} T = 
\frac{2 \pi I}{l} \frac{-i l^2}{\hbar I},
\]
from which
\[
l = -\hbar n.
\]