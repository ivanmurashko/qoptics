%% -*- coding:utf-8 -*- 
\section{Dirac Formulation of Quantum Mechanics}
\label{AddDirac}
In the course of lectures on quantum optics, we will everywhere use the 
Dirac formalism \cite{bDiracPrincipleQuantumMechanic}. In the usual formulation of quantum mechanics, we deal with wave functions, for example $\psi\left(q, t\right)$ - the wave function in the coordinate representation. The same state of the system can be described by wave functions in different representations related by linear transformations. For instance, the wave function in the momentum representation is related to the wave function in the coordinate representation by the equality 
\begin{equation}
\phi\left(p, t\right) = \frac{1}{2 \pi \hbar} \int_{-\infty}^{+\infty}
\psi \left(q, t\right) e^{-i \frac{p q}{\hbar}} dq
\end{equation}
The main point here is that the same state can be described by wave functions expressed through different variables. From this, it follows that one can introduce a more general entity that characterizes the state of the system independently of the representation. For such an entity, Dirac introduced the concept of the wave vector, or state vector, denoted as: 
\begin{equation}
\left| \dots \right>
\end{equation}
and called a ket-vector.

\subsection{Ket-vector}
$\left| \dots \right>$ is a general notation for a ket-vector; $\left| a \right>$,  $\ket{ x }$, $\left| \psi \right>$, etc. denote ket-vectors describing some particular states, whose symbols are written inside the brackets.

\subsection{Bra-vectors}
Each ket-vector has a conjugate bra-vector. The bra-vector is denoted: 
\begin{equation}
\left< \dots \right|, \quad 
\bra{ a }, \quad  
\left< \psi \right|.
\end{equation}

The names bra- and ket-vectors come from the first and second halves of the English word {\itshape bra-cket}.

Thus, to bra-vectors
$\bra{ a }$,  $\bra{ x }$, $\bra{ \psi }$
correspond conjugate ket-vectors  
$\ket{ a }$,  $\ket{ x }$, $\ket{ \psi }$
and vice versa. For state vectors, the same fundamental relations hold as for wave functions:  
\begin{equation}
\ket{ u } = \ket{ a }  + \ket{ b }, \quad 
\bra{ u } = \bra{ a }  + \bra{ b }, \quad 
\ket{ v } = l \ket{ a }, \quad  
\bra{ v } = l \bra{ a }.
\end{equation}
Bra- and ket-vectors are connected by the operation of Hermitian conjugation:
\begin{equation}
\ket{ u } = \left( \bra{ u } \right)^{\dag}, \quad 
\bra{ u } = \left( \ket{ u } \right)^{\dag}.
\end{equation}

In known cases, this reduces to the following relations:
\[
\left( \psi\left( q \right) \right)^{\dag} = \psi^{*}\left( q \right)
\]
for the wave function in the coordinate representation;
\[
\left(
\begin{array} {c} 
a_1 \\
a_2 \\
\vdots \\
a_n
\end{array}
 \right)^{\dag} = 
\left( a_1^{*}, a_2^{*}, \cdots, a_n^{*}\right)
\]
in the matrix representation.

Using bra- and ket-vectors, we can define the scalar product
\begin{equation}
\bra{ v }\ket{ u } = \bra{ u }\ket{ v }^{*}.
\label{eqAddDirac_swap}
\end{equation}

In specific cases, this means:
\[
\left< \psi \right|\left. \phi \right> = 
\int \psi^{*} \phi dq
\]
in the coordinate representation;
\[
\bra{ a }\ket{ b } = 
\left( a_1^{*}, a_2^{*}, \cdots, a_n^{*}\right) 
\left(
\begin{array} {c} 
b_1 \\
b_2 \\
\vdots \\
b_n
\end{array}
 \right) = 
a_1^{*} b_1 +  a_2^{*} b_2 + \cdots + a_n^{*} b_n
\]
in the matrix representation.

From relation \eqref{eqAddDirac_swap} it follows that the norm of a vector is real. Additionally, we assume that the norm of the vector is positive or zero: 
$\bra{ a }\ket{ a } \geq 0$.

\subsection{Operators}
In quantum mechanics, linear operators are used. Operators connect one state vector with another: 
\begin{equation}
\ket{ q } = \hat{L}\ket{ p }
\label{eqAddDirac_operator_property1}
\end{equation}
The conjugate equality has the form
\begin{equation}
\bra{ q } = \bra{ p }  \hat{L}^{\dag}
\label{eqAddDirac_operator_property2}
\end{equation}
where $\hat{L}^{\dag}$ is the operator conjugate to $\hat{L}$.

Let us give some relations valid for linear 
operators:
\begin{eqnarray}
\hat{L}^{++} = \hat{L}, \quad
\left(l \hat{L} \ket{ a } \right)^{\dag} = 
l^{*} \bra{ a } \hat{L}^{\dag}, 
\nonumber \\
\left(\left(\hat{L_1} + \hat{L_2} \right) \ket{ a } \right)^{\dag} = 
\bra{ a } \left(\hat{L_1}^{\dag} + \hat{L_2}^\dag \right), 
\nonumber \\
\left(\left(\hat{L_1} \hat{L_2} \right) \ket{ a } \right)^{\dag} = 
\bra{ a } \left(\hat{L_2}^{\dag} \hat{L_1}^\dag \right),
\nonumber \\
\left(\left(\hat{L_1} \hat{L_2} \hat{L_3}\right) \ket{ a } \right)^{\dag} = 
\bra{ a } \left(\hat{L_3}^{\dag} \hat{L_2}^\dag \hat{L_1}^\dag \right), 
\mbox{ and so on}
\label{eqAddDirac_propert}
\end{eqnarray}

Note that the algebra of operators coincides with the algebra of square matrices. The matrix elements of operators are denoted as follows: 
\begin{equation}
\bra{a}\hat{L}\ket{b} = L_{ab}
\end{equation}

For matrix elements, the following equalities hold:
\begin{equation}
\bra{a}\hat{L}\ket{b}^{*} = 
\bra{b}\hat{L}^{\dag}\ket{a}, \quad
\bra{a}\hat{L_1}\hat{L_2}\ket{b}^{*} = 
\bra{b}\hat{L_2}^{\dag}\hat{L_1}^\dag\ket{a}
\end{equation}


\subsection{Eigenvalues and eigenvectors of operators} 
Eigenvalues and eigenvectors of operators are defined by the equality
\begin{equation}
\hat{L} \ket{l_n} = l_n \ket{l_n},
\end{equation}
where $l_n$ is an eigenvalue; $\ket{l_n}$ an eigenvector.

For bra-vectors we have analogous equalities:
\begin{equation}
\bra{d_n} \hat{D}  = d_n \bra{d_n}.
\end{equation}

If operators correspond to observables, they must be self-adjoint:
\begin{equation}
\hat{L}  = \hat{L}^{\dag}.
\label{eqAddDirac_ermit}
\end{equation}

The eigenvalues of a self-adjoint (Hermitian) operator are real. Indeed, from 
\[
\hat{L} \ket{ l } = l \ket{ l }
\]
it follows that 
\[
\bra{ l } \hat{L} \ket{ l } = l \bra{ l }
\ket{ l }.
\]
On the other hand, recalling \eqref{eqAddDirac_propert}:
$\bra{ l } \hat{L}^{\dag} = l^{*} \bra{ l }$, from
\eqref{eqAddDirac_ermit} we have
\[
\bra{ l } \hat{L} \ket{ l } = l^{*} \bra{ l }
\ket{ l }.
\] 
Thus $l\bra{ l }
\ket{ l } = l^{*}\bra{ l }
\ket{ l }$, i.e. $l  = l^{*}$.

Eigenvectors of a self-adjoint operator are orthogonal. 
Indeed, consider two eigenvectors 
$\ket{ l_1 }$ and $\ket{ l_2 }$:
\[
\hat{L} \ket{ l_1 } = l_1 \ket{ l_1 }, \quad
\hat{L} \ket{ l_2 } = l_2 \ket{ l_2 }
\]
From the second relation we get
\[
\bra{ l_1 } \hat{L} \ket{ l_2 } = l_2 \bra{ l_1 } \ket{ l_2 }
\]
Taking into account the reality of eigenvalues and relation
\eqref{eqAddDirac_ermit} for the vector $\ket{ l_1 }$ we obtain:
\[
\bra{ l_1 } \hat{L} = l_1 \bra{ l_1 }.
\]
Hence
\[
\bra{ l_1 } \hat{L} \ket{ l_2 } = l_1 \bra{ l_1 } \ket{ l_2 }.
\] 
Thus
\[
\left(l_1 - l_2\right) \bra{ l_1 } \ket{ l_2 } = 0, 
\quad \mbox{i.e. } \bra{ l_1 } \ket{ l_2 } = 0,
\mbox{ since } l_1 \neq l_2.
\] 

\subsection{Observables. Expansion in eigenvectors. Completeness of set of eigenvectors}
Operators corresponding to physical observables are self-adjoint operators. This ensures the reality of the values of the observable physical quantity. We have a set of eigenstates of some Hermitian operator  
$\ket{ l_n }$,  $\hat{L} \ket{ l_n } = l_n \ket{ l_n }$.  If the set of eigenstates is complete, according to the principles of quantum mechanics any state can be represented as a superposition
of states $\ket{ l_n }$:
\begin{equation}  
\left| \psi \right> = \sum_{(n)} c_n \ket{ l_n }.
\end{equation}  

From here for the expansion coefficients we have:  
$c_n = \bra{ l_n } \left. \psi \right>$, and thus,
the equality holds 
\begin{equation}  
\left| \psi \right> = \sum_{(n)} \bra{ l_n } \left. \psi
\right> \ket{ l_n } = 
\sum_{(n)} \ket{ l_n } \bra{ l_n } \left. \psi
\right>.
\label{eqAddDirac_full}
\end{equation}  

From equality \ref{eqAddDirac_full} follows an important relation:
\begin{equation}  
\sum_{(n)} \ket{ l_n } \bra{ l_n } = \hat{I}.
\label{eqAddDiracI}
\end{equation}  
where $\hat{I}$ is the identity operator. This equality is the condition of completeness of the set of eigenvectors (the condition of decomposability).

\subsection{Projection operator}
\label{AddDiracProjector}

Consider the operator \(\hat{P}_n = \ket{ l_n } \left< l_n
\right|\). 
The result of action of this operator on the state 
\(\left| \psi \right>\) will be
\begin{equation}
\hat{P}_n \left| \psi \right> = \sum_{(k)} \ket{ l_n } \left<
l_n \right| c_k \ket{ l_k } = c_n \ket{ l_n }.
\label{eqDiracProektor}
\end{equation}
The operator \(\hat{P}_n = \ket{ l_n } \bra{ l_n }\) is called
the projection operator.

One can write the following properties of this operator
\begin{equation}  
\sum_{(n)} \hat{P}_n = \hat{I}.
\end{equation}  

\begin{equation}  
\hat{P}_n^2 = \hat{P}_n.
\end{equation}  

\input ./add/quant/figproject.tex
The action of the projection operator has a simple geometric
interpretation (see \autoref{figAddProject}):
\[
\hat{P}_n\left|\psi\right> = \cos{\theta} \ket{l_n},
\]
where $\cos{\theta} = \left<\psi|l_n\right> = c_n$. 

\subsection{Trace of an operator}
\label{AddDiracTrace}
In an orthonormal basis \(\left\{\ket{l_n}\right\}\), 
the quantity 
\begin{equation}  
Sp \hat{L} = \sum_n \bra{l_n} \hat{L} \ket{l_n}
\label{eqAddDiracTr}
\end{equation}  
is called the trace of the operator \(\hat{L}\). Under certain conditions
\cite{bTraceClassOperatorAdd1}, the series \ref{eqAddDiracTr}
converges absolutely and does not depend on the choice of basis.

If we use the matrix representation 
\[
L_{kn} = \bra{l_k} \hat{L} \ket{l_n}, 
\]
then the trace of the operator is the sum of the diagonal elements of the matrix 
representation
\[
Sp \hat{L} = \sum_n L_{nn}
\]

One can write the following properties of the trace:
\begin{eqnarray}
Sp\left(l \hat{L} + m \hat{M}\right) = 
l Sp \hat{L} + m Sp \hat{M},
\nonumber \\
Sp\left(\hat{L}\hat{M}\right) = 
Sp\left(\hat{M}\hat{L}\right).
\label{eqAddDiracTrProperty}
\end{eqnarray}

\subsection{Mean values of operators}
The mean value of operator $\hat{L}$ in the state $\left| \psi \right>$ is given by the equality 
\begin{equation}  
\left< \hat{L} \right>_{\psi} = \bra{\psi}\hat{L}\ket{\psi}
\label{eqAddDiracMid}
\end{equation}  
under the condition
\[
\bra{\psi}\ket{\psi} = 1.
\]

Indeed, if one assumes that $\ket{\psi}$ is expanded in the series of eigenfunctions of the operator $\hat{L}$ as follows:
\[
\ket{\psi} = \sum_n c_n \ket{l_n},
\]
then $\hat{L}\left|\psi\right>$ can be written as
\[
\hat{L}\ket{\psi} = \sum_n l_n c_n \ket{l_n},
\]
where $l_n$ is the eigenvalue corresponding to the eigenstate 
$\ket{l_n}$. 
If now the last two expressions are substituted into \eqref{eqAddDiracMid}
one obtains:
\[
\bra{\psi}\hat{L}\ket{\psi} = \sum_{n,m} 
l_n c_n c_m^{*} \bra{l_m}\ket{l_n}=
\sum_n l_n c_n c_n^{*} = 
\sum_n l_n \left|c_n\right|^2, 
\]
which (under the condition $\left<\psi\right.\left|\psi\right> = 1$) proves
that expression \eqref{eqAddDiracMid} indeed 
represents the expression for the mean value of the operator 
$\hat{L}$   in the state $\left|\psi\right>$.
\footnote{For this, it is enough to recall that $\left|c_n\right|^2$
  gives the probability to find the system in the state $\ket{l_n}$,
  i.e., to obtain a reading of the measuring device equal to $l_n$}

If one takes some orthonormal basis $\{\ket{ n }\}$,
forming a complete set, i.e., satisfying the condition
\eqref{eqAddDiracI}: $\sum_n \ket{ n }\bra{ n } =
\hat{I}$, then expression \eqref{eqAddDiracMid}
can be rewritten as follows:
\begin{eqnarray}
\left< \hat{L} \right>_{\psi} = 
\bra{\psi}\hat{L}\ket{\psi} = 
\bra{\psi}\hat{I}\hat{L}\ket{\psi} = 
\nonumber \\
= 
\sum_n \bra{\psi}\ket{n}\bra{n}
\hat{L}\ket{\psi} = 
\sum_n \bra{n}
\hat{L}\ket{\psi}\bra{\psi}\ket{n} = 
Sp \left(\hat{L} \hat{\rho} \right),
\nonumber
\end{eqnarray}
where 
\(
\hat{\rho} = \ket{\psi}\bra{\psi} = \hat{P}_{\psi}
\) is the projection operator onto the state 
$\left| \psi \right>$.
Taking into account \eqref{eqAddDiracTrProperty}, one can write
\begin{equation}
\left< \hat{L} \right>_{\psi} = Sp \left(\hat{\rho} \hat{L} \right).
\label{eqAddDiracMidViaRho}
\end{equation}

\subsection{Representation of operators through outer products of eigenvectors}
Using the completeness condition \eqref{eqAddDirac_full} twice, we get
\begin{equation}
\hat{A} = \hat{I} \hat{A} \hat{I} = \sum_{(l)}\sum_{(l')} 
\ket{l}\bra{l} \hat{A} \ket{l'}\bra{l'} = 
\sum_{(l)}\sum_{(l')} 
\ket{l}\bra{l'} A_{ll'},
\end{equation}  
where $A_{ll'} = \bra{l} \hat{A} \ket{l'}$ is the matrix element of the operator $\hat{A}$ in the $\ket{l}$ representation.
 
The operator expressed through its own eigenvectors can be represented by the expansion \footnote{Under the condition
 of normalization of eigenvectors: $\bra{l}\ket{l} = 1$} 
\begin{equation}
\hat{L} = \sum_{(l)} 
l \ket{l}\bra{l}.
\end{equation}  

The generalization of this equality for an operator function has the form
\begin{equation}
F\left(\hat{L}\right) = \sum_{(l)} 
F\left(l\right) \ket{l}\bra{l}.
\label{eqAddDiracFL}
\end{equation}  

\subsection{Wave functions in coordinate and momentum representations}
Transition from the state vector to the wave function is carried out 
by scalar multiplying this state vector by the eigenvector of the corresponding observable. For example, for the wave function in the coordinate representation 
\begin{equation}
\phi\left(q\right) = \bra{q}\left.\psi\right>.
\end{equation}  
where $\bra{q}$ is the eigenvector of the coordinate operator. 
In the momentum representation, we get
\begin{equation}
\phi\left(p\right) = \bra{p}\left.\psi\right>.
\end{equation}  
where $\bra{p}$ is the eigenvector of the momentum operator.
