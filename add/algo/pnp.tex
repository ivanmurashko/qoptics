%% -*- coding:utf-8 -*-
\section{Complexity Classes $P$, $NP$}

When solving a particular problem, two questions arise primarily. Does the given problem have a solution in principle, i.e. can an algorithm be constructed that solves the given problem? Once a positive answer to the first question is found, i.e. an algorithm solving the problem is constructed, the next issue is the practical feasibility of the proposed solution. Here, the theory of algorithm complexity helps us address the feasibility of various algorithms.

In the following table \cite{bSchneier} 
% p. 34
some large numbers useful for analyzing algorithm complexity are given
\begin{table}
\centering
\begin{tabular}{|p {10cm}|p {5cm}|}
\hline
Physical analogy & Number \\ \hline
Probability of being struck by lightning (in one day) &
1 in 9 billion ($2^{33}$) \\ 
Probability of winning the grand prize in the US state lottery &
1 in 4,000,000 ($2^{22}$) \\
Probability of winning the grand prize in the US state lottery and being
struck by lightning on the same day &
1 in $2^{61}$ \\
Probability of drowning (in the USA within a year) &
1 in 59,000 ($2^{16}$) \\
Probability of dying in a car accident (in the USA in a year) &
1 in 6,100 ($2^{13}$) \\
Probability of dying in a car accident (in the USA during a lifetime) &
1 in 88 ($2^7$) \\
Time until next glaciation &
14,000 ($2^{14}$) years \\
Time until the Sun becomes a supernova &
$10^9$ ($2^{30}$) years \\
Age of the planet &
$10^9$ ($2^{30}$) years \\
Age of the Universe & 
$10^{10}$ ($2^{34}$) years \\
Number of atoms in the planet &
$10^{51}$ ($2^{170}$) \\
Number of atoms in the Sun &
$10^{57}$ ($2^{190}$) \\
Number of atoms in the galaxy & 
$10^{67}$ ($2^{223}$) \\
Number of atoms in the Universe &
$10^{77}$ ($2^{265}$) \\
Volume of the Universe &
$10^{84}$ ($2^{280}$) $\mbox{cm}^3$ \\
\hline
\multicolumn{2}{|l|}{If the Universe is finite:} \\
\hline
Total lifetime of the universe &
$10^{11}$ ($2^{37}$) years \\
& $10^{18}$ ($2^{61}$) seconds \\
\hline
\multicolumn{2}{|l|}{If the Universe is infinite:} \\
\hline
Time until cooling of low-mass stars &
$10^{14}$ ($2^{47}$) years \\
Time until planets detach from stars &
$10^{15}$ ($2^{50}$) years \\
Time until stars detach from galaxies & 
$10^{19}$ ($2^{64}$) years \\
Time until orbital decay by gravitational radiation &
$10^{20}$ ($2^{67}$) years \\
Time until black holes evaporate by Hawking processes &
$10^{64}$ ($2^{213}$) years \\
Time until matter becomes liquid at zero temperature &
$10^{65}$ ($2^{216}$) years \\
\hline
\end{tabular}
\caption{Large numbers \cite{bSchneier}}
\label{tblBigNumber}
\end{table}

\begin{example}
\emph{Prime factorization of integers by trial division}
\label{exAddAlgoTrialDivision}
Suppose we have the problem of factorizing an integer $x$
(see example \ref{exFactor}). 
In binary representation, $x$ consists of $N$ digits, i.e. 
$2^N \le x < 2^{N+1}$. In the simplest approach (trial division), we
can try all integers from 2 up to $2^{\frac{N}{2}}$, and for
each such integer $2 < y \le 2^{\frac{N}{2}}$ find the remainder of
division of $x$ by $y$. If $x$ is divisible by $y$, the problem is solved and $y$
is declared a divisor of $x$. If none of the numbers $y$ divide
$x$, then $x$ is declared prime.
Thus, the original problem is divided into $2^{\frac{N}{2}}$ subproblems involving computing
remainders.

Assume $x$ has 20 digits, i.e. $N=20$. In that case,
$2^{10} = 1024$ division operations need to be performed to find
a solution to the original problem, which can be easily done on simple
computers. 

However, if the number of digits increases 30-fold to $N=600$, then to
solve the given problem, one must perform $2^{300}$ division operations. If you look at Table \ref{tblBigNumber}, you can see
that this number is much larger than the number of atoms in the universe,
which is on the order of $2^{265}$. That is, even if each atom in the universe represents
a very simple computational device that computes the remainder of division of two integers,
using all atoms in the universe would be insufficient to solve the problem.

Thus, despite the existence of a simple algorithm, the problem of prime factorization,
using this algorithm, becomes practically infeasible for numbers of length 600 bits,
even though the number 600 does not seem very large compared to 20 (where the task is practically feasible).
\end{example}


\subsection{Hierarchy of Complexity Classes}

Below are formal definitions for problems that are
practically feasible, $P$. The class of problems that are
practically infeasible (until it is proven that
$P = NP$) is also defined.

\begin{definition}[Problem size]
The size of the problem $N$ is defined as the number of symbols from the alphabet of the Turing machine
contained on the tape in the initial state, that is, in other
words, the size of the problem is considered to be the length of the string input to the Turing machine.
\end{definition}

For example, in example \ref{exAddAlgoTrialDivision}, the problem size is
the length of number $x$ (whose divisors are being sought) in binary
representation. 

\begin{definition}[Class $P$]
A problem (algorithm) belongs to class $P$ (Polynomial) if it can be solved in 
 $O\left(N^k\right)$ steps on a deterministic Turing machine.
Here $N$ is the size of the original problem, $k$ is any integer
independent of $N$.  
\end{definition}

The class $P$ defines the class of problems that are practically
feasible. For example, the problem of sorting unordered data
(a list) can be solved in 
$O\left(N \log N \right) = O\left(N^2\right)$ steps, where $N$ is the number
of elements in the list to be sorted (the problem size).

\begin{definition}[Class $NP$]
A problem (algorithm) belongs to class $NP$ (Nondeterministic Polynomial) if
it can be solved in  $O\left(N^k\right)$ steps on
a nondeterministic Turing machine. Here $N$ is the size of the original
problem, $k$ is any integer independent of $N$. 
\end{definition}

One example of an $NP$ problem is integer factorization.
A nondeterministic Turing machine can guess a solution (in
one step) and verify it (multiply the two guessed numbers and compare
the product to the original number) in polynomial time. At the same
time, the question of whether the factorization problem belongs to class $P$
remains open.

\input ./add/algo/figpnp.tex

The problem of equality of classes $P$ and $NP$ is one of the most important
unsolved problems in mathematics and probably the most important unsolved problem
in algorithm theory. To show how an $NP$ problem can be reduced to a $P$ problem,
it is necessary to introduce the class of $NP$-complete problems,
which are, in some sense, the hardest $NP$ problems.

\begin{definition}[Class $NPC$ ($NP$-complete)]
A problem $M$ belongs to class $NPC$ if any $NP$ problem
can be reduced to $M$ (see \autoref{figAddAlgoNPC}).
\end{definition}

\input ./add/algo/fignpc.tex

Thus, if a polynomial algorithm is found for at least
one $NPC$ problem, then polynomial algorithms will be found
for all $NP$ problems (see \autoref{figAddAlgoPNP}). 
 

\input ./add/algo/cooklevin.tex