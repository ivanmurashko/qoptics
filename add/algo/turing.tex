%% -*- coding:utf-8 -*-
\section{Turing Machine}
\label{addTuring}

A Turing Machine $TM$ is a mathematical object consisting of
a control device (head) and an infinite tape divided into
cells, each of which can hold one symbol.

\input ./add/algo/figturing.tex

The control device $TM$ can be in one of the states $q_i
\in Q$, where $Q$ 
is a set of different states. The state in which $TM$ was at
the initial time $q_0 \in Q$ is called the initial state.

Among the elements of the set $Q$, a subset $F \subset Q$
of final states is selected. If $TM$ is in a state $f \in F \subset
Q$, it is said that the Turing Machine has terminated its operation.

The symbols recorded on the tape form a certain set (alphabet)
$\Gamma = \left\{\gamma_i\right\}$.

The input alphabet $\Sigma$ of the Turing Machine is a 
subset of the elements $\Gamma$ with which the control device $TM$
operates, i.e., the symbols it can recognize on the tape. Those symbols
that the control device cannot recognize will be denoted by the symbol
$\beta$ (blank).

In the process of operation, $TM$ performs the following actions:
\begin{itemize}
\item{reads the element pointed to by the control device
  (see \autoref{figAddAlgoTuring})}
\item{changes (or leaves unchanged) this symbol to an element
  $\sigma \in \Sigma \subset \Gamma$}
\item{changes the position of the control device by one cell to the right
  ($R$), to the left ($L$) or does not change it ($S$)}
\end{itemize}

\input ./add/algo/figturingtrans.tex

The change of state $TM$ is described by a transition table
$\Delta$. An element of the transition table is recorded in the following form: 
$\delta_j\left(q_{old}, \gamma_{old}\right)$ where $q_{old} \in Q$ is the current state,
$\gamma_{old} \in \Gamma$ is the selected element on the tape. The value
of the transition can be 
undefined or recorded in the form $\left\{q_{new}, \gamma_{new},
d\right\}$ (see \autoref{figAddAlgoTuringTrans}), where
$q_{new} \in Q$ is the new state of the Turing Machine, $\gamma_{new} \in \Gamma$ is the
new value of the element on the tape pointed to by the control device,
i.e., $\gamma_{old}$ is replaced by $\gamma_{new}$. $d \in D =
\left\{L, R, S\right\}$ is the direction of 
movement of the control device ($L$ left, $R$ right, $S$
stay in the same place). In the case of an undefined
transition, it is said that the Turing Machine has halted in state $q_{old}$.

\begin{definition}[Language of the Turing Machine $L\left(M\right)$]
The original symbols on the tape can be interpreted as a certain string $x$
that is fed to the input of the Turing Machine $M$. All those strings that
take the machine $M$ to a certain final state $f \in F$
are called the language $L\left(M\right)$ of machine $M$
\end{definition}

\begin{definition}[Deterministic Turing Machine]
If in the transition table the mapping
\(
\left(q_{old}, \gamma_{old}\right) \rightarrow 
\left\{q_{new}, \gamma_{new}, d\right\}
\)
is a bijection 
\footnote{Bijection is a one-to-one mapping, i.e., in our
  case, each state $\left(q_{old}, \gamma_{old}\right)$
  corresponds to only one element of the transition table 
  $\left\{q_{new}, \gamma_{new}, d\right\}$
},
then the corresponding Turing Machine
is called a deterministic Turing Machine
\label{defAlgoDMT}
\end{definition}

\begin{definition}[Non-deterministic Turing Machine]
If in the transition table the mapping
\(
\left(q_{old}, \gamma_{old}\right) \rightarrow 
\left\{q_{new}, \gamma_{new}, d\right\}
\)
is not a bijection, i.e., the state $\left(q_{old}, \gamma_{old}\right)$
may correspond to several elements of the transition table 
$\left\{q_{new}, \gamma_{new}, d\right\}$,
then the corresponding Turing Machine
is called a non-deterministic Turing Machine
\label{defAlgoNDMT}
\end{definition}