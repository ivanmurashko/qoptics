%% -*- coding:utf-8 -*-
\section{Boolean Logic. Boolean Formulas in CNF}

\begin{definition}[Conjunction (logical "AND")]
For the logical operation "AND" ($a \& b$), the following truth table \ref{tblAddAlgoAND} holds.
\begin{table}
\centering
\begin{tabular}{|c|c|c|}
\hline
$a$ & $b$ & $a \& b$ \\ \hline
0  & 0 & 0 \\
0  & 1 & 0 \\
1  & 0 & 0 \\
1  & 1 & 1 \\ \hline
\end{tabular}
\caption{Conjunction $a \& b$}
\label{tblAddAlgoAND}
\end{table}
\end{definition}

\begin{definition}[Disjunction (logical "OR")]
For the logical operation "OR" ($a \| b$), the following truth table \ref{tblAddAlgoOR} holds.
\begin{table}
\centering
\begin{tabular}{|c|c|c|}
\hline
$a$ & $b$ & $a \| b$ \\ \hline
0  & 0 & 0 \\
0  & 1 & 1 \\
1  & 0 & 1 \\
1  & 1 & 1 \\ \hline
\end{tabular}
\caption{Disjunction $a \| b$}
\label{tblAddAlgoOR}
\end{table}
\end{definition}

\begin{definition}[Boolean Formula]
A Boolean formula is a set of Boolean literals combined by logical operations.
\end{definition}

\begin{definition}[CNF - Conjunctive Normal Form (SAT)]
In Boolean logic, a conjunctive normal form is a Boolean formula that is a conjunction of disjunctions of literals.
\end{definition}

\begin{example}
\emph{SAT Problem}
\label{exAddAlgoSAT}
The SAT problem is the problem of determining the satisfiability of CNF. Suppose the following Boolean formula is given in CNF form:
\begin{equation}
\left(x_1 \| x_2 \| \bar{x_3} \right) \& \left(x_1 \| \bar{x_2} \| x_3 \right)
\nonumber
\end{equation}
It is satisfiable; it is sufficient to set $x_1 = 1$, and the values of $x_{2,3}$ can be arbitrary.

Meanwhile, the following formula is unsatisfiable:
\begin{equation}
\left(x_1 \| x_2 \| \bar{x_3} \right) \& 
\left(\bar{x_1} \| \bar{x_2} \| x_3 \right).
\nonumber
\end{equation}

\end{example}

\begin{theorem}[Reducibility to CNF]
Any Boolean formula can be reduced to CNF. 
\end{theorem}

\begin{proof}
For the proof, you may use: the law of double negation, De Morgan's laws, distributivity.
\end{proof}

\begin{theorem}[Reducibility to 3-CNF (3-SAT)]
Any Boolean formula in conjunctive normal form can be reduced to 3-CNF.  
\end{theorem}

\begin{proof}
TBD
\end{proof}