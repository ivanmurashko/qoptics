%% -*- coding:utf-8 -*-
\subsection{Basic Concepts and Axioms of Probability Theory}

Classical probability theory deals with set theory and is based on several simple axioms. This axiomatics was proposed in the 1930s by Kolmogorov A. N. \cite{bKolmogorov74basic}. 

First of all, a few definitions of the objects with which we will be dealing. 

\begin{definition}[Event Space]
\label{def:events_set}  
  $\mathcal{F}$ is some $\sigma$-algebra defined on some
set $\Omega$ and is called an event space. 
\end{definition}

\begin{remark}[Event Space]
  By virtue of the properties of a $\sigma$-algebra, we have the following facts: if
  $\Omega \subset \Omega$, $\emptyset \subset \Omega$ and for any
  $A,B \subset \Omega$ it holds that
  $A \cup B \subset \Omega$, $A \cap B \subset \Omega$ and
  $A \setminus B \subset \Omega$.  An element of the set
  $\omega \in \Omega$ is called an elementary event. A subset 
  $A \subset \Omega$ or $A \in \mathcal{F}$ is called an event.
\end{remark}

\begin{definition}[Probability $P$]
  \label{def:probability}
  A finite measure defined on $\mathcal{F}$ over the set $\Omega$
  is called a probability if $P\left(\Omega\right) = 1$.
\end{definition}

\begin{example}[Event Space]
\input ./add/probability/figdefinition.tex
In \autoref{figAddProbabilityDefinition}, the set
$\Omega$ is depicted, which we call the event space. Elements
of the set $\omega_i \in \Omega$ (black dots in
\autoref{figAddProbabilityDefinition}) are elementary
events. A subset $A \subset \Omega$ is an event.
\end{example}

Now, strictly speaking, the axioms.

\begin{axiom}[Non-negativity]
  \label{axProbabilityKolmogorovNonNegativity}
  The probability of an event $A \subset \Omega$ is a non-negative
  real number, i.e. $P\left(A\right) \ge 0$
\end{axiom}

\begin{axiom}[Normalization]
  \label{ax:ProbabilityNormalization}
  The probability of the event space $\Omega$ is $1$, i.e.
  $P\left(\Omega\right) = 1$
\end{axiom}

\begin{axiom}[Additivity]
\label{axProbabilityAdditivity}
If $A_i \cap A_j = \emptyset$, then 
$P\left(A_i \cup A_j\right) = P\left(A_i\right) + P\left(A_j\right)$
\end{axiom}

\begin{example}[Axioms of Kolmogorov's Probability Theory]
\input ./add/probability/figaxioms.tex
Assign to each elementary event $\omega_i \in \Omega$
in \autoref{figAddProbabilityAxioms} a non-negative number 
$P\left(\omega_i\right) = \frac{1}{12}$. The probability of event $A$ is 
$P\left(A\right) = \frac{5}{12}$ and for event $B$ we have
$P\left(B\right) = \frac{4}{12}$. Thus, since 
$A \cap B = \emptyset$, using axiom
\ref{axProbabilityAdditivity}, we get:
\[
P\left(A\cup B\right) = 
P\left(A\right) + P\left(B\right) = 
\frac{5}{12} + \frac{4}{12} = \frac{3}{4}.
\]
For the event $\Omega$, by axiom \ref{ax:ProbabilityNormalization}
we have $P\left(\Omega\right) = 1$. On the other hand, 
$\Omega = \cup_i \omega_i$, i.e. 
\[
P\left(\Omega\right) = \sum_i P\left(\omega_i\right) =
12\cdot\frac{1}{12} = 1.
\]
\end{example}

All known facts of probability theory are derived from these three axioms.  

\input ./add/probability/figcond.tex

TBD