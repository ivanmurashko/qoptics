%% -*- coding:utf-8 -*-
\section{Quantum Probability Theory}
\label{sec:add:quantprobability}
When speaking about quantum probability theory, we will rely on the classical
(Kolmogorov) theory, while trying to choose a formulation
that is applicable in both cases. As a
model, we will consider a system with $n$ possible outcomes, i.e.
one where the classical probability distribution is given by
the following distribution:
\begin{equation}
p = \{p_1, p_2, \dots, p_n\}
\label{eq:quantprob_class}
\end{equation}

A quantum state is fully described in the general case (both pure
and mixed states) by the density matrix
\rindex{Density matrix}
\begin{equation}
\hat{\rho} = \begin{bmatrix}
\rho_{11} & \rho_{12} & \cdots & \rho_{1n} \\
\rho_{21} & \rho_{22} & \cdots & \rho_{2n} \\
\vdots & \vdots & \ddots & \vdots \\
\rho_{n1} & \rho_{n2} & \cdots & \rho_{nn} \\
\end{bmatrix}.
\nonumber
\end{equation}
In the classical case, this object corresponds
to the probability distribution \eqref{eq:quantprob_class}, which
can also be represented as a certain matrix
\begin{equation}
\hat{p} = \begin{bmatrix}
p_1 & 0 & \cdots & 0 \\
0 & p_2 & \cdots & 0 \\
\vdots & \vdots & \ddots & \vdots \\
0 & 0 & \cdots & p_n \\
\end{bmatrix}.
\nonumber
\end{equation}
It is worth noting that for both cases the equality holds
\[
Sp\left(\hat{\rho}\right) =
Sp\left(\hat{p}\right) = 1,
\]
which is a rewritten version of axiom
\ref{ax:ProbabilityNormalization}. 

An observable $\hat{x}$ corresponds to a certain matrix
\begin{equation}
\hat{x} = \begin{bmatrix}
x_{11} & x_{12} & \cdots & x_{1n} \\
x_{21} & x_{22} & \cdots & x_{2n} \\
\vdots & \vdots & \ddots & \vdots \\
x_{n1} & x_{n2} & \cdots & x_{nn} \\
\end{bmatrix},
\label{eq:quantprob_obser_quant}
\end{equation}
which in the classical case is diagonal 
\begin{equation}
\hat{x} = \begin{bmatrix}
x_1 & 0 & \cdots & 0 \\
0 & x_2 & \cdots & 0 \\
\vdots & \vdots & \ddots & \vdots \\
0 & 0 & \cdots & x_n \\
\end{bmatrix}.
\label{eq:quantprob_obser_class}
\end{equation}
It is also worth noting that the calculation of the mean then takes the familiar classical form
\begin{equation}
\bar{x} = Sp\left(\hat{p} \hat{x}\right) = \sum_{i=1}^n p_i x_i.
\nonumber
\end{equation}
\begin{remark}
  It is worth noting that for any observable $\hat{X}$, due to
  self-adjointness $\hat{X} = \hat{X}^\dag$, one can choose
  a basis in which the corresponding matrix will be diagonal, with
  eigenvalues on the diagonal. Moreover, if two
  observables $\hat{X}$ and $\hat{Y}$ commute:  
  \[
  \hat{X} \hat{Y} = \hat{Y} \hat{X},
  \]
  then they can be diagonalized in a common basis.
\end{remark}

The difference between quantum probability theory and classical theory can be
shown in several ways \cite{bHolevo2003, bHolevo2003add},
one of which is based on the noncommutativity of the quantum description.
Indeed, it can be noted that in the general case quantum observables,
represented by matrices of the form \eqref{eq:quantprob_obser_quant}, do not
commute with each other. At the same time, classical observables,
written as diagonal matrices
\eqref{eq:quantprob_obser_class} always commute with each other. From
this, many facts follow TBD


