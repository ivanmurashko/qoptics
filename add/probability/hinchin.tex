%% -*- coding:utf-8 -*- 
\subsection{Random Processes}
To fully describe a random process $x\left(t\right)$,
it is necessary to know all the joint probability density functions
\[
p_n\left(\left\{x\right\}_n,\left\{t\right\}_n\right) = 
p_n\left(x_1,x_2,\dots,x_n,t_1,t_2,\dots,t_n\right).
\] 
For example, using $p_1 = p_1\left(x,t\right)$, one can calculate the mean 
value of the random variable
\begin{equation}
\left<x\left(t\right)\right> = \int x p_1\left(x,t\right) dx,
\label{eqAddHinchin1}
\end{equation}
and using $p_2 = p_2\left(x_1,x_2,t_1,t_2\right)$ — the so-called
two-time autocorrelation function  
\begin{equation}
r\left(t_1, t_2\right) = \left<x^{*}\left(t_1\right) x\left(t_2\right)\right> = \int
\int x_1 x_2 p_2\left(x_2,t_2,x_1,t_1\right)dx_1 dx_2.
\label{eqAddHinchin2}
\end{equation}

In the case where the mean of the random variable \eqref{eqAddHinchin1} does not
depend on $t$:
\begin{equation}
\left<x\left(t\right)\right> = const_t,
\label{eqAddHinchinStatWide1}
\end{equation}
and the autocorrelation function \eqref{eqAddHinchin2} depends only on
the difference $t_1 - t_2 = \tau$:
\begin{equation}
r\left(t_1, t_2\right) = r\left(t_1 - t_2\right) = r\left(\tau\right),
\label{eqAddHinchinStatWide2}
\end{equation}
then the random process is called wide-sense stationary. Obviously, for such processes, the following
identity is valid:
\begin{eqnarray}
r\left(- \tau\right) = r^{*}\left(\tau\right),
\label{eqAddHinchinStatWide3}
\end{eqnarray}
indeed, from \eqref{eqAddHinchinStatWide2} we have
\begin{eqnarray}
r\left(- \tau\right) = r\left(t_2 - t_1\right) = r\left(t_2,
t_1\right) = r^{*}\left(t_1, t_2\right) = r^{*}\left(\tau\right).
\nonumber
\end{eqnarray}


For wide-sense stationary random processes, the following theorem holds:

\begin{theorem}[Khinchin-Wiener]
\label{thm:khinchin_wiener}
The autocorrelation function of wide-sense stationary random processes
and their spectral density are linked by the direct and inverse
Fourier transforms.

\begin{proof}
The proof will be informal and serves as an attempt to explain the meaning
of this theorem. Consider the following integral
\begin{equation}
\tilde{x}\left(\omega\right) = \frac{1}{2 \pi}
\int_{-\infty}^{\infty}x\left(t\right)e^{i \omega t}dt,
\label{eqAddHinchinFourier1}
\end{equation}
which can be interpreted as the Fourier transform (see \autoref{sec:cont_fourier})
of the random process
$x\left(t\right)$. For stationary random processes, it is obvious that
the integral \eqref{eqAddHinchinFourier1} generally does not exist.
Indeed, if this integral existed, the Parseval relation would hold:
\begin{equation}
\int_{-\infty}^\infty \left|\tilde{x}\left(\omega\right)\right|^2 d
\omega = \int_{-\infty}^\infty \left|x\left(t\right)\right|^2 dt,
\label{eq:add:hinchin:parseval}
\end{equation}
which obviously diverges in the case of a stationary random
process. At the same time, by virtue of \eqref{eq:add:hinchin:parseval}, 
$S\left(\omega\right) = \left|\tilde{x}\left(\omega\right)\right|^2$ represents
the distribution of energy along the frequency axis for a particular realization
of the process, and the averaged value
$\left<S\left(\omega\right)\right>$ represents a certain
characteristic of the random process that describes the energy distribution
over frequencies.

Although the integral \eqref{eq:add:hinchin:parseval} diverges for
the considered random processes, as shown by Khinchin and Wiener,
the integral
\begin{equation}
\left<\tilde{x}^{*}\left(\omega\right)\tilde{x}\left(\omega'\right)\right>
= 
\frac{1}{\left(2 \pi\right)^2}
\int_{-\infty}^{\infty}\int_{-\infty}^{\infty}\left<x^{*}\left(t\right)x\left(t'\right)\right>e^{i
  \left(\omega' t' - \omega t\right)}dtdt',
\label{eqAddHinchinFourier2}
\end{equation}
exists and can be interpreted as the power spectral density $S\left(\omega,\omega'\right)$
of the considered random process. Thus, taking into account stationarity,
\[
\left<x^{*}\left(t\right)x\left(t'\right)\right> = r\left(t' - t\right) = r\left(\tau\right),
\]
we obtain
\begin{eqnarray}
S\left(\omega, \omega'\right) =
\left<\tilde{x}^{*}\left(\omega\right)\tilde{x}\left(\omega'\right)\right>
= 
\nonumber \\
=
\frac{1}{\left(2 \pi\right)^2}
\int_{-\infty}^{\infty}\int_{-\infty}^{\infty}\left<x^{*}\left(t\right)x\left(t'\right)\right>e^{i
  \left(\omega' t' - \omega t\right)}dtdt' =
\nonumber \\
=
\frac{1}{\left(2 \pi\right)^2}
\int_{-\infty}^{\infty}\int_{-\infty}^{\infty}
r\left(\tau\right)
e^{i\left(\omega' - \omega\right) t}
e^{i\omega'\left(t' - t\right) }
dtd\tau =
\nonumber \\
=
\frac{1}{\left(2 \pi\right)^2}
\int_{-\infty}^{\infty}
e^{i \left(\omega' - \omega\right) t}dt 
\int_{-\infty}^{\infty}r\left(\tau\right)
e^{i \omega' \tau}d\tau =
\nonumber \\
= \tilde{r}\left(\omega'\right)\delta\left(\omega' - \omega\right),
\label{eqAddHinchinFourier3}
\end{eqnarray}
where the integral representation of the delta function
\eqref{eq:delta_from_integral} has been used.

Thus, from \eqref{eqAddHinchinFourier3} it follows that the Fourier transform
of the autocorrelation function $\tilde{r}\left(\omega\right)$ can
be regarded as the spectral density of the random process
\eqref{eqAddHinchinFourier2}. 
\end{proof}
\end{theorem}