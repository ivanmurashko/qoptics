%% -*- coding:utf-8 -*-
\chapter{Density Matrix and Operator}
\label{AddState}

\subsection{Pure and Mixed States}
\rindex{Mixed state}
\rindex{Pure state}
In general, the state of a quantum system is considered defined if the wave function describing this state is known. The wave function can be obtained as follows. We perform measurements of eigenvalues corresponding to a complete set of operators. The number of these measurable quantities (quantum numbers) is equal to the number of degrees of freedom of the system. With the help of the obtained quantum numbers, we find the wave function that is an eigenfunction for each of the operators in the complete set, with the eigenvalues corresponding to the measured quantum numbers. In general (up to a constant factor), such a function is unique. It can be assumed that this function describes the state at the initial moment of time. The state at subsequent moments of time can be found using the Schrödinger equation. Such a state with a well-defined wave function is called a \textbf{pure state}.

In some cases, the wave function of the system cannot be uniquely determined, for example, for a system with a large number of degrees of freedom. In this case, we consider a statistical mixture of states where each wave function has its statistical weight. That is, the system can be in states described by wave functions \(\left\{\left|\psi_n\right>\right\}\). The probability of the system being in some state \(\left|\psi_n\right>\) from this set is \(p_n\). It is obvious that
\[
\sum_n p_n = 1.
\]
A state described by a mixture of pure states is called a \textbf{mixed state}.

\subsection{Density Matrix}
For a system in a mixed state, to calculate the averages of some operator, it is convenient to use the density matrix formalism, which was proposed by John von Neumann and independently by Landau and Bloch in 1927.
\rindex{Density matrix!definition}

In \ref{eqAddDiracMidViaRho} it was shown that the average value of some operator \(\hat{L}\) in a pure state \(\left|\psi_n\right>\) can be written as
\[
\left< \hat{L} \right>_{\psi_n} = Sp \left(\hat{\rho_n} \hat{L} \right),
\]
where
\[
\hat{\rho_n} = \hat{P}_n = \left|\psi_n\right>\left<\psi_n\right|.
\]

For a mixed state, the formula for computing averages can be written as a sum of averages in pure states with a given weight:
\[
\left< \hat{L} \right>_{mix} = \sum_n p_n \left< \hat{L} \right>_{\psi_n}. 
\]
Thus, the average in a mixed state can be written in the following form
\begin{eqnarray}
\left< \hat{L} \right>_{mix} = Sp \left(\hat{\rho} \hat{L} \right),
\end{eqnarray}
where
\begin{eqnarray}
\hat{\rho} = \sum_n p_n \hat{\rho_n} = 
\sum_n p_n \left|\psi_n\right>\left<\psi_n\right|.
\end{eqnarray}