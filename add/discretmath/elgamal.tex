%% -*- coding:utf-8 -*-
\section{Elgamal Scheme}
\label{sec:add:dm:elgamal}
One of the variations of the Diffie-Hellman protocol is the Elgamal scheme. It is important to distinguish between the Elgamal encryption algorithm and the Elgamal digital signature algorithm. The Elgamal digital signature underlies the US Digital Signature Algorithm (DSA) and the Russian standard (GOST R 34.10-94).

Below we will consider the algorithm in encryption mode.

\subsection{Key Generation}

\begin{itemize}
\item Generate a prime number $p$.
\item Choose an integer $g$.
\item Choose a random integer $x: 1 < x < p$.
\item Compute $y = g^x \mod p$.
\end{itemize}

The public key is the triple $p, g, y$. The private key is the number $x$.

\begin{example}[Key Generation (Elgamal)]
Choose $p = 21, g = 10, x = 3$. $y = 10^3 \mod 21 = 13$.
\label{ex:add:dm:elgamal_gen}
\end{example}

\subsection{Encryption}
The message to be encrypted $M$ must satisfy $0 < M < p$.
\begin{itemize}
\item Choose a session key — a random number $k: 1 < k < p - 1$.
\item Compute $a = g^k \mod p$.
\item Compute $b = y^k M \mod p$.
\end{itemize}

The pair of numbers $(a, b)$ is considered the ciphertext.
\begin{example}[Encryption (Elgamal)]
Suppose we want to encrypt $M=6$.
Choose $k = 7$. Then $a = 10^7 \mod 21 = 10$ and $b = 13^7 \cdot 6 \mod 21 = 15$.
\label{ex:add:dm:elgamal_crypt}
\end{example}

\subsection{Decryption}
Knowing the private key $x$, we can recover the original message by
\begin{equation}
M = b \cdot \left(a^x\right)^{-1} \mod p,
\label{eq:add:dm:elgamal_decrypt}
\end{equation}
indeed, since
\[
\left(a^x\right)^{-1} = g^{-kx} \mod p,
\]
we have
\[
b \cdot \left(a^x\right)^{-1} = 
y^k M g^{-kx} = g^{kx} M g^{-kx} = M \mod p.
\]
\begin{example}[Decryption (Elgamal)]
The encrypted message from example \ref{ex:add:dm:elgamal_crypt} 
is $C = (a=10, b=15)$. Using \eqref{eq:add:dm:elgamal_decrypt} we have
\[
M = 15 \cdot 13 \mod 21 = 6,
\]
where
\[
\left(10^3\right)^{-1} \equiv 13 \mod 21,
\]
since $13 \cdot 10^3 = 1 \mod 21$.
Thus, we have recovered the message $M=6$ encrypted in example \ref{ex:add:dm:elgamal_crypt}.
\label{ex:add:dm:elgamal_crypt}
\end{example}