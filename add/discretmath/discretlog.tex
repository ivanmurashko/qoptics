%% -*- coding:utf-8 -*-
\section{Discrete Logarithm}
\label{AddDiscretLog}

\begin{definition}[Discrete Logarithm]
Consider some abelian multiplicative group $G$ and the equation 
\begin{equation}
g^x = a
\label{eq:add:dm:discretlog}
\end{equation}
A solution of this equation, i.e., a non-negative integer $x$
satisfying the equality \eqref{eq:add:dm:discretlog} is called
a discrete logarithm.
\end{definition}

The equation \eqref{eq:add:dm:discretlog} does not, in general, have a solution
for any value of $a$. But in the case when $g$ is a generating
element of $G$, i.e., $G=\langle g\rangle$, then the solution always exists and it is
unique. Further, when speaking about the discrete logarithm, we will
assume that $g$ is chosen such that $G=\langle g\rangle$.

In applied cryptography, a special kind of
discrete logarithm in the ring of residues modulo $p$ is often used:
\begin{definition}[Discrete Logarithm in the Ring of Residues modulo $p$]
The discrete logarithm $ind_g\left(a\right) \mod{p}$
\footnote{From the word {\bf ind}ex - an alternative name for the discrete logarithm}
is called
the minimal number $x$ that satisfies the following equation
(if such a number exists): 
\begin{equation}
g^x \equiv a \mod{p}
\end{equation}
\end{definition}

\begin{example}
\emph{($ind_3{14} \mod{17}$)}
Let us solve the problem by brute force \cite{bWikiDiscretLog}:
\begin{eqnarray}
3^1 \equiv 3 \mod{17},\: 
3^2 \equiv 9 \mod{17},\: 
3^3 \equiv 10 \mod{17},\:
3^4 \equiv 13 \mod{17}, 
\nonumber \\
3^5 \equiv 5 \mod{17},\: 
3^6 \equiv 15 \mod{17},\: 
3^7 \equiv 11 \mod{17},\: 
3^8 \equiv 16 \mod{17}, 
\nonumber \\
3^9 \equiv 14 \mod{17},\: 
3^{10} \equiv 8 \mod{17},\: 
3^{11} \equiv 7 \mod{17},\: 
3^{12} \equiv 4 \mod{17}, 
\nonumber \\
3^{13} \equiv 12 \mod{17},\: 
3^{14} \equiv 2 \mod{17},\:
3^{15} \equiv 6 \mod{17},\: 
3^{16} \equiv 1 \mod{17},
\nonumber
\end{eqnarray}
thus, one can see that $ind_3{14} \mod{17} = 9$, 
since $3^9 \equiv 14 \mod{17}$. 
\label{ex:dm:discretlog}
\end{example}

The problem of finding the discrete logarithm is a hard
problem. The fastest known algorithm
\cite{bGordon93discretelogarithms} solves it in time approximately
\(
O\left(c \cdot
\exp\left(\log{p}^{\frac{1}{3}}\log{\log{p}}^{\frac{2}{3}}
\right)\right)
\), where $c$ is some constant,
which explains the widespread use of algorithms that utilize
the discrete logarithm in cryptography.
