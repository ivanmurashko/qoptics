%% -*- coding:utf-8 -*-
\section{Euler's Function}

\subsection{Definition}
\begin{definition}[Euler's Function]
Euler's function $\phi\left(n\right)$ shows how many numbers $k \in \{1, ... n-1\}$ are coprime with $n$, i.e. $\gcd\left(k, n\right) = 1$.
\label{def:add:discretmath:eulerfun}
\end{definition}

\begin{example}[Euler's Function]
  If we take the number $n=15$, there are 8 numbers coprime with $15$: $1, 2, 4, 7, 8, 11, 13, 14$. The remaining 7 numbers are not coprime with $n$ because they have a greatest common divisor other than 1, for example $\gcd\left(6, 15\right) = 3$. Thus, $\phi\left(15\right) = 8$. 
\end{example}

\subsection{Properties}

\begin{property}[Euler's Function of a Prime Number]
If $p$ is a prime number, then $\phi(p) = p - 1$
\begin{proof}
This follows from definition \ref{def:add:discretmath:eulerfun}.
\end{proof}
\label{prop:add:discretmath:eulerfun1}
\end{property}

\begin{property}[Euler's Function of a Product (Generalized Multiplicativity)]
If $\gcd\left(n, m\right) = 1$, then $\phi\left(n \cdot m\right) = \phi\left(n\right) \phi\left(m\right)$
\begin{proof}
TBD
\end{proof}
\label{prop:add:discretmath:eulerfun2}
\end{property}

\begin{remark}[On the Complexity of Calculating Euler's Function]
\label{rem:add:discretmath:eulerfuncomplex}
Calculating Euler's function for large numbers is a very complex task. It is often computed using property \ref{prop:add:discretmath:eulerfun1} in combination with \ref{prop:add:discretmath:eulerfun2}. Applying these properties to an arbitrary number requires its factorization, so the complexity of calculating Euler's function is comparable to the complexity of the factorization problem.
\end{remark}