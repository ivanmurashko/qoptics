\begin{itemize}
\item Two prime numbers $p$ and $q$ are chosen.
\item The product of the chosen prime numbers is calculated $n = p\cdot q$.
\item Euler's \myref{def:add:discretmath:eulerfun}{function} is calculated (see properties \ref{prop:add:discretmath:eulerfun1} and \ref{prop:add:discretmath:eulerfun2})
\(
\phi\left(n\right)=\left(p - 1 \right)\left(q - 1 \right)
\)
\item An integer $e$ is chosen such that 
\(
1 < e < \phi\left(n\right)
\) and $e$ and $\phi\left(n\right)$ are coprime, i.e. 
\(
\gcd\left( e, \phi\left(n\right) \right) = 1.
\)
\item Calculate $d \equiv e^{-1} \mod{\phi\left(n\right)}$, i.e. $d \cdot e \equiv 1 \mod{\phi\left(n\right)}$.
\end{itemize}

The public key consists of two numbers: the modulus $n$ and the public exponent $e$. These two numbers are used to encrypt the original message.

The private key also consists of two numbers: the modulus $n$ and the private exponent $d$.