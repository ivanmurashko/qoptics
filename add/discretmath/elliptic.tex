%% -*- coding:utf-8 -*-
\section{Elliptic Cryptography}

\subsection{Elliptic Curves over the Field $\mathbb{R}$}

\input ./add/discretmath/figelliptic.tex

In elliptic cryptography, certain sets of objects that form a group are considered (see \autoref{sec:add:group}). As such a set, we will consider points belonging to a certain curve (see \autoref{fig:add:ellipticR}): 
\[
E: y^2 = x^3 +a x + b,
\]
where the coefficients $a,b$ must satisfy the following relation
\[
4 a^3 + 27 b^2 \ne 0,
\]
in this case the cubic equation $x^3 + a x + b = 0$ will have 3 distinct real roots \cite{Washington:2008:ECN:1388394}. 

\input ./add/discretmath/figellipticsum.tex

According to definition \ref{def:add:group}, for points on the curve there is a certain binary operation which assigns a third point $r$ with coordinates $(r_x, r_y)$ to two points $p, q$ with coordinates $(p_x, p_y)$ and $(q_x, q_y)$, respectively. We will call this operation addition:
\[
p + q = r.
\]
There is a simple geometric interpretation of the addition operation (see \autoref{fig:add:ellipticRsum}). Suppose there are 2 points on the curve that we want to add: $p$ and $q$ with coordinates $(x_p, y_p), (x_q, y_q)$ respectively. If $x_p \ne x_q$, then through these points one can draw a straight line with slope
\[
m = \frac{y_p - y_q}{x_p - x_q}.
\]
and intersects the curve at a point $r'$. If this point has coordinates $(x_{r'}, y_{r'})$, then since it lies on the line with slope $m$:
\[
m = \frac{y_{r'} - y_p}{x_{r'} - x_p},
\]
therefore
\[
y_{r'} = y_p + m \left(x_{r'} - x_p\right).
\]
This point must belong to the curve, i.e.
\begin{eqnarray}
y_{r'}^2 = \left(y_p + m \left(x_{r'} - x_p\right)\right)^2 = x_{r'}^3
+ a x_{r'} + b
\nonumber
\end{eqnarray}
which can be rewritten as
\begin{eqnarray}
x_{r'}^3 - m^2 x_{r'}^2 + \dots = 0.
\nonumber
\end{eqnarray}
The equation $x^3 - m^2 x^2 + \dots = 0$ has 3 roots: $x_p, x_q,
x_{r'}$, i.e., it can also be rewritten as
\begin{eqnarray}
(x - x_{r'})(x - x_p)(x - x_q) = x^3 - (x_{r'} + x_p + x_q) x^2 +
  \dots = 0.
\nonumber
\end{eqnarray}
Thus, 
\[
x_{r'} + x_p + x_q = m^2.
\]
Therefore
\begin{eqnarray}
x_{r'} = m^2 - x_p - x_q,
\nonumber \\
y_{r'} = y_p + m \left(x_{r'} - x_p\right),
\nonumber
\end{eqnarray}
Reflecting this point relative to the X-axis we get the final point $r$
which we will call the sum of the original points (see
\autoref{fig:add:ellipticRsum}). The coordinates of this point $x_r, y_r$
can be obtained by the following formulas
\begin{eqnarray}
x_{r} = m^2 - x_p - x_q,
\nonumber \\
y_{r} = - y_p + m \left(x_p - x_r\right).
\label{eq:add:discretmath:elliptic:add}
\end{eqnarray}

\input ./add/discretmath/figellipticsumeq2.tex
\input ./add/discretmath/figellipticsumeq.tex

In the case $x_p = x_q$ there are two possibilities:
\begin{enumerate}
\item $y_p = y_q$ (see \autoref{fig:add:ellipticRsumEq2}). When the points on the curve approach each other, the straight line drawn through them tends to the tangent line. The coefficient $m$ can be found by the following formula: $m = \frac{dy}{dx}$. Taking into account $2 y dy = 3 x^2 dx + a dx$ we have $m = \frac{dy}{dx} = \frac{3 x^2 + a}{2 y}$. Further calculation is carried out using formulas \eqref{eq:add:discretmath:elliptic:add}. 
\item $y_p \ne y_q$ (see \autoref{fig:add:ellipticRsumEq}). In this case, due to the symmetry of the curve with respect to the X-axis, only one possibility exists: $y_p = y_q$ and the line passing through these two points does not intersect the curve at a third point. For this case, an additional point $0$ is introduced, where the line passing through the two points goes. Thus, in this case we have $p + q = 0, q = -p$.
\end{enumerate}

Thus, we can define an elliptic curve over the field of real numbers $\mathbb{R}$
\rindex{Field $\mathbb{R}$}
as the following set of points
\begin{equation}
E\left(\mathbb{R}\right) = \left\{ (x,y) \in \mathbb{R} \times \mathbb{R}:
y^2 = x^3 +ax +b, 
\right\} \cup \{0\}.
\label{eq:add:discretmath:elliptic:er}
\end{equation}
where $a,b \in \mathbb{R}$.

For these points, a binary operation is defined, which we called addition. There is a zero element on the set, and for each element an inverse element can be defined. It can be proven that the introduced operation is associative: $(a+b) + c = a + (b+c)$ \cite{Washington:2008:ECN:1388394}. Thus, the set $E\left(\mathbb{R}\right)$ forms a group with respect to addition. Considering the obvious equality $p + q = q + p$, this group will be commutative, i.e., Abelian (see definition \ref{def:add:abeliangroup}). \rindex{Abelian group}  

\begin{remark}[On the Addition Operation]
The introduced addition operation for points on an elliptic curve initially looks unnatural; why not use more obvious operations of addition of points on the plane, for example, vector addition rules. In this case, if $p,q \in E\left(\mathbb{R}\right)$, then it is quite possible that $p + q \not\in E\left(\mathbb{R}\right)$ and, consequently, the main property of the group — closure — would be violated. Meanwhile, for the binary operation \eqref{eq:add:discretmath:elliptic:add} defined by us, all group properties hold, and accordingly, all properties following from this, such as Lagrange's theorem (see theorem \ref{thm:lagrange}), \rindex{Lagrange's theorem} which can be applied to the introduced objects.
\end{remark}


\subsection{Elliptic Curves over the Field $\mathbb{F}_p$}
The set \eqref{eq:add:discretmath:elliptic:er} together with the addition operation \eqref{eq:add:discretmath:elliptic:add} can be defined over an arbitrary field (see definition \ref{def:field}), \rindex{Field}, i.e., not only over $\mathbb{R}$. From the cryptographic point of view, special interest lies in the field $\mathbb{F}_p$ (see \autoref{sec:add:diskretmath:mod:fp}). It is possible to define the set of elements of an elliptic curve over the field $\mathbb{F}_p$ similarly to expression \eqref{sec:add:diskretmath:mod:fp}:
\begin{equation}
E\left(\mathbb{F}_p\right) = \left\{ (x,y) \in 
\mathbb{F}_p \times \mathbb{F}_p: 
y^2 \equiv x^3 +ax +b, 
\right\} \cup \{0\}.
\label{eq:add:discretmath:elliptic:fp}
\end{equation}
where $a,b \in \mathbb{F}_p$.

\begin{definition}[Order of an Elliptic Curve]
\label{def:elliptic_curve_order}
The number of points of an elliptic curve $E$ is called its order and is denoted as $\left|E\right|$
\end{definition}

\input ./add/discretmath/figellipticfp.tex

\autoref{fig:add:ellipticFp} shows such a set for the field
$\mathbb{F}_{19}$, i.e., $p = 19$. The curve equation is $y^2 \equiv x^3 -7 x + 10 \mod 19$. 

For each point $a$ with coordinates $x_a, y_a$, the inverse element $-a$ is defined with coordinates $x_{-a} = x_a, y_{-a} \equiv -y_a \mod p$.

For points on this curve, the following addition law is defined $a + b = c,
b \ne -a$
\begin{eqnarray}
x_{c} \equiv m^2 - x_a - x_b \mod p,
\nonumber \\
y_{c} \equiv - y_a + m \left(x_a - x_c\right) \mod p,
\label{eq:add:discretmath:elliptic:addfp}
\end{eqnarray}
where $(x_{a,b,c}, y_{a,b,c})$ are the coordinates of points $a,b$ and $c$ respectively. For the coefficient $m$, the following relations are used: 
\begin{eqnarray}
m = \left(y_a - y_b\right)\left(x_a - x_b\right)^{-1} \mod p, \mbox{ if } x_a \ne x_b
\nonumber \\
m = \left(3x^2 + a\right)\left(2y\right)^{-1} \mod p, \mbox{ if }
x_a = x_b.
\nonumber
\end{eqnarray}
Obviously, if $b = -a$, then $a + b = a + (-a) = 0$.

\subsection{Scalar Multiplication and Discrete Logarithm}

Assume $n$ is some natural number $n \in \mathbb{N}$ and $a \in
E\left(\mathbb{F}_p\right)$. Define scalar multiplication as follows:
\[
n \cdot a = a + a + \dots + a = \sum_{k=1}^n a
\] 

A naive implementation requires $O\left(n\right)$ addition operations, but using the divide and conquer paradigm (see \autoref{addDivideAndConquer}) it can be reduced to $O\left(\log n\right)$ addition operations.

%% *Elliptic> c = Curve (-7) 10 19
%% *Elliptic> c
%% y^2 = (x^3 + -7x + 10) mod 19
%% *Elliptic> points c
%% [0,(1,2),(1,17),(2,2),(2,17),(3,4),(3,15),(5,9),(5,10),(7,0),(9,7),(9,12),(10,3),(10,16),(12,1),(12,18),(13,8),(13,11),(16,2),(16,17),(17,4),(17,15),(18,4),(18,15)]
%% *Elliptic> p = Point 13 8 c
%% *Elliptic> 0 .*. p
%% 0
%% *Elliptic> 1 .*. p
%% (13,8)
%% *Elliptic> 2 .*. p
%% (16,17)
%% *Elliptic> 3 .*. p
%% (18,15)
%% *Elliptic> 4 .*. p
%% (12,1)
%% *Elliptic> 5 .*. p
%% (5,10)
%% *Elliptic> 6 .*. p
%% (7,0)
%% *Elliptic> 7 .*. p
%% (5,9)
%% *Elliptic> 8 .*. p
%% (12,18)
%% *Elliptic> 9 .*. p
%% (18,4)
%% *Elliptic> 10 .*. p
%% (16,2)
%% *Elliptic> 11 .*. p
%% (13,11)
%% *Elliptic> 12 .*. p
%% 0


\begin{example}[Scalar Multiplication]
Consider the elliptic curve 
\[
E\left(\mathbb{F}_{19}\right) = 
\{(x,y) \in \mathbb{F}_{19} \times \mathbb{F}_{19}: y^2 \equiv x^3 -7 x +
10 \} \cup \{0\}, 
\] 
shown in \autoref{fig:add:ellipticFp}. Choose the point $p = (13, 8)$, then
\begin{eqnarray}
0 \cdot p = 0, \nonumber \\
1 \cdot p = p = (13,8), \nonumber \\
2 \cdot p = p + p = (16,7), \nonumber \\
3 \cdot p = 2 \cdot p + p = (18,5), \nonumber \\
4 \cdot p = 3 \cdot p + p = (12,1), \nonumber \\
5 \cdot p = 4 \cdot p + p = (5,10), \nonumber \\
6 \cdot p = 5 \cdot p +p = (7,0), \nonumber \\
7 \cdot p = 6 \cdot p + p = (5,9), \nonumber \\
8 \cdot p = 7 \cdot p +p = (12,18), \nonumber \\
9 \cdot p = 8 \cdot p + p = (18,4), \nonumber \\
10 \cdot p = 9 \cdot p + p = (16,2), \nonumber \\
11 \cdot p = 10 \cdot p + p = (13,11), \nonumber \\
12 \cdot p = 11 \cdot p + p = 0
\nonumber 
\end{eqnarray}
\label{ex:add:discretmath:elliptic:subgroup}
\end{example}
As seen from example \ref{ex:add:discretmath:elliptic:subgroup}, every element of an elliptic curve is a generator of some cyclic subgroup. At the same time, the entire group of points on the elliptic curve is not necessarily cyclic 
(TBD).
%https://math.stackexchange.com/questions/2323595/under-what-conditions-do-all-the-points-on-an-elliptic-curve-form-a-cyclic-group
On the other hand, for formulating the discrete logarithm problem, we need exactly a cyclic group. Thus, for a given elliptic curve, first its order (see def. \ref{def:elliptic_curve_order}) is calculated, for which there exists an efficient Schoof's algorithm \cite{ReneSchoof:1985}. Then a prime divisor of the found order is located and a point is sought which is a generator of a subgroup of the chosen order. To do this, the following fact is used. For any point 
$g \in E$, the relation holds
\[
N g = 0,
\]
where $N = \left|E\right|$ is the order (number of points) of the elliptic curve.
Assume $p$ is a prime divisor of $N$:
\[
N = h p
\]
then
\[
N g = p \left( h g \right) = 0.
\]
Thus, if $hg \ne 0$, then the point $g' = h g$ will be a generator of a cyclic subgroup of order $p$.
\begin{remark}
It makes sense to choose the order of the subgroup to be a prime number, since if the order of the group is prime then by Lagrange's theorem (see theorem \ref{thm:lagrange}) \rindex{Lagrange's theorem} it has only trivial subgroups — the group itself and the subgroup consisting of the identity element. Thus, due to $hg \ne 0$, only the group itself remains, which is cyclic with order $p$. 
\end{remark}

\input ./add/discretmath/figellipticfp97.tex

%% *Elliptic> c = Curve (-7) 10 97
%% *Elliptic> c
%% y^2 = (x^3 + -7x + 10) mod 97
%% *Elliptic> cord c
%% 82
%% *Elliptic> g = Point 1 2 c
%% *Elliptic> pord g
%% 82
%% *Elliptic> g' = 2 .*. g
%% *Elliptic> g'
%% (96,93)
%% *Elliptic> pord g'
%% 41
%% *Elliptic> 

\begin{example}[Choosing a Base Point]
\label{ex:add:elliptic:basepoint}
Consider the elliptic curve 
\[
E = E\left(\mathbb{F}_{97}\right) = 
\{(x,y) \in \mathbb{F}_{97} \times \mathbb{F}_{97}: y^2 \equiv x^3 -7 x + 10 \} \cup \{0\},
\] 
shown in \autoref{fig:add:ellipticFp97}. The order of this curve is:
\[
N = \left|E\right| = 82.
\]
The number 82 has 2 divisors: 41 and 2. Thus, there exists a cyclic subgroup of order 41, i.e., $h = 2$. 

Take the point $g = (1,2) \in E$, its order:
$\left|\left<g\right>\right| = 82$, i.e., this point is not suitable. 
Calculate 
\[
g' = h g = 2 g = (96,93) \ne 0.
\]
At the same time, $\left|\left<g'\right>\right| = 41$, i.e., we have found the required base point.
\end{example}

If there are two points on the curve $a, b \in E\left(\mathbb{F}_p\right)$,
the question of the existence of such $x \in \mathbb{N}$ makes sense:
\[
x \cdot a = b
\]
this problem is called the discrete logarithm problem on elliptic
curves. 

