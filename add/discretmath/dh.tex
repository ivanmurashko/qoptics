%% -*- coding:utf-8 -*-
\section{Diffie-Hellman Protocol (DH)}
\label{sec:add:dm:dh}
Suppose there are two parties, Alice and Bob. They are aware of two numbers
$g$ and $p$, which are not secret.

Alice chooses a random number $a$ and sends the following value to Bob
\begin{equation}
A \equiv g^a \mod{p}.
\label{eqAddDiscretIndDHA}
\end{equation}

Bob calculates the following number (using a secret random variable
$b$)
\begin{equation}
B \equiv g^b \mod{p}.
\label{eqAddDiscretIndDHB}
\end{equation}

Alice, using only the number $a$ known to her, calculates the key
\begin{equation}
K \equiv B^a\mod{p} \equiv g^{ab} \mod{p}.
\label{eqAddDiscretIndDHKey}
\end{equation}

Bob can obtain the same key value using his
secret number $b$:
\begin{equation}
K \equiv A^b\mod{p} \equiv g^{ab} \mod{p}.
\label{eqAddDiscretIndDHKeyB}
\end{equation}

Thus, Alice and Bob obtain the same key, which can later be used for message transfer using
symmetric encryption algorithms (for example, AES). \rindex{AES}

\begin{example}
\emph{(Diffie-Hellman)}
Initial data (public information): $g = 2$, $p = 23$. Alice
chooses a random number $a = 6$ and calculates the number
\eqref{eqAddDiscretIndDHA} 
%power_mod(2, 6, 23);
$A = 18$ and sends it to Bob.
Bob chooses a random number $b=9$ and, using the formula 
\eqref{eqAddDiscretIndDHB}, gets
%power_mod(2, 9, 23);
$B = 6$ and sends this number to Alice.

Alice calculates the key 
% power_mod(6, 6, 23);
$K = 12$ using the formula \eqref{eqAddDiscretIndDHKey}. Bob can obtain
the same key value 
% power_mod(18, 9, 23);
$K = 12$ using \eqref{eqAddDiscretIndDHKeyB}
\nonumber
\end{example}

The eavesdropper Eve knows the numbers $g$, $p$, $A$, and $B$. To obtain
the key $K$, Eve needs to get one of the secret numbers $a$ or $b$:
\begin{equation}
a \equiv ind_g\left( A \right) \mod{p},
\nonumber
\end{equation}
from which, using \eqref{eqAddDiscretIndDHKey}, the desired
value $K$ is obtained.