%% -*- coding:utf-8 -*- 
\chapter{Quantum Theory of the Laser}
\label{chLaser}

The semiclassical laser theory cannot answer all questions
arising in connection with its operation. According to this theory, the laser does not generate at all
before reaching the threshold, and upon exceeding the threshold begins
to generate a classical electromagnetic field (light). 

In reality, well below the threshold the laser generates chaotic
light, and significantly above the threshold its radiation is close to
classical. At the threshold and near it lies the transitional region from
chaotic light to ordered radiation. Only a fully quantum theory can adequately describe this. 

Another problem, also requiring quantization of the electromagnetic field,
is determining the ultimate (natural) linewidth of the laser radiation,
when the linewidth is determined by quantum fluctuations of the field,
while various external influences, in principle removable, are not
taken into account. 

The laser is an open system in which the active atoms and the resonator field
are connected with large external systems, which we will call reservoirs,
providing pumping and losses. 
It follows that the laser, as an open system, should be considered
using the density matrix.  

\input ./part2/laser/model.tex
\input ./part2/laser/theory.tex
\input ./part2/laser/stat.tex
\input ./part2/laser/theorycoh.tex
\input ./part2/laser/questions.tex

%% \begin{thebibliography}{99}
%% \bibitem{bCh1LaserMandel} L. Mandel, E. Wolf. Optical Coherence and
%%   Quantum Optics. Translated from English / Edited by V.V. Samartsev - M.:
%%   Nauka. Fizmatlit, 2000.- 896 p. 
%% \bibitem{bCh1Laser_Lamb} Lectures by W. Lamb // Quantum optics and quantum
%%   radiophysics: Lectures at summer school. M.: Mir, 1966. 
%% \bibitem{bCh1Laser_Scally} M. Scally. Quantum Laser Theory -
%%   The Problem of Nonequilibrium Statistical Mechanics. Quantum Fluctuations
%%   of Laser Radiation / Editors: F. Arekki, M. Scally, G. Haken, W. Weidlich. M.:
%%   Mir, 1974. 
%% \bibitem{bCh1Laser_Haken} G. Haken. Laser Light Dynamics. M.: Mir,
%%   1988.
%% \bibitem{bCh1LaserSkalliZubari} M. O. Scally, M. S. Zubairy. Quantum
%%   Optics. M. Fizmatlit, 2003.
%% \end{thebibliography}

