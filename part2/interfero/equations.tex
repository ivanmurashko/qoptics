%% -*- coding:utf-8 -*- 
\section{Operator relations}
We will consider the incident and transmitted fields as two-dimensional,
see \autoref{figPart2Interfero_2}. The operators $\hat{a}_0$ and 
$\hat{a}_1$ are operators of the incident field, corresponding to two different modes
(zero and first). The operators of the transmitted field are $\hat{a}_2$ and 
$\hat{a}_3$ (second and third modes). Naturally, the operators must
satisfy the commutation relations:
\begin{eqnarray}
\left[\hat{a}_0, \hat{a}_0^{\dag}\right] = 
\left[\hat{a}_1, \hat{a}_1^{\dag}\right] = 
\left[\hat{a}_2, \hat{a}_2^{\dag}\right] = 
\left[\hat{a}_3, \hat{a}_3^{\dag}\right] = 1,
\nonumber \\
\left[\hat{a}_0, \hat{a}_1^{\dag}\right] = 
\left[\hat{a}_2, \hat{a}_3^{\dag}\right] = 0.
\label{eqPart2Interfero5}
\end{eqnarray}

\input ./part2/interfero/fig2.tex

The relation between the modes arises upon reflection from the mirror and
passing through it. It is described by the following equations:
\begin{eqnarray}
\hat{a}_2 = t' \hat{a}_0 + r \hat{a}_1,
\nonumber \\
\hat{a}_3 = r' \hat{a}_0 + t \hat{a}_1.
\label{eqPart2Interfero6}
\end{eqnarray}

The incident and transmitted fields must satisfy the commutation relations \eqref{eqPart2Interfero5}, from which, taking into account 
\eqref{eqPart2Interfero3}, we have
\begin{eqnarray}
\left[\hat{a}_2, \hat{a}_2^{\dag}\right] = 
\left[t' \hat{a}_0 + r \hat{a}_1, t'^{*} \hat{a}_0^{\dag} + r^{*}
  \hat{a}_1^{\dag}\right] =
\nonumber \\ = 
\left|r\right|^2 + \left|t\right|^2 = 1.
\label{eqPart2InterferoTask2a}
\end{eqnarray}
Similarly, we get 
\begin{equation}
\left[\hat{a}_3, \hat{a}_3^{\dag}\right] = 1 
\label{eqPart2InterferoTask2b}
\end{equation}
and further
\begin{equation}
\left[\hat{a}_2, \hat{a}_3^{\dag}\right] = 
\left[t' \hat{a}_0 + r \hat{a}_1, 
r'^{\dag} \hat{a}_0^{\dag} + t^{*} \hat{a}_1^{\dag}\right] = 
t' r'^{*} + r t^{*} = 0.
\label{eqPart2InterferoTask2c}
\end{equation}
Thus, we have obtained the fulfillment of all the required commutation relations. Note that if only mode 1 (input 1) is excited, then the field of mode zero (input 0) cannot be canceled as is done in
the classical case. In the quantum case, the mode is in the vacuum state, and its field is not zero. If this is not taken into account, then the commutation relations \eqref{eqPart2Interfero5} will be violated.

\input ./part2/interfero/fig3.tex

Let us assume that the mirror (beam splitter) shown in
\autoref{figPart2Interfero_3} receives at input 1 (first mode) -
a coherent state $\left|\alpha\right>$, and at the zero input
(zero mode) a vacuum state $\ket{0}$, i.e.
\begin{eqnarray}
\hat{a}_0 \ket{0} = 0,
\nonumber \\
\hat{a}_1 \left|\alpha\right> = \alpha \left|\alpha\right>.
\nonumber
\end{eqnarray}
Thus, at the input, we have a two-mode state 
\[
\left|\psi\right> = \ket{0}_0 \left|\alpha\right>_1.
\]
From \eqref{eqPart2Interfero6} it follows
\begin{equation}
\hat{a}_3\left|\psi\right> = 
\left(r' \hat{a}_0 + t \hat{a}_1\right)\ket{0}_0
\left|\alpha\right>_1 = 
t \alpha \left|\psi\right>,
\nonumber
\end{equation}
from which it is seen that the state at the output 3 is a coherent state
with a reduced amplitude $\alpha_3 = t \alpha$. 
Similarly, we get
\begin{equation}
\hat{a}_2\left|\psi\right> = 
r \alpha \left|\psi\right>, \, \alpha_2 = r\alpha.
\label{eqPart2InterferoTask3}
\end{equation}
Thus, the change in amplitude $\alpha$ is the same as in
the classical case, and the state remains coherent.

