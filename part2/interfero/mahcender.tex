%% -*- coding:utf-8 -*- 
\section{Mach-Zehnder Interferometer. Error of Phase Measurements.}
\subsection{Interferometer Equation}
\rindex{Mach-Zehnder Interferometer}
Let us now consider in more detail the operation of the Mach-Zehnder interferometer,
the schematic of which is shown in \autoref{figPart2Interfero_4}.

\input ./part2/interfero/fig4.tex

Assume that the input $M_1$ and output $M_2$ mirrors are partially transparent
(50:50), i.e.
\[
t = r = \frac{1}{\sqrt{2}},
\]
and the other two mirrors are fully reflective
\[
t = 0, \, r = 1.
\]
The ``scattering matrix'' of the beam splitters has the form
\begin{equation}
\hat{S} = \frac{1}{\sqrt{2}} 
\left(
\begin{array}{cc}
1 & i \\
i & 1 \\
\end{array}
\right).
\nonumber
\end{equation}
Here we used the notation \eqref{eqPart2Interfero4a}. 
The interferometer equation is:
\begin{eqnarray}
\hat{a}_2 = \frac{1}{\sqrt{2}} \left(\hat{a}_0 + i \hat{a}_1\right),
\,
\hat{a}_3 = \frac{1}{\sqrt{2}} \left(i \hat{a}_0 + \hat{a}_1\right),
\nonumber \\
\hat{a}_4 = \frac{1}{\sqrt{2}} \left(i \hat{a}_2 + e^{i \varphi}
\hat{a}_3\right) = 
\frac{1}{2}\left[
i \left(1 + e^{i \varphi}\right)\hat{a}_0 -
\left(1 - e^{i \varphi}\right)\hat{a}_1
\right],
\nonumber \\
\hat{a}_5 = \frac{1}{\sqrt{2}} \left(\hat{a}_2 + i e^{i \varphi}
\hat{a}_3\right) = 
\frac{1}{2}\left[
\left(1 - e^{i \varphi}\right)\hat{a}_0 +
i \left(1 + e^{i \varphi}\right)\hat{a}_1
\right],
\label{eqPart2Interfero11}
\end{eqnarray}
where $\varphi$ is the difference in optical path lengths of the interferometer arms (the phase difference accrued in the arms).

Formulas \eqref{eqPart2Interfero11} connect the input fields with
the output fields. Let us assume that the input field at port zero is in
the vacuum state $\ket{0}$, and the input field at port one is in
a coherent state $\left|\alpha\right>$. Thus the input
field is in a two-mode state $\left|\psi\right> =
\ket{0}_0\left|\alpha\right>_1$. Next, one can find
the mean number of photons at outputs 4 and 5 depending on
$\varphi$. The result will be indistinguishable from the classical one.
Indeed,
\begin{eqnarray}
  \hat{a}_5^{\dag}\hat{a}_5 =
  \frac{1}{4}
  \left[
    \left(1 - e^{-i \varphi}\right)\hat{a}_0^{\dag} -
    i \left(1 + e^{-i \varphi}\right)\hat{a}_1^{\dag}
    \right]
  \nonumber \\
  \left[
    \left(1 - e^{i \varphi}\right)\hat{a}_0 +
    i \left(1 + e^{i \varphi}\right)\hat{a}_1
    \right] = 
  \nonumber \\
  = \frac{1}{2}
  \left[
    \left(
    1 -  \cos \varphi 
    \right)
    \hat{a}_0^{\dag}\hat{a}_0 +
    \left(
    1 +  \cos \varphi 
    \right)
    \hat{a}_1^{\dag}\hat{a}_1
    \right] -
  \frac{\sin \varphi}{2}
  \left[
    \hat{a}_0^{\dag}\hat{a}_1 +
    \hat{a}_1^{\dag}\hat{a}_0
    \right]
  \label{eqPart2InterferoA55}
\end{eqnarray}
Thus, for the state $\left|\psi\right> =
\ket{0}_0\left|\alpha\right>_1$ the detector at port 5 produces
the following signal
\[
\left<\psi\right|\hat{a}^{\dag}_5 \hat{a}_5\left|\psi\right> =
\frac{1}{2}\left(1+\cos \varphi \right) \left|\alpha\right|^2.
\]

For the photodetector at port 4 similarly we obtain
\begin{eqnarray}
  \hat{a}_4^{\dag}\hat{a}_4 =
  \frac{1}{4}
  \left[
    - i \left(1 + e^{-i \varphi}\right)\hat{a}_0^{\dag} -
    \left(1 - e^{-i \varphi}\right)\hat{a}_1^{\dag}
    \right]
  \nonumber \\
  \left[
    i \left(1 + e^{i \varphi}\right)\hat{a}_0 -
    \left(1 - e^{i \varphi}\right)\hat{a}_1
    \right] = 
  \nonumber \\
  = \frac{1}{2}
  \left[
    \left(
    1 +  \cos \varphi 
    \right)
    \hat{a}_0^{\dag}\hat{a}_0 +
    \left(
    1 -  \cos \varphi 
    \right)
    \hat{a}_1^{\dag}\hat{a}_1
    \right] +
  \frac{\sin \varphi}{2}
  \left[
    \hat{a}_0^{\dag}\hat{a}_1 +
    \hat{a}_1^{\dag}\hat{a}_0
    \right]
  \label{eqPart2InterferoA44}
\end{eqnarray}
At photodetector 4 we have
the following signal
\[
\left<\psi\right|\hat{a}^{\dag}_4 \hat{a}_4\left|\psi\right> =
\frac{1}{2}\left(1-\cos \varphi \right) \left|\alpha\right|^2.
\]

\input ./part2/interfero/fig5.tex

We consider a more complex balanced detection scheme,
shown in \autoref{figPart2Interfero_5}. Each channel
is detected by its own photodetector. The signals coming from the photodetectors
are subtracted and recorded. This detection scheme essentially is a
\rindex{homodyne!homodyne detector}
synchronous detector (homodyne detector), where the local oscillator is
the field in the coherent state, and the signal consists of vacuum fluctuations.

Using (\ref{eqPart2InterferoA44},
\ref{eqPart2InterferoA55}) one can calculate
the signal at the input of the balanced detector, which is defined as the expectation value
of the operator  
\begin{equation}
\hat{R} = 
\hat{a}_5^{\dag} \hat{a}_5 - 
\hat{a}_4^{\dag} \hat{a}_4 =
\left(
\hat{a}_1^{\dag} \hat{a}_1 - 
\hat{a}_0^{\dag} \hat{a}_0
\right) \cos\,\varphi -
\left(
\hat{a}_0^{\dag} \hat{a}_1 + 
\hat{a}_1^{\dag} \hat{a}_0
\right) \sin\,\varphi.
\label{eqPart2Interfero12}
\end{equation}
In deriving \eqref{eqPart2Interfero12} the relations
\eqref{eqPart2Interfero11} were used.

\rindex{homodyne}
If the mode of the local oscillator (input 1, mode $\hat{a}_1$) is in
a coherent state with a large amplitude $\alpha$, while the signal mode
(input 0, mode $\hat{a}_0$) is in the vacuum state, then the output signal
is
\begin{eqnarray}
\left<\hat{R}\right> = 
\left<\psi\right|\hat{R} \left|\psi\right> = 
\bra{0}_0\left<\alpha\right|_1
\left(
\hat{a}_1^{\dag} \hat{a}_1 - 
\hat{a}_0^{\dag} \hat{a}_0
\right)
\ket{0}_0\left|\alpha\right>_1
\cos\,\varphi
-
\nonumber \\
-
\bra{0}_0\left<\alpha\right|_1
\left(
\hat{a}_0^{\dag} \hat{a}_1 + 
\hat{a}_1^{\dag} \hat{a}_0
\right) 
\ket{0}_0\left|\alpha\right>_1 = 
\nonumber \\
= \left|\alpha\right|^2 \cos\,\varphi.
\label{eqPart2Interfero13}
\end{eqnarray}
In deriving \eqref{eqPart2Interfero13} it was taken into account that the operators
of the first and zero modes act only on the state of their
own mode. Therefore,
\[
\hat{a}_0\ket{0} = 0, \, 
\bra{0}\hat{a}_0^{\dag} = 0,
\]
and hence 
\[
\bra{0}_0\left<\alpha\right|_1
\left(
\hat{a}_0^{\dag} \hat{a}_1 + 
\hat{a}_1^{\dag} \hat{a}_0
\right) 
\ket{0}_0\left|\alpha\right>_1 = 0,
\]
while
\begin{eqnarray}
\left<\psi\right|\hat{R} \left|\psi\right>= 
\bra{0}_0\left<\alpha\right|_1
\left(
\hat{a}_1^{\dag} \hat{a}_1 - 
\hat{a}_0^{\dag} \hat{a}_0
\right)
\ket{0}_0\left|\alpha\right>_1
=
\nonumber \\
=
\bra{0}_0\left<\alpha\right|_1
\left(
\hat{a}_1^{\dag} \hat{a}_1
\right)
\ket{0}_0\left|\alpha\right>_1
= \left|\alpha\right|^2 = 
\left<\hat{n}\right>
\nonumber
\end{eqnarray}
\rindex{homodyne}
is equal to the mean photon number in the local oscillator mode.

If $\varphi = \frac{\pi}{2}$, then $\left<\hat{R}\right> = 0$ and
the output signal is absent. At the same time, the operator $\hat{R} \ne 0$:
\[
\hat{R} = 
-
\left(
\hat{a}_0^{\dag} \hat{a}_1 + 
\hat{a}_1^{\dag} \hat{a}_0
\right).
\]
Although the mean of this operator is zero, the mean of its square will
be nonzero and will describe the noise which limits the accuracy
of measuring $\varphi$, which itself describes the difference
in optical path lengths of the interferometer arms.

\subsection{Measurement Accuracy of the Interferometer}

First, let us find the mean square of the noise term:
\begin{eqnarray}
\left<\psi\right|\hat{R}\hat{R}^{\dag}\left|\psi\right> = 
\left<\psi\right|
\left(
\hat{a}_0^{\dag} \hat{a}_1 + 
\hat{a}_1^{\dag} \hat{a}_0
\right)
\left(
\hat{a}_0 \hat{a}_1^{\dag} +
\hat{a}_1 \hat{a}_0^{\dag}
\right)
\left|\psi\right> = 
\nonumber \\
\left<\psi\right|
\left(
\hat{a}_0^{\dag} \hat{a}_1 
\hat{a}_0 \hat{a}_1^{\dag} 
+ 
\hat{a}_0^{\dag} \hat{a}_1 
\hat{a}_1 \hat{a}_0^{\dag}
+
\hat{a}_1^{\dag} \hat{a}_0
\hat{a}_0 \hat{a}_1^{\dag} 
+
\hat{a}_1^{\dag} \hat{a}_0
\hat{a}_1 \hat{a}_0^{\dag}
\right)
\left|\psi\right>.
\label{eqPart2Interfero15}
\end{eqnarray}
Of the four terms in \eqref{eqPart2Interfero15}, the first three vanish upon averaging, while the last one is nonzero.
For example, for the first term we have
\begin{eqnarray}
\left<\psi\right|
\hat{a}_0^{\dag} \hat{a}_1 
\hat{a}_0 \hat{a}_1^{\dag} 
\left|\psi\right> = 
\bra{0}_0\left<\alpha\right|_1
\hat{a}_0^{\dag} \hat{a}_1 
\hat{a}_0 \hat{a}_1^{\dag} 
\ket{0}_0\left|\alpha\right>_1 =
\nonumber \\
=
\bra{0}_0\left<\alpha\right|_1
\hat{a}_0^{\dag} \hat{a}_0 
\hat{a}_1 \hat{a}_1^{\dag} 
\ket{0}_0\left|\alpha\right>_1 =
\bra{0}_0
\hat{a}_0^{\dag} \hat{a}_0 
\ket{0}_0
\left<\alpha\right|_1
\hat{a}_1 \hat{a}_1^{\dag} 
\left|\alpha\right>_1 =
0.
\nonumber
\end{eqnarray}
The last term equals
\begin{eqnarray}
\left<\psi\right|
\hat{a}_1^{\dag} \hat{a}_0
\hat{a}_1 \hat{a}_0^{\dag}
\left|\psi\right> = 
\bra{0}_0\left<\alpha\right|_1
\hat{a}_1^{\dag} \hat{a}_0
\hat{a}_1 \hat{a}_0^{\dag}
\ket{0}_0\left|\alpha\right>_1 =
\nonumber \\
=
\bra{0}_0\left<\alpha\right|_1
\hat{a}_1^{\dag} \hat{a}_1 
\hat{a}_0 \hat{a}_0^{\dag}
\ket{0}_0\left|\alpha\right>_1 =
\left<\alpha\right|_1
\hat{a}_1^{\dag} \hat{a}_1 
\left|\alpha\right>_1 
\bra{0}_0
\hat{a}_0 \hat{a}_0^{\dag}
\ket{0}_0 =
\nonumber \\
=
\left<\alpha\right|_1
\hat{a}_1^{\dag} \hat{a}_1 
\left|\alpha\right>_1 
\bra{1}
\ket{1}_0 =
\left<\alpha\right|_1
\hat{a}_1^{\dag} \hat{a}_1 
\left|\alpha\right>_1 
= \left|\alpha\right|^2.
\end{eqnarray}
Thus, we have found that the root mean square value of the noise is equal to
the mean photon number in the mode that is in the coherent state:
\begin{equation}
\bar{\left|R\right|^2} = 
\left<\hat{R}\hat{R}^{\dag}\right> = 
\left|\alpha\right|^2 =
\bar{n}.
\nonumber
\end{equation}
The standard deviation is
\begin{equation}
\sqrt{\bar{\left|R\right|^2}} = 
\sqrt{\bar{n}}.
\nonumber
\end{equation}
The signal arises due to the variation
\[
\varphi = \frac{\pi}{2} + \Delta \varphi.
\]
Its magnitude is
\[
\bar{R} = \bar{n} \cos\left(\frac{\pi}{2} + \Delta \varphi\right) \approx
- \bar{n} \Delta \varphi.
\]
The minimal distinguishable signal, i.e., the threshold signal, will be
considered the signal equal to the noise standard deviation:
\[
\left|\bar{n} \Delta \varphi\right| \ge
\sqrt{\bar{\left|R\right|^2}} = 
\sqrt{\bar{n}},
\]
from which it follows that
\[
\Delta \varphi  \ge \frac{1}{\sqrt{\bar{n}}}.
\]
Equality corresponds to the threshold value of $\Delta \varphi$.

As we will see later (see ch. \ref{chSqueezed}), the measurement accuracy
can be significantly improved if the ``squeezed vacuum'' is supplied
to the signal arm of the interferometer.  


