%% -*- coding:utf-8 -*- 
\section{Classical description of polarization properties of light}
The field of an arbitrary monochromatic plane wave can be represented as
follows:
\begin{equation}
\vec{E} = \vec{E_0} e^{-i \left(\omega t - \vec{k}\vec{r}\right)},
\nonumber
\end{equation}
where the vector 
\[
\vec{E_0} = E_{x}\vec{e}_x + E_{y}\vec{e}_y
\]
defines the intensity and polarization properties of the electromagnetic
radiation. For describing polarization properties, the complex polarization vector— the Jones vector—is used:
\rindex{Jones vector}
\begin{equation}
\vec{e} = \alpha \vec{e}_x + \beta \vec{e}_y.
\label{eqEntangJones}
\end{equation}
The components of this vector, $\alpha$ and $\beta$, can be represented
as points on a certain sphere called the Poincaré sphere. The coordinates
of these points are determined by two angles $\theta$ and $\varphi$
\begin{eqnarray}
\alpha = \frac{E_x}{\left|E_x\right|^2 + \left|E_y\right|^2} = 
 \cos \, \frac{\theta}{2},
\nonumber \\
\beta = \frac{E_y}{\left|E_x\right|^2 + \left|E_y\right|^2} = 
 e^{i\varphi} \sin \, \frac{\theta}{2}.
%\label{eqEntangJones2}
\nonumber
\end{eqnarray}
To measure the components of the Jones vector $\alpha$ and $\beta$, the scheme shown in
\autoref{figPart3EntangJones} can be used. Here, a beam of polarized light
is directed onto a Nicol prism $P$, which separates the $x$ and $y$ components
of this beam, which are then sent to two photodetectors $D_x$ and $D_y$.

\input ./part2/polarization/figjones.tex

In addition to the Jones vector, the Stokes vector is often used to describe polarization properties. Its four components
have the dimension of intensity and can be easily measured
experimentally. The Stokes vector can be defined as follows:
\rindex{Stokes vector}
\rindex{Stokes parameters!classical case}
\begin{eqnarray}
S_0 = \left|E_x\right|^2 + \left|E_y\right|^2,
\nonumber \\
S_1 = \left|E_x\right|^2 - \left|E_y\right|^2,
\nonumber \\
S_2 = E_x^{*} E_y + E_x E_y^{*} = 2 \mathrm{Re} \left(E_x^{*} E_y\right),
\nonumber \\
S_3 = \frac{E_x^{*} E_y - E_x E_y^{*}}{i} = 2 \mathrm{Im} \left(E_x^{*}
E_y\right),
\label{eqEntangStokes}
\end{eqnarray}
where the amplitude values $E_{x,y}$ are taken at some moment in time $t$. 
The parameter $S_0$ in \eqref{eqEntangStokes} defines the intensity of the wave
at time $t$,
and the other three parameters $S_1$, $S_2$, and $S_3$ describe the polarization
properties. 

\input ./part2/polarization/figstokes.tex

To measure the Stokes parameters, the scheme
shown in \autoref{figPart3EntangStokes} can be used
\cite{bEntangKlyshko}. To measure $S_1$, $S_2$, and $S_3$, it is necessary
to split the original beam into three parts, each of which is sent to
a detector depicted in \autoref{figPart3EntangStokes}.

\input ./part2/polarization/figs2.tex

To measure $S_1$, a Nicol prism $P$ is used, which separates 
the $x$ and $y$ polarized components of the beam,
so that the difference of the currents of the two photodetectors is proportional to $S_1$, while
the sum of the currents is proportional to $S_0$. 
To measure $S_2$, the prism is rotated by an angle $\xi =
\frac{\pi}{4}$, so that the difference of the currents is proportional to $S_2$
(see \autoref{figPart3EntangS2}). 
To measure $S_3$, a quarter-wave plate
$\frac{\lambda}{4}$ is placed before the prism with an orientation at $\frac{\pi}{4}$, resulting
in the difference of the photodetector currents being proportional to $S_3$.

The detector readings are averaged over time. Thus, one can
speak of measuring time-averaged Stokes parameters - 
$\left<S_k\right>$, which can also be used to describe
partially polarized light. For quantitative characterization
of such radiation, a quantity called
the degree of polarization is used:
\begin{equation}
P = \frac{\sqrt{\left<S_1\right>^2 + \left<S_2\right>^2 +
    \left<S_3\right>^2}}{\left<S_0\right>}.
\label{eqEntangPolyarDegree}
\nonumber
\end{equation}
The degree of polarization of fully polarized light is $P = 1$. For completely
unpolarized light, $P = 0$.
