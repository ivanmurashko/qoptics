\section{Laser Model}

The scheme of laser levels is shown in \autoref{figPart2Laser2_1}. The level $c$ is the main level, level $b$ is the lower working level, $a$ is the upper working level. In fact, the scheme is four-level. Pumping is carried out by incoherent light through a rather wide absorption line, depicted by a dashed line in the figure. But since a high rate of non-radiative transition to the upper working level is assumed, it can be considered that pumping occurs directly to level $a$. The presence of the fourth level allows a significant reduction of the reverse transition from level $a$ to the main level $c$. \autoref{figPart2Laser2_1} shows the transitions that will be considered in the theory. $\gamma_a$ and $\gamma_b$ characterize the relaxation of populations from levels $a$ and $b$ due to the connection with the dissipative system. $r_a$ is the rate of pumping of the upper working level $a$ due to incoherent optical pumping. The transition $a \rightarrow c$ is an induced transition caused by the laser field being generated.

\input ./part2/laser2/fig1.tex

The laser scheme is presented in \autoref{figPart2Laser2_2}. The scheme contains a resonator $F$ in which the generated mode is excited, interacting with the active atoms of the working medium, the scheme of which is presented in \autoref{figPart2Laser2_1}. In addition, there are two reservoirs at temperature $T$: $R_{a}$ (associated with the active atoms) and $R_{F}$ (associated with the mode), causing relaxation of the atoms and the mode field.

\input ./part2/laser2/fig2.tex