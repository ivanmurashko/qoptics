%% -*- coding:utf-8 -*- 
\section{Photocount distribution for coherent and chaotic
  light}
To get an idea of how the statistics of photo counts are related to the statistics of light incident on the photodetector,
we apply formula \eqref{eqCh4_49} to two limiting cases:
light constant in amplitude (intensity) and chaotic light with
fluctuating
intensity. Let us start with the simplest case when the intensity of the incident
light is constant. In this case, $\bar{I}\left(t, T\right)$ does not depend
on $t$ and $T$, $\bar{I}\left(t, T\right) = I_0 = const$,  so
the second averaging is not necessary. The same result can be obtained if in
\eqref{eqCh4_49} we consider $P\left(\bar{I}\right) =
\delta\left(\bar{I} - I_0\right)$.  The final expression for the case
of intensity-constant light is:
\begin{equation}
P_m = \frac{\bar{m}^m}{m!}e^{- \bar{m}}.
\label{eqCh4_50}
\end{equation}
where $\bar{m} = \xi I_0 T$ is the mean number of photo counts registered
by the photodetector during time $T$.  Thus, for constant
intensity, the photo count distribution is a Poisson distribution. \rindex{Poisson distribution}
Note that for sufficiently large $T$ (much greater than
the correlation time $\tau_c$), $\bar{I}\left(t, T\right)$ in any case\footnote{including for chaotic light} will tend to
a constant value, and the photo count distribution approaches a Poisson distribution. \rindex{Poisson distribution}

In the other limiting case, when $T \ll \tau_c$, one can consider
$\bar{I}\left(t, T\right) = I\left(t\right)$ - the instantaneous
intensity. In the case of chaotic light, $P\left(I\right) =
\frac{1}{\bar{I}} e^{- \frac{I}{\bar{I}}}$ \cite{bLoudon1976}.
According to Mandel's formula we have:
\begin{eqnarray}
P_m\left(T\right) = \frac{1}{\bar{I}}\int_0^{\infty} e^{- \frac{I}{\bar{I}}}
\frac{\left(\xi I T\right)^m}{m!} e^{-
  \xi I T} d I = 
\nonumber \\
= \int_0^{\infty} \frac{y^m\left(\bar{I} \xi
  T\right)^m}{m!\left(1 + \bar{m}\right)^{m + 1}} e^{-y} dy.
\label{eqCh4_51}
\end{eqnarray}
Here a change of variables was made:
\[
y = I \left(\frac{1}{\bar{I}} + \xi T\right),
\]
\[
I = \frac{y \bar{I}}{1 + \xi \bar{I} T},
\]
\[
\bar{m} = \xi \bar{I} T.
\]
Finally, we obtain
\begin{equation}
P_m\left(T\right) = 
\frac{\bar{m}^m}{\left(1 + \bar{m}\right)^{m + 1} m!}
\int_0^{\infty}y^m e^{-y}dy = 
\frac{\bar{m}^m}{\left(1 + \bar{m}\right)^{m + 1}}
\label{eqCh4_52}
\end{equation}
since
\[
\int_0^{\infty}y^m e^{-y}dy = m!.
\]

We have obtained that the photo count distribution replicates the
photon statistics for chaotic fields, but with a modified scale. It is much
more complicated to determine the photo count distribution for a counting time $T$
corresponding to the intermediate case $T \approx \tau_c$. Only numerical calculations
are possible here. The results of such calculations are shown
in \autoref{figPart4Ch2_6} \cite{bLoudon1976}. They allow judging the nature
of changes in the photo count distribution as the ratio
$T/\tau_c$ increases. From the graphs, it is clear that the character of the distribution changes
near $T = \tau_c$.  This allows, by conducting measurements for various
$T$, to estimate $\tau_c$ and the width of the spectrum of the incident
light $\sim 1/\tau_c$. 

\input ./part2/optic/fig6.tex