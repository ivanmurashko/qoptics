%% -*- coding:utf-8 -*- 
\section{Distribution of Photo-Counts for Coherent and Chaotic Light}

To understand how the statistics of photo-counts relate to the statistics of light incident on the photodetector, we apply formula \eqref{eqCh4_49} to two extreme cases: constant amplitude (intensity) light and chaotic light with fluctuating intensity. Let's start with the simplest case, when the intensity of the incident light is constant. In this case, $\bar{I}\left(t, T\right)$ does not depend on $t$ and $T$, $\bar{I}\left(t, T\right) = I_0 = const$, so the second averaging is not needed. We get the same result if in \eqref{eqCh4_49} we assume $P\left(\bar{I}\right) = \delta\left(\bar{I} - I_0\right)$. The final expression for the case of light with constant intensity is: 
\begin{equation}
P_m = \frac{\bar{m}^m}{m!}e^{- \bar{m}}.
\label{eqCh4_50}
\end{equation}
where $\bar{m} = \xi I_0 T$ is the average number of photo-counts recorded by the photodetector over time $T$. Thus, for constant intensity, the distribution of photo-counts is a Poisson distribution. \rindex{Poisson distribution}
Note that for sufficiently large $T$ (significantly larger than the correlation time $\tau_c$), $\bar{I}\left(t, T\right)$ in any case\footnote{including chaotic light} will tend to a constant value, and the distribution of photo-counts will become a Poisson distribution. \rindex{Poisson distribution}
  
In another extreme case, when $T \ll \tau_c$, we can consider $\bar{I}\left(t, T\right) = I\left(t\right)$ - the instantaneous intensity. In the case of chaotic light, $P\left(I\right) = \frac{1}{\bar{I}} e^{- \frac{I}{\bar{I}}}$ \cite{bLoudon1976}.  
By Mandel's formula, we have:
\begin{eqnarray}
P_m\left(T\right) = \frac{1}{\bar{I}}\int_0^{\infty} e^{- \frac{I}{\bar{I}}}
\frac{\left(\xi I T\right)^m}{m!} e^{-
  \xi I T} d I = 
\nonumber \\
= \int_0^{\infty} \frac{y^m\left(\bar{I} \xi
  T\right)^m}{m!\left(1 + \bar{m}\right)^{m + 1}} e^{-y} dy.
\label{eqCh4_51}
\end{eqnarray}
Here a change of variables is made:
\[
y = I \left(\frac{1}{\bar{I}} + \xi T\right),
\]
\[
I = \frac{y \bar{I}}{1 + \xi \bar{I} T},
\]
\[
\bar{m} = \xi \bar{I} T.
\]
Finally, we obtain
\begin{equation}
P_m\left(T\right) = 
\frac{\bar{m}^m}{\left(1 + \bar{m}\right)^{m + 1} m!}
\int_0^{\infty}y^m e^{-y}dy = 
\frac{\bar{m}^m}{\left(1 + \bar{m}\right)^{m + 1}}
\label{eqCh4_52}
\end{equation}
since
\[
\int_0^{\infty}y^m e^{-y}dy = m!.
\]

We have found that the distribution of photo-counts follows the distribution of photons for a chaotic field, but with a modified scale. It is much more difficult to determine the distribution of photo-counts for the counting time $T$, corresponding to the intermediate case $T \approx \tau_c$. Here, only numerical calculations are possible. The results of such calculations are provided in \autoref{figPart4Ch2_6} \cite{bLoudon1976}. They allow us to understand the nature of the change in the distribution of photo-counts with an increasing ratio of $T/\tau_c$. From the graphs, it is evident that the nature of the distribution changes near $T = \tau_c$. This allows, by conducting measurements at different $T$, to estimate $\tau_c$ and the width of the incident light spectrum $\sim 1/\tau_c$. 

\input ./part2/optic/fig6.tex