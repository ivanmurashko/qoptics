\section{Photon Counting Experiments. Application of Photon Counting Technique for Spectral Measurements}

We have seen that the distribution of photocounts depends on the counting time $T$. When the counting time is compared with the correlation time, the nature of the distribution changes, and a peak appears on the curve. With further increase in counting time, the nature of the curve is preserved, and for large counting times, the distribution will tend towards the Poisson distribution. \rindex{Poisson distribution}

Therefore, by measuring $P_m\left(T\right)$ at different $T$, one can estimate $\tau_c$ and the width of the light beam spectrum $\sim 1/\tau_c$. Such methods, where spectral parameters of light are determined from photon counting experiments, are called intensity fluctuation spectroscopy. The minimum counting time is determined by the resolution of the photodetector, which is about $10^{-8} - 10^{-9}$ seconds, corresponding to a frequency of $10^{8}\div 10^{9}$ Hz. This is the upper limit of the frequency changes of the method. The lower limit is determined by the maximum counting time, which is usually $1$ second, corresponding to a resolution of $1$ Hz.

Thus, the photon counting method can investigate the frequency interval from $1$ Hz to $10^8$ Hz. Consequently, this method complements conventional spectroscopy, which operates in the frequency range from $10^7$ Hz to $10^{15}$ Hz. It is more convenient to use another type of photon counting experiment. In these, the correlation between the numbers of photons $m_1$ and $m_2$, that is $\left<m_1 m_2\right>$, recorded in two short time intervals $\Delta t_1 = \Delta t_2 = \Delta t$, delayed relative to each other by time $\tau$, is measured. Both intervals have the same duration $\Delta t$, less than $\tau$, and the correlation time $\tau_c$. In this case, the second-order coherence is measured, defined by the formula 
\[
G_{12}^{(2)} = \frac{\left<m_1 m_2\right>}{\left(\bar{m}\right)^2} = 
1 + \left(G_{12}^{(1)}\right)^2
\]
where $\bar{m}$ is the average number of counts over time $\Delta t$.

Knowing $G_{12}^{(2)}$ for chaotic light, one can calculate the first-order coherence function
\[
G_{12}^{(1)} = \sqrt{G_{12}^{(2)} - 1}
\]
and the form and width of the spectral line of the light beam related to it by Fourier transformation. For chaotic light with a Lorentzian spectral line, we have:
\begin{equation}
\left<m_1 m_2\right> = \bar{m}^2\left(e^{-2 \gamma \left|\tau\right|}
+ 1\right),
\label{eqCh4_68}
\end{equation}
and for a Doppler line -
\begin{equation}
\left<m_1 m_2\right> = \bar{m}^2\left(e^{-\delta^2 \tau^2}
+ 1\right).
\label{eqCh4_69}
\end{equation}
Here $\delta$ and $\gamma$ define the line widths in these cases. By measuring these parameters in the experiment, the spectral width of the lines can be determined.