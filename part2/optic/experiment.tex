%% -*- coding:utf-8 -*- 
\section{Experiments on photon counting. Application of photon counting technique for spectral measurements}
We have seen that the distribution of photo counts depends on the counting time
$T$. When the counting time is compared with the correlation time, the nature of
the distribution changes, a maximum appears on the curve. With further
increase in counting time, the nature of the curve is preserved, and at large
counting times the distribution tends to the Poisson distribution. \rindex{Poisson distribution}
Therefore, by measuring $P_m\left(T\right)$ at different
$T$, one can estimate $\tau_c$ and the spectral width of the light beam $\sim
1/\tau_c$. Such methods, in which spectral parameters
of light are determined from photon counting experiments, are called intensity fluctuation spectroscopy. The minimum
counting time is determined by the resolution of the photodetector, which is on the order of
$10^{-8} - 10^{-9}$ sec, corresponding to frequencies of $10^{8}\div
10^{9}$ Hz. This is the upper frequency limit of the method. The lower
limit is determined by the maximum counting time, which is usually
$1$ sec, corresponding to a resolution of $1$ Hz.

Thus, the photon counting method can be used to study the frequency range
from $1$ Hz to $10^8$ Hz. Therefore, this method complements conventional
spectroscopy, which operates in the frequency range from $10^7$ Hz to
$10^{15}$ Hz. It is more convenient to use photon counting experiments
of another type. They measure the correlation between photon numbers $m_1$ and
$m_2$, i.e., $\left<m_1 m_2\right>$, recorded in two
short time intervals $\Delta t_1 = \Delta t_2 = \Delta t$,
delayed relative to each other by time $\tau$. Both intervals
have the same duration $\Delta t$, which is less than $\tau$ and the
correlation time $\tau_c$. In this case, the second-order coherence is measured,
defined by the formula
\[
G_{12}^{(2)} = \frac{\left<m_1 m_2\right>}{\left(\bar{m}\right)^2} = 
1 + \left(G_{12}^{(1)}\right)^2
\]
where $\bar{m}$ is the average number of counts in time $\Delta t$.

Knowing $G_{12}^{(2)}$ for chaotic light, the first-order coherence function can be calculated as
\[
G_{12}^{(1)} = \sqrt{G_{12}^{(2)} - 1}
\]
and via its Fourier transform, the shape and width of the spectral line of the light beam can be obtained. For chaotic light with
a Lorentzian spectral line we have:
\begin{equation}
\left<m_1 m_2\right> = \bar{m}^2\left(e^{-2 \gamma \left|\tau\right|}
+ 1\right),
\label{eqCh4_68}
\end{equation}
and for a Doppler line -
\begin{equation}
\left<m_1 m_2\right> = \bar{m}^2\left(e^{-\delta^2 \tau^2}
+ 1\right).
\label{eqCh4_69}
\end{equation}
Here $\delta$ and $\gamma$ determine the line widths in these
cases. By measuring these parameters experimentally, one can determine
the spectral width of the lines.
%One of the schemes suitable for such
%measurements is shown in Fig. 7 (shifted coincidence scheme).  
%Fig. 7.