\section{Photon Counting and Statistics}

We have verified that the quantization of the optical field and the existence of light quanta (photons) preserve the general picture of interference phenomena if the observation is conducted long enough for the averaged picture to emerge. In our formulas, this corresponds to averaging using the density matrix. However, the existence of photons allows for new types of experiments based on photon counting and the investigation of their statistical patterns. These methods are known as photon counting methods. The essence is as follows: the studied light is fed to a photodetector connected to a counter that counts the number of photoelectrons recorded over a certain period. A shutter in front of the photodetector (or circuit lock) controls the counting duration. With the counter set to zero, the shutter opens for a time $T$, and the number of photoelectrons is recorded. After a time longer than the correlation time $\tau_c$, everything repeats many times. Based on the measurements, one can determine $P_m\left(T\right)$ - the probability of registering $m$ counts of photoelectrons over time $T$:
\begin{equation}
P_m\left(T\right) = \frac{N_m}{N},
\label{eqCh4_40}
\end{equation}
where $N_m$ is the number of measurements in which $m$ photoelectrons were recorded, $N$ is the total number of measurements, which should be large. Obviously, it is assumed that the light flow is stationary. The obtained distribution contains information about the spectral properties of the light beams. The primary task here is to find the photon statistics from the photo-count statistics, which we can measure, required to obtain information about the properties of the light beams.