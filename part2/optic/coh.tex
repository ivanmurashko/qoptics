%% -*- coding:utf-8 -*- 
\section{Coherent Properties of Light}
The coherent properties of light determine the interference phenomena in
optics, which are well described by classical wave theory. In
classical wave optics, the coherent properties of light are described by
the corresponding coherence functions
\cite{bAhmanovNikitinPhysicalOptics2004, bMandel2000, bKilinQuantumOptics2003}.
%\cite{bKaluderSudershan1970}. FIXME!!! check it 
Let us consider the phenomenon of 
interference from a quantum point of view. To this end, consider
Young's interference experiment (see \autoref{figPart4Ch2_2}). On the screen
(photographic plate),
the arrival of a photon is recorded as a dark spot. To obtain
an interference pattern similar to the classical one, a sufficient
amount of time is needed. Then individual points merge into fringes and the pattern
will not differ from the classical one (\autoref{figPart4Ch2_3}). 

\input ./part2/optic/fig2.tex

\input ./part2/optic/fig3.tex

Field operators on the screen can be represented as a superposition of fields
passing through holes 1 and 2:
\begin{equation}
\hat{E}^{(+)}\left(\vec{r}, t\right) = 
\gamma \left[ 
\hat{E}^{(+)}\left(\vec{r}_1, t_1\right) +
\hat{E}^{(+)}\left(\vec{r}_2, t_2\right)
\right],
\label{eqCh4_11}
\end{equation}
where $\vec{r}_1$, $\vec{r}_2$ are the coordinates of the holes;  $t_1 = t -
\tau_1$,  $t_2 = t - \tau_2$;  $\tau_1$, $\tau_2$ are the delays due to
propagation of light from the holes to the screen; the coefficient $\gamma$
characterizes the attenuation of the field during propagation from the holes to
the screen. Using photodetector $D$, we examine the intensity of light at
different points on the screen.  

The photon counting rate will be, as we know, proportional to
\begin{eqnarray}
W = 
\alpha Sp\left(\hat{\rho}\hat{E}^{(-)}\left(\vec{r},
t\right)\hat{E}^{(+)}\left(\vec{r}, t\right)\right) = 
\nonumber \\
\alpha \left|\gamma\right|^2
Sp \left[\hat{\rho}
\left( 
\hat{E}^{(-)}\left(\vec{r}_1, t_1\right) +
\hat{E}^{(-)}\left(\vec{r}_2, t_2\right)
\right)
\left( 
\hat{E}^{(+)}\left(\vec{r}_1, t_1\right) +
\hat{E}^{(+)}\left(\vec{r}_2, t_2\right)
\right)
\right].
\nonumber
\end{eqnarray}

Multiplying out term by term, we obtain:
\begin{eqnarray}
W = 
g Sp \left[\hat{\rho}
\left( 
\hat{E}^{(-)}\left(\vec{r}_1, t_1\right) \hat{E}^{(+)}\left(\vec{r}_1,
t_1\right) +
\hat{E}^{(-)}\left(\vec{r}_2, t_2\right) \hat{E}^{(+)}\left(\vec{r}_2,
t_2\right) +
\right.
\right.
\nonumber \\
\left.
\left.
+
\hat{E}^{(-)}\left(\vec{r}_1, t_1\right) \hat{E}^{(+)}\left(\vec{r}_2,
t_2\right) +
\hat{E}^{(-)}\left(\vec{r}_2, t_2\right) \hat{E}^{(+)}\left(\vec{r}_1,
t_1\right)
\right)
\right],
\label{eqCh4_12}
\end{eqnarray}
where the following notation is introduced: $g = \alpha \left|\gamma\right|^2$.

The first two terms give the intensity of the fields passing through the first and
second holes when the other hole is closed. The last two terms
describe interference. Since  
\[
\left(\hat{E}^{(-)}\left(\vec{r}_2, t_2\right) \hat{E}^{(+)}\left(\vec{r}_1,
t_1\right)\right)^{\dag} = 
\left(\hat{E}^{(-)}\left(\vec{r}_1, t_1\right) \hat{E}^{(+)}\left(\vec{r}_2,
t_2\right)\right)
\]
then  
\(
Sp\left(\hat{\rho}\hat{E}^{(-)}\left(\vec{r}_2, t_2\right) \hat{E}^{(+)}\left(\vec{r}_1,
t_1\right)\right)
\)
and 
\(
Sp\left(\hat{\rho}\hat{E}^{(-)}\left(\vec{r}_1, t_1\right) \hat{E}^{(+)}\left(\vec{r}_2,
t_2\right)\right)
\)
will be complex conjugates. The term 
\[
Sp 
\left[
\hat{\rho}
\left(
\hat{E}^{(-)}\left(\vec{r}_1, t_1\right) \hat{E}^{(+)}\left(\vec{r}_2,
t_2\right) +
\hat{E}^{(-)}\left(\vec{r}_2, t_2\right) \hat{E}^{(+)}\left(\vec{r}_1,
t_1\right)
\right)
\right]
\]
gives
the oscillating interference term (\autoref{figPart4Ch2_3}). The envelope of the interference term is proportional to the expression 
\begin{equation}
\left|
Sp \left[
\hat{\rho}
\hat{E}^{(-)}\left(\vec{r}_1, t_1\right) \hat{E}^{(+)}\left(\vec{r}_2,
t_2\right)
\right]
\right|
\label{eqCh4_13}
\end{equation}
From this follows the definition of the first-order coherence function
\begin{eqnarray}
G\left(\vec{r}_1, t_1, \vec{r}_2, t_2\right) = 
G_{12}^{1} = 
\nonumber \\
= \frac{\left|
Sp \left[
\hat{\rho}
\hat{E}^{(-)}\left(\vec{r}_1, t_1\right) \hat{E}^{(+)}\left(\vec{r}_2,
t_2\right)
\right]
\right|}
{\sqrt{
Sp \left[
\hat{\rho}
\hat{E}^{(-)}\left(\vec{r}_1, t_1\right) \hat{E}^{(+)}\left(\vec{r}_1,
t_1\right)
\right]
Sp \left[
\hat{\rho}
\hat{E}^{(-)}\left(\vec{r}_2, t_2\right) \hat{E}^{(+)}\left(\vec{r}_2,
t_2\right)
\right]
}}
\label{eqCh4_14}
\end{eqnarray}

This expression resembles the classical definition of the coherence function,
but here instead of the analytic signal we have field operators and
quantum averaging is performed using the density matrix. 

Let us consider some particular cases as examples. We start with a single-mode
field in the states $\ket{n}$ or $\left|\alpha\right>$. In this case, we have: 
\begin{eqnarray}
\hat{\vec{E}}^{(+)}\left(\vec{r}, t\right) = \sqrt{\frac{\hbar \omega_k}{2 \varepsilon_0
    V}} \hat{a}_k \vec{e}_k e^{-i \omega_k t + i \left(\vec{k}\vec{r}
  \right)},
\nonumber \\
\hat{\vec{E}}^{(-)}\left(\vec{r}, t\right) = \sqrt{\frac{\hbar \omega_k}{2 \varepsilon_0
V}} \hat{a}_k^{\dag} \vec{e}_k^{*} e^{i \omega_k t - i \left(\vec{k}\vec{r}
  \right)} = \left(\hat{\vec{E}}^{(+)}\left(\vec{r}, t\right)\right)^{\dag}.
\label{eqCh4_15}
\end{eqnarray}

Using these expressions to calculate the coherence function
\eqref{eqCh4_14}, we get $G_{12}^{1} = 1$ since
$\bra{n}\hat{a}^{\dag}\hat{a}\ket{n} = n$   and
$\left|e^{-i x}\right| = 1$. The same result is obtained for the coherent state case: $G_{12}^{1} = 1$, as
$\left<\alpha\right|\hat{a}^{\dag}\hat{a}\left|\alpha\right> =
\left|\alpha\right|^2$. 

In general, a single-mode field excited in an arbitrary 
pure state \rindex{Pure state}
possesses full first-order coherence. Moreover,
a single-mode field excited in an arbitrary statistical mixed
state also exhibits full coherence. In this case, one needs to compute
$Sp \left(\hat{\rho} \hat{a}^{\dag}\hat{a}\right)$, where  $\hat{\rho} =
\sum_{(m)}\sum_{(l)}\rho_{ml}\ket{m}\bra{n}$.  We have  
\begin{eqnarray}
Sp \left(\hat{\rho} \hat{a}^{\dag}\hat{a}\right) = 
\sum_{(n)}\sum_{(m)}\sum_{(l)}\rho_{ml}\bra{n}\ket{m}\bra{l}
\hat{a}^{\dag}\hat{a}\ket{n} = 
\nonumber \\
= \sum_{(n)}\sum_{(l)}\rho_{nl}\bra{l}\ket{n}n = 
\sum_{(n)}\rho_{nn}n = \bar{n}.
\label{eqCh4_16}
\end{eqnarray}

A similar expression appears in the denominator, therefore,
$G_{12}^{(1)} = 1$.

More commonly, one deals with multimode states of the field, so let us consider
the coherence of light in this case. Consider two limiting cases:
light in a coherent state and fully chaotic
light. 

In the first case, the state vector is
\[
\left|\left\{\alpha_{k}\right\}\right> = 
\left|\left\{\alpha_{k_1}\right\}\right>
\left|\left\{\alpha_{k_2}\right\}\right>
\dotsc
\left|\left\{\alpha_{k_s}\right\}\right>
\dots 
\]
Using the properties of coherent states,
\[
\hat{E}^{(+)}\left(\vec{r},
t\right)\left|\left\{\alpha_{k}\right\}\right> = 
E\left(\vec{r},
t\right)
\left|\left\{\alpha_{k}\right\}\right>
\]
and
\[
\left<\left\{\alpha_{k}\right\}\right|\hat{E}^{(-)}\left(\vec{r},
t\right) = 
E^{*}\left(\vec{r},
t\right)\left<\left\{\alpha_{k}\right\}\right|,
\]
we get:
\[
\left<\left\{\alpha_{k}\right\}\right|\hat{E}^{(-)}\left(\vec{r}_1,
t_1\right)\hat{E}^{(+)}\left(\vec{r}_2,
t_2\right)\left|\left\{\alpha_{k}\right\}\right> = 
E^{*}\left(\vec{r}_1, t_1\right)E\left(\vec{r}_2,t_2\right) 
\]
where $E$ is the eigenvalue of the operator $\hat{E}^{(+)}$, being
the analytic signal of the classical field. From this, we have 
\begin{equation}
G_{12}^{(1)} = 
\frac{\left|
E^{*}\left(\vec{r}_1, t_1\right)E\left(\vec{r}_2,t_2\right)
\right|}{\sqrt{
E^{*}\left(\vec{r}_1, t_1\right)E\left(\vec{r}_1,t_1\right)
E^{*}\left(\vec{r}_2, t_2\right)E\left(\vec{r}_2,t_2\right)
}} 
= 1
\label{eqCh4_17}
\end{equation}
that is, a multimode field in a coherent state
is fully coherent.  

In the case of an arbitrary multimode field state, we have only
partial coherence. Consider the most common
case of chaotic light (radiation from a heated body, gas discharge,
etc.), where light is emitted by many independent sources
(atoms, ions, molecules). The expression for the density matrix in this
case has the form \eqref{eqCh1_102}: 
\begin{eqnarray}
\hat{\rho} = 
\sum_{\left\{n_k\right\}} 
 \left|\left\{n_k\right\}\right>\left<\left\{n_k\right\}\right|
\prod_{\left\{n_k\right\}} 
\frac{\bar{n}_k^{n_k}}{\left(1 + \bar{n}_k\right)^{n_k+1}} =
\nonumber \\
= \sum_{\left\{n_k\right\}} \rho_{\left\{n_k\right\},
  \left\{n_k\right\}}
\left|\left\{n_k\right\}\right>\left<\left\{n_k\right\}\right|  
\label{eqCh4_18}
\end{eqnarray}
where
\begin{eqnarray}
\rho_{\left\{n_k\right\},
  \left\{n_k\right\}} =
P_{\left\{n_k\right\}} = 
\prod_k 
\frac{\bar{n}_k^{n_k}}{\left(1 + \bar{n}_k\right)^{n_k+1}} = 
\prod_k P_{n_k},
\nonumber \\
\sum_{\left\{n_k\right\}} \dots = 
\sum_{n_1} \sum_{n_2} \dots \sum_{n_k} \dots. 
\nonumber
\end{eqnarray}
The electric field operators are
\begin{eqnarray}
\hat{\vec{E}}^{(+)}= \sum_{(k)}\sqrt{\frac{\hbar \omega_k}{2
    \varepsilon_0 V}} \vec{e}_k \hat{a}_k e^{-i \omega_k t + i
    \left(\vec{k} \vec{r}\right)},
\nonumber \\
\hat{\vec{E}}^{(-)}= \sum_{(k)}\sqrt{\frac{\hbar \omega_k}{2
    \varepsilon_0 V}} \vec{e}_k^{*} \hat{a}_k^{\dag} e^{i \omega_k t - i
    \left(\vec{k} \vec{r}\right)},
\label{eqCh4_19}
\end{eqnarray}
from which it follows:
\begin{eqnarray}
Sp \left(
\hat{\rho}\hat{E}^{(-)}\left(x_1\right)
\hat{E}^{(+)}\left(x_2\right)
\right) = 
\nonumber \\
=\sum_{\left\{n_{k'}\right\}}\sum_{\left\{n_{k}\right\}}
\left<\left\{n_{k'}\right\}\right|\left.\left\{n_{k}\right\}\right>
P_{\left\{n_k\right\}}
\left<\left\{n_{k}\right\}\right|
\hat{E}^{(-)}\left(x_1\right)
\hat{E}^{(+)}\left(x_2\right)
\left|\left\{n_{k'}\right\}\right> =
\nonumber \\
= \sum_{\left\{n_{k}\right\}}
\left<\left\{n_{k}\right\}\right|
\hat{E}^{(-)}\left(x_1\right)
\hat{E}^{(+)}\left(x_2\right)
\left|\left\{n_{k}\right\}\right>
P_{\left\{n_k\right\}}.
\label{eqCh4_20}
\end{eqnarray}
The product of operators equals
\begin{equation}
\hat{E}^{(-)}\left(x_1\right)
\hat{E}^{(+)}\left(x_2\right) = \sum_{(k)}\sum_{(k')}
\frac{\hbar \sqrt{\omega_k \omega_{k'}}}{2 \varepsilon_0 V}
\left(\vec{e}_k\vec{e}_{k'}\right)
\hat{a}_k^{\dag}\hat{a}_{k'}
e^{-i x_2 + i x_1},
\label{eqCh4_21}
\end{equation}
where  $x_1 = \omega_k t_1 - \left(\vec{k}\vec{r}_1\right)$,
$x_2 = \omega_k t_2 - \left(\vec{k}\vec{r}_2\right)$.
The operator $\hat{a}_k^{\dag}\hat{a}_{k'}$ is averaged. Since
\[\left|\left\{n_{k}\right\}\right> = 
\ket{n_{k_1}}
\ket{n_{k_2}} \dots,
\] 
we have 
\[
\left<\left\{n_{k}\right\}\right|
\hat{a}_k^{\dag}\hat{a}_{k'}
\left|\left\{n_{k}\right\}\right> = 
n_k \delta_{kk'},
\]
and, therefore, 
\begin{eqnarray}
Sp \left(
\hat{\rho}\hat{E}^{(-)}\left(x_1\right)
\hat{E}^{(+)}\left(x_2\right)
\right) = 
\nonumber \\
=\sum_{k}\sum_{\left\{n_{k}\right\}}
\frac{\hbar \omega_k}{2 \varepsilon_0 V} n_k e^{-i \left(x_2 - x_1
  \right)} 
P_{\left\{n_k\right\}} =
\nonumber \\
= \sum_{k}
\frac{\hbar \omega_k}{2 \varepsilon_0 V}
\sum_{\left\{n_{k}\right\}} n_k
\prod_k P_{n_k} =
\sum_{k} 
\frac{\hbar \omega_k}{2 \varepsilon_0 V}
\bar{n}_k e^{-i \left(x_2 - x_1\right)},
\nonumber
\end{eqnarray}
since
\begin{eqnarray}
\sum_{\left\{n_{k}\right\}} n_k
\prod_kP_{n_k} = 
\sum_{n_1}P_{n_1} 
\sum_{n_2}P_{n_2}
\dots
\sum_{n_k}n_kP_{n_k}
\dots = 
\nonumber \\
=   \sum_{n_k}n_kP_{n_k} = \bar{n}_k,
\nonumber
\end{eqnarray}
because
\[
\sum_{n_k}P_{n_k} = 1.
\]

From this we get
\begin{equation}
G_{12}^{(1)} = \frac{\left|
\sum_{k} 
\frac{\hbar \omega_k}{2 \varepsilon_0 V}
\bar{n}_k e^{-i \omega_k \left(t_2 - t_1\right) + 
i \left(\vec{k}, \vec{r}_2 - \vec{r}_1\right)}
\right|}{
\sum_{k}
\frac{\hbar \omega_k}{2 \varepsilon_0 V}
\bar{n}_k
}
\label{eqCh4_22}
\end{equation}
The summation over $k$ can be replaced by integration over
frequency. Using the expression for the density of states, we get: 
\begin{eqnarray}
\sum_{k}\left(\dots\right) = \frac{L^3}{\left(2 \pi\right)^3}
\int \int \int \left(\dots\right) d k_x d k_y d k_z =
\nonumber \\
= \frac{V}{\left(2 \pi\right)^3}
\int_{\Omega} d \Omega \int \omega^2 \left(\dots\right) d \omega.
\nonumber
\end{eqnarray}
This leads us to the expression
\begin{equation}
G_{12}^{(1)} = \frac{\left|
\int d \Omega \int \omega^3 \bar{n}\left(\omega, \Omega\right) 
e^{-i \omega\left(t_2 - t_1\right) + i \left(\vec{k}, \vec{r}_2 -
  \vec{r}_1 \right)}
d \omega
\right|}{
\int d \Omega \int \omega^3 \bar{n}\left(\omega, \Omega\right) d \omega
}.
\label{eqCh4_23}
\end{equation}
If the line is narrow compared to the carrier frequency, that is 
$\bar{n}\left(\omega, \Omega\right)$ has
a narrow peak near the frequency  $\sim \omega_0$,  the function $\omega^3$ can be
taken out of the integral and canceled; moreover, if the light beam is narrow,
that is, the integration domain over $\Omega$ is small, the expression
\eqref{eqCh4_23} can be simplified to 
\begin{equation}
G_{12}^{(1)} = \frac{\left|
\int \bar{N}\left(\omega\right) 
e^{-i \omega\left(t_2 - t_1\right) + i \left(\vec{k}, \vec{r}_2 -
  \vec{r}_1 \right)}
d \omega
\right|}{
\int \bar{N}\left(\omega\right) d \omega
},
\label{eqCh4_24}
\end{equation}
where $\bar{N}\left(\omega\right) = \int_{\Delta \Omega}
\bar{n}\left(\omega, \Omega\right) d \Omega$.  For example, when
the spectral line is Lorentzian, we have:
\[
\bar{N}\left(\omega\right) = \bar{N}_0\frac{\gamma}{\left(\omega_0 -
  \omega\right)^2 + \gamma^2} .
\]  

The problem considered can be solved using the representation of coherent
states. Then
\begin{equation}
\hat{\rho} = \int \dots \int P\left(\left\{\alpha_k\right\}\right)
\left|\left\{\alpha_k\right\}\right>\left<\left\{\alpha_k\right\}\right|d^2 \left\{\alpha_k\right\},
\nonumber
\end{equation}
where
\begin{equation}
P\left(\left\{\alpha_k\right\}\right) = \prod_k\frac{1}{\pi
  \bar{n}_k}e^{-\frac{\left|\alpha_k\right|^2}{\bar{n}_k}}=
\prod_k P_k\left(\alpha_k\right).
\nonumber
\end{equation}
We have
\begin{eqnarray}
Sp \left(
\hat{\rho}\hat{E}^{(-)}\left(x_1\right)
\hat{E}^{(+)}\left(x_2\right)
\right) = 
\nonumber \\
= \int \dots \int
P\left(\left\{\alpha_k\right\}\right)
\left<\left\{\alpha_k\right\}\right|
\hat{E}^{(-)}\left(x_1\right)
\hat{E}^{(+)}\left(x_2\right)
\left|\left\{\alpha_k\right\}\right>
d^2 \left\{\alpha_k\right\}.
\label{eqCh4_coh1_add1}
\end{eqnarray}
The product of operators equals
\begin{equation}
\hat{E}^{(-)}\left(x_1\right)
\hat{E}^{(+)}\left(x_2\right) = 
\sum_k \sum_{k'}
\frac{\hbar\sqrt{\omega_{k}\omega_{k'}}}{2 \varepsilon_0 V}
\left(\vec{e}_{k}\vec{e}_{k'}\right)\hat{a}^{\dag}_{k}\hat{a}_{k'}
e^{-i x_2 + i x_1}.
\nonumber
\end{equation}
The product of operators is written in normal order (annihilation operators to the right of creation operators), so
the matrix element appearing in \eqref{eqCh4_coh1_add1} can be easily written,
replacing $\hat{a}_{k'} \rightarrow \alpha_{k'}$, $\hat{a}^{\dag}_{k}
\rightarrow \alpha^{*}_{k}$. From this, we have
\begin{eqnarray}
Sp \left(
\hat{\rho}\hat{E}^{(-)}\left(x_1\right)
\hat{E}^{(+)}\left(x_2\right)
\right) = 
\nonumber \\
= 
\sum_{k \ne k'}
\frac{\hbar\sqrt{\omega_{k}\omega_{k'}}}{2 \varepsilon_0 V}
\left(\vec{e}_{k}\vec{e}_{k'}\right)
e^{-i x_2 + i x_1}
\cdot
\nonumber \\
\cdot
\int 
P\left(\alpha_{k}\right)
\alpha^{*}_{k}
d^2 \alpha_{k} 
\int 
P\left(\alpha_{k'}\right)
\alpha_{k'}
d^2 \alpha_{k'} +
\nonumber \\
+
\sum_k 
\frac{\hbar\omega_{k}}{2 \varepsilon_0 V}
e^{-i x_2 + i x_1}
\int 
\left|
\alpha_{k}
\right|^2
d^2 \alpha_{k}.
\label{eqCh4_coh1_add2}
\end{eqnarray}
In the derivation of \eqref{eqCh4_coh1_add2}, the following relation was used:
\[
\int 
P\left(\alpha_{k}\right)
d^2 \alpha_{k} = 1.
\]
Further, the following relations are used:
\begin{eqnarray}
\int P\left(\alpha_k\right)\alpha_k d^2\alpha_k = 0,
\nonumber \\
\int \left|\alpha_k\right|^2 d^2\alpha_k = \bar{n}_k,
\nonumber
\end{eqnarray}
which can be easily verified by switching to polar coordinates:
\begin{equation}
\alpha = \left|\alpha\right|e^{i\theta} = r e^{i\theta}.
\nonumber
\end{equation}
As a result, for \eqref{eqCh4_coh1_add2} we obtain
\begin{eqnarray}
Sp \left(
\hat{\rho}\hat{E}^{(-)}\left(x_1\right)
\hat{E}^{(+)}\left(x_2\right)
\right) = 
\sum_k 
\frac{\hbar\omega_{k}}{2 \varepsilon_0 V}
e^{-i x_2 + i x_1}
\bar{n}_k.
\nonumber
\end{eqnarray}
Using these results, which coincide with those obtained previously using the number state representation $n$, we
get the same final expressions \eqref{eqCh4_23} and
\eqref{eqCh4_24}. 