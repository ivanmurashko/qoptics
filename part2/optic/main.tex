%% -*- coding:utf-8 -*- 
\chapter{Photon Optics (Quantum Phenomena in Optics)}
\label{chOptic}
We will consider those optical phenomena in which the quantum properties of light manifest to some degree.

Although a large number of optical phenomena can be analyzed from a classical perspective, many phenomena can only be fully understood and described within the framework of a fully quantum description.

The quantum consideration allows a more complete understanding of the essence of interference experiments and on this basis to understand the connection between classical and quantum descriptions. In addition, the quantum approach allows for the consideration of new types of experiments that study the statistics of photons in light beams and its connection with the spectral properties of light.

\input ./part2/optic/photoeff.tex
\input ./part2/optic/coh.tex
\input ./part2/optic/coh2.tex
\input ./part2/optic/cohhigh.tex
\input ./part2/optic/calc.tex
\input ./part2/optic/statdep.tex
\input ./part2/optic/destrib.tex
\input ./part2/optic/statdeterm.tex
\input ./part2/optic/quant.tex
\input ./part2/optic/experiment.tex

\section{Exercises}
\begin{enumerate}
\item Show that formula \eqref{eqCh4_47} indeed provides the solution of the problem.
\item Derive expression \eqref{eqCh4_52} from formula \eqref{eqCh4_51}.
\item Obtain formula \eqref{eqCh4_66} using the coherent states representation.
\item Prove the orthogonality conditions for Laguerre polynomials \eqref{eqCh4_TaskLager1} and \eqref{eqCh4_TaskLager2}.
\item Expand $P_T\left(u\right)$ in a series of Laguerre polynomials: \eqref{eqCh4_55}-\eqref{eqCh4_56}.
\end{enumerate}

%% \begin{thebibliography}{99}
%% \bibitem{bCh1Optic_MandelVolf} Mandel L., Wolf E. Optical Coherence and Quantum Optics. V.: FML,  2000
%% \bibitem{bCh1Optic_Sudershan} Clauder J., Sudarshan E. Fundamentals of Quantum Optics. M.: Mir, 1970.
%% \bibitem{bCh1Optic_Loudon} Loudon R. Quantum Theory of Light. M.: Mir, 1976.
%% \bibitem{bCh1Optic_Dvait} Dwight H.B. Tables of Integrals and Other Mathematical Formulas. M.: Nauka, 1973.
%% \end{thebibliography}