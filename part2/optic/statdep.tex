%% -*- coding:utf-8 -*- 
\section{Connection of photon statistics with photoelectron count statistics}
Let us start with a semiclassical consideration and then generalize the results to a fully quantum treatment. We have seen that the photoemission rate is proportional to the intensity of the light flux, which is determined by the mean value of the operator $\left<\hat{E}^{(-)}\hat{E}^{(+)}\right>$. Since the quantum-mechanical intensity is analogous, in the classical case, to the intensity $I\left(t\right)$ averaged over a period, the electron emission rate can be considered proportional to $I\left(t\right)$ in the semiclassical approximation. Let $P\left(t\right)dt$ denote the probability of the appearance of a photoelectron in the time interval $t$ to $t + dt$. Then, $P\left(t\right)dt = \xi I\left(t\right)dt$, where $\xi$ characterizes the efficiency of the photocathode. Denote the probability of registering $m$ photo counts in the time interval $t$ to $t' + dt'$ by $P_m\left(t, t' + dt'\right)$. There are two possibilities to obtain $m$ counts in the given time interval, shown in \autoref{figPart4Ch2_5}:

\input ./part2/optic/fig5.tex

\begin{enumerate}
\item In the interval $t$ to $t'$, $m$ counts were made, and in the interval $dt$ - none.
\item In the interval $t$ to $t'$, $m - 1$ counts were made, and in the interval $dt'$ - one count.
\end{enumerate}
Note: the interval is so small that the probabilities of two or more counts are negligibly small. For these two cases, we can write
\begin{eqnarray}
P_m^{(1)}\left(t, t' + dt'\right) = 
P_m\left(t, t'\right)\left(1 - P\left(t'\right)dt'\right),
\nonumber \\
P_m^{(2)}\left(t, t' + dt'\right) = 
P_{m - 1}\left(t, t'\right)P\left(t'\right)dt',
\label{eqCh4_41}
\end{eqnarray}
where $P\left(t\right)dt = \xi I\left(t\right)dt$.  
The total probability will be the sum of the probabilities of these two events:
\begin{eqnarray}
P_m\left(t, t' + dt'\right) = 
P_m\left(t, t'\right)\left(1 - P\left(t'\right)dt'\right) +
\nonumber \\
+
P_{m - 1}\left(t, t'\right)P\left(t'\right)dt'.
\label{eqCh4_42}
\end{eqnarray}
From this, we get a chain of differential recurrence equations
\begin{equation}
\frac{dP_m}{dt'} = \xi I\left(t'\right)\left\{P_{m - 1}\left(t'\right)
- P_m\left(t'\right)\right\},
\label{eqCh4_43}
\end{equation}
which can be solved by sequentially integrating, starting with $m = 0$. The initial equation has the form:
\begin{equation}
\frac{dP_0}{dt'} = - \xi I\left(t'\right) P_0\left(t'\right),
\label{eqCh4_44}
\end{equation}
with obvious initial conditions $P_0\left(t'\right) = 1$ at $t' = t$. The solution of the equation under these initial conditions is
\begin{equation}
P_0\left(t, T\right) = e^{- \xi \int_t^{t + T} I\left(t'\right) d t'}, 
\label{eqCh4_45}
\end{equation}
where $T$ is the counting time. If we introduce the intensity averaged over the counting time
\[
\bar{I}\left(t, T\right) = \frac{1}{T}
\int_t^{t + T}I\left(t'\right)dt',
\]
then equation \eqref{eqCh4_45} can be rewritten as
\begin{equation}
P_0\left(t, T\right) = e^{- \xi \bar{I} T}
\label{eqCh4_46}
\end{equation}
The other probabilities can be successively expressed through $P_0\left(t, T\right)$. By induction it is easy to show that
\begin{equation}
P_m\left(t, T\right) = \frac{\left(\xi \bar{I} T\right)^m}{m!} e^{-
  \xi \bar{I} T} 
\label{eqCh4_47}
\end{equation}
One can verify the validity of this solution by substituting it into the original equation \eqref{eqCh4_43}. Generally, the intensity $\bar{I}\left(t, T\right)$ fluctuates from one counting interval to another. To account for this, an averaging over the ensemble of measurements needs to be done. The result is Mandel's formula:
\begin{equation}
P_m\left(T\right) = 
\left<P_m\left(t, T\right)\right> = 
\left<
\frac{\left(\xi \bar{I}\left(t, T\right) T\right)^m}{m!} e^{-
  \xi \bar{I}\left(t, T\right) T} 
\right>.
\label{eqCh4_48}
\end{equation}
Alternatively, this formula can be written as
\begin{equation}
P_m\left(T\right) = 
\left<P_m\left(t, T\right)\right> = 
\int_0^{\infty}
P\left(\bar{I}\right)
\frac{\left(\xi \bar{I}T\right)^m}{m!} e^{-
  \xi \bar{I} T} 
d \bar{I},
\label{eqCh4_49}
\end{equation}
where $P\left(\bar{I}\right)$ is the probability density for $\bar{I}$.