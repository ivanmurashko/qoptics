\chapter{\nameref{part1}}

\section{\nameref{chQuantel}} 

\subsection{\ref{qQuantelNumberMods}: Number of modes}
How many electromagnetic field modes with wavelength
  $\lambda \ge 500 \mbox{nm}$ are in the quantization cube with side
$L=1 \mbox{mm}$?

The wavelength range $\lambda \ge \lambda_0 = 500 \mbox{nm}$
corresponds to frequency range
$\omega \le \omega_0 = \frac{2 \pi c}{\lambda_0}$.

Using formula \ref{eqCh1_modenumber_1}
\[
d N = 2 \left(\frac{L}{2 \pi c} \right)^3 \omega^2 d \omega d \Omega
\]
hence
\begin{eqnarray}
  N = \int_0^{\omega_0} 2 \left(\frac{L}{2 \pi c} \right)^3 \omega^2 d
  \omega d \int_{4 \pi} d \Omega =
  \nonumber \\
  = 8 \pi \int_0^{\omega_0} \left(\frac{L}{2 \pi c} \right)^3 d \omega
  = 8 \pi \left(\frac{L}{2 \pi c}\right)^3 \frac{\omega_0^3}{3} =
  \nonumber \\
  = \frac{8 \pi}{3} \left(\frac{L}{\lambda_0}\right)^3 \approx 67
  \cdot 10^9
  \nonumber
\end{eqnarray}


\section{\nameref{chInteraction}}

\subsection{\ref{qInteractionFreq}: Determination of lithium atom transition frequency} 

%\input ./part1/interaction/figfreq.tex

Let us denote by $\Omega_R^{(1,2)}$ the effective Rabi frequencies for
measurements 1 and 2, respectively. The frequency detunings for the two
experiments are $\delta_{1,2}$. Thus, we have
\begin{eqnarray}
  \Omega_R^{(1)} = \sqrt{\omega_R^2 + \delta_1^2},
  \nonumber \\
  \Omega_R^{(2)} = \sqrt{\omega_R^2 + \delta_2^2}
  \label{eqAnswersInteraction40}
\end{eqnarray}

On the other hand, from the graphs
\autoref{figPart1InteractionQuestionFreq} it is seen that the maximum
probability to detect lithium atoms in state $\ket{2}$ for
frequency $\omega_2 = 2 \pi \cdot 228.4 \mbox{MHz}$ will be $P_2 = 0.49$,
while from \eqref{eqPart1InteractionRabiProbability} we have
\begin{equation}
  P_2 = \left(\frac{\omega_R}{\Omega_R^{(2)}}\right)^2 =
  \frac{1}{1 + \frac{\delta_2^2}{\omega_R^2}},
  \nonumber
\end{equation}
from which
\begin{equation}
  \omega_R = 0.98 \cdot \delta_2 .
  \label{eqAnswersInteraction41}
\end{equation}

On the other hand, one can conclude that
\[
\frac{\Omega_R^{(1)} \cdot 6 \mbox{µs}}{2} = \pi
\]
and
\[
\frac{\Omega_R^{(2)} \cdot 12.5 \mbox{µs}}{2} = \pi,
\]
thus
\[
\Omega_R^{(1)} = 2 \pi \cdot 0.167 \mbox{MHz},
\]
and
\[
\Omega_R^{(2)} = 2 \pi \cdot 0.08 \mbox{MHz},
\]

From \eqref{eqAnswersInteraction40} and  \eqref{eqAnswersInteraction41}
one can find
\[
\delta_2 = 2 \pi \cdot 0.057 \mbox{MHz},
\]
i.e. two candidates
\begin{equation}
  f_0 = 228.4 \pm 0.057 \mbox{MHz} = 228.343 \mbox{MHz},\,
  228.457 \mbox{MHz} 
\label{eqAnswersInteraction4Res1}
\end{equation}

For $\delta_1$ we have
\[
\delta_1 = \sqrt{\left(\Omega_R^{(1)}\right)^2 - \omega_R^2} =
2 \pi \sqrt{0.167^2 - (0.98 \cdot 0.057)^2} \mbox{MHz} =
2 \pi \cdot 0.157 \mbox{MHz}
\]
thus we get the next two candidates
\begin{equation}
  f_0 = 228.5 \pm 0.157 \mbox{MHz} = 228.343 \mbox{MHz},\,
  228.657 \mbox{MHz} 
\label{eqAnswersInteraction4Res2}
\end{equation}

From \eqref{eqAnswersInteraction4Res1} and
\eqref{eqAnswersInteraction4Res2} we finally have
\[
f_0 = 228.343 \mbox{MHz}.
\]

It is worth noting that the obtained value differs from the real $f_0 =
228.205 \mbox{MHz}$. This deviation is caused by the influence of the Zeeman effect from
the magnetic field, which is usually present in such experiments.



