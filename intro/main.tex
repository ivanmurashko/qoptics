%% -*- coding:utf-8 -*- 
\chapter*{Introduction}

Quantum optics studies optical phenomena where the quantum nature of light is evident. It can be said that quantum optics examines optical phenomena in which light and the system interacting with it must be described by quantum equations.

In quantum optics, the frequency range is approximately from 
\(f_1 \simeq 10^{13} \text{Hz}\) to \(f_2 \simeq 10^{18} \text{Hz}\), that is, from the infrared range to X-rays. The lower limit is determined by the condition that the quantum energy exceeds the thermal motion energy: 
\(\omega_1 \hbar > k T\). The upper limit is set as in quantum optics, non-relativistic electron energies are generally considered, and therefore the quantum energy should be noticeably less than the electron rest energy: \(\omega_2 \hbar \ll m c^2\).

Currently, there are few educational materials dedicated to quantum optics. Among domestic works, one should mention \cite{bTarasovQuantumOpticsIntro2008}. The proposed material reflects a number of questions of the quantum theory of light, presented in the course "Quantum Optics." The manual consists of three parts.

The first part of the manual provides an introduction to quantum electrodynamics. It examines the simplest tasks related to the interaction of the light field and matter. A quantum theory of the laser is built on a simple model. The quantum description of light coherence and its connection with the classical description is considered. Chapter \ref{chQuantel} gives a brief introduction to quantum electrodynamics within the scope necessary for presenting the course on quantum optics. Various quantum states of the light field used in quantum optics are considered. Special attention is paid to coherent states, which allow the quantum description of optical phenomena to be as close as possible to the classical one. In Chapter \ref{chInteraction}, the simplest tasks for the light field associated with both individual atoms and a large number of atoms in thermal equilibrium are considered. Special attention is given to the connection of the field with the thermostat, leading to relaxation processes that play a significant role in quantum optics and quantum electronics.

The second part of the manual deals with some applied issues of quantum optics. In Chapter \ref{chLaser}, a simple model is used to examine the quantum theory of the laser. An expression for the statistics of laser photons and a formula estimating the "natural" width of the laser generation line are obtained. Chapter \ref{chLaser2} reviews the quantum theory of the laser in the Heisenberg representation.

Chapter \ref{chOptic} is dedicated to photon optics. It outlines the quantum description of light coherence and its connection with classical description, coherence functions of different orders, the problem of photon statistics, and its connection with the statistics of photo-counts (Mandel's formula). Numerous examples are provided. Further, the connection of photon statistics with the spectral properties of light beams and its application for spectral measurements are considered.

In Chapter \ref{chInterfero}, a quantum description of interferometric experiments is given. The Mach-Zehnder interferometer and the error of interference measurements are examined. It will be shown later in Chapter \ref{chSqueezed} how to improve the accuracy of measurements using squeezed states.

Chapter \ref{chPolarization} is devoted to the quantum description of the polarization properties of light. Later, this material is used in the description of entangled states in Chapter \ref{ch:entangl}.

The third part of the manual discusses non-classical states of light. The definition of a non-classical state is provided (see Chapter \ref{chNonClass}). Squeezed (Chapter \ref{chSqueezed}) and entangled (see \autoref{ch:entangl}) non-classical states of light are described. For each type of non-classical state, the definition, ways of obtaining, application, and proof of non-classicality are given.

The fourth part of the manual considers quantum information theory. In particular, Chapter \ref{chQuantInfo} introduces quantum information theory. The concept of the amount of information in the classical and quantum case is defined. Shannon's classical coding theorem and its quantum analogue are considered. Chapter \ref{chQuantCrypto} is dedicated to information security theory (cryptography): the classical theory of cryptography and its main drawbacks are described. A description of quantum cryptography is given. Chapter \ref{chQuantComp} describes the main quantum algorithms: Shor's algorithm for factoring integers and Grover's algorithm for searching in an unstructured data array.

The appendices contain material that complements the main course and mainly serves a reference function, allowing one not to resort to special literature.

The manual is intended for senior students specializing in the field of quantum electronics.