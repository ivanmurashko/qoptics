%% -*- coding:utf-8 -*- 
\chapter*{Introduction}

Quantum optics studies optical phenomena in which the quantum nature of light is manifested. It can be said that quantum optics considers optical phenomena for which light and the system interacting with it must be described by quantum equations.

Quantum optics deals with the frequency range roughly from 
\(f_1 \simeq 10^{13} \mbox{Hz}\) to \(f_2 \simeq 10^{18}
\mbox{Hz}\), i.e., from the infrared range to
X-rays. The lower limit is determined by the condition that the energy of the quantum exceeds the thermal motion energy: 
\(\omega_1 \hbar > k T\). The upper limit is set based on the fact that in quantum optics
non-relativistic electron energies are usually considered and, accordingly, the quantum energy should be much less than the electron rest energy: \(\omega_2 \hbar \ll m c^2\).

Currently, there are few educational materials devoted to quantum optics. Among domestic ones, note \cite{bTarasovQuantumOpticsIntro2008}. The presented material reflects a number of issues of quantum theory of light, presented in the course "Quantum Optics". The manual consists of three parts.

The first part of the manual gives an introduction to quantum electrodynamics. The simplest problems related to the interaction of the light field and matter are considered. The quantum theory of the laser is constructed on the simplest model. The quantum description of light coherence and its connection with the classical description are examined.
Chapter \ref{chQuantel} provides a brief introduction to quantum electrodynamics at the level necessary for the exposition of the quantum optics course. Various quantum states of the light field used in quantum optics are considered. Special attention is given to coherent states, which allow the quantum description of optical phenomena to be approximated as closely as possible to the classical one.
Chapter \ref{chInteraction} considers the simplest problems for the light field associated with both individual atoms and a large number of atoms in thermal equilibrium. Special attention is paid to the connection of the field with the thermostat, leading to relaxation processes that play an important role in quantum optics and quantum electronics.

The second part of the manual addresses some applied questions of quantum optics. Chapter \ref{chLaser} discusses the quantum theory of the laser on a simple model. An expression for the statistics of laser photons and a formula evaluating the "natural" linewidth of the laser generation are obtained.
Chapter \ref{chLaser2} considers the quantum theory of the laser in the Heisenberg picture.

Chapter \ref{chOptic} is devoted to photon optics. The quantum description of light coherence and its connection with the classical description, coherence functions of various orders, the problem of photon statistics and its relation to the statistics of photo-detections (Mandel's formula) are presented. Numerous examples are given. Further, the connection between photon statistics and the spectral properties of light beams and its application to spectral measurements are considered.

Chapter \ref{chInterfero} provides a quantum description of interferometric experiments. The Mach-Zehnder interferometer and the error of interference measurements are discussed. Later, in chapter \ref{chSqueezed}, it will be shown how to increase measurement accuracy using squeezed states.

Chapter \ref{chPolarization} is dedicated to the quantum description of polarization properties of light. Later, this material is used in the description of entangled states in chapter \ref{ch:entangl}.

The third part of the manual discusses nonclassical states of light. A definition of a nonclassical state is given (see chapter \ref{chNonClass}). Squeezed (chapter \ref{chSqueezed}) and entangled (see \autoref{ch:entangl}) nonclassical states of light are described. For each type of nonclassical state, a definition, methods of obtaining, applications, and a proof of nonclassicality are provided.

In the fourth part of the manual, the quantum theory of information is considered. In particular, chapter \ref{chQuantInfo} presents an introduction to quantum information theory. The concept of information quantity in both classical and quantum cases is defined. The classical Shannon coding theorem and its quantum analogue are discussed. Chapter \ref{chQuantCrypto} is devoted to information protection theory (cryptography): the classical cryptography theory and its main drawbacks are described. A description of quantum cryptography is given. Chapter \ref{chQuantComp} describes the main quantum algorithms: Shor's algorithm for integer factorization and Grover's algorithm for searching in an unstructured data array.

The appendices contain material that complements the main course and is mostly of a reference nature, allowing one to avoid resorting to special literature.

The manual is intended for senior students specializing in quantum electronics.