%% -*- coding:utf-8 -*- 
\section{Hamiltonian of the atom-field system}
\rindex{Hamiltonian!atom-field system}
The Hamiltonian of the atom-field system can be represented as
\begin{equation}
\hat{\mathcal{H}} = \hat{\mathcal{H}}^{(A)} +
\hat{\mathcal{H}}^{(F)} + \hat{V}
\nonumber
\end{equation}
where
$\hat{\mathcal{H}}^{(A)}$ is the Hamiltonian of the two-level atom,
$\hat{\mathcal{H}}^{(F)}$ is the Hamiltonian of the electromagnetic field, and
$\hat{V}$ is the interaction Hamiltonian.

For the Hamiltonian of the electromagnetic field, we have the expression
\eqref{eqCh1_quant_stoyachie_volny}:
\begin{equation}
\hat{\mathcal{H}}^{(F)} = \hbar \omega 
\left(\hat{a}^{\dag}\hat{a} + \frac{1}{2}\right).
\nonumber
\end{equation}

In the case of the Hamiltonian of the two-level atom, we use the following
relations:
\begin{eqnarray}
\hat{\mathcal{H}}^{(A)}\ket{a} = \hbar \omega_a \ket{a},
\nonumber \\
\hat{\mathcal{H}}^{(A)}\ket{b} = \hbar \omega_b \ket{b},
\nonumber \\
\ket{a}\bra{a} + \ket{b}\bra{b} = \hat{I},
\nonumber
\end{eqnarray}
from which we obtain
\begin{eqnarray}
\hat{\mathcal{H}}^{(A)} = \hat{\mathcal{H}}^{(A)}\hat{I} = 
\hat{\mathcal{H}}^{(A)}
\ket{a}\bra{a} + \hat{\mathcal{H}}^{(A)}
\ket{b}\bra{b} =
\nonumber \\
 = 
\hbar\omega_a
\ket{a}\bra{a} + \hbar\omega_b
\ket{b}\bra{b} =
\nonumber \\
= \hbar \omega_a \hat{\sigma}_a +
\hbar \omega_b \hat{\sigma}_b,
\label{eqCh2HamiltonianAOperator}
\end{eqnarray}
where the following notations are introduced
\begin{equation}
\hat{\sigma}_a = \ket{a}\bra{a},
\quad
\hat{\sigma}_b = \ket{b}\bra{b}.
\nonumber
\end{equation}
The expression \eqref{eqCh2HamiltonianAOperator} can be rewritten in
matrix form:
\begin{equation}
\hat{\mathcal{H}}^{(A)} = \hbar 
\left(
\begin{array} {cc}
\omega_a & 0  
\\
0 & \omega_b 
\end{array}
\right).
\label{eqCh2HamiltonianA}
\end{equation}

For the interaction Hamiltonian $\hat{V}$ within the semiclassical
approach in the dipole approximation, we have:
\begin{equation}
\hat{V} = - e \left(\hat{\vec{r}} \vec{E}\left(t\right)\right).
\label{eqCh2HamiltonianVsemiclass}
\end{equation}
Applying the completeness condition twice to \eqref{eqCh2HamiltonianVsemiclass}, we obtain
\begin{eqnarray}
\hat{V} = \hat{I}\hat{V}\hat{I} = - e \left(\vec{E}\left(t\right)
\hat{I}\hat{\vec{r}}\hat{I}\right) = 
\nonumber \\
= - e \left(\vec{E}\left(t\right) 
\left(\ket{a}\bra{a} +
  \ket{b}\bra{b}\right)\hat{\vec{r}}
\left(\ket{a}\bra{a} +
  \ket{b}\bra{b}\right)\right) = 
\nonumber \\
= - \left(\vec{E}\left(t\right)
  \left(\vec{P}_{ab}\ket{a}\bra{b} +
    \vec{P}_{ba}\ket{b}\bra{a}\right)\right),
\label{eqCh2HamiltonianVsemiclass2}
\end{eqnarray}
where
\[
\vec{P}_{ab} = \vec{P}_{ba}^{*} = e \bra{a}\hat{\vec{r}}\ket{b}
\]
is the matrix element of the electric dipole moment.
For simplicity, we assume $\vec{P}_{ab} = \vec{P}_{ba} = \vec{p}$ to be a real quantity. As a result, expression
\eqref{eqCh2HamiltonianVsemiclass2} can be written as
\begin{eqnarray}
\hat{V} 
= - \left(\vec{E}\left(t\right)\vec{p}\right)
\left(
  \ket{a}\bra{b} +
  \ket{b}\bra{a}\right) = 
\nonumber \\
= - \left(\vec{E}\left(t\right)\vec{p}\right) \left(\hat{\sigma}^{\dag} + \hat{\sigma}\right),
\label{eqCh2HamiltonianVsemiclass3}
\end{eqnarray}
where  $\hat{\sigma}^{\dag} = \ket{a}\bra{b}$ and 
$\hat{\sigma} = \ket{b}\bra{a}$ 
are the transition operators. 
$\hat{\sigma}^{\dag}$ is the raising operator corresponding to the transition from the lower to the upper state:
\[
\hat{\sigma}^{\dag} \ket{b} = 
\ket{a}\bra{b}\ket{b} = 
\ket{a},
\]
$\hat{\sigma}$
is the lowering operator corresponding to the transition from the upper to the lower state:
\[
\hat{\sigma}
\ket{a} = 
\ket{b}\bra{a}\ket{a} = 
\ket{b}.
\]
In addition, the following equalities hold:
\[
\hat{\sigma}^{\dag} \ket{a} = 
\ket{a}\bra{b}\ket{a} = 
0
\]
and
\[
\hat{\sigma}
\ket{b} = 
\ket{b}\bra{a}\ket{b} = 
0.
\]
For the operators $\hat{\sigma}$ and $\hat{\sigma}^{\dag}$, the following relation can be written: 
\begin{eqnarray}
\hat{\sigma}\hat{\sigma}^{\dag} + \hat{\sigma}^{\dag}\hat{\sigma} =
\nonumber \\
= \ket{b}\bra{a}\ket{a}\bra{b} + 
 \ket{a}\bra{b}\ket{b}\bra{a} = 
\nonumber \\
= \ket{b}\bra{b} + \ket{a}\bra{a} = \hat{I}, 
\label{eqCh2_task1}
\end{eqnarray}
where $\hat{I}$ is the identity operator.

Expression \eqref{eqCh2HamiltonianVsemiclass3} can also be
written in matrix form:
\begin{equation}
\hat{V} = - \left(\vec{p} \vec{E}\right)
\left(
\begin{array} {cc}
0 & 1  
\\
1 & 0 
\end{array}
\right).
\label{eqCh2_8}
\end{equation}

Transition from the semiclassical interaction Hamiltonian to the fully
quantum expression is made by replacing the classical
electric field with the operator of the electric field. Thus, we obtain 
\begin{equation}
\hat{V} = - p E_1 \sin k z \left(\hat{a} + \hat{a}^{\dag}\right)
\left(\hat{\sigma} + \hat{\sigma}^{\dag}\right)
\label{eqCh2_8_add1}
\end{equation}
Here, the simplest expression for the mode electric field operator \eqref{eqCh1_EH_simple} is used.

The total Hamiltonian of the atom-field system has the form
\begin{equation}
\hat{\mathcal{H}}_{AF} = 
\hbar \omega_a \sigma_a + \hbar \omega_b \sigma_b +
\hbar \omega 
\left(\hat{a}^{\dag}\hat{a} + \frac{1}{2}\right)
+ \hbar g \left(\hat{a} + \hat{a}^{\dag}\right)
\left(\hat{\sigma} + \hat{\sigma}^{\dag}\right)
\nonumber
\end{equation}
where $g = -\frac{p}{\hbar}E_1 \sin k z$ is the interaction constant,
depending on the spatial distribution of the mode field.  

\input ./part1/interaction/fig2_1.tex

\input ./part1/interaction/fig2_2.tex

\input ./part1/interaction/fig2_3.tex

\input ./part1/interaction/fig2_4.tex


Not all terms entering the interaction Hamiltonian
\eqref{eqCh2_8_add1} are equivalent. When multiplying the brackets, we obtain
terms: 
\begin{enumerate}
\item $\hat{a}\hat{\sigma}$ - this term corresponds to the absorption of
  a photon and the transition of the atom from the upper state to the lower state. This process is conditionally depicted in \autoref{figPart1Ch2_2_1};  
\item $\hat{a}\hat{\sigma}^{\dag}$ - this term corresponds to the absorption
  of a photon and the transition of the atom from the lower state to the upper state (see \autoref{figPart1Ch2_2_2});  
\item $\hat{a}^{\dag}\hat{\sigma}$ - this term corresponds to the emission
  of a photon and the transition of the atom from the upper state to the lower state (see \autoref{figPart1Ch2_2_3}); 
\item $\hat{a}^{\dag}\hat{\sigma}^{\dag}$ - this term corresponds to
  the emission of a photon and the transition of the atom from the lower state to the upper state (see \autoref{figPart1Ch2_2_4}). 
\end{enumerate}


Cases 1 and 4 correspond to processes (see \autoref{figPart1Ch2_2_1} and
\ref{figPart1Ch2_2_4}) where the energy conservation law is violated.
These processes can be classified as so-called virtual processes. Their
probability tends to zero as time increases, and
their contribution to the interaction Hamiltonian can be
neglected. Remaining are terms 2 and 3 (see \autoref{figPart1Ch2_2_2} and
\ref{figPart1Ch2_2_3}), corresponding to processes
that occur without violation of the energy conservation law. Considering all the above, the interaction Hamiltonian takes the form: 
\begin{equation}
\hat{V} = g \hbar \left(
\hat{a}\hat{\sigma}^{\dag} + 
\hat{a}^{\dag}\hat{\sigma}
\right)
\end{equation}

It is convenient to switch to the interaction picture (see \autoref{AddWaveFuncInter}),
using the well-known 
formula \eqref{eqAddWaveFunc_VInter}
\begin{equation}
\hat{V}_I = 
e^{i \frac{\hat{\mathcal{H}}_0 t}{\hbar}}
\hat{V}
e^{- i \frac{\hat{\mathcal{H}}_0 t}{\hbar}}
\label{eqCh2_task21}
\end{equation}
where 
\(
\hat{\mathcal{H}}_0 = 
\hat{\mathcal{H}}^{\left(A\right)} +
\hat{\mathcal{H}}^{\left(F\right)}
\)
is the unperturbed Hamiltonian of the atom-field system. It can be shown that in
the interaction picture, the interaction Hamiltonian has the form 
\begin{equation}
\hat{V}_I = 
g \hbar \left(
\hat{a}^{\dag}\hat{\sigma} e^{i \left(\omega - \omega_{ab}\right)t} +
\hat{\sigma}^{\dag} \hat{a} e^{-i \left(\omega - \omega_{ab}\right)t}
\right).
\label{eqCh2_task22}
\end{equation}
This is most easily done by calculating the matrix elements 
$\left<n +
1\right|\bra{b}\hat{V}_I\ket{a}\ket{n}$ and
$\bra{n}\bra{a}\hat{V}_I\ket{b}\left|n +
1\right>$ for the operator $\hat{V}_I$, which can be represented in
the form \eqref{eqCh2_task21} and \eqref{eqCh2_task22}, and verifying that they
are the same in both cases.