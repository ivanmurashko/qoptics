%% -*- coding:utf-8 -*- 
\section{Ramsey Interferometer. Quantum Non-Demolition Measurements}

Observation of a photon \rindex{photon!non-demolition measurements} usually implies its destruction, for example, when using a photodetector the photon's energy is converted into an electrical signal and the photon itself disappears.

An interesting experiment \cite{Nogues1999Nv400p239} using Rabi oscillations demonstrated the possibility of observing a photon without destroying it. For this, a Ramsey interferometer was used (see \autoref{figPart1RamsiInterferometer})

\input ./part1/interaction/figramsi.tex

\input ./part1/interaction/figramsia.tex

The scheme works as follows. Atoms with the energy level structure shown in \autoref{figPart1RamsiAtom} are used. The transition frequency $\omega_{ab}$ between levels $a$ and $b$ matches the radiation frequency in regions $R_1$ and $R_2$ of the interferometer (see \autoref{figPart1RamsiInterferometer}). Region $C$ contains the investigated electromagnetic field for which we want to answer the question whether there is one photon in it or not, i.e., whether it is in state $\ket{1}$ or in the vacuum state $\ket{0}$. The investigated field in region $C$ has a frequency matching the transition frequency $\omega_{ac}$ between levels $a$ and $c$.

The size of regions $R_{1,2}$ is chosen so that the interaction time $t_R$ of the atom with the electromagnetic field in those regions satisfies the relation
\begin{equation}
  \omega_R^{(R)} t_R = \frac{\pi}{2},
  \nonumber
\end{equation}
where $\omega_R^{(R)}$ is the Rabi frequency corresponding to the transition frequency $\omega_{ab}$. The interaction time $t_C$ in region $C$ is given by
\begin{equation}
  \omega_R^{(C)} t_C = 2 \pi,
  \nonumber
\end{equation}
where $\omega_R^{(C)}$ is the Rabi frequency corresponding to the transition frequency $\omega_{ac}$.

For interaction in regions $R_{1,2}$, the following relations \eqref{eqPart1RabiAbsorbtion}, \ref{eqPart1RabiEmission} hold:
\begin{eqnarray}
  \ket{a} \rightarrow \frac{1}{\sqrt{2}}\left(
  \ket{a} - i \ket{b}  
  \right),
  \nonumber \\
  \ket{b} \rightarrow \frac{1}{\sqrt{2}}\left(
  -i \ket{a} + \ket{b}  
  \right).
  \label{eqPart1RamsiRzone}
\end{eqnarray}
Expression \eqref{eqPart1RamsiRzone} can be rewritten in matrix form as $\left|\psi\right> \rightarrow R \left|\psi\right>$, where
\[
R = \frac{1}{\sqrt{2}} \left(
\begin{array} {cc}
1 & -i
\\
-i & 1 
\end{array}
\right).
\]

Region $C$ does not affect state $\ket{b}$; for state $\ket{a}$ in the case of the vacuum state $\ket{0}$ (absence of a photon)
\begin{eqnarray}
  \ket{a} \rightarrow \ket{a},
  \nonumber
\end{eqnarray}
i.e., if there is no photon, the atom's state does not change. Indeed, if we consider the two-level system formed by states $\ket{a}$ and $\ket{c}$, then $\ket{a}\ket{0}$ is the minimal possible energy state from which the system (two-level atom and electromagnetic field) cannot transition into any other state.
In this case, for $C$ we have
\[
C_0 = \left(
\begin{array} {cc}
1 & 0
\\
0 & 1 
\end{array}
\right).
\]

For the case of a photon presence, from \eqref{eqPart1RabiAbsorbtion}
\begin{eqnarray}
  \ket{a} \rightarrow -\ket{a},
  \nonumber
\end{eqnarray}
and in this case $C$ has the form
\[
C_1 = \left(
\begin{array} {cc}
-1 & 0
\\
0 & 1 
\end{array}
\right).
\]

Thus, if initially the atom emitted by the source $S$ is in state $\ket{a} = \left(
\begin{array} {c}
1
\\
0
\end{array}
\right)$, then in the absence of a photon in region $C$, on detector $D$ we will detect the atom in state
\begin{eqnarray}
  R_2 C_0 R_1 \ket{a} =
  \nonumber \\
  =
  \frac{1}{2}
  \left(
  \begin{array} {cc}
    1 & -i
    \\
    -i & 1 
  \end{array}
  \right)
  \left(
  \begin{array} {cc}
    1 & 0
    \\
    0 & 1 
  \end{array}
  \right)
  \left(
  \begin{array} {cc}
    1 & -i
    \\
    -i & 1 
  \end{array}
  \right)
  \left(
  \begin{array} {c}
    1
    \\
    0
  \end{array}
  \right) =
  \nonumber \\
  =
  %% >> R=[1,-i;-i,1];
  %% >> C0=[1,0;0,1];
  %% >> a=[1;0];
  %% >> 1/2*R*C0*R*a
  %% ans =
  %% 0 + 0i
  %% 0 - 1i
  %% >>
  -i 
  \left(
  \begin{array} {c}
    0
    \\
    1
  \end{array}
  \right) =
  -i \ket{b},
  \nonumber
\end{eqnarray}
i.e., the atom will be detected in the unexcited state
$\ket{b}$.

For the case of one photon present in region $C$ we have
\begin{eqnarray}
  R_2 C_1 R_1 \ket{a} =
  \nonumber \\
  =
  \frac{1}{2}
  \left(
  \begin{array} {cc}
    1 & -i
    \\
    -i & 1 
  \end{array}
  \right)
  \left(
  \begin{array} {cc}
    -1 & 0
    \\
    0 & 1 
  \end{array}
  \right)
  \left(
  \begin{array} {cc}
    1 & -i
    \\
    -i & 1 
  \end{array}
  \right)
  \left(
  \begin{array} {c}
    1
    \\
    0
  \end{array}
  \right) =
  \nonumber \\
  =
  %% >> C1=[-1,0;0,1];
  %% >> R=[1,-i;-i,1];
  %% >> a=[1;0];
  %% >> 1/2*R*C1*R*a
  %% ans =
  %% -1
  %% 0
  %% >> 
  - 
  \left(
  \begin{array} {c}
    1
    \\
    0
  \end{array}
  \right) =
  - \ket{a},
  \nonumber
\end{eqnarray}
i.e., the atom will be detected in the excited state
$\ket{a}$. Thus, it can be concluded that the presence of a photon in region $C$ can be determined without destructive interaction with it.