\section{Ramsey Interferometer. Quantum Non-Demolition Measurements}

Observation of a photon \rindex{photon!non-destructive measurements} typically involves its destruction, for example when using a photodetector, the energy of the photon is converted into an electrical signal, causing the photon itself to disappear.

An intriguing experiment \cite{Nogues1999Nv400p239} using Rabi oscillations demonstrated the possibility of observing a photon without destroying it. A Ramsey interferometer was used in this experiment (see \autoref{figPart1RamsiInterferometer}).

\input ./part1/interaction/figramsi.tex

\input ./part1/interaction/figramsia.tex

The operation scheme is as follows. Atoms with level structures depicted in \autoref{figPart1RamsiAtom} are used. The transition frequency $\omega_{ab}$ between levels $a$ and $b$ matches the frequency of the radiation in zones $R_1$ and $R_2$ of the interferometer (see \autoref{figPart1RamsiInterferometer}). Zone $C$ contains the electromagnetic field under investigation, for which we want to determine whether it contains one photon or not, i.e., whether it is in state $\ket{1}$ or in vacuum state $\ket{0}$. The field under investigation in zone $C$ has a frequency that matches the transition frequency $\omega_{ac}$ between levels $a$ and $c$.  

The size of zones $R_{1,2}$ is chosen such that the interaction time $t_R$ of the atom with the electromagnetic field in them corresponds to the following relation
\begin{equation}
  \omega_R^{(R)} t_R = \frac{\pi}{2},
  \nonumber
\end{equation}
where $\omega_R^{(R)}$ is the Rabi frequency corresponding to the transition frequency $\omega_{ab}$.
The interaction time $t_C$ in zone $C$ is determined by the expression
\begin{equation}
  \omega_R^{(C)} t_C = 2 \pi,
  \nonumber
\end{equation}
where $\omega_R^{(C)}$ is the Rabi frequency corresponding to the transition frequency $\omega_{ac}$.

For interaction in zones $R_{1,2}$, the following relations hold \eqref{eqPart1RabiAbsorbtion}, \ref{eqPart1RabiEmission}:
\begin{eqnarray}
  \ket{a} \rightarrow \frac{1}{\sqrt{2}}\left(
  \ket{a} - i \ket{b}  
  \right),
  \nonumber \\
  \ket{b} \rightarrow \frac{1}{\sqrt{2}}\left(
  -i \ket{a} + \ket{b}  
  \right).
  \label{eqPart1RamsiRzone}
\end{eqnarray}
The expression \eqref{eqPart1RamsiRzone} can be rewritten in matrix form $\left|\psi\right> \rightarrow R \left|\psi\right>$,
where
\[
R = \frac{1}{\sqrt{2}} \left(
\begin{array} {cc}
1 & -i
\\
-i & 1 
\end{array}
\right).
\]

Zone $C$ does not affect state $\ket{b}$; for state $\ket{a}$ in the case of vacuum state $\ket{0}$ (absence of a photon),
\begin{eqnarray}
  \ket{a} \rightarrow \ket{a},
  \nonumber
\end{eqnarray}
i.e., in the absence of a photon, the state of the atom does not change. Indeed, if we consider a two-level system formed by states $\ket{a}$ and $\ket{c}$, then $\ket{a}\ket{0}$ will be the minimally possible energy state, from which the system (two-level atom and electromagnetic field) cannot transition to any other state.
In this case, for $C$ we have
\[
C_0 = \left(
\begin{array} {cc}
1 & 0
\\
0 & 1 
\end{array}
\right).
\]

In the presence of a photon, from \eqref{eqPart1RabiAbsorbtion}
\begin{eqnarray}
  \ket{a} \rightarrow -\ket{a},
  \nonumber
\end{eqnarray}
and in this case $C$ is represented as
\[
C_1 = \left(
\begin{array} {cc}
-1 & 0
\\
0 & 1 
\end{array}
\right).
\]

Thus, if initially the atom emitted by source $S$ is in state $\ket{a} = \left(
\begin{array} {c}
1
\\
0
\end{array}
\right)$, then in the absence of a photon in zone $C$, we will detect the atom in state
\begin{eqnarray}
  R_2 C_0 R_1 \ket{a} =
  \nonumber \\
  =
  \frac{1}{2}
  \left(
  \begin{array} {cc}
    1 & -i
    \\
    -i & 1 
  \end{array}
  \right)
  \left(
  \begin{array} {cc}
    1 & 0
    \\
    0 & 1 
  \end{array}
  \right)
  \left(
  \begin{array} {cc}
    1 & -i
    \\
    -i & 1 
  \end{array}
  \right)
  \left(
  \begin{array} {c}
    1
    \\
    0
  \end{array}
  \right) =
  \nonumber \\
  =
  -i 
  \left(
  \begin{array} {c}
    0
    \\
    1
  \end{array}
  \right) =
  -i \ket{b},
  \nonumber
\end{eqnarray}
i.e., the atom will be observed in the unexcited state $\ket{b}$.

In the presence of one photon in zone $C$, we have
\begin{eqnarray}
  R_2 C_1 R_1 \ket{a} =
  \nonumber \\
  =
  \frac{1}{2}
  \left(
  \begin{array} {cc}
    1 & -i
    \\
    -i & 1 
  \end{array}
  \right)
  \left(
  \begin{array} {cc}
    -1 & 0
    \\
    0 & 1 
  \end{array}
  \right)
  \left(
  \begin{array} {cc}
    1 & -i
    \\
    -i & 1 
  \end{array}
  \right)
  \left(
  \begin{array} {c}
    1
    \\
    0
  \end{array}
  \right) =
  \nonumber \\
  =
  - 
  \left(
  \begin{array} {c}
    1
    \\
    0
  \end{array}
  \right) =
  - \ket{a},
  \nonumber
\end{eqnarray}
i.e., the atom will be observed in the excited state $\ket{a}$. Thus, it is possible to determine the presence of a photon in zone $C$ without destructively affecting it.