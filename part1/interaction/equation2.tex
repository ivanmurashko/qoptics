%% -*- coding:utf-8 -*- 
\section{Equation of Motion for the Statistical Operator of the Field Mode in the Coherent State Representation}
In quantum optics and quantum electronics, it is often convenient to use coherent states $\left|\alpha\right>$. In this case, the statistical operator can be written in diagonal representation \eqref{eqCh1_rhorepresent}  
\begin{equation}
\hat{\rho}\left(t\right) = \int
P\left(\alpha, t\right)\left|\alpha\right>\left<\alpha\right| d^2 \alpha,
\label{eqCh2_65}
\end{equation}
Substituting this expression into the equation of motion for the statistical operator \eqref{eqCh2_rho_final2}, \ref{eqCh2_64}, we get: 
\begin{eqnarray}
\int \dot{P}\left(\alpha, t\right)\left|\alpha\right>\left<\alpha\right|
d^2 \alpha  = 
\nonumber \\
= -\frac{1}{2}\int P\left(\alpha, t\right)\left[R_a
\left(\hat{a}\hat{a}^{\dag}\left|\alpha\right>\left<\alpha\right|-\hat{a}^{\dag}\left|\alpha\right>\left<\alpha\right|\hat{a}\right)
\right. +
\nonumber \\
+
\left.
R_b
\left(\hat{a}^{\dag}\hat{a}\left|\alpha\right>\left<\alpha\right|-\hat{a}\left|\alpha\right>\left<\alpha\right|\hat{a}^{\dag}\right)
\right]d^2 \alpha + \text{h.c.}
\label{eqCh2_66}
\end{eqnarray}

The following equalities hold
\begin{eqnarray}
\hat{a}^{\dag}\left|\alpha\right>\left<\alpha\right| = 
\left(\frac{\partial}{\partial \alpha} +
\alpha^{*}\right)\left|\alpha\right>\left<\alpha\right|, 
\nonumber \\
\left|\alpha\right>\left<\alpha\right|\hat{a} = 
\left(\frac{\partial}{\partial \alpha^{*}} +
\alpha\right)\left|\alpha\right>\left<\alpha\right|, 
\label{eqCh2_67}
\end{eqnarray}

These equalities are easily proven if we use one of the forms of the definition of the coherent state
\[
\left|\alpha\right> = e^{\alpha \hat{a}^{\dag} -
  \frac{1}{2}\alpha\alpha^{*}} \ket{0}.
\]
From here we have 
\[
\left|\alpha\right>\left<\alpha\right| = e^{\alpha \hat{a}^{\dag} -
  \frac{1}{2}\alpha\alpha^{*}} \ket{0}
\bra{0}e^{\alpha^{*} \hat{a} -
  \frac{1}{2}\alpha^{*}\alpha}.
\]
Differentiating this expression by $\alpha$, we get:
\begin{eqnarray}
\frac{\partial}{\partial \alpha}\left|\alpha\right>\left<\alpha\right|
= \left(\hat{a}^{\dag} -
\alpha^{*}\right) e^{\alpha \hat{a}^{\dag} -
  \frac{1}{2}\alpha\alpha^{*}} \ket{0}
\bra{0}e^{\alpha^{*} \hat{a} -
  \frac{1}{2}\alpha^{*}\alpha} = 
\nonumber \\
= \left(\hat{a}^{\dag} -
\alpha^{*}\right)\left|\alpha\right>\left<\alpha\right| 
\nonumber
\end{eqnarray}
thus,
\begin{equation}
\hat{a}^{\dag}\left|\alpha\right>\left<\alpha\right| = 
\left(\frac{\partial}{\partial \alpha} +
\alpha^{*}\right)\left|\alpha\right>\left<\alpha\right|.
\label{eqCh2_68}
\end{equation}
In the same way, the validity of the second equality is shown
\begin{eqnarray}
\frac{\partial}{\partial \alpha^{*}}\left|\alpha\right>\left<\alpha\right|
=  e^{\alpha \hat{a}^{\dag} -
  \frac{1}{2}\alpha\alpha^{*}} \ket{0}
\bra{0}e^{\alpha^{*} \hat{a} -
  \frac{1}{2}\alpha^{*}\alpha} 
\left(\hat{a} -
\alpha\right). 
\nonumber
\end{eqnarray}
Thus,
\begin{equation}
\left|\alpha\right>\left<\alpha\right|\hat{a} = 
\left(\frac{\partial}{\partial \alpha^{*}} +
\alpha\right)\left|\alpha\right>\left<\alpha\right|, 
\label{eqCh2_68a}
\end{equation}

Using \eqref{eqCh2_67} and the known relations for the creation and annihilation operators $\hat{a}\left|\alpha\right> = \alpha\left|\alpha\right>$,
$\left<\alpha\right|\hat{a}^{\dag} = \alpha^{*}\left<\alpha\right|$,
we transform the equation \eqref{eqCh2_66} to the form:   
\begin{eqnarray}
\int \dot{P}\left(\alpha,
t\right)\left|\alpha\right>\left<\alpha\right| d^2 \alpha  = 
\nonumber \\
= -\frac{1}{2}R_a\int
P\left(\alpha,t\right)
\left\{
-\left(
\frac{\partial^2}{\partial \alpha \partial \alpha^{*}} +
\alpha^{*}\frac{\partial}{\partial \alpha^{*}}
\right)
\left|\alpha\right>\left<\alpha\right| 
\right\}
d^2 \alpha - 
\nonumber \\
-\frac{1}{2}R_b\int P\left(\alpha,t\right)
\alpha \frac{\partial}{\partial \alpha}
\left|\alpha\right>\left<\alpha\right| 
d^2 \alpha + \text{h.c.}
\label{eqCh2_69}
\end{eqnarray}

We need to consider the integrals involved in \eqref{eqCh2_69},
transforming them using integration by parts.
In the integrals involved in \eqref{eqCh2_69} $d^2\alpha = dx dy$, if $\alpha = x + i y$, $x = \text{Re}\,\alpha$, $x = \text{Im}\,\alpha$.
Thus, for integration by parts, it is necessary to express the underintegral function in terms of $x$, $y$, perform integration by parts, and then return to the previous variables. But we can proceed
differently. Formally consider $\alpha$ and $\alpha^{*}$ independent variables and express $dx dy$ through $d\alpha d\alpha^{*}$.
\[
d^2\alpha = dx dy = d\alpha d\alpha^{*} \left|J\right|,
\]
where $\left|J\right|$ is the Jacobian of the transformation, which in our case is a constant $\left|J\right| = \frac{1}{2}$. Now we can integrate by parts, and in the final expression, switch back to the previous notation 
\[
d\alpha d\alpha^{*} \left|J\right| \rightarrow d^2\alpha.
\]
We have  
\begin{eqnarray}
\int P \alpha \frac{\partial}{\partial \alpha}
\left|\alpha\right>\left<\alpha\right| 
d^2 \alpha = 
\nonumber \\
= \left.\alpha P \left|\alpha\right>\left<\alpha\right|
\right|_{-\infty}^{+\infty} - 
\int \frac{\partial}{\partial \alpha}\left(\alpha P\right)
\left|\alpha\right>\left<\alpha\right|d^2 \alpha = 
\nonumber \\
= 
- \int \frac{\partial}{\partial \alpha}\left(\alpha P\right)
\left|\alpha\right>\left<\alpha\right|d^2 \alpha.
\label{eqCh2_70}
\end{eqnarray}

Here, we assumed that $\alpha P\left(\alpha, t\right) \rightarrow 0$, as $\left|\alpha\right| \rightarrow \infty$. The integral with the second derivative, after double integration by parts, is transformed to the form: 
\begin{eqnarray}
\int P \frac{\partial^2}{\partial \alpha \partial \alpha^{*}}
\left|\alpha\right>\left<\alpha\right| 
d^2 \alpha = 
\nonumber \\
= \int \left(\frac{\partial^2}{\partial \alpha \partial \alpha^{*}}
P\right) \left|\alpha\right>\left<\alpha\right|  
d^2 \alpha
\label{eqCh2_71}
\end{eqnarray}
Substituting all this into equality \eqref{eqCh2_69}, we obtain:
\begin{eqnarray}
\int \dot{P} \left(\alpha, t\right) 
\left|\alpha\right>\left<\alpha\right| 
d^2 \alpha = 
\nonumber \\
= - \int \left\{
\frac{1}{2}\left(R_a - R_b\right)
\left[
\frac{\partial}{\partial \alpha}
\left(P \alpha\right) + \text{c.c.}
\right]
\right.
-
\nonumber \\
- \left.
R_a \frac{\partial^2 P}{\partial \alpha \partial \alpha^{*}}
\right\}
\left|\alpha\right>\left<\alpha\right| 
d^2 \alpha
\label{eqCh2_72}
\end{eqnarray}

This equality will be satisfied if we set the multiplier for  
$\left|\alpha\right>\left<\alpha\right|$ equal to zero. Thus, we obtain 
\begin{equation}
\dot{P}\left(\alpha, t\right) = 
-\frac{1}{2}\left(R_a - R_b\right)
\left(
\frac{\partial}{\partial \alpha}\left(\alpha P\right) + \text{c.c.}
\right)
+ R_a
\frac{\partial^2 P}{\partial \alpha \partial \alpha^{*}}
\label{eqCh2_73}
\end{equation}
This equation is of the Fokker-Planck type, which is used in solving classical problems of statistical physics, such as problems about Brownian motion and other similar tasks. By solving \eqref{eqCh2_73} with appropriate initial and boundary conditions, it is easy to use \eqref{eqCh2_65} to write an expression for the statistical operator 
\[
\hat{\rho}\left(t\right) = 
\int P\left(\alpha, t\right)
\left|\alpha\right>\left<\alpha\right| 
d^2 \alpha
\]
and using \eqref{eqCh1_middleO} and \eqref{eqCh1_113} to calculate the average values of observable quantities related to the system under consideration   
\[
\left<O\right>= \int P\left(\alpha, t\right) \left<\alpha\right|\hat{O}\left|\alpha\right>d^2
\alpha =
\int  P\left(\alpha, t\right)O^{\left(n\right)}\left(\alpha, \alpha^{*}\right)d^2\alpha,
\]
where $O^{\left(n\right)}$ is the normal representation of the operator $\hat{O}$ \eqref{eqCh1_normalO}.
The coefficients $R_a$ and  $R_b$, involved in \eqref{eqCh2_73}, can, as before \eqref{eqCh2_RabQw}, be expressed through $Q$  and $\bar{n}_T$, then the equation takes the form 
\begin{equation}
\frac{\partial}{\partial t}P\left(\alpha, t\right) = 
\frac{1}{2}\frac{\omega}{Q}\left[
\frac{\partial}{\partial \alpha}\left(
\alpha P\left(\alpha, t\right)
\right) + \text{c.c.}
\right]
+
\frac{\omega \bar{n}_T}{Q} \frac{\partial^2  P\left(\alpha,
  t\right)}{\partial \alpha \partial \alpha^{*}}.
\label{eqCh2_74}
\end{equation}

As an example, we will calculate the average value of the operator of the positive frequency part of the electric field $E_1\hat{a}$.   
\begin{eqnarray}
\frac{d \left<E_1\hat{a}\right>}{d t} = 
E_1\int \alpha \dot{P}d^2 \alpha =
\nonumber \\
=E_1 \frac{1}{2}\frac{\omega}{Q}
\int \left[
\frac{\partial}{\partial \alpha}\left(\alpha P\right) + \text{c.c.}
\right]\alpha d^2 \alpha +
\nonumber \\
+E_1\frac{\omega \bar{n}_T}{Q}\int \alpha
\frac{\partial^2 P}{\partial \alpha \partial \alpha^{*}}
d^2 \alpha
\label{eqCh2_75}
\end{eqnarray}
Integrating the second term by $\alpha^{*}$ we get
\begin{eqnarray}
\int\alpha\frac{\partial^2 P}{\partial \alpha \partial \alpha^{*}}
d^2 \alpha = 
\left.\frac{\partial P}{\partial
  \alpha}\alpha\right|_{-\infty}^{\infty} - 
\nonumber \\
- \int \frac{\partial P}{\partial \alpha} \alpha
\frac{\partial \alpha}{\partial \alpha^{*}} = 0,
\nonumber
\end{eqnarray}
since $\frac{\partial P}{\partial \alpha}$ is assumed to rapidly
approach 0 at infinity, and 
$\frac{\partial \alpha}{\partial \alpha^{*}} = 0$, since 
$\alpha$ and $\alpha^{*}$ are independent variables.
The first term can be written as follows
\begin{eqnarray}
\int \alpha \left(
\frac{\partial}{\partial \alpha}\left(\alpha P\right) + 
\frac{\partial}{\partial \alpha^{*}}\left(\alpha^{*} P\right)
\right) d^2 \alpha = 
\nonumber \\
= - \int \alpha P \frac{\partial \alpha}{\partial \alpha} d^2 \alpha -
\int \alpha^{*} P \frac{\partial \alpha}{\partial \alpha^{*}} d^2 \alpha  
= - \int \alpha P  d^2 \alpha
= - \left<\hat{a}\right>.
\nonumber
\end{eqnarray}
Thus, finally, we have: 
\begin{equation}
\frac{d \left<E_1\hat{a}\right>}{d t} = 
-\frac{\omega}{2 Q}\left<E_1\hat{a}\right>
\label{eqCh2_76}
\end{equation}
- the result we obtained earlier by another method. Equation \eqref{eqCh2_74} can be written in various coordinate systems, for example, in polar $\alpha = r e^{i \theta}$, $\alpha^{*} = r e^{- i \theta}$. By performing the appropriate transformations, we obtain 
\begin{eqnarray}
\frac{\partial}{\partial t}P\left(r, \theta, t\right) = 
\frac{1}{2}\frac{\omega}{Q}
\frac{\partial}{\partial r}
\left(r^2 P\left(r, \theta, t\right)\right) +
\nonumber \\
+ \frac{1}{2}\frac{\omega}{Q}
\frac{\bar{n}_T}{2 r^2}
\left(
r \frac{\partial}{\partial r} r \frac{\partial}{\partial r} +
\frac{\partial^2}{\partial \theta^2}
\right)
P\left(r, \theta, t\right).
\label{eqCh2_77}
\end{eqnarray}
If we use the coordinates  $x$ and $y$, 
$\alpha = x + i y$, $\alpha^{*} = x - i y$, that is 
$x = \frac{1}{2}\left(\alpha + \alpha^{*}\right)$,  
$y = - \frac{i}{2}\left(\alpha - \alpha^{*}\right)$, the equation can be represented as 
\begin{eqnarray}
\frac{\partial}{\partial t}P\left(x, y, t\right) = 
\frac{1}{2}\frac{\omega}{Q}
\left(
\frac{\partial}{\partial x} x +
\frac{\partial}{\partial y} y
\right)
P\left(x, y, t\right) +
\nonumber \\
+
\frac{1}{4}
\frac{\omega}{Q}\bar{n}_T
\left(
\frac{\partial^2}{\partial x^2} +
\frac{\partial^2}{\partial y^2}
\right)
P\left(x, y, t\right).
\label{eqCh2_77a}
\end{eqnarray}

In solving a specific problem, it is convenient to use the equation written either in rectangular or in polar coordinate systems.