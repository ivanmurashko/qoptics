\section{Interaction of an Atom with a Multimode Field. Induced and Spontaneous Transitions}

Previously, we established that the probability of an atom transitioning from a lower to an upper state when interacting with a single mode of the field in a state with $n$ photons is given by \eqref{eqCh2_prob_C_an} 
\begin{equation}
\left|C_{a,n}\left(t\right)\right|^2 = 4 g^2 \left(n+1\right)
  \frac{sin^2\left(\left(\omega - 
  \omega_{ab}\right)t/2\right)} {\left(\omega - 
  \omega_{ab}\right)^2}
  \nonumber
\end{equation}

We can express this probability as follows
\begin{equation}
  P_a = \left|C_{a,n}\left(t\right)\right|^2 =
  P_a^{\text{ind}} + P_a^{\text{spont}},
  \label{eqCh2_prob_C_an_2_1}
\end{equation}
where
\begin{eqnarray}
  P_a^{\text{ind}} = 4 g^2 n
  \frac{sin^2\left(\left(\omega - 
  \omega_{ab}\right)t/2\right)} {\left(\omega - 
    \omega_{ab}\right)^2},
  \nonumber \\
  P_a^{\text{spont}} = 4 g^2 
  \frac{sin^2\left(\left(\omega - 
  \omega_{ab}\right)t/2\right)} {\left(\omega - 
  \omega_{ab}\right)^2}
  \label{eqCh2_prob_C_an_2}
\end{eqnarray}
are the probabilities of induced ($P_a^{\text{ind}}$) and spontaneous ($P_a^{\text{spont}}$) transitions.

\subsection{Induced Transitions}

We will now focus on the probability of the induced transition $P_a^{\text{ind}}$. To obtain the transition probability considering interactions with all modes, the probabilities \eqref{eqCh2_prob_C_an_2} corresponding to each mode need to be summed. 

In the case of an atom interacting with a free space field (field expansion in plane waves), the coupling constant is given by \cite{bLuisell1972}: 
\begin{equation}
g = - \frac{\left(\vec{p}\vec{e}\right)}{\hbar}
\sqrt{\frac{\hbar \omega}{2 \varepsilon_0 V}}
\end{equation}
where $\vec{p}$ is the matrix element of the dipole moment operator, and $\vec{e}$ is the field's polarization vector.

If in expression \eqref{eqCh2_prob_C_an_2} time $t$ is not very small (but significantly less than the characteristic time of probability changes), the dependence of $\left|C_{a,n}\left(t\right)\right|^2$ on the frequency will have a sharp peak at $\omega = \omega_{ab}$, indicating energy conservation in the elementary act (the energy of the absorbed photon equals the change in the atom's energy). Assume that the incident field in the direction $\vec{k}$ within the solid angle $d \Omega$ has a spectrum that changes little near frequency $\omega_{ab}$. 

The number of modes corresponding to the frequency interval $d \omega$ near $\omega_{ab}$ and the solid angle $d \Omega$ around the direction $\vec{k}$ is \eqref{eqCh1_modenumber_1},
\begin{equation}
d N = 2 \left(\frac{L}{2 \pi c} \right)^3 \omega^2 d \omega d \Omega
\end{equation}

The probability of photon absorption during atom interaction with a single mode is given by formula \eqref{eqCh2_prob_C_an_2}. The total probability can be obtained by summing over all modes. Assuming the mode spectrum is quasi-continuous (see \eqref{eqCh1_modenumber_kvazy_contig}), summation can be replaced by integration
\begin{equation}
W_{b \rightarrow a} = \int_{\Omega} 8 \left|g\right|^2 n(\vec{k})
\left(\frac{L}{2 \pi c}\right)^3 \omega_{ab}^2 d \Omega 
\int_{-\infty}^{+\infty} 
\frac{sin^2\left(\left(\omega - 
  \omega_{ab}\right)t/2\right)} {\left(\omega - 
  \omega_{ab}\right)^2} d \omega.
\label{eqCh2Wba_1}
\end{equation}

It's noted that $\frac{\sin^2 x}{x^2}$ has a narrow peak near $x = 0$, while in expression \eqref{eqCh2Wba_1} we assume that $t$ is small compared to the characteristic time change of $C$, but sufficiently large to exhibit the filter properties of the integral. All slowly varying terms are taken at $\omega = \omega_{ab}$ and factored out of the integral. It is known that $\int_{-\infty}^{+\infty} \frac{\sin^2 x}{x^2} dx = \pi$. Thus, we easily obtain  
\[
\int_{-\infty}^{+\infty} \frac{sin^2\left(\left(\omega - 
  \omega_{ab}\right)t/2\right)} {\left(\omega - 
  \omega_{ab}\right)^2} d \omega = \frac{\pi t }{2}.
\]
Therefore, we have   
\begin{equation}
W_{b \rightarrow a} = t \frac{\omega^2 2 \pi \omega}
{\hbar \varepsilon_0 \left(2 \pi c\right)^3}\int_{\Omega} 
n(\vec{k})\left|\left(\vec{p} \vec{e}\right)\right|^2
d \Omega.
\label{eqCh2_Wab} 
\end{equation}

Note that the quantization volume does not enter into the final expression, and the number of photons $n$ depends on the direction, i.e., the solid angle $\Omega$. Formula \eqref{eqCh2_Wab} shows that the transition probability is proportional to time. This allows the introduction of the transition rate, i.e., the probability of transition per unit time 
\begin{equation}
w_{b \rightarrow a} = \frac{W_{b \rightarrow a}}{t} = \frac{2 \pi
  \omega^3 }  
{\hbar \varepsilon_0 \left(2 \pi c\right)^3}\int_{\Omega} 
n(\vec{k}) \left|\left(\vec{p} \vec{e}\right)\right|^2
d \Omega.
\label{eqCh2_wab} 
\end{equation}

The transition rate or photon absorption rate can be expressed \cite{bLuisell1972} in terms of the energy flux (photon flux) traveling in the direction $\vec{k}$ within the solid angle $d \Omega$.  The energy flux is defined as the energy transferred per unit area $dS$ per unit time $dt$, i.e.
\[
d I = \frac{dH}{dS dt}.
\]

From \autoref{figPart1Ch2_add1}, we have that the number of photons passing through area $dS$ per unit time $dt$ is equal to the number contained in the cylinder shown in \autoref{figPart1Ch2_add1}, i.e., the transferred energy can be written as
\[
dH = n \hbar \omega d N ,
\]
where $dN$ is the number of modes corresponding to the interval $d \omega d \Omega$ in the considered volume $c \cdot dS  dt$ \eqref{eqCh1_modenumber_1}:
\[
d N = 2 \left(\frac{1}{2 \pi c} \right)^3 \omega^2 
c \cdot dS  dt
d \omega d \Omega,
\]
from which we obtain 
\begin{equation}
d I = I\left(\omega, \vec{k}\right) d \omega d \Omega = 
\frac{2 n \hbar \omega c}{\left(2 \pi c\right)^3}
\omega^2 d \omega d \Omega,
\label{eqCh2_dI}
\end{equation}
where $I\left(\omega, \vec{k}\right)$ is the photon energy flux in the direction $\vec{k}$ per unit frequency interval and unit solid angle. Thus, from \eqref{eqCh2_dI}, we obtain 
\begin{equation}
I\left(\omega, \vec{k}\right) = 
\frac{2 n \hbar \omega^3 c}{\left(2 \pi c\right)^3}
\nonumber
\end{equation}

Substituting this into expression \eqref{eqCh2_wab}, we obtain:
\begin{equation}
w_{ab} = \frac{\pi}{\hbar^2}\sqrt{\frac{\mu_o}{\varepsilon_0}}
\int_{\Omega}I\left(\omega, \vec{k}\right)
\left|\left(\vec{p} \vec{e}\right)\right|^2
d \Omega.
\label{eqCh2_WWab}
\end{equation}

The relation $\mu_0 \varepsilon_0 = \frac{1}{c^2}$ is used here.

\input ./part1/interaction/fig6.tex

To find the total absorption rate, integrate \eqref{eqCh2_WWab} over all wave propagation directions. Additionally, the incident radiation is usually unpolarized. To account for this, all polarization directions should be averaged. We use the coordinate system shown in \autoref{figPart1Ch2_6}. Direction $\vec{k}$ is chosen as the polar axis $z$. Vectors $\vec{e}_1$ and $\vec{e}_2$ are directed along $x$ and $y$, respectively. Angles $\varphi$ and $\theta$ define the direction of $\vec{p}$. The angle $\varphi'$ defines the polarization direction of the incident wave. From the figure, it is seen that
\[
\left|\left(\vec{p} \vec{e}\right)\right|^2 = 
\left|p\right|^2 \sin^2 \theta \cos^2 \varphi'. 
\]
Averaging over all polarizations gives: 
\begin{equation}
\frac{\left|p\right|^2}{2 \pi} \int_0^{2 \pi}
\cos^2 \varphi' d \varphi' = \frac{\left|p\right|^2}{2}.
\label{eqCh2_PolyarMedian}
\end{equation}

Summation over all wave arrival directions leads to expression
\begin{equation}
w_{ab} = \frac{\pi}{2 \hbar^2}\sqrt{\frac{\mu_o}{\varepsilon_0}}
\left|p\right|^2
\int_{\Omega}I\left(\omega, \vec{k}\right)
\sin^3 \theta d \theta d \varphi.
\end{equation}

Assuming radiation comes from all directions and is isotropic, obtain: 
\begin{equation}
w_{ab} = \frac{\pi}{2 \hbar^2}\sqrt{\frac{\mu_o}{\varepsilon_0}}
\left|p\right|^2 I\left(\omega\right)
\int_{0}^{2 \pi}d \varphi \int_0^{\pi}
\sin^3 \theta d \theta = 
\frac{\pi}{ \hbar^2}\sqrt{\frac{\mu_o}{\varepsilon_0}}
\frac{\left|p\right|^2}{3}I_0,
\label{eqCh2_Wab_1}
\end{equation}
where $I_0 = 4 \pi I\left(\omega\right)$ is the total energy flux incident on the atom. In deriving \eqref{eqCh2_Wab_1}, the following relation was used:
\[
 \int_0^{\pi}
\sin^3 \theta d \theta = \frac{4}{3}
\]

If, however, the radiation comes from a small region $\Delta \Omega$ of directions near angles $\theta_0$, $\varphi_0$, then we can write:  
\begin{eqnarray}
w_{ab} = \frac{\pi}{2 \hbar^2}\sqrt{\frac{\mu_o}{\varepsilon_0}}
\left|p\right|^2 \sin^2 \theta_0
\int_{\Delta \Omega} I\left(\theta, \varphi\right)
\delta \Omega = 
\nonumber \\
= 
\frac{\pi}{2 \hbar^2}\sqrt{\frac{\mu_o}{\varepsilon_0}}
\left|p\right|^2  I_0 \sin^2 \theta_0,
\label{eqCh2_Wab_2}
\end{eqnarray}
where $I_0 = \int_{\Delta \Omega} I\left(\theta, \varphi\right)
d \Omega$ is the total flux illuminating the atom.

\subsection{Spontaneous Transitions}

Now, consider another problem. Determine the probability of photon emission or the transition of an excited atom to a lower state.  

For the probability of atom emission of a photon into one mode, we have the expression \eqref{eqCh2_prob_C_bn}:
\[
\left|C_{b, n + 1}\left(t\right)\right|^2 = 4 g^2 \left(n + 1\right)
\frac{sin^2\left(\left(\omega - \omega_{ab}\right)t/2\right)}
{\left(\omega - \omega_{ab}\right)^2},
\]
which, like \eqref{eqCh2_prob_C_an_2_1}, can be considered as a sum of two terms
\[
P_b = \left|C_{b, n + 1}\left(t\right)\right|^2 =
P_b^{\text{ind}} + P_b^{\text{spont}}
\]
where the first term
\[
P_b^{\text{ind}} = 4 g^2 n
\frac{sin^2\left(\left(\omega - \omega_{ab}\right)t/2\right)}
{\left(\omega - \omega_{ab}\right)^2}
\]
corresponds to induced emission, and the second
\[
P_b^{\text{spont}} = 4 g^2 
\frac{sin^2\left(\left(\omega - \omega_{ab}\right)t/2\right)}
{\left(\omega - \omega_{ab}\right)^2}
\]
to spontaneous emission. The term corresponding to induced emission is considered in the same way as for photon absorption. The result will obviously be the same. The corresponding formulas will match \eqref{eqCh2_Wab_1}, \eqref{eqCh2_Wab_2}. It follows that the probabilities of induced absorption and emission processes are equal. The probability of spontaneous emission per unit time will obviously be equal to 
\begin{equation}
w_{\text{spont}} = 
\frac{2 \pi \omega^3}
{\hbar \varepsilon_0 \left(2 \pi c\right)^3}
\int_{\Omega}
\left|\left(\vec{p} \vec{e}\right)\right|^2
d \Omega
\label{eqCh2_Wspon}
\end{equation}

To obtain the total probability, \eqref{eqCh2_Wspon} must be integrated over all directions, since spontaneous emission can occur into any mode. For polarizations, use the average value \eqref{eqCh2_PolyarMedian}. Thus, the entire procedure reduces to calculating the integral 
\begin{equation}
w_{\text{spont}} = 
\frac{\left|p\right|^2 \omega^3}
{\varepsilon_0 \hbar \left(2 \pi\right)^2 c^3}
\int_{0}^{2 \pi}d \varphi \int_0^{\pi}
\sin^3 \theta d \theta
= 
\frac{4 \pi}{3}\frac{\left|p\right|^2 \omega^3}
{\hbar \varepsilon_0 \left(2 \pi\right)^2 c^3}
\end{equation}

Ultimately, we get
\begin{equation}
w_{\text{spont}} = 
\frac{\left|p\right|^2 \omega^3}
{3 \pi c^2 \hbar}
\sqrt{\frac{\mu_0}{\varepsilon_0}}
\label{eqCh2_Wspon_final}
\end{equation}

The relation 
\[
\frac{1}{c} = \sqrt{\mu_0 \varepsilon_0}
\]
is used here.

Formula \eqref{eqCh2_Wspon_final} indicates a strong dependence of the spontaneous transition probability on frequency, as $\omega^3$. Thus, based on quantum electrodynamics equations, we directly, without invoking external considerations, obtained expressions for the probabilities of induced and spontaneous atomic transitions per unit time. A shortcoming of this consideration is the use of perturbation theory, necessitating limitations to small times. For instance, when solving the problem of the lifetime of an excited atom, one cannot restrict to short times alone. In this case, another approximation known as the Weisskopf-Wigner approximation is applied \cite{bLuisell1972}. Interestingly, the results obtained in these two cases do not contradict each other.
