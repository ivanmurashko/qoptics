%% -*- coding:utf-8 -*- 
\section{Interaction of an atom with a multimode field. Induced and
  spontaneous transitions}
Above, we established that the probability of the atom transitioning from the lower to
the upper state when interacting with a single mode of the field, which is in
a state with $n$ photons, is given by \eqref{eqCh2_prob_C_an} 
\begin{equation}
\left|C_{a,n}\left(t\right)\right|^2 = 4 g^2 \left(n+1\right)
  \frac{\sin^2\left(\left(\omega - 
  \omega_{ab}\right)t/2\right)} {\left(\omega - 
  \omega_{ab}\right)^2}
  \nonumber
\end{equation}
Let us write this probability in the following form
\begin{equation}
  P_a = \left|C_{a,n}\left(t\right)\right|^2 =
  P_a^{\mbox{ind}} + P_a^{\mbox{sp}},
  \label{eqCh2_prob_C_an_2_1}
\end{equation}
where
\begin{eqnarray}
  P_a^{\mbox{ind}} = 4 g^2 n
  \frac{\sin^2\left(\left(\omega - 
  \omega_{ab}\right)t/2\right)} {\left(\omega - 
    \omega_{ab}\right)^2},
  \nonumber \\
  P_a^{\mbox{sp}} = 4 g^2 
  \frac{\sin^2\left(\left(\omega - 
  \omega_{ab}\right)t/2\right)} {\left(\omega - 
  \omega_{ab}\right)^2}
  \label{eqCh2_prob_C_an_2}
\end{eqnarray}
are the probabilities of induced ($P_a^{\mbox{ind}}$)
and spontaneous ($P_a^{\mbox{sp}}$) transitions.

\subsection{Induced transitions}
Further, we will focus on the probability of the induced transition
$P_a^{\mbox{ind}}$.
To obtain the transition probability considering interaction with all
modes, one needs to sum the probabilities \eqref{eqCh2_prob_C_an_2},
corresponding to each mode. 

In the case of an atom interacting with the free space field
(expansion of the field into plane waves), the interaction constant is
\cite{bLuisell1972}: 
\begin{equation}
g = - \frac{\left(\vec{p}\cdot\vec{e}\right)}{\hbar}
\sqrt{\frac{\hbar \omega}{2 \varepsilon_0 V}}
\end{equation}
where $\vec{p}$ is the matrix element of the dipole moment operator,
$\vec{e}$ is the polarization vector of the field. 

If in expression \eqref{eqCh2_prob_C_an_2} the time $t$ is not very small (but
significantly less than the characteristic time for the change of probabilities),
the dependence of $\left|C_{a,n}\left(t\right)\right|^2$ on frequency will
have a sharp peak at $\omega = \omega_{ab}$, which indicates
energy conservation in the elementary act (the energy of the absorbed photon
equals the change in the atom's energy). Let us assume that the field incident in the direction
$\vec{k}$ within the solid angle $d \Omega$ has a spectrum that varies little near the frequency $\omega_{ab}$. 
 
The number of modes within the frequency interval $d \omega$ near
$\omega_{ab}$ and the solid angle $d \Omega$ around the direction
$\vec{k}$, as is known from \eqref{eqCh1_modenumber_1}, is
\begin{equation}
d N = 2 \left(\frac{L}{2 \pi c} \right)^3 \omega^2 d \omega d \Omega.
\end{equation}

The probability of photon absorption \rindex{photon} during atom interaction with one mode
is given by formula \eqref{eqCh2_prob_C_an_2}. The total probability can
be obtained by summing over all modes. Treating the mode spectrum
as quasi-continuous (see \eqref{eqCh1_modenumber_kvazy_contig}),
the summation can be replaced by integration
\begin{equation}
W_{b \rightarrow a} = \int_{\Omega} 8 \left|g\right|^2 n(\vec{k})
\left(\frac{L}{2 \pi c}\right)^3 \omega_{ab}^2 d \Omega 
\int_{-\infty}^{+\infty} 
\frac{\sin^2\left(\left(\omega - 
  \omega_{ab}\right)t/2\right)} {\left(\omega - 
  \omega_{ab}\right)^2} d \omega.
\label{eqCh2Wba_1}
\end{equation}
Here it is acknowledged that $\frac{\sin^2 x}{x^2}$ has a narrow peak near $x = 0$, and in expression \eqref{eqCh2Wba_1} we assume that $t$ is small compared to the characteristic time of change in $C$, but large enough for the filtering properties of the integral to manifest. All slowly varying terms are taken at $\omega = \omega_{ab}$ and pulled out of the integral. It is known that $\int_{-\infty}^{+\infty} \frac{\sin^2 x}{x^2} dx = \pi$. Hence,
\[
\int_{-\infty}^{+\infty} \frac{\sin^2\left(\left(\omega - 
  \omega_{ab}\right)t/2\right)} {\left(\omega - 
  \omega_{ab}\right)^2} d \omega = \frac{\pi t }{2}.
\]
Thus, we have   
\begin{equation}
W_{b \rightarrow a} = t \frac{\omega^2 2 \pi \omega}
{\hbar \varepsilon_0 \left(2 \pi c\right)^3}\int_{\Omega} 
n(\vec{k})\left|\left(\vec{p} \cdot \vec{e}\right)\right|^2
d \Omega.
\label{eqCh2_Wab} 
\end{equation}
Note that the quantization volume does not appear in the final expression,
and the photon number $n$ depends on direction, i.e. on solid angle $\Omega$. Formula \eqref{eqCh2_Wab} shows that the transition probability 
is proportional to time. This allows introducing the concept of the transition rate, i.e., transition probability per unit time
\begin{equation}
w_{b \rightarrow a} = \frac{W_{b \rightarrow a}}{t} = \frac{2 \pi
  \omega^3 }  
{\hbar \varepsilon_0 \left(2 \pi c\right)^3}\int_{\Omega} 
n(\vec{k}) \left|\left(\vec{p} \cdot \vec{e}\right)\right|^2
d \Omega.
\label{eqCh2_wab} 
\end{equation}
The transition rate or photon absorption rate can be expressed
\cite{bLuisell1972} through the energy flux (photon flux),
propagating in direction $\vec{k}$ in  
the solid angle $d \Omega$. Energy flux is defined as energy
transferred through a unit area $dS$ per unit time $dt$, i.e.
\[
d I = \frac{dH}{dS dt}.
\]
\input ./part1/interaction/fig_add1.tex

From \autoref{figPart1Ch2_add1} we have that through the area $dS$ during unit
time $dt$ passes as many photons as are contained in the cylinder
depicted in \autoref{figPart1Ch2_add1}, i.e., the transferred energy can be written as
\[
dH = n \hbar \omega d N ,
\]
where $dN$ is the number of modes within the interval $d \omega d \Omega$ in
the volume $c \cdot dS  dt$ \eqref{eqCh1_modenumber_1}:
\[
d N = 2 \left(\frac{1}{2 \pi c} \right)^3 \omega^2 
c \cdot dS  dt
d \omega d \Omega,
\]
from which we have 
\begin{equation}
d I = I\left(\omega, \vec{k}\right) d \omega d \Omega = 
\frac{2 n \hbar \omega c}{\left(2 \pi c\right)^3}
\omega^2 d \omega d \Omega,
\label{eqCh2_dI}
\end{equation}
where $I\left(\omega, \vec{k}\right)$ is the energy flux of photons in the direction $\vec{k}$ within unit frequency interval and unit solid angle. Thus, from
\eqref{eqCh2_dI} we obtain 
\begin{equation}
I\left(\omega, \vec{k}\right) = 
\frac{2 n \hbar \omega^3 c}{\left(2 \pi c\right)^3}
\nonumber
\end{equation}
Substituting this into expression \eqref{eqCh2_wab}, we get:
\begin{equation}
w_{ab} = \frac{\pi}{\hbar^2}\sqrt{\frac{\mu_0}{\varepsilon_0}}
\int_{\Omega}I\left(\omega, \vec{k}\right)
\left|\left(\vec{p} \cdot \vec{e}\right)\right|^2
d \Omega.
\label{eqCh2_WWab}
\end{equation}
Here the relation $\mu_0 \varepsilon_0 = \frac{1}{c^2}$ is used.

\input ./part1/interaction/fig6.tex

To find the total absorption rate, one must
integrate \eqref{eqCh2_WWab} over all propagation directions.
Moreover, the incoming radiation is usually
unpolarized. To account for this, averaging over all polarization directions is performed.
We use the coordinate system shown in \autoref{figPart1Ch2_6}. The polar axis $z$
is chosen along $\vec{k}$. The polarization vectors $\vec{e}_1$ and $\vec{e}_2$ are directed along $x$ and $y$ respectively.
The angles $\varphi$ and $\theta$ define the direction of $\vec{p}$. 
The angle $\varphi'$ defines the polarization direction of the incoming wave. From the figure, it follows that
\[
\left|\left(\vec{p} \cdot \vec{e}\right)\right|^2 = 
\left|p\right|^2 \sin^2 \theta \cos^2 \varphi'. 
\]
Averaging over all polarizations gives: 
\begin{equation}
\frac{\left|p\right|^2}{2 \pi} \int_0^{2 \pi}
\cos^2 \varphi' d \varphi' = \frac{\left|p\right|^2}{2}.
\label{eqCh2_PolarMedian}
\end{equation}

Summing over all directions of wave incidence leads to the expression
\begin{equation}
w_{ab} = \frac{\pi}{2 \hbar^2}\sqrt{\frac{\mu_0}{\varepsilon_0}}
\left|p\right|^2
\int_{\Omega}I\left(\omega, \vec{k}\right)
\sin^3 \theta d \theta d \varphi.
\end{equation}
Assuming the radiation arrives from all directions and is isotropic, 
we have: 
\begin{equation}
w_{ab} = \frac{\pi}{2 \hbar^2}\sqrt{\frac{\mu_0}{\varepsilon_0}}
\left|p\right|^2 I\left(\omega\right)
\int_{0}^{2 \pi}d \varphi \int_0^{\pi}
\sin^3 \theta d \theta = 
\frac{\pi}{ \hbar^2}\sqrt{\frac{\mu_0}{\varepsilon_0}}
\frac{\left|p\right|^2}{3}I_0,
\label{eqCh2_Wab_1}
\end{equation}
where $I_0 = 4 \pi I\left(\omega\right)$ is the total energy flux incident on the atom. In deriving  \eqref{eqCh2_Wab_1}
we used the following relation:
\[
 \int_0^{\pi}
\sin^3 \theta d \theta = \frac{4}{3}.
\]

If the radiation arrives from a small solid angular region
$\Delta \Omega$ near angles $\theta_0$, $\varphi_0$, then 
we can write:  
\begin{eqnarray}
w_{ab} = \frac{\pi}{2 \hbar^2}\sqrt{\frac{\mu_0}{\varepsilon_0}}
\left|p\right|^2 \sin^2 \theta_0
\int_{\Delta \Omega} I\left(\theta, \varphi\right)
d \Omega = 
\nonumber \\
= 
\frac{\pi}{2 \hbar^2}\sqrt{\frac{\mu_0}{\varepsilon_0}}
\left|p\right|^2  I_0 \sin^2 \theta_0,
\label{eqCh2_Wab_2}
\end{eqnarray}
where $I_0 = \int_{\Delta \Omega} I\left(\theta, \varphi\right)
d \Omega$ is the total flux irradiating the atom.

\subsection{Spontaneous transitions}
Now consider a different problem. Define the probability of photon emission
or transition of the excited atom to the lower state.  

For the probability of photon emission by the atom into one mode we had expression
\eqref{eqCh2_prob_C_bn}:
\[
\left|C_{b, n + 1}\left(t\right)\right|^2 = 4 g^2 \left(n + 1\right)
\frac{\sin^2\left(\left(\omega - \omega_{ab}\right)t/2\right)}
{\left(\omega - \omega_{ab}\right)^2},
\]
which similarly to \eqref{eqCh2_prob_C_an_2_1} can be considered as a sum of two terms
\[
P_b = \left|C_{b, n + 1}\left(t\right)\right|^2 =
P_b^{\mbox{ind}} + P_b^{\mbox{sp}}
\]
where the first term
\[
P_b^{\mbox{ind}} = 4 g^2 n
\frac{\sin^2\left(\left(\omega - \omega_{ab}\right)t/2\right)}
{\left(\omega - \omega_{ab}\right)^2}
\]
corresponds to induced emission, and
the second
\[
P_b^{\mbox{sp}} = 4 g^2 
\frac{\sin^2\left(\left(\omega - \omega_{ab}\right)t/2\right)}
{\left(\omega - \omega_{ab}\right)^2}
\]
to spontaneous emission. The term corresponding to induced emission is analyzed similarly to the case of photon absorption. The result will, of course, be the same. Corresponding formulas coincide with \eqref{eqCh2_Wab_1}, 
\eqref{eqCh2_Wab_2}. Hence, the probabilities of induced processes of absorption and emission are equal to each other. The probability of spontaneous emission per unit time will evidently be 
\begin{equation}
w_{\mbox{sp}} = 
\frac{2 \pi \omega^3}
{\hbar \varepsilon_0 \left(2 \pi c\right)^3}
\int_{\Omega}
\left|\left(\vec{p} \cdot \vec{e}\right)\right|^2
d \Omega
\label{eqCh2_Wspon}
\end{equation}
To obtain the total probability, one must integrate \eqref{eqCh2_Wspon}
over all directions, since spontaneous emission may occur into any mode. For polarizations, the average value \eqref{eqCh2_PolarMedian} is used. Thus, the entire procedure reduces to computing the integral 
\begin{equation}
w_{\mbox{sp}} = 
\frac{\left|p\right|^2 \omega^3}
{\varepsilon_0 \hbar \left(2 \pi\right)^2 c^3}
\int_{0}^{2 \pi}d \varphi \int_0^{\pi}
\sin^3 \theta d \theta
= 
\frac{4 \pi}{3}\frac{\left|p\right|^2 \omega^3}
{\hbar \varepsilon_0 \left(2 \pi\right)^2 c^3}.
\end{equation}
Finally, we obtain
\begin{equation}
w_{\mbox{sp}} = 
\frac{\left|p\right|^2 \omega^3}
{3 \pi c^2 \hbar}
\sqrt{\frac{\mu_0}{\varepsilon_0}}
\label{eqCh2_Wspon_final}
\end{equation}
Here the relation  
\[
\frac{1}{c} = \sqrt{\mu_0 \varepsilon_0}
\]
was used.
Formula \eqref{eqCh2_Wspon_final} indicates a strong
dependence of the spontaneous transition probability on frequency as  
$\omega^3$. Thus,
based on quantum electrodynamics equations, without invoking extraneous considerations, we directly obtained
expressions for the probabilities of induced and spontaneous atomic transitions per
unit time. A drawback of this consideration is the use of perturbation theory and, therefore, the necessity to restrict to
small times. For example, when solving the problem of the lifetime
of an excited atom, one cannot restrict to small times. In this case, another approximation called the Weisskopf-Wigner approximation \cite{bLuisell1972} is used. Interestingly, the results
obtained in these two cases are consistent with each other.  

