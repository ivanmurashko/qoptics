%% -*- coding:utf-8 -*- 
\section{Resonator Mode Damping. Heisenberg-Langevin Equation.}

\subsection{Problem Statement.}
The problem of field relaxation (mode) of a resonator is considered using the density matrix method, i.e., the Schrödinger representation. There is another approach, attributed to Langevin, where the equations of a dynamic system for accounting for the influence of a dissipative system are supplemented by random forces with known statistical properties. In the quantum case, the dynamic equations are Heisenberg equations for operators of observable quantities, and random influences associated with the thermostat are accounted for by adding a certain noise operator. The resulting equations are known as Heisenberg-Langevin equations.

We consider the problem of relaxing the mode of a resonator (harmonic oscillator) interacting with a thermostat using this method. In the Heisenberg representation, operators depend on time and satisfy the Heisenberg equations:
\begin{equation}
\frac{d \hat{O}}{dt} = \frac{i}{\hbar}
\left[\hat{\mathcal{H}}, \hat{O}\right],
\nonumber
\end{equation}
\rindex{Hamiltonian}
where $\hat{\mathcal{H}}$ is the Hamiltonian of the system considered. For example, for the free field, we previously obtained the equations \eqref{eqCh1_54}:
\begin{equation}
\frac{d \hat{a}}{d t} = \frac{i}{\hbar}\left[\hat{\mathcal{H}},
  \hat{a}\right] = -i \omega \hat{a}, \quad
\frac{d \hat{a}^{\dag}}{dt} = \frac{i}{\hbar}\left[\hat{\mathcal{H}},
  \hat{a}^{\dag}\right] = i \omega \hat{a}^{\dag},
\nonumber
\end{equation}
where $\omega$ is the mode frequency, and $\hat{a}$ and $\hat{a}^{\dag}$ are the annihilation and creation operators.

\input ./part1/interaction/fig_add3.tex

The problem in question is schematically shown in
\autoref{figPart1Ch2_add3}. The electromagnetic field mode, considered as a dynamic system, interacts with a vast reservoir consisting of a large number of harmonic oscillators (phonons) in equilibrium at temperature $T$. The Hamiltonian of such a system can be represented as
\begin{equation}
\hat{\mathcal{H}} = \hat{\mathcal{H}}_0 + \hat{V},
\nonumber
\end{equation}
where
\begin{equation}
 \hat{\mathcal{H}}_0 = \hbar\omega \hat{a}^{\dag}\hat{a} +
\sum_{(k)} \hbar\omega_k \hat{b}_k^{\dag} \hat{b}_k
\nonumber
\end{equation}
is the Hamiltonian of the system without taking into account the interaction between its parts,
\begin{equation}
 \hat{V} = \hbar \sum_{(k)}g_k\left(
\hat{b}_k^{\dag}\hat{a} + \hat{a}^{\dag}\hat{b}_k
\right)
\label{eqPart1Ch2_LanzgevenV}
\end{equation}
is the Hamiltonian of interaction between the dynamic system and the thermostat. In \eqref{eqPart1Ch2_LanzgevenV}, $g_k$ denotes the interaction constant for the $k$-th mode of the reservoir. $\hat{a}$ and $\hat{a}^{\dag}$ are annihilation and creation operators for the field mode, $\hat{b}_k$ and $\hat{b}_{k}^{\dag}$ are annihilation and creation operators for the $k$-th mode of the reservoir (phonons). Vacuum terms in the Hamiltonian are not accounted for, as they cancel out during the derivation of the equations.

\subsection{Heisenberg Equations of Motion for Operators.}
The Heisenberg equations of motion for $\hat{a}$ and $\hat{b}_{k}$ are
\begin{eqnarray}
\frac{d \hat{a}}{d t} = \frac{i}{\hbar}\left[
 \hat{\mathcal{H}}, \hat{a}
\right],
\nonumber \\
\frac{d \hat{b}_k}{d t} = \frac{i}{\hbar}\left[
 \hat{\mathcal{H}}, \hat{b}_k
\right]
\nonumber
\end{eqnarray}
using the commutation relations 
$\left[\hat{a}, \hat{a}^{\dag}\right] = 1$ and 
$\left[\hat{b}_k, \hat{b}_{k}^{\dag}\right] = 1$ are reduced to the form
\begin{eqnarray}
\frac{d \hat{a}\left(t\right)}{d t} = -i \omega \hat{a}\left(t\right) - i\sum_{(k)}g_k
\hat{b}_k\left(t\right),
\nonumber \\
\frac{d \hat{b}_k\left(t\right)}{d t} = -i \omega_k
\hat{b}_{k}\left(t\right) - i g_k \hat{a}\left(t\right).
\label{eqPart1Ch2_LanzgevenAB}
\end{eqnarray}
From the system \eqref{eqPart1Ch2_LanzgevenAB} $\hat{b}_k$ can be eliminated by integrating the second equation and substituting the result into the first equation. Integration is easiest to perform using operational calculus. We obtain
\begin{equation}
\hat{b}_k\left(t\right) = 
\hat{b}_k\left(0\right) e^{-i \omega_k t} 
- i g_k \int_0^t d t' \hat{a}\left(t'\right)e^{-i \omega_k\left(t - t'\right)}
\nonumber
\end{equation}
and further
\begin{eqnarray}
\frac{d \hat{a}\left(t\right)}{dt} = 
- i \omega \hat{a}\left(t\right) - \sum_{(k)} g_k^2 \int_0^t
d t'  \hat{a}\left(t'\right)e^{-i \omega_k\left(t - t'\right)}
+ \hat{f}\left(t\right),
\nonumber \\
\hat{f}\left(t\right) = -i \sum_{(k)} g_k \hat{b}_k\left(0\right)
e^{-i \omega_k t}.
\label{eqPart1Ch2_LanzgevenA}
\end{eqnarray}
The operator $\hat{f}\left(t\right)$ depends on the thermostat variables and is a noise operator describing the impact of the reservoir on the dynamic system.

The equation \eqref{eqPart1Ch2_LanzgevenA} is integro-differential. Under certain approximations, it can be transformed into a purely differential one. First, we switch to slow variables by writing
\begin{eqnarray}
\hat{a}\left(t\right) = \hat{A}\left(t\right)e^{-i \omega t},
\nonumber \\
\hat{f}\left(t\right) = \hat{F}\left(t\right)e^{-i \omega t},
\nonumber
\end{eqnarray}
where $\omega$ is the mode frequency. From this, we have
\begin{equation}
\frac{d \hat{A}\left(t\right)}{dt} = 
- \sum_{(k)} g_k^2 \int_0^t
d t'  \hat{A}\left(t'\right)e^{i \left(\omega -
  \omega_k\right)\left(t - t'\right)} 
+ \hat{F}\left(t\right).
\label{eqPart1Ch2_LanzgevenAA}
\end{equation}
% where 
% $\hat{F}\left(t\right) = \hat{f}\left(t\right) e^{i \omega t}$. 
The first term of expression \eqref{eqPart1Ch2_LanzgevenAA} resembles the form encountered when considering spontaneous emission using the Weisskopf-Wigner method. The approach used there can also be applied here. From expression \eqref{eqPart1Ch2_LanzgevenAA} it is clear that the main contribution comes from terms with frequencies close to the mode frequency: $\omega_k \approx \omega$. For this reason, all slowly varying terms can be taken at frequency $\omega$ and taken out of the integral. Summation over $k$, assuming the phonon spectrum is quasi-continuous, can be replaced by integration over frequency:
\begin{eqnarray}
\sum_{(k)}g_k^2\int_0^t d t' \hat{A}\left(t'\right) e^{i\left(\omega -
  \omega_k\right)\left(t - t'\right)} = 
\nonumber \\
=
\int_0^t d t'
  \hat{A}\left(t'\right)\int_0^{\infty}g^2\left(\omega\right) 
D\left(\omega\right) e^{i\left(\omega -
  \omega'\right)\left(t - t'\right)} d \omega' = 
\nonumber \\
= g^2\left(\omega\right) 
D\left(\omega\right)
\int_0^t d t'
  \hat{A}\left(t'\right)
\int_0^{\infty}e^{i\left(\omega -
  \omega'\right)\left(t - t'\right)} d \omega',
\nonumber
\end{eqnarray}
where $D\left(\omega\right)$ is the frequency density of states.
$D\left(\omega\right)$ and $g^2\left(\omega\right)$ are taken at the field mode frequency and factored out of the integral.

Consider the frequency integral. Proceeding as in the consideration of spontaneous emission using the Weisskopf-Wigner method, we obtain
\begin{eqnarray}
\int_0^{\infty}e^{- i\left(\omega' -
  \omega\right)\left(t - t'\right)} d \omega' = 
\int_{-\omega}^{\infty} e^{-i \nu \left(t - t'\right)} d \nu \approx 
\int_{-\infty}^{\infty} e^{-i \nu \left(t - t'\right)} d \nu = 
\nonumber \\
= \int_{-\infty}^{\infty} e^{i \nu \left(t' - t\right)} d \nu = 
2 \pi \delta\left(t' - t\right).
\nonumber
\end{eqnarray}
Consequently, we can write
\begin{eqnarray}
\frac{d \hat{A}\left(t\right)}{d t} = 
- 2 \pi g^2\left(\omega\right) 
D\left(\omega\right)
\int_0^t d t'
  \hat{A}\left(t'\right)
\delta\left(t - t'\right)  + \hat{F}\left(t\right) = 
\nonumber \\
=
- 2 \pi g^2\left(\omega\right) 
D\left(\omega\right) \hat{A}\left(t\right)
 + \hat{F}\left(t\right).
\nonumber
\end{eqnarray}
 Let us denote 
\(
\left.2 \pi D g^2\right|_{\omega} = \frac{\gamma}{2}
\) - the damping coefficient. We obtain
\begin{equation}
\frac{d \hat{A}\left(t\right)}{d t} = 
- \frac{\gamma}{2} \hat{A}\left(t\right) + 
\hat{F}\left(t\right),
\label{eqPart1Ch2_LanzgevenAeq}
\end{equation}
where $\hat{F}\left(t\right)$ is a random noise operator whose properties need to be determined. As will be seen later, the noise operator ensures the preservation of commutation relations for the field mode operator.

Let us now examine the statistical properties characterized by the correlation functions of $\hat{F}$. Suppose the reservoir is in thermal equilibrium at temperature $T$. Then we have:
\begin{eqnarray}
\left<\hat{b}_k\left(0\right)\right>_R = 
\left<\hat{b}_k^{\dag}\left(0\right)\right>_R = 0, 
\nonumber \\
\left<\hat{b}_k^{\dag}\left(0\right)\hat{b}_{k'}\left(0\right)\right>_R = 
\bar{n}_k \delta_{k k'},
\nonumber \\
\left<\hat{b}_k\left(0\right)\hat{b}_{k'}^{\dag}\left(0\right)\right>_R = 
\left(\bar{n}_k + 1\right)\delta_{k k'},
\nonumber \\
\left<\hat{b}_k\left(0\right)\hat{b}_{k'}\left(0\right)\right>_R = 
\left<\hat{b}_k^{\dag}\left(0\right)\hat{b}_{k'}^{\dag}\left(0\right)\right>_R
= 0,
\label{eqPart1Ch2_LanzgevenPropertyF}
\end{eqnarray}
where $\left<\dots\right>_R$ means averaging over the thermostat variables, and $\bar{n}_k$ is the average number of phonons in mode $k$. The noise operator was defined by the formula 
\begin{equation}
\hat{F}\left(t\right) = -i \sum_{(k)}g_k 
\hat{b}_k\left(0\right)
e^{-i \left(\omega_k - \omega\right)t}.
\label{eqPart1Ch2_LanzgevenDefenitionF}
\end{equation}
Using \eqref{eqPart1Ch2_LanzgevenPropertyF}, it is easy to see
\begin{eqnarray}
\left<\hat{F}\left(t\right)\right>_R =
\left<\hat{F}^{\dag}\left(t\right)\right>_R  = 0,
\nonumber \\
\left<\hat{F}^{\dag}\left(t\right)\hat{F}\left(t'\right)\right>_R = 
\nonumber \\
= \sum_{k}\sum_{k'}g_k g_{k'}
\left<\hat{b}_k^{\dag}\left(0\right)\hat{b}_{k'}\left(0\right)\right>_R 
e^{i\left(\omega_k - \omega\right)t} 
 e^{-i\left(\omega_{k'} - \omega\right)t'} =
\nonumber \\
= \sum_{k}g_k^2 \bar{n}_k 
e^{i\left(\omega_k - \omega\right)\left(t - t'\right)} = 
\nonumber \\
=
\int_0^\infty
g^2\left(\omega\right)D\left(\omega\right)\bar{n}_T
e^{i\left(\omega' - \omega\right)\left(t - t'\right)}d \omega' = 
\nonumber \\
= 2 \pi
g^2\left(\omega\right)D\left(\omega\right)\bar{n}_T
\delta\left(t - t'\right) =
\frac{\gamma \bar{n}_{T}}{2} \delta\left(t - t'\right),
\label{eqPart1Ch2_LanzgevenCorrelations}
\end{eqnarray}
where $\bar{n}_T$ is the average number of phonons in the reservoir mode at temperature $T$, determined by Planck's formula, $\gamma = 4 \pi g^2\left(\omega\right)D\left(\omega\right)$.
If, according to Langevin, we introduce the diffusion coefficient $\mathcal{D} =
\frac{\gamma \bar{n}_{T}}{2}$, then the last equation of the system
\eqref{eqPart1Ch2_LanzgevenCorrelations} can be written as
\begin{equation}
\left<\hat{F}^{\dag}\left(t\right)\hat{F}\left(t'\right)\right>_R = 
\mathcal{D} \delta\left(t - t'\right). 
\nonumber
\end{equation}
Note that $\left<\dots\right>_R$ means averaging over the thermostat variables, so the result of averaging will be a correlation function that depends only on the variables of the dynamic system (field).

Similarly proved
\begin{equation}
\left<\hat{F}\left(t\right)\hat{F}^{\dag}\left(t'\right)\right>_R = 
\frac{\gamma\left(\bar{n}_{T} + 1\right)}{2} \delta\left(t - t'\right)
\label{eqPart1Ch2_Lanzgeven_Task1}
\end{equation}
and
\begin{equation}
\left<\hat{F}\left(t\right)\hat{F}\left(t'\right)\right>_R = 
\left<\hat{F}^{\dag}\left(t\right)\hat{F}^{\dag}\left(t'\right)\right>_R = 0.
\label{eqPart1Ch2_Lanzgeven_Task2}
\end{equation}
Thus, in our approximation, the noise is 
$\delta$-correlated.

We will also need a correlation function of the form
\(
\left<\hat{F}^{\dag}\left(t\right)\hat{A}\left(t'\right)\right>_R 
\) 
- necessary to derive the equation that 
\(
\left<\hat{A}^{\dag}\left(t\right)\hat{A}\left(t'\right)\right>_R 
\) satisfies. For this, we integrate the equation for $\hat{A}\left(t\right)$
\eqref{eqPart1Ch2_LanzgevenAeq} and, as a result, using
operational calculus, we obtain
\begin{equation} 
\hat{A}\left(t\right) = \hat{A}\left(0\right)e^{-\frac{\gamma}{2}t} +
\int_0^t d t' e^{-\frac{\gamma}{2}\left(t - t'\right)} \hat{F}\left(t'\right).
\nonumber
\end{equation} 
Multiply this expression by $\hat{F}^{\dag}\left(t\right)$ and average over
the thermostat. We obtain
\begin{eqnarray} 
\left<\hat{F}^{\dag}\left(t\right)\hat{A}\left(t\right)\right>_R = 
\int_0^t  
\left<\hat{F}^{\dag}\left(t\right)\hat{F}\left(t'\right)\right>_R 
e^{-\frac{\gamma}{2}\left(t - t'\right)}
d t' +
\nonumber \\
+ \left<\hat{F}^{\dag}\left(t\right)\hat{A}\left(0\right)\right>_R e^{-\frac{\gamma}{2}t}.
\label{eqPart1Ch2_LanzgevenFACorr}
\end{eqnarray} 
The last term in \eqref{eqPart1Ch2_LanzgevenFACorr} is $0$, as 
$\hat{F}^{\dag}\left(t\right)$ and $\hat{A}\left(0\right)$ are independent,
%\footnote{$\hat{F}^{\dag}\left(t\right)$ at a later time cannot affect $\hat{A}\left(0\right)$ at the initial time} 
and their averages are $0$. Taking into account the second equation of the system 
\eqref{eqPart1Ch2_LanzgevenCorrelations} i.e.,
\(
\left<\hat{F}^{\dag}\left(t\right)\hat{F}\left(t'\right)\right>_R = 
\frac{\gamma \bar{n}_{T}}{2} \delta\left(t - t'\right),
\)
we obtain
\begin{equation}
\left<\hat{F}^{\dag}\left(t\right)\hat{A}\left(t\right)\right>_R = 
\frac{\gamma \bar{n}_{T}}{2}.
\nonumber
\end{equation}
Now we can consider the correlation function 
\(
\left<\hat{A}^{\dag}\left(t\right)\hat{A}\left(t\right)\right>_R
\). Differentiating this function over time, we obtain
\begin{equation}
\frac{d \left<\hat{A}^{\dag}\left(t\right)\hat{A}\left(t\right)\right>_R}{d
t}=
\left<\frac{d \hat{A}^{\dag}\left(t\right)\hat{A}\left(t\right)}{d
t}\right>_R =
\left<\frac{d \hat{A}^{\dag}\left(t\right)}{d
t} \hat{A}\left(t\right) \right>_R + 
\left<\hat{A}^{\dag}\left(t\right)\frac{d \hat{A}\left(t\right)}{d
t} \right>_R.
\label{eqPart1Ch2_Lanzgeven_dAA}
\end{equation}
Then, using the equation \eqref{eqPart1Ch2_LanzgevenAeq} and its conjugate, we obtain
\begin{equation}
\frac{d \left<\hat{A}^{\dag}\left(t\right)\hat{A}\left(t\right)\right>_R}{d
t}=
- \gamma \left<\hat{A}^{\dag}\left(t\right)\hat{A}\left(t\right)\right>_R
+ \left<\hat{F}^{\dag}\left(t\right)\hat{A}\left(t\right)\right>_R + 
\left<\hat{A}^{\dag}\left(t\right)\hat{F}\left(t\right)\right>_R.
\nonumber
\end{equation}
Considering the expressions 
\begin{equation}
\left<\hat{F}^{\dag}\left(t\right)\hat{A}\left(t\right)\right>_R =
\left<\hat{A}^{\dag}\left(t\right)\hat{F}\left(t\right)\right>_R = 
\frac{\gamma \bar{n}_T}{2},
\nonumber
\end{equation}
finally, we have
\begin{equation}
\frac{d \left<\hat{A}^{\dag}\left(t\right)\hat{A}\left(t\right)\right>_R}{d
t}=
- \gamma \left<\hat{A}^{\dag}\left(t\right)\hat{A}\left(t\right)\right>_R
+ \gamma \bar{n}_T.
\nonumber
\end{equation}
Similarly, we obtain
\begin{equation}
\frac{d \left<\hat{A}\left(t\right)\hat{A}^{\dag}\left(t\right)\right>_R}{d
t}=
- \gamma \left<\hat{A}\left(t\right)\hat{A}^{\dag}\left(t\right)\right>_R
+ \gamma \left(\bar{n}_T + 1\right).
\label{eqPart1Ch2_Lanzgeven_Task3}
\end{equation}
Subtracting these expressions from each other, we arrive at the commutator
\begin{equation}
\frac{d \left<\left[\hat{A}\left(t\right),\hat{A}^{\dag}\left(t\right)\right]\right>_R}{d
t}=
\gamma \left( 1 - \left<\left[\hat{A}\left(t\right),\hat{A}^{\dag}\left(t\right)\right]\right>_R
\right).
\label{eqPart1Ch2_Lanzgeven_CommutatorAA}
\end{equation}
From equation \eqref{eqPart1Ch2_Lanzgeven_CommutatorAA} it follows that
if at the initial time 
\[
\left<\left[\hat{A}\left(0\right),\hat{A}^{\dag}\left(0\right)\right]\right>_R
= 1,
\]
i.e., commutation relations hold, then these relations will always hold. In other words, commutation relations averaged over the thermostat variables are preserved over time. 

The picture would be completely different if quantum noises were not taken into account. In this case, the equation
\eqref{eqPart1Ch2_Lanzgeven_CommutatorAA} would take the form
\begin{equation}
\frac{d \left<\left[\hat{A}\left(t\right),\hat{A}^{\dag}\left(t\right)\right]\right>_R}{d
t}=
- \gamma \left<\left[\hat{A}\left(t\right),\hat{A}^{\dag}\left(t\right)\right]\right>_R,
\nonumber
\end{equation}
whose solution is
\begin{equation}
\left<\left[\hat{A}\left(t\right),\hat{A}^{\dag}\left(t\right)\right]\right>_R
= 
\left<\left[\hat{A}\left(0\right),\hat{A}^{\dag}\left(0\right)\right]\right>_R
e^{- \gamma t}= e^{- \gamma t}.
\nonumber
\end{equation}
This means that, over a long enough time, the operators $\hat{A}$ and
$\hat{A}^{\dag}$ commute, which is not observed in reality. Therefore, taking into account the quantum noise term is mandatory.


The equation of motion for the field averaged over the thermostat variables is easily obtained from \eqref{eqPart1Ch2_LanzgevenAeq}. To do this, we first write the averaged field
\begin{eqnarray}
\left<\hat{E}\left(t\right)\right>_R 
= \left< E_0 \sin\,kz \left( \hat{a}\left(t\right) +
\hat{a}^{\dag}\left(t\right)\right)\right>_R  = 
\nonumber \\
= E_0 \sin\,kz
\left(\left<\hat{A}\left(t\right)\right>_R e^{-i \omega t}
+
\left<\hat{A}^{\dag}\left(t\right)\right>_R e^{i \omega t}
\right),
\nonumber
\end{eqnarray}
where it is considered $\hat{a} = \hat{A}e^{-i \omega t}$. From
\eqref{eqPart1Ch2_LanzgevenAeq} we have 
\begin{equation}
\frac{d \left<\hat{A}\left(t\right)\right>_R}{d t} = 
- \frac{\gamma}{2} \left<\hat{A}\left(t\right)\right>_R,
\nonumber
\end{equation}
since $\left<\hat{F}\left(t\right)\right>_R = 0$. Thus, over time, $\left<\hat{A}\left(t\right)\right>_R$ and
$\left<\hat{A}^{\dag}\left(t\right)\right>_R$ tend to zero along with  
$\left<\hat{E}\left(t\right)\right>_R$:
\begin{equation}
\left<\hat{E}\left(t\right)\right>_R \rightarrow 0.
\nonumber
\end{equation}

\subsection{Fluctuation-Dissipation Formula}
We obtained expression \eqref{eqPart1Ch2_LanzgevenCorrelations}:
\begin{equation}
\left<\hat{F}^{\dag}\left(t\right)\hat{F}\left(t'\right)\right>_R = 
\frac{\gamma \bar{n}_{T}}{2} \delta\left(t - t'\right).
\nonumber
\end{equation}
Integrating both sides by $d t'$, we obtain
\begin{equation}
\gamma =
\frac{2}{\bar{n}_T}\int_0^{\infty}\left<\hat{F}^{\dag}\left(t\right)\hat{F}\left(t'\right)\right>_R
d t',
\label{eqPart1Ch2_LanzgevenGamma}
\end{equation}
i.e., the damping rate and noise fluctuations are related by the relation \eqref{eqPart1Ch2_LanzgevenGamma}. Greater damping corresponds to greater noise.