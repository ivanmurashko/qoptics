%% -*- coding:utf-8 -*- 
\section{Spontaneous Emission. \\
  Weisskopf-Wigner Approximation.}

Above, we derived a formula for the rate of spontaneous emission, valid for sufficiently small times. At the same time, when solving problems such as the lifetime of an excited atom, one cannot limit oneself to small times. In this case, a different method is applied, called the Weisskopf-Wigner approximation.

Let's consider this method. We found that an atom transitions from an excited state to the ground state due to interaction with all modes of space, even if they are unexcited and in a vacuum state. The interaction Hamiltonian in this case has the form
\begin{equation}
\hat{V}_I = \hbar \sum_{k} g_k \hat{\sigma}^{\dag}\hat{a}_k e^{i\left(
\omega_{ab} - \omega_k
\right)t} + \text{h. c.},
\label{eqCh2VaiskopfV}
\end{equation}
where $g_k$ is the coupling constant with the $k$-th mode. The Hamiltonian \eqref{eqCh2VaiskopfV} differs from the one we previously used \eqref{eqCh2_task22} by summing over all quantization space modes.

Suppose at the initial moment $t=0$ the atom is excited, and there are no photons in the field. Then the initial state is
\begin{equation}
\left|\psi\left(0\right)\right> =
 \left|a, \left\{0\right\}\right>,
\nonumber
\end{equation}
meaning the atom is excited, and all modes are in the vacuum state. At later times, due to the atom-field interaction, the state of the system will be
\begin{equation}
\left|\psi\left(t\right)\right> =
\sum_{k} C_{bk}\left(t\right) \ket{b, 1_k}
+
C_{a}\left(t\right) \left|a, \left\{0\right\}\right>,
\label{eqCh2Vaiskopf3}
\end{equation}
where $\ket{b, 1_k}$ is the state when the atom is in the ground state, and there is one photon excited in the $k$-th mode.

The Schrödinger equation in the interaction representation is
\begin{equation}
\frac{d}{dt} \left|\psi\left(t\right)\right> =
- \frac{i}{\hbar} \hat{V}_I \left|\psi\left(t\right)\right>.
\label{eqCh2Vaiskopf4}
\end{equation}
Substituting \eqref{eqCh2Vaiskopf3} and multiplying \eqref{eqCh2Vaiskopf4} sequentially by $\left<a, \left\{0\right\}\right|$ and $\bra{b, 1_k}$, we obtain a system of equations for the probability amplitudes $C_{a}\left(t\right)$ and $C_{bk}\left(t\right)$.
\begin{eqnarray}
\dot{C}_{a}\left(t\right) = - i \sum_{k} g_k e^{i \left(\omega_{ab} - 
  \omega_k\right)t} C_{bk}\left(t\right),
\nonumber \\
\dot{C}_{bk}\left(t\right) = - i g_k e^{- i \left(\omega_{ab} -
  \omega_k\right)t} C_{a}\left(t\right).
\label{eqCh2Vaiskopf5}
\end{eqnarray}
Integrating the second equation of \eqref{eqCh2Vaiskopf5} over time from $0$ to $t$. 
\begin{equation}
C_{bk}\left(t\right) = - i g_k \int_0^{t} e^{- i \left(\omega_{ab} -
  \omega_k\right)t'} C_{a}\left(t'\right) dt'.
\label{eqCh2Vaiskopf6}
\end{equation}
Substituting \eqref{eqCh2Vaiskopf6} into the first equation of the system \eqref{eqCh2Vaiskopf5}, we get the following integro-differential equation
\begin{equation}
\dot{C}_{a}\left(t\right) = - \sum_{k} g_k^2 
\int_0^t
e^{i \left(\omega_{ab} - \omega_k\right)\left(t - t'\right)}  
C_{a}\left(t'\right) dt'.
\label{eqCh2Vaiskopf7}
\end{equation}

Let's make some simplifications (Weisskopf-Wigner approximation). We will consider the distribution of modes as quasi-continuous and replace the summation over $k$ with integration, using the relation \eqref{eqCh1_modenumber_1pre}:
\begin{equation}
d N = 2 \left(\frac{L}{2 \pi} \right)^3 k^2 d k d \Omega = 
2 \frac{V}{\left(2 \pi\right)^3}  k^2 d k d \Omega
\nonumber
\end{equation}
Then we obtain \eqref{eqCh1_modenumber_kvazy_contig}
\begin{eqnarray}
\sum_{k} g_k^2 
\int_0^t
e^{i \left(\omega_{ab} - \omega_k\right)\left(t - t'\right)}  
C_{a}\left(t'\right) dt' = 
\nonumber \\
= 2 \frac{V}{\left(2 \pi\right)^3} \int_{4\pi}d \Omega \int_0^{\infty}
g_k^2 k^2 dk  \int_0^t dt'
e^{i \left(\omega_{ab} - \omega_k\right)\left(t - t'\right)}  
C_{a}\left(t'\right).
\label{eqCh2Vaiskopf8pre}
\end{eqnarray}
In \eqref{eqCh2Vaiskopf8pre} 
\begin{equation}
g_k^2 = \frac{\omega_k\left|p\right|^2 \sin^2 \theta}{4 \hbar
  \varepsilon_0 V},
\label{eqCh2VaiskopfGk}
\end{equation}
where $\theta$ is the angle between the directions $\vec{k}$ and $\vec{p}$.
The expression \eqref{eqCh2VaiskopfGk} is obtained considering the averaging over polarizations \eqref{eqCh2_PolyarMedian}.

Transitioning in \eqref{eqCh2Vaiskopf8pre} from $k$ to $\omega$ using the relations 
\begin{equation}
k = \frac{\omega_k}{c}, \quad k^2 dk = \frac{\omega_k^2 d \omega_k}{c^3}
\nonumber
\end{equation}
and denoting for convenience \(\omega_k = \omega\),
we obtain
\begin{equation}
\dot{C}_{a}\left(t\right) = - 
2 \frac{V}{\left(2 \pi c\right)^3} \int_{4\pi}d \Omega \int_0^{\infty}
g^2\left(\omega\right) \omega^2 d\omega  \int_0^t dt'
e^{i \left(\omega_{ab} - \omega\right)\left(t - t'\right)}  
C_{a}\left(t'\right).
\label{eqCh2Vaiskopf8}
\end{equation}

For the approximate calculation of the integral in \eqref{eqCh2Vaiskopf8}, we further apply a series of simplifying assumptions. First, integrate \eqref{eqCh2Vaiskopf8} over frequency (i.e., change the order of integration, assuming it is possible). From the structure \eqref{eqCh2Vaiskopf8}, it is evident that the main contribution to the time integral is given by frequency regions $\omega \approx \omega_{ab}$. For this reason, we can assume 
\[
\omega^2 g^2\left(\omega\right) \approx 
\omega_{ab}^2 g^2\left(\omega_{ab}\right).
\]
In this approximation, the frequency integral will look like (see the integral representation of the delta function \eqref{eq:delta_from_integral}):  
\begin{eqnarray}
\int_0^{\infty}d \omega e^{i\left(\omega_{ab} - \omega\right)\left(t -
  t'\right)}  = \left|\nu = \omega - \omega_{ab}\right| =
\int_{- \omega_{ab}}^{\infty}d \nu e^{-i \nu\left(t - t'\right)} \approx
\nonumber \\
\approx \int_{- \infty}^{\infty} d \nu e^{-i \nu\left(t - t'\right)} = 
\int_{- \infty}^{\infty} d \nu e^{i \nu\left(t' - t\right)} =
2 \pi \delta\left(t' - t\right).
\label{eqCh2Vaiskopf9}
\end{eqnarray}

Substituting \eqref{eqCh2Vaiskopf9} into equation \eqref{eqCh2Vaiskopf8} and integrating over time, we obtain
\begin{equation}
\dot{C}_{a}\left(t\right) = - 
2 \frac{V}{\left(2 \pi c\right)^3} \int_{4\pi}d \Omega 
g^2\left(\omega_{ab}\right) \omega_{ab}^2   
2 \pi C_{a}\left(t\right) = - \frac{\Gamma}{2} C_{a}\left(t\right).
\label{eqCh2Vaiskopf10}
\end{equation}
Substituting here the expression for $g_k^2$ \eqref{eqCh2VaiskopfGk} and performing integration over angles (over $d \Omega$), we find the expression for the damping coefficient $\Gamma$:
\begin{equation}
\Gamma = \frac{\omega_{ab}^3 \left|p\right|^2}{3 \pi c^2 \hbar}
\sqrt{\frac{\mu_0}{\varepsilon_0}}. 
\label{eqCh2Vaiskopf11}
\end{equation}
Note that the expression \eqref{eqCh2Vaiskopf11} coincides with the one obtained earlier by another method \eqref{eqCh2_Wspon_final}.

From \eqref{eqCh2Vaiskopf10}, it follows that $\Gamma$ characterizes the rate of change of the probability $\left|C_{a}\right|^2$. Indeed, from \eqref{eqCh2Vaiskopf10} we have:
\begin{eqnarray}
\dot{C}_{a}C_{a}^{*} = - \frac{\Gamma}{2}C_{a}C_{a}^{*},
\nonumber \\
\dot{C}_{a}^{*}C_{a} = - \frac{\Gamma}{2}C_{a}^{*}C_{a},
\nonumber
\end{eqnarray}
from which we have
\begin{equation}
\frac{d C_{a}C_{a}^{*}}{dt} = -\Gamma \left(C_{a}C_{a}^{*}\right),
\nonumber
\end{equation}
or otherwise
\begin{equation}
\frac{d \left|C_{a}\right|^2}{dt} = -\Gamma \left|C_{a}\right|^2.
\nonumber
\end{equation}
The solution has the form (see \autoref{fig:part1:vaickopf})
\[
\left|C_{a}\right|^2 = e^{- \Gamma t},
\]
if the initial value $\left.\left|C_{a}\right|^2\right|_{t = 0} =
1$, i.e., the atom was initially excited.

\input ./part1/interaction/figvaickopf.tex 

\begin{remark}[Weisskopf-Wigner Approximation and Rabi Oscillations]
If we compare \autoref{fig:part1:rabi} and
\autoref{fig:part1:vaickopf}, we can see that these two graphs are very different, despite describing seemingly the same model system. This raises the question of when we can apply the single-mode approximation, and hence obtain Rabi oscillations (see \autoref{fig:part1:rabi}), and in which cases it is necessary to use multi-mode approximations like Weisskopf-Wigner (see \autoref{fig:part1:vaickopf}). 

Formally, we can consider that interaction with each mode of the electromagnetic field contributes to the resulting probabilities; however, according to \eqref{eqCh2_prob_C_bn}, the change in probability (for small $t$)
\[
\Delta \left|C_{a, n}\left(t\right)\right|^2 = -\Delta \left|C_{b, n +
  1}\left(t\right)\right|^2 \sim \left(n + 1\right).
\]
Thus, those modes in which the number of photons is large contribute more than those where it is small. Consequently, if there is a mode with a large number of photons $n \gg 1$, the influence of vacuum modes with a photon number $n = 0$ can be neglected in this case. Otherwise, especially if all modes are equally significant, it is necessary to consider all modes and apply approximations like Weisskopf-Wigner.
\end{remark}