%% -*- coding:utf-8 -*- 
\section{Interaction of an atom with a mode of the electromagnetic field}
The Schrödinger equation in the interaction representation has the form
\begin{equation}
i \hbar \frac{\partial}{\partial t} \left|\psi\left(r, t\right)\right>_I
= \hat{V}_I \left|\psi\left(r, t\right)\right>_I,
\label{eqCh2_Shredinger_inter}
\end{equation}
 where the wave function in the interaction representation is written as
\(
\left|\psi\left(r, t\right)\right>_I
\):
\begin{equation}
\left|\psi\left(r, t\right)\right>_I = e^{i \frac{\hat{\mathcal{H}}_0
    t}{\hbar}}
\left|\psi\left(r, t\right)\right>.
\label{eqCh2_Psi_inter}
\end{equation}
Let us solve this equation for the initial state 
$\left|\psi\left(r, 0\right)\right>_I = \ket{b, n + 1}_I$, i.e. 
we assume that at the initial moment the atom is in the lower
state, and the field contains $n + 1$ photons. The other case
corresponds to the initial state of the system  
$\ket{a, n}_I$, i.e. the atom is in the upper
state, and the field mode contains $n$ photons. After some time, as a result of the interaction of the atom and the field, the system will be in a
superposition state  
\begin{equation}
\left|\psi\left(r, t\right)\right>_I = 
C_{a,n}\left(t\right)_I\ket{a, n}_I +
C_{b,n+1}\left(t\right)_I\ket{b, n + 1}_I.
\label{eqCh2_OurPsi_inter}
\end{equation}

Using \eqref{eqCh2_Psi_inter} on the left-hand side of expression
\eqref{eqCh2_OurPsi_inter}, we get 
\begin{eqnarray}
C_{a,n}\left(t\right)_I\ket{a, n}_I +
C_{b,n+1}\left(t\right)_I\ket{b, n + 1}_I =
\nonumber \\
= C_{a,n}\left(t\right)_I e^{i \frac{\hat{\mathcal{H}}_0
    t}{\hbar}} \ket{a, n} +
C_{b,n+1}\left(t\right)_I e^{i \frac{\hat{\mathcal{H}}_0
    t}{\hbar}} \ket{b, n + 1} = 
\nonumber \\
=
C_{a,n}\left(t\right)_I \exp\left\{i \left(
 \omega_a +  \omega 
\left( n + \frac{1}{2}\right) 
\right) t
\right\} \ket{a, n} +
\nonumber \\
+
C_{b,n+1}\left(t\right)_I \exp\left\{i 
\left(
\omega_b +  \omega 
\left( n + 1 + \frac{1}{2}\right) 
\right)
    t \right\} \ket{b, n + 1}
\nonumber \\
=
C_{a,n}\left(t\right) \ket{a, n} +
C_{b,n+1}\left(t\right) \ket{b, n + 1},
\nonumber
\end{eqnarray}
where $C_{a,n}\left(t\right)$ and $C_{b,n+1}\left(t\right)$ are defined as
\begin{eqnarray}
C_{a,n}\left(t\right) = 
C_{a,n}\left(t\right)_I \exp\left\{i 
\left(
\omega_a +  \omega 
\left( n + \frac{1}{2}\right) 
\right) 
    t \right\},
\nonumber \\
C_{b,n+1}\left(t\right) =
C_{b,n+1}\left(t\right)_I \exp\left\{i \left(
\omega_b + \omega 
\left( n + 1 + \frac{1}{2}\right) 
\right)
    t\right\}.
\nonumber
\end{eqnarray}

Thus, equation \eqref{eqCh2_Shredinger_inter} takes the following form   
\begin{eqnarray}
i \left( 
{\dot C}_{a,n}\left(t\right)\ket{a, n} +
{\dot C}_{b,n+1}\left(t\right)\ket{b, n + 1}
\right) =
\nonumber \\
= g \left(
\hat{a}\hat{\sigma}^{\dag} e^{-i \left(\omega - \omega_{ab}\right)t} +
\hat{a}^{\dag}\hat{\sigma} e^{i \left(\omega - \omega_{ab}\right)t}
\right) 
\left( 
C_{a,n}\ket{a, n} +
C_{b,n+1}\ket{b, n + 1}
\right) = 
\nonumber \\
= g \sqrt{n + 1} \left(
C_{b, n+1} e^{-i \left(\omega - \omega_{ab}\right)t} \ket{a, n} + 
C_{a, n} e^{i \left(\omega - \omega_{ab}\right)t} \ket{b, n + 1}
\right).
\end{eqnarray}
Multiply this equation from the left by 
$\bra{a, n}$ and $\bra{b, n + 1}$ respectively. Taking into account the properties
of the operators $\hat{a}$, $\hat{a}^{\dag}$, $\hat{\sigma}$,
$\hat{\sigma}^{\dag}$ and the orthogonality of the state vectors, we obtain  
a system of equations for probability amplitudes 
\begin{eqnarray}
{\dot C}_{a,n}\left(t\right) = -i g \sqrt{n + 1}
e^{-i \left(\omega - \omega_{ab}\right)t} 
C_{b, n + 1}\left(t\right),
\nonumber \\
{\dot C}_{b, n + 1}\left(t\right) = -i g \sqrt{n + 1}
e^{i \left(\omega - \omega_{ab}\right)t} 
C_{a, n}\left(t\right).
\label{eqCh2_task3}
\end{eqnarray}

\input ./part1/interaction/fig3.tex

For the case when the atom is in the lower state (case of photon absorption) we have at the initial moment  
$C_{a, n}\left(0\right) = 0$, $C_{b, n + 1}\left(0\right) = 1$
(\autoref{figPart1Ch2_3}). Formal integration of \eqref{eqCh2_task3} gives  
\begin{equation}
C_{a,n}\left(t\right) = -i g \sqrt{n + 1}
\int_0^t e^{-i \left(\omega - \omega_{ab}\right)t'} 
C_{b, n + 1}\left(t'\right) dt'
\label{eqCh2_ampl_prob_int}
\end{equation}
In the first approximation for small times, one can put
$C_{b, n + 1}\left(t'\right) = 1$ under the integral in \eqref{eqCh2_ampl_prob_int}. Integrating then yields  
\begin{equation}
C_{a,n}\left(t\right) = -i g \sqrt{n + 1}
\frac{e^{-i \left(\omega - \omega_{ab}\right)t} - 1}
{-i \left(\omega - \omega_{ab}\right)}
\end{equation}
from which the probability of excitation of the atom and photon absorption is 
\begin{equation}
\left|C_{a,n}\left(t\right)\right|^2 = g^2 \left(n + 1\right) t^2
\frac{\sin^2\left(\left(\omega - \omega_{ab}\right)t/2\right)}
{\left(\omega - \omega_{ab}\right)^2\left(t/2\right)^2}. 
\label{eqCh2_prob_C_an}
\end{equation}

\input ./part1/interaction/fig4.tex

In this approximation, one can also solve the second problem (\autoref{figPart1Ch2_4}) 
$C_{a, n}\left(0\right) = 1$, $C_{b, n + 1}\left(0\right) = 0$. 
We obtain an expression analogous to \eqref{eqCh2_prob_C_an} for the probability of atom transition to the lower state and emission of a photon  
\begin{equation}
\left|C_{b, n + 1}\left(t\right)\right|^2 = g^2 \left(n + 1\right) t^2
\frac{\sin^2\left(\left(\omega - \omega_{ab}\right)t/2\right)}
{\left(\omega - \omega_{ab}\right)^2\left(t/2\right)^2}.
\label{eqCh2_prob_C_bn}
\end{equation}

In the case when the field contains no photons ($n = 0$),
a nonzero probability is obtained 
$\left|C_{b, 1}\left(t\right)\right|^2 \cong g^2 t^2$
($\frac{\sin^2 x}{x^2} \cong 1$, if
$x \ll 1$), which shows that an excited atom can transition to the lower state even in the absence of photons. This process is known as spontaneous emission or spontaneous transition. It
is related to the zero-point fluctuations of the field. Note that in the semiclassical
treatment spontaneous emission does not follow from semiclassical equations and is introduced by additional considerations. From quantum equations,
spontaneous emission naturally follows. 
The system of equations \eqref{eqCh2_task3} can be solved exactly. For simplicity,
we will consider the resonant case $\omega = \omega_{ab}$.
Eliminate $C_{b, n + 1}\left(t\right)$ from the equations:   
\[
{\ddot C}_{a,n}\left(t\right) = -i g \sqrt{n + 1}
{\dot C}_{b, n + 1}\left(t\right) = -g^2 \left(n + 1\right)
C_{a,n}\left(t\right). 
\]
The solution of the obtained equation:
\begin{equation}
C_{a,n}\left(t\right) = A \sin\left(g \sqrt{n + 1} t\right) +
B \cos\left(g \sqrt{n + 1} t\right)
\nonumber
\end{equation}

For absorption, the initial conditions ($C_{a,n}\left(0\right) = 0$, $C_{b,n+1}\left(0\right)
= 1$) give $A = i$, $B = 0$, hence  
\begin{equation}
C_{a,n}\left(t\right) = -i \sin\left(\frac{\omega_R t}{2}\right), \quad
C_{b, n + 1}\left(t\right) = \cos\left(\frac{\omega_R t}{2}\right),
\label{eqPart1RabiAbsorbtion}
\end{equation}
where $\omega_R = 2g\sqrt{n + 1}$ is the Rabi frequency in the quantum case. For emission, from initial conditions ($C_{a,n}\left(0\right) = 1$,
$C_{b,n+1}\left(0\right) = 0$) we have $A = 0$, $B = 1$, hence 
\begin{equation}
C_{a,n}\left(t\right) = \cos\left(\frac{\omega_R t}{2}\right), \quad
C_{b, n + 1}\left(t\right) = -i \sin\left(\frac{\omega_R t}{2}\right),
\label{eqPart1RabiEmission}
\end{equation}
On \autoref{fig:part1:rabi}, the dependencies of probabilities 
$\left|C_{a,n}\left(t\right)\right|^2$ and 
$\left|C_{b, n + 1}\left(t\right)\right|^2$ on time are shown for the case
of photon emission.

\input ./part1/interaction/fig5.tex
\rindex{Rabi frequency}
In the semiclassical expression, the Rabi frequency $\omega_R$ equals 
$\frac{\left|p E\right|}{\hbar}$, while in the quantum case it is $2g\sqrt{n + 1}$,
i.e. $E$ corresponds to $E_1\sqrt{n + 1}$, where $E_1$ is the field
corresponding to one photon in the mode. \rindex{photon} At large $n$, these two
expressions practically coincide. However, for small $n$ the difference can be 
significant. From the quantum expression it follows that even when there are no photons in the mode, the probabilities oscillate with frequency $g$.  In fact, 
it is known that Rabi oscillations damp out. The discrepancy arises because the atom interacts with all modes of the space, but we considered only interaction with one mode.  

Let's now try to find the general solution of the system of equations
\eqref{eqCh2_task3}. The relations can be written as
\begin{eqnarray}
  {\dot C}_{a,n}\left(t\right) = -i \frac{\omega_R}{2} e^{-i \delta t} 
C_{b, n + 1}\left(t\right),
\nonumber \\
{\dot C}_{b, n + 1}\left(t\right) = -i \frac{\omega_R}{2} e^{i \delta t} 
C_{a, n}\left(t\right),
  \nonumber
\end{eqnarray}
where $\delta = \omega - \omega_{ab}$ is the frequency detuning.
We will solve the problem only for the emission case, i.e., the initial
conditions are
\begin{eqnarray}
  C_{a,n}\left(0\right) = 1,
  \nonumber \\
  C_{b,n+1}\left(0\right) = 0.
  \nonumber
\end{eqnarray}
We are mainly interested in the behavior of the probabilities
$\left|C_{b, n + 1}\left(t\right)\right|^2$. Denoting $x\left(t\right) = C_{b,n+1}\left(t\right)$, we get
\begin{eqnarray}
  \ddot{x} =  -i \frac{\omega_R}{2} \frac{d \left(e^{i \delta t} 
    C_{a, n}\left(t\right)\right) }{d t} =
  \nonumber \\
  = i \delta \left(- i \frac{\omega_R}{2} e^{ i\delta t} 
  C_{a, n}\left(t\right) \right) -i \frac{\omega_R}{2} e^{i \delta t}
  \frac{d C_{a, n}\left(t\right)}{dt} =
  \nonumber \\
  = i \delta \dot{x} - \frac{\omega_R^2}{4} e^{-i \delta t} e^{i \delta t} x =
  \nonumber \\
  = i \delta \dot{x} - \frac{\omega_R^2}{4} x,
  \nonumber
\end{eqnarray}
i.e.
\begin{equation}
  \ddot{x} - i \delta \dot{x} + \frac{\omega_R^2}{4} x = 0.
  \label{eqPart1RabiCommon}
\end{equation}
Equation \eqref{eqPart1RabiCommon} must be solved with the following
initial conditions:
\begin{eqnarray}
  \left.x\right|_{t=0} = C_{b,n+1}\left(0\right) = 0,
  \nonumber \\
  \left.\dot{x}\right|_{t=0} = -i \frac{\omega_R}{2} C_{a,n}\left(0\right) =
  -i \frac{\omega_R}{2}.
  \nonumber
\end{eqnarray}
%% (%i1) eq: 'diff(x(t),t,2) - 2*%i*d*'diff(x(t),t) + w^2*x(t)/4 = 0;
%% (%i2) atvalue(x(t),t=0,0);
%% (%i3) atvalue('diff(x(t),t),t=0,-%i*w/2);
%% (%i4) desolve(eq, x(t));
%% positive;
%% (%i5) 
The result:
\begin{equation}
  x\left(t\right) = -  i e^{\frac{ i \delta t}{2}}
  \frac{\omega_R}{\sqrt{\omega_R^2 + \delta^2}}
  \sin{\left(\frac{\sqrt{\omega_R^2 + \delta^2}}{2}t\right)},
  \nonumber
\end{equation}
denoting $\Omega_R = \sqrt{\omega_R^2 + \delta^2}$ as
the generalized Rabi frequency, for the desired probability
$\left|C_{b,n+1}\right|^2$ we get (see \autoref{figPart1InteractionRabiDelta})
\begin{eqnarray}
  \left|C_{b,n+1}\right|^2 =  \frac{\omega_R^2}{\Omega_R^2}
  \sin^2{\frac{\Omega_R}{2} t}.
  \label{eqPart1InteractionRabiProbability}
\end{eqnarray}
%% TODO FIXME wrong result - should be
%%  \frac{\omega_R^2}{\Omega_R^2}
%%  \sin^2{\frac{\Omega_R t}{2}}

\input ./part1/interaction/figrabi.tex

\rindex{AC Stark effect}
\begin{remark}[AC Stark effect]
  It is worth noting that in the case $\delta \ne 0$ the states we
  used in expression \ref{eq:part1:rabi_solution} do not
  satisfy the law of energy conservation. Indeed, if the system
  initially was in the state $\ket{a}\ket{n}$ with
  energy $E_a + \hbar n \omega$, then after some time it,
  with nonzero probability,
  can
  be registered in the state
  $\ket{b}\ket{n+1}$ with energy
  \[
  E_b + \hbar (n+1)
  \omega \ne E_a + \hbar n \omega.
  \]
  That is, a violation of the law of
  energy conservation appears. This contradiction is resolved as follows. Due to the AC Stark effect
  (also called the Autler-Townes effect in the English literature)\cite{wiki:autler_townes_effect},
  the atomic energy levels shift, 
  which ensures that the law of conservation of energy is fulfilled.

  For example,
  at small detunings $\delta \approx 0$, this shift equals
  \cite{wiki:autler_townes_effect} 
  \[
  \Delta E \approx \pm \frac{\hbar \omega_R}{2}.
  \]
\end{remark}