%% -*- coding:utf-8 -*- 
\section{Interaction of Electromagnetic Field of the Resonator with a Reservoir of Atoms at Temperature $T$}
\label{ch2_6}
We apply the general approach outlined above to a system consisting of a harmonic oscillator (electromagnetic field mode) interacting with a reservoir in the form of a beam of two-level atoms in equilibrium at temperature $T$. As we will see later, the specific model of the reservoir does not matter for the final result.\footnote{If the interaction is weak, i.e., the binding energy is small compared to the energy of the dynamic system.} The considered model is depicted in \autoref{figPart1Ch2_7}.  

\input ./part1/interaction/fig7.tex

The initial atomic density matrix \rindex{Density matrix} corresponds to the Boltzmann distribution at temperature $T$:  
\begin{equation}
\hat{\rho}_{at} = 
\left(
\begin{array} {cc}
\rho_{aa} & 0  
\\
0 & \rho_{bb} 
\end{array}
\right)
=
z^{-1}
\left(
\begin{array} {cc}
e^{-\frac{\hbar \omega_a}{k_B T}} & 0  
\\
0 & e^{-\frac{\hbar \omega_b}{k_B T}} 
\end{array}
\right).
\end{equation}
Here $\hbar \omega_a = E_a$ is the energy of the upper level;  
$\hbar \omega_b = E_b$ is the energy of the lower level; 
$z = e^{-\frac{\hbar \omega_a}{k_B T}} + e^{-\frac{\hbar \omega_b}{k_B T}}$ is the partition function; $T$ is the temperature of the reservoir. 
We take the interaction Hamiltonian in the form of \eqref{eqCh2_task22}. For simplicity, assume that $\omega_{ab} = \omega$, i.e., the transition frequency equals the mode frequency. Then  
\begin{equation}
\hat{V} = \hbar g \left(\hat{\sigma}\hat{a}^{\dag} + 
\hat{\sigma}^{\dag}\hat{a} \right)= 
\hbar g 
\left(
\begin{array} {cc}
0 & \hat{a}  
\\
\hat{a}^{\dag} & 0 
\end{array}
\right)
\end{equation}
since  
\(
\hat{\sigma} = 
\left(
\begin{array} {cc}
0 & 0  
\\
1 & 0 
\end{array}
\right)
\);
\(
\hat{\sigma}^{\dag} = 
\left(
\begin{array} {cc}
0 & 1  
\\
0 & 0 
\end{array}
\right)
\); $g$ is the interaction constant. 

We retain terms up to the second order in the iterative series \eqref{eqCh2_rho_sequance}. In our case, after integration, we get a simple expression  
\begin{eqnarray}
\hat{\rho}_{f}\left(t\right) =
\hat{\rho}_{f}\left(t_0 + \tau\right) = 
\nonumber \\
= \hat{\rho}_{f}\left(t_0\right) +
Sp_{at}
\left\{
- \frac{i}{\hbar}\tau\left[\hat{V}, \hat{\rho}_{at, f}\right]
- \frac{1}{2} \left(\frac{\tau}{\hbar}\right)^2
\left[\hat{V},\left[\hat{V}, \hat{\rho}_{at, f}\right]\right]
\right\}, 
\label{eqCh2_task4}
\end{eqnarray}
where $\hat{\rho}_{at, f}$ is the statistical operator of the atom-field system; $\hat{\rho}_{f}$ is the statistical operator of the field, folded over atomic variables.
  
In deriving \eqref{eqCh2_task4}, it was assumed that the atom beam interacts with the field for a time $\tau$ (time of flight through the resonator). From the initial conditions (at time $t_0$), we have:  
\begin{equation}
\hat{\rho}_{at, f} = \hat{\rho}_{f}
\left(
\begin{array} {cc}
\rho_{aa} & 0  
\\
0 & \rho_{bb} 
\end{array}
\right) = 
\left(
\begin{array} {cc}
\rho_{aa}\hat{\rho}_{f} & 0  
\\
0 & \rho_{bb}\hat{\rho}_{f}
\end{array}
\right)
\end{equation}
The first-order term in \eqref{eqCh2_task4} gives a matrix with zero diagonal elements in the reservoir variables. The trace of the matrix in these variables is obviously zero  
\begin{eqnarray}
Sp_{at}\left[\hat{V}, \hat{\rho}_{at, f}\right] = 
\hbar g Sp_{at}
\left\{
\left(
\begin{array} {cc}
0 & \hat{a} \rho_{bb}\hat{\rho}_{f}  
\\
\hat{a}^{\dag} \rho_{aa}\hat{\rho}_{f}   & 0
\end{array}
\right)
\right.
-
\nonumber \\
-
\left.
\left(
\begin{array} {cc}
0 & \rho_{aa} \hat{\rho}_{f} \hat{a}   
\\
\rho_{bb} \hat{\rho}_{f} \hat{a}^{\dag}   & 0
\end{array}
\right)
\right\}
= 0.
\label{eqCh2_sp_1}
\end{eqnarray}
The matrix included in the second-order term has non-zero diagonal elements, its trace in the reservoir variables is non-zero and can be calculated: 
\begin{eqnarray}
Sp_{at}\left[\hat{V},\left[\hat{V}, \hat{\rho}_{at, f}\right] \right]
= \hbar^2 g^2 \cdot
\nonumber \\
\cdot Sp_{at}
\left(
\begin{array} {cc}
\hat{a} \hat{a}^{\dag}\rho_{aa} \hat{\rho}_{f} -
\hat{a}\rho_{bb} \hat{\rho}_{f}\hat{a}^{\dag}  & 0
\\
0 & 
\hat{a}^{\dag} \hat{a}\rho_{bb} \hat{\rho}_{f} -
\hat{a}^{\dag}\rho_{aa} \hat{\rho}_{f}\hat{a} 
\end{array}
\right) + \mbox{c.c.} 
\label{eqCh2_sp_2}
\end{eqnarray}
From here, using equations \eqref{eqCh2_task4}, \eqref{eqCh2_sp_1}, \eqref{eqCh2_sp_2}, we obtain the final equation for the statistical operator of the field $\hat{\rho}$:  
\begin{eqnarray}
\hat{\rho}_{f}\left(t_0 + \tau\right) =
\hat{\rho}_{f}\left(t_0\right) - \frac{1}{2}g^2 \tau^2
\left\{
\left(\hat{a}\hat{a}^{\dag}\hat{\rho}_{f} - 
\hat{a}^{\dag}\hat{\rho}_{f}\hat{a}
\right)\rho_{aa} +
\right.
\nonumber \\
\left.
+
\left(\hat{a}^{\dag}\hat{a}\hat{\rho}_{f} - 
\hat{a}\hat{\rho}_{f}\hat{a}^{\dag}
\right)\rho_{bb}
\right\} + \mbox{c.c.}
\label{eqCh2_rho_final}
\end{eqnarray}
Thus, the statistical operator of the mode due to interaction with one atom changes by an amount 
\(
\delta\hat{\rho}_{f}=
\hat{\rho}_{f}\left(t_0 + \tau\right) -
\hat{\rho}_{f}\left(t_0\right)
\), determined by \eqref{eqCh2_rho_final}. If the injection rate of atoms into the resonator is $r$ atoms per second, during the time $\Delta t$ $r \Delta t$ atoms will interact. The change in the density operator over this time will be 
$\Delta \hat{\rho}_{f} = \delta\hat{\rho}_{f}r \Delta t$. We define a smoothed derivative as 
\[
\dot{\hat{\rho}}_{f} = \frac{\Delta \hat{\rho}_{f}}{\Delta t} = 
\delta\hat{\rho}_{f} r = r 
\left(
\hat{\rho}_{f}\left(t_0 + \tau\right) -
\hat{\rho}_{f}\left(t_0\right)
\right).
\]
Let us denote: $r_a = r \rho_{aa}$ as the injection rate of atoms in the upper state; $r_b = r \rho_{bb}$ as the injection rate of atoms in the lower state. This leads us to the final form of the equation for the statistical operator of the resonance mode field 
\begin{eqnarray}
\dot{\hat{\rho}}_{f} =
- \frac{1}{2}R_a
\left(\hat{a}\hat{a}^{\dag}\hat{\rho}_{f} - 
\hat{a}^{\dag}\hat{\rho}_{f}\hat{a}
\right)
- 
\nonumber \\
- \frac{1}{2}R_b
\left(\hat{a}^{\dag}\hat{a}\hat{\rho}_{f} - 
\hat{a}\hat{\rho}_{f}\hat{a}^{\dag}
\right)
 + \mbox{c.c.}
\label{eqCh2_rho_final2}
\end{eqnarray}
Here it is denoted 
\begin{eqnarray}
R_a=r_a g^2\tau^2,
\nonumber \\
R_b=r_b g^2\tau^2.
\label{eqCh2_RaRbDefenition}
\end{eqnarray}
Equation \eqref{eqCh2_rho_final2} is written in operator form. It can be written in different representations.   