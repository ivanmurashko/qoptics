\section{Emission and Absorption of Light by an Atom}
Consider the simplest problem of the interaction of single-mode radiation with a two-level atom. The simplifications are justified because laser radiation can often be considered single-mode, and in resonant interaction, all levels of the atom except two can be neglected, with the transition frequency between them close to the frequency of the mode interacting with the atom. This simplest situation is depicted in \autoref{figPart1Ch2_1}

\input ./part1/interaction/fig1.tex

Assume that the atom and the field are initially in the states
\begin{equation}
\left|\psi_A\right> = C_a\ket{a} + C_b\ket{b}, \,
\left|\psi_F\right> = \sum_{(n)}C_n\ket{n}
\end{equation}
where: $\ket{a}$, $\ket{b}$ are vectors of the upper and lower states of the atom, respectively, $\ket{n}$ is the state vector of the mode containing $n$ photons.

The complete state vector of the atom-field system is
\begin{equation}
\left|\psi_{AF}\right> = \sum_{(n)} 
\left(
C_{an}\ket{a}\ket{n} + 
C_{bn}\ket{b}\ket{n}
\right).
\end{equation}
Here $C_{an}$, $C_{bn}$ are the corresponding probability amplitudes. Due to the interaction between the atom and the field, the probability amplitudes change over time. For example, if the atom was initially in the upper level and the field mode contained $n$ photons, i.e.,
$\left|\psi_{AF}\right> = \ket{a}\ket{n}$,
then at subsequent moments in time the field will be in the state
\begin{equation}
\left|\psi_{AF}\right> =
C_{an}\ket{a}\ket{n} + 
C_{b,n + 1}\ket{b}\ket{n + 1}.
\label{eq:part1:rabi_solution}
\end{equation}