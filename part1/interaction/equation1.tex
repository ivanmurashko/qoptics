\section{Equation for the Density Matrix of the Field in the Occupation Number Representation}
\label{ch2_7}
In the occupation number representation (photon numbers), the matrix elements are defined by the equality
\[
\rho_{nm} = \bra{n}\hat{\rho}\ket{m}.
\]
To obtain the required equation, multiply \eqref{eqCh2_rho_final2} on the left by $\bra{n}$ and on the right by $\ket{m}$. Using the properties of the operators $\hat{a}$ and $\hat{a}^{\dag}$, we obtain a system of equations relating the matrix elements
\begin{eqnarray}
\dot{\rho}_{nm} = - \frac{1}{2}
\left(
R_a\left(n + 1 + m + 1\right) + 
R_b\left(n + m\right)
\right)\rho_{nm} +
\nonumber \\
+ 
R_a\sqrt{nm}\rho_{n - 1, m - 1} +
R_b\sqrt{\left(n + 1\right)\left(m + 1\right)}\rho_{n + 1, m + 1}.
\label{eqCh2_task5}
\end{eqnarray}
For diagonal elements $n = m$, therefore
\begin{eqnarray}
\dot{\rho}_{nn} = - 
\left(
R_a\left(n + 1\right) + 
R_b\left(n\right)
\right)\rho_{nn} +
\nonumber \\
+ 
R_a n \rho_{n - 1, n - 1} +
R_b\left(n + 1\right)\rho_{n + 1, n + 1}.
\label{eqCh2_51}
\end{eqnarray}

\input ./part1/interaction/fig8.tex

This equation can be considered as the balance of probability flows of photon numbers. This is conditionally depicted in \autoref{figPart1Ch2_8}. In thermal equilibrium, the flows should be equal. Hence, using the principle of detailed balance, we obtain the equalities:
\begin{eqnarray}
R_a n \rho_{n - 1, n - 1} = R_b n \rho_{nn},
\nonumber \\
R_a \left(n + 1\right) \rho_{n n} = R_b 
\left(n + 1\right) \rho_{n + 1, n + 1}
\label{eqCh2_52}
\end{eqnarray}
From \eqref{eqCh2_52} we have
\begin{equation}
\rho_{n + 1, n + 1} = \frac{R_a}{R_b}\rho_{nn} = 
e^{-\frac{\hbar \omega}{k_B T}}\rho_{nn},
\label{eqCh2_53}
\end{equation}
because  
\(
\frac{R_a}{R_b} = 
\exp \left(-\frac{\hbar \omega}{k_B T}\right)
\)
(the reservoir is in equilibrium at temperature $T$). Using \eqref{eqCh2_53} successively, starting with $n = 0$, we obtain:  
\begin{equation}
\rho_{nn} = \rho_{00} 
e^{-\frac{n \hbar \omega}{k_B T}} =
\left(1 - e^{-\frac{\hbar \omega}{k_B T}}\right) 
e^{-\frac{n \hbar \omega}{k_B T}},
\end{equation}
where $\rho_{00}$ is found from the condition $\sum_{(n)}\rho_{nn} = 1$.

The average number of photons in the mode, as expected, is determined by the Planck formula
\begin{equation}
\bar{n} = \left<n\right> = 
\sum_{(n)} n \rho_{nn} = 
\frac{1}{ e^{\frac{\hbar \omega}{k_B T}} - 1}.
\end{equation}

From all said, it follows that over time the temperature of the radiation becomes equal to the temperature of the atomic beam (reservoir).  
The change in the average number of photons over time is obtained from the equation \eqref{eqCh2_51}
\begin{eqnarray}
\frac{d}{d t}\left<n\left(t\right)\right> = \sum_{(n)}n \dot{\rho} = 
\nonumber \\
= \sum_{(n)}\left(-R_a\left(n^2 + n\right)\rho_{nn} - R_b n^2
\rho_{nn} + \right.
\nonumber \\
+ \left.
R_b \left(n^2 + n\right) \rho_{n +1, n+ 1} +
R_a n^2 \rho_{n - 1, n - 1}
\right).
\end{eqnarray}
Replace the summation variables $m= n + 1$ in the third sum and $m = n - 1$ in the fourth sum. We get:
\begin{eqnarray}
\frac{d}{d t}\left<n\right> = 
-R_a \sum_{(n)}\left(n^2 + n\right)\rho_{nn}
 - R_b \sum_{(n)} n^2 \rho_{nn} +
\nonumber \\
+ R_b\sum_{(m)}\left(m^2 - m\right)\rho_{m,m} 
+ R_a\sum_{(m)}\left(m^2 +2 m + 1\right)\rho_{m,m} = 
\nonumber \\
= R_a \sum_{(m)}\left( m + 1\right)\rho_{m,m} - R_b\sum_{(m)} m
\rho_{m,m} =
\nonumber \\
= \left(R_a - R_b\right) \left<n\right> + R_a.
\label{eqCh2_57}
\end{eqnarray}
At equilibrium
\[
\frac{d}{d t}\left<n\right> = 0,
\]  
and we obtain the previous relation
\begin{equation}
\left<n_{(\infty)}\right> = \bar{n} = 
\frac{R_a}{R_b - R_a} = \frac{1}{\frac{R_b}{R_a} - 1} = 
\frac{1}{e^{\frac{\hbar \omega}{k_B T}} - 1}.
\end{equation}

\begin{definition}[Q-factor]
\label{def:Qfactor}
The Q-factor of an oscillatory system $Q$ is called \cite{bKarlov2003} the number of oscillations performed by the system during the characteristic time of their damping $\tau$. That is, if the change in the energy of the system $U$ (complex amplitude) obeys the equation
\begin{equation}
\label{eq:part1:q:remark}
\frac{d U}{d t} = - \frac{1}{\tau} \left(U - U_0\right),
\end{equation}
then
\[
U = \left(\left.U\right|_{t=0} - U_0 \right)e^{-\frac{t}{\tau}} + U_0
\]
Thus
\[
Q = \omega \tau
\]
therefore the equation \eqref{eq:part1:q:remark} can be rewritten as 
\[
\frac{d U}{d t} = - \frac{\omega}{Q} \left(U-U_0\right)
\]
\end{definition}


Equation \eqref{eqCh2_57} describes the change in the number of photons (energy) over time as a result of relaxation (interaction with a dissipative system). The classical equation describing this process has the form (see def. \ref{def:Qfactor})
\begin{equation}
\frac{d}{d t}\left<n\right> = 
- \frac{\omega}{Q}\left<n\right> + \frac{\omega}{Q} \left<n\right>_{\mbox{eq.}},
\nonumber
\end{equation}
where $\left<n\right>_{\mbox{eq.}} = \bar{n}_T$ is the equilibrium value of $\left<n\right>$ at temperature $T$, and $Q$ is the Q-factor of the resonator (see def. \ref{def:Qfactor}).
\rindex{Q-factor}
Therefore, it is possible to assume 
\begin{eqnarray}
R_b - R_a = \frac{\omega}{Q},
\nonumber \\
R_a = \bar{n}_T \frac{\omega}{Q},
\nonumber \\
R_b = \frac{\omega}{Q} \left(1 + \bar{n}_T\right),
\label{eqCh2_RabQw}
\end{eqnarray}
i.e., the quantities $R_a$ and $R_b$ introduced by us are expressed through the classical quantity $\frac{\omega}{Q}$, characterizing the losses in the resonator. 

Another quantity of interest is the average electromagnetic field
\begin{eqnarray}
\bar{E}_{(f)} = E_0 \sin k z
Sp\left(\hat{\rho}\left(\hat{a}^{\dag} + \hat{a}\right)\right) = 
\nonumber \\
= E_0 \sin k z Sp\left(\hat{\rho}\hat{a}\right) + \text{c.c.} = 
\nonumber \\
= E_1 Sp\left(\hat{\rho}\hat{a}\right) + \text{c.c.} =
\left<E\right> + \text{c.c.}, 
\nonumber
\end{eqnarray}
where $\left<E\right>$ is the analytical signal of the classical field. The equation satisfied by the field can be obtained using the equation of motion for the density matrix. Write the equation for the statistical operator of the field mode \eqref{eqCh2_rho_final2} and using for $R_a$, $R_b$ their expressions through $Q$ and $\bar{n}_T$ \eqref{eqCh2_RabQw}, we have
\begin{eqnarray}
\dot{\hat{\rho}} =
- \frac{\omega}{2Q}\bar{n}_T
\left(\hat{a}\hat{a}^{\dag}\hat{\rho} - 
2 \hat{a}^{\dag}\hat{\rho}\hat{a} + \hat{\rho}\hat{a}\hat{a}^{\dag}
\right)
- 
\nonumber \\
- \frac{\omega}{2Q}\left(\bar{n}_T + 1\right)
\left(\hat{a}^{\dag}\hat{a}\hat{\rho} - 
2 \hat{a}\hat{\rho}\hat{a}^{\dag}
+ \hat{\rho}\hat{a}^{\dag}\hat{a}
\right)
\label{eqCh2_eq1_add1}
\end{eqnarray}

Since 
\begin{equation}
\left<E\right> = E_1 Sp\left(\hat{\rho}\hat{a}\right), \quad
\frac{d \left<E\right>}{d t} = E_1 Sp\left(\frac{d \hat{\rho}}{dt}\hat{a}\right),
\label{eqCh2_eq1_add2}
\end{equation}
Therefore, equation \eqref{eqCh2_eq1_add1} must be multiplied by $E_1 \hat{a}$ and $Sp$ taken. As a result, we obtain:
\begin{eqnarray}
\dot{\left<E\right>} =
- \frac{\omega E_1}{2Q}\bar{n}_T
\left\{Sp\left(\hat{a}\hat{a}^{\dag}\hat{\rho}\hat{a} - 
2 \hat{a}^{\dag}\hat{\rho}\hat{a}\hat{a} +
\hat{\rho}\hat{a}\hat{a}^{\dag}\hat{a}\right) + 
\right.
\nonumber \\
+\left.
Sp\left(\hat{a}^{\dag}\hat{a}\hat{\rho}\hat{a} - 
2 \hat{a}\hat{\rho}\hat{a}^{\dag}\hat{a}
+ \hat{\rho}\hat{a}^{\dag}\hat{a}\hat{a}
\right)
\right\}
- 
\nonumber \\
- \frac{\omega E_1}{2Q}
Sp\left(\hat{a}^{\dag}\hat{a}\hat{\rho}\hat{a} - 
2 \hat{a}\hat{\rho}\hat{a}^{\dag}\hat{a}
+ \hat{\rho}\hat{a}^{\dag}\hat{a}\hat{a}
\right)
\label{eqCh2_eq1_add3}
\end{eqnarray}
It is known that under the sign $Sp$, circular permutation of operators can be performed. Apply this to \eqref{eqCh2_eq1_add3}.
For example
\begin{eqnarray}
Sp\left(\hat{a}\hat{a}^{\dag}\hat{\rho}\hat{a} - 
2 \hat{a}^{\dag}\hat{\rho}\hat{a}\hat{a} +
\hat{\rho}\hat{a}\hat{a}^{\dag}\hat{a}\right) = 
\nonumber \\
= Sp\left(\hat{a}\hat{a}^{\dag}\hat{\rho}\hat{a} - 
2 \hat{a}\hat{a}^{\dag}\hat{\rho}\hat{a} +
\hat{\rho}\hat{a}\hat{a}^{\dag}\hat{a}\right) = 
\nonumber \\
= Sp\left(\hat{a}\hat{a}^{\dag}\hat{\rho}\hat{a} - 
2 \hat{a}\hat{a}^{\dag}\hat{\rho}\hat{a} +
\hat{\rho}\hat{a}\left(\hat{a}\hat{a}^{\dag} - 1\right)\right) = 
\nonumber \\
Sp\left(\hat{a}\hat{a}^{\dag}\hat{\rho}\hat{a} - 
2 \hat{a}\hat{a}^{\dag}\hat{\rho}\hat{a} +
\left(\hat{a}\hat{a}^{\dag} - 1\right)\hat{\rho}\hat{a}\right) = 
- Sp\left(\hat{\rho}\hat{a}\right).
\label{eqCh2_eq1_add4}
\end{eqnarray}
Doing the same with the second bracket in \eqref{eqCh2_eq1_add3}, we get
\begin{eqnarray}
Sp\left(\hat{a}^{\dag}\hat{a}\hat{\rho}\hat{a} - 
2 \hat{a}\hat{\rho}\hat{a}^{\dag}\hat{a}
+ \hat{\rho}\hat{a}^{\dag}\hat{a}\hat{a} \right) = 
\nonumber \\
= 
Sp\left(\hat{\rho}\hat{a}\hat{a}^{\dag}\hat{a} - 
2 \hat{\rho}\hat{a}^{\dag}\hat{a}\hat{a}
+ \hat{\rho}\hat{a}^{\dag}\hat{a}\hat{a} \right) = 
\nonumber \\
= 
Sp\left(\hat{\rho}\left(\hat{a}^{\dag}\hat{a} + 1\right)\hat{a} - 
2 \hat{\rho}\hat{a}^{\dag}\hat{a}\hat{a}
+ \hat{\rho}\hat{a}^{\dag}\hat{a}\hat{a} \right) = 
Sp\left(\hat{\rho}\hat{a}\right).
\label{eqCh2_eq1_add5}
\end{eqnarray}
Substituting \eqref{eqCh2_eq1_add4} and \eqref{eqCh2_eq1_add5} into \eqref{eqCh2_eq1_add3}, we have
\begin{eqnarray}
\dot{\left<E\right>} =
- \frac{\omega E_1}{2Q}\bar{n}_T
\left\{Sp\left(\hat{\rho}\hat{a}\right) -
Sp\left(\hat{\rho}\hat{a}\right)\right\} -
\nonumber \\
- \frac{\omega E_1}{2Q}Sp\left(\hat{\rho}\hat{a}\right) = 
- \frac{\omega}{2Q}\left<E\right>.
\label{eqCh2_61}
\end{eqnarray}

Thus, we have related the parameters of the reservoir (parameters of the beam) with the classical value $Q$ and with the average number of photons in the mode at the reservoir temperature. This allows us to write the equations of motion for the density matrix of the field in general form, while the specific model of the reservoir does not matter
\begin{eqnarray}
\dot{\rho}_{nm} = - \frac{\omega}{2 Q}
\left(2 \bar{n}_T\left( n + m + 1\right) + n + m \right)\rho_{nm} +
\nonumber \\
+ \frac{\omega \bar{n}_T}{Q}\sqrt{nm}\rho_{n - 1, m - 1} +
\frac{\omega}{Q}\left(\bar{n}_T + 1\right)
\sqrt{\left(n + 1\right)\left(m + 1\right)}
\rho_{n + 1, m + 1}
\label{eqCh2_63}
\end{eqnarray}
In operator form, this equation is written as
\begin{eqnarray}
\dot{\rho} = - \frac{\omega}{2 Q}
\left\{
\bar{n}_T\left(\hat{a}\hat{a}^{\dag}\hat{\rho} - 
\hat{a}^{\dag}\hat{\rho}\hat{a}\right)
\right. +
\nonumber \\
+
\left .
\left(\bar{n}_T + 1\right)\left(\hat{a}^{\dag}\hat{a}\hat{\rho} - 
\hat{a}\hat{\rho}\hat{a}^{\dag}\right)
\right\} + \text{h.c.}
\label{eqCh2_64}
\end{eqnarray}
The recording of the equation \eqref{eqCh2_rho_final2}, \ref{eqCh2_64} in the occupancy number representation is one of many. It is often convenient to use other representations.  