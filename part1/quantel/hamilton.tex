%% -*- coding:utf-8 -*- 
\section{Hamiltonian Form of the Equations for the Free Electromagnetic Field}
Let us represent the field in the volume $V$ in the form of expansions \eqref{eqCh1_sep1}:
\begin{equation}
\vec{E}\left(r, t\right) = \sum_{(n)}
\frac{q_n\left(t\right) \omega_n}{\sqrt{\varepsilon_0}} \vec{E}_n\left(r\right),
\quad
\vec{H}\left(r, t\right) = \sum_{(n)}
\frac{p_n\left(t\right)}{\sqrt{\mu_0}} \vec{H}_n\left(r\right),
\label{eqCh1_sep1_1}
\end{equation}
where $p_n = \frac{d q_n}{d t}$, and $q_n$ satisfies the equation
\[
\frac{d^2 q_n}{d t^2} + \omega_n^2 q_n = 0.
\]
This is the equation of the harmonic oscillator. Its solution is:
\begin{equation}
q_n\left(t\right) = q_n\left(0\right) \cos \omega_n t + 
\frac{\dot{q_n}\left(0\right)}{\omega_n} \sin \omega_n t 
\end{equation}
$q_n\left(0\right)$ and 
$\dot{q_n}\left(0\right)$ 
are determined from the expansion of the initial field over  
$\vec{E_n}$, $\vec{H_n}$.
 
Now let us write the electrodynamics equations in Hamiltonian
form. The Hamiltonian function for the electromagnetic field 
(the field energy)
\begin{equation}
\mathcal{H} = \frac{1}{2}
\int_{\nu}\left( \varepsilon_0\left(\vec{E}\right)^2 + \mu_0
\left(\vec{H}\right)^2\right) d\nu
\label{eqCh1_hamilton}
\end{equation}
Substituting expansions \eqref{eqCh1_sep1_1} into \eqref{eqCh1_hamilton} and
using the orthogonality of the eigenfunctions, we get 
\begin{eqnarray}
\mathcal{H} = \frac{1}{2} 
\int_{\nu}\left( \sum_{(n)} \sum_{(m)}\varepsilon_0
\frac{q_n q_m \omega_n \omega_m}{\varepsilon_0}
\left( \vec{E_n} \vec{E_m}\right)
\right) d\nu +
\nonumber \\
+  \frac{1}{2} 
\int_{\nu}\left( \sum_{(n)} \sum_{(m)}\mu_0
\frac{p_n p_m}{\mu_0}
\left( \vec{H_n} \vec{H_m}\right)
\right) d\nu =
\nonumber \\
= \frac{1}{2}\sum_{(n)}\left(\omega_n^2 q_n^2 + p_n^2\right).
\end{eqnarray}
This expression corresponds to the Hamiltonian function for a collection
of independent harmonic oscillators. 

For a single mode, the Hamiltonian function is  
\begin{equation}
\mathcal{H}_n = \frac{1}{2}\left(\omega_n^2 q_n^2 + p_n^2\right).
\label{eqCh1_hamilton_one_mode}
\end{equation}
then
\begin{equation}
\mathcal{H} = \sum_{(n)} \mathcal{H}_n
\label{eqCh1_hamilton_sum_mode}
\end{equation}
The equations of motion are obtained from the Hamiltonian function by the standard procedure
\eqref{eq:add:quantel:hamilton}:
\begin{equation}
\dot{q}_n = \frac{\partial \mathcal{H}_n}{\partial p_n} = p_n,
\quad
\dot{p}_n = - \frac{\partial \mathcal{H}_n}{\partial q_n} =
- \omega_n^2 q_n = \ddot{q}_n.
\end{equation}
The resulting equations coincide with equations \eqref{eqCh1_after_sep}.  
The Hamiltonian function of a mechanical oscillator is
\[
\mathcal{H} = \frac{1}{2}\left(M \Omega^2 x^2 + \frac{p^2}{M}\right)
\]

Therefore, \eqref{eqCh1_hamilton_one_mode} formally corresponds
to an oscillator with unit mass and eigenfrequency $\omega_n$.
The Hamiltonian formulation of the electromagnetic field equations
is convenient for the field quantization procedure. This
approach is often suitable for solving various classical
problems as well. More details can be found in the book
\cite{bCh1Quantel_Gin}.  
