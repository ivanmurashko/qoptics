%% -*- coding:utf-8 -*- 
\section{Coherent States}
States $\ket{n}$ do not tend with increasing $n$ to the classical solution for the oscillator. This property is possessed by the so-called coherent states. They were introduced by E. Schrödinger and later developed in our time by R. Glauber \cite{bQuantumOpticsAndRadioPhisicsLecture1966} and E. Sudarshan \cite{bKaluderSudershan1970} in relation to problems of quantum optics. A coherent state can be defined in various ways. We will use the definition in the form of a superposition of states $\ket{n}$:
\begin{equation}
\left|\alpha\right> = e^{-\frac{1}{2} \left|\alpha\right|^2} \sum_{(n)} \frac{\alpha^n}{\sqrt{n!}}\ket{n},
\quad
\left<\alpha\right| = e^{-\frac{1}{2} \left|\alpha\right|^2} \sum_{(n)} \frac{\alpha^{*n}}{\sqrt{n!}}\bra{n},
\label{eqCh1_coh_def}
\end{equation}
where $\alpha$ is some complex parameter, the meaning of which will become clear later.

From the definition \eqref{eqCh1_coh_def} follows a number of properties of coherent states. First of all,
\begin{equation}
\left<\alpha\right|\left.\alpha\right> = e^{-\left|\alpha\right|^2}\sum_{(n)}
\frac{\left(\alpha\alpha^{*}\right)^n}{n!} = e^{\left|\alpha\right|^2} e^{-\left|\alpha\right|^2} = 1,
\end{equation}
i.e., coherent states are normalized. However, these states are not orthogonal:
\begin{equation}
\left<\alpha\right|\left.\beta\right> = 
e^{-\frac{1}{2}\left(\left|\alpha\right|^2 + \left|\beta\right|^2\right)}\sum_{(n)} 
\frac{\left(\alpha^{*}\beta\right)^n}{n!} = 
e^{
-\frac{1}{2} \left|\alpha\right|^2  -\frac{1}{2} \left|\beta\right|^2
+
\alpha^{*} \beta 
}
\label{eqCh1_ortog}
\end{equation}
Hence
\begin{equation}
\left|\left<\alpha\right|\left.\beta\right>\right|^2 = 
e^{
-\left|\alpha\right|^2  - \left|\beta\right|^2
+
\alpha^{*} \beta  + \alpha \beta^{*}} = 
e^{-\left|\alpha - \beta\right|^2}, 
\end{equation}
i.e., these states can be considered approximately orthogonal if $\left|\alpha - \beta\right|$ is sufficiently large.

The non-orthogonality of coherent states is a consequence of the over-completeness of the system ($\alpha$ is any complex number, whereas $n$ are only integers).

Coherent states are eigenstates of the annihilation operator with an eigenvalue of $\alpha$:
\begin{equation}
\hat{a}\left|\alpha\right> = e^{-\frac{1}{2} \left|\alpha\right|^2}
\sum_{(n)} \frac{\alpha^n}{\sqrt{n!}}\sqrt{n}\ket{n - 1} = 
e^{-\frac{1}{2} \left|\alpha\right|^2}
\sum_{(m = n - 1)} \frac{\alpha \alpha^m}{\sqrt{m!}}\ket{m} = 
\alpha\left|\alpha\right>
\end{equation}

The conjugate relationship looks like
\begin{equation}
\left<\alpha\right|\hat{a}^{\dag} =  
\alpha^{*}\left<\alpha\right|
\end{equation}

The probability of detecting (measuring) $n$ photons in a coherent state
\begin{equation}
\left|\bra{n}\left.\alpha\right>\right|^2 =
e^{-\left|\alpha\right|^2}
\frac{\left(\left|\alpha\right|^2\right)^n}{n!}
\label{eqCh1_PuassonCoh}
\end{equation}
corresponds to the Poisson distribution
with an average number of photons  
$\left<n\right> = \left|\alpha\right|^2$. Using the known relationship 
\begin{equation}
\ket{n} = \frac{1}{\sqrt{n!}}
\left(\hat{a}^{\dag}\right)^n\ket{0}
\label{eqCh1_nstate}
\end{equation}
the coherent state can be represented as
\begin{equation}
\left|\alpha\right> = e^{-\frac{1}{2} \left|\alpha\right|^2}\sum_{(n)}
\frac{\left(\alpha \hat{a}^{\dag}\right)^n}{n!}\ket{0} = 
e^{\alpha \hat{a}^{\dag} -
  \frac{1}{2}\left|\alpha\right|^2}\ket{0}
\label{eqCh1_astate}
\end{equation}

Rewriting \eqref{eqCh1_astate} in another form, we use the following relation
\[
e^{- \alpha^{*} \hat{a}} \ket{0} = \left(1 - \alpha^{*} \hat{a} +
\dots\right) \ket{0} = \ket{0},
\]
resulting in
\begin{equation}
\left|\alpha\right> = 
e^{\alpha \hat{a}^{\dag}} 
e^{-\frac{1}{2}\left|\alpha\right|^2}
e^{- \alpha^{*} \hat{a}} 
\ket{0}.
\label{eqCh1_astate_add}
\end{equation}
Next, we apply the operator identity (Baker-Hausdorff formula)
\begin{equation}
e^{\hat{c} + \hat{d}} = e^{- \frac{1}{2}\left[\hat{c},
    \hat{d}\right]}e^{\hat{c}} e^{\hat{d}} 
\label{eqPart1Ch1_BeikerHausdorf}
\end{equation}
valid under the condition  
$\left[\hat{c},\left[\hat{c}, \hat{d}\right]\right] =
\left[\hat{d},\left[\hat{c}, \hat{d}\right]\right] = 0$.
If we let $\hat{c} = \alpha\hat{a}^{\dag}$, $\hat{d} = -\alpha^*\hat{a}$
then from \eqref{eqCh1_astate_add} we get 
\begin{eqnarray}
\left|\alpha\right> =  
e^{\alpha \hat{a}^{\dag}} 
e^{-\frac{1}{2}\left|\alpha\right|^2}
e^{- \alpha^{*} \hat{a}} 
\ket{0} =
e^{-\frac{1}{2}\left|\alpha\right|^2}
e^{\alpha \hat{a}^{\dag}} 
e^{- \alpha^{*} \hat{a}} 
\ket{0} = 
\nonumber \\
=
e^{\alpha \hat{a}^{\dag} - \alpha^{*} \hat{a}}\ket{0}.
\label{eqCh1_astate4squeezed}
\end{eqnarray}

In order to vividly represent coherent states, in \autoref{AddQCoh} a coordinate representation of the coherent state is given.

Here are a few more useful relationships obtained from \eqref{eqCh1_astate}. Differentiating \eqref{eqCh1_astate} by $\alpha$,
we have:  
\[
\frac{\partial}{\partial \alpha}\left|\alpha\right> = 
\left( \hat{a}^{\dag} - \frac{1}{2}\alpha^{*}\right)\left|\alpha\right>,
\]
or equivalently 
\[
\hat{a}^{\dag}\left|\alpha\right> = \left(\frac{\partial}{\partial
  \alpha} +  \frac{1}{2}\alpha^{*}\right)\left|\alpha\right>.
\]
The conjugate equality looks like
\[
\left<\alpha\right|\hat{a} = \left(\frac{\partial}{\partial
  \alpha^{*}} +  \frac{1}{2}\alpha\right)\left<\alpha\right|
\]

In general, the operator of the positive frequency part of the electric field looks like
\[
\hat{\vec{E}}^{(+)} = \sum_{(k)} \sqrt{\frac{\hbar \omega_k}{2 \varepsilon_0
V}} \hat{a}_k\left(t\right) \vec{e}_k e^{-i \omega_k t + i \left(\vec{k}\vec{r}
  \right)}, 
\]

Suppose that the field (multimode) is in the state
\[
\left| \left\{\alpha_s\right\}\right> = 
\left| \alpha_1, \alpha_2, \dots, \alpha_s, \dots\right> = 
\left| \alpha_1\right>
\otimes
\left| \alpha_2\right>
\otimes
\dots
\otimes
\left| \alpha_s\right>
\otimes
\dots,
\]
where $\left\{\alpha_s\right\}= \left\{\alpha_{k_s}\right\}$
denotes a certain set of parameters $\alpha$. Applying the operator $\hat{\vec{E}}^{(+)}$ to the state $\left|\left\{\alpha_{k_s}\right\}\right>$, we obtain 
\begin{eqnarray}
\hat{\vec{E}}^{(+)}\left|\left\{\alpha_{k_s}\right\}\right> = 
\left\{ \sum_{(k_s)} \sqrt{\frac{\hbar \omega_k}{2 \varepsilon_0
V}} \hat{a}_{k_s} \vec{e}_k e^{-i \omega_k t + i \left(\vec{k}\vec{r}
  \right)}\right\}\left|\left\{\alpha_{k_s}\right\}\right> = 
\nonumber \\
= 
\sum_{(k_s)} \sqrt{\frac{\hbar \omega_k}{2 \varepsilon_0
V}} \alpha_{k_s} \vec{e}_k e^{-i \omega_k t + i \left(\vec{k}\vec{r}
  \right)}\left|\left\{\alpha_{k_s}\right\}\right>,
\end{eqnarray}
i.e., $\left|\left\{\alpha_{k_s}\right\}\right>$ is an eigenvector of the positive-frequency part of the electric field operator, and the eigenvalue is a classical field (analytical signal), whose complex mode amplitudes are  
\(
\sqrt{\frac{\hbar \omega_k}{2 \varepsilon_0
V}} \alpha_k.
\)
From this, it follows that each classical field corresponds to some coherent state.

The average value of the electric field operator $\hat{\vec{E}} =
\hat{\vec{E}}^{(+)} + \hat{\vec{E}}^{(-)}$ in a coherent state is obtained by first considering a single mode: 
\[
\left<\alpha_{k}\right|\hat{\vec{E}}_k\left|\alpha_{k}\right> = 2
\sqrt{\frac{\hbar \omega_k}{2 \varepsilon_0
V}}\left|\alpha_{k}\right|\cos 
\left(\omega_k t - \left(\vec{k} \vec{r} + \theta_k\right) \right),
\]
where $\alpha_k = \left|\alpha_{k}\right| e^{i \theta_k}$, which follows from the relations
\[
\left<\alpha_k\right|\hat{a}_k\left|\alpha_k\right> = \alpha_k, \quad
\left<\alpha_k\right|\hat{a}^{\dag}_k\left|\alpha_k\right> = \alpha^{*}_k
\]
Generalizing this to a multimode case, we obtain
\begin{equation}
\left<\left\{\alpha_{k}\right\}\right|\hat{\vec{E}}\left|\left\{\alpha_{k}\right\}\right>
= 2 \sum_{(k)} \sqrt{\frac{\hbar \omega_k}{2 \varepsilon_0
V}} \left|\alpha_k\right|\cos \left(\omega_k t -
\left(\vec{k}\vec{r}\right) + \theta_k \right),
\end{equation}
which corresponds to a classical multimode field, i.e., the average value of $\hat{\vec{E}}$ in a coherent state is a classical field with corresponding mode amplitudes and phases.

Coherent states, as we have seen, are not orthogonal due to the over-completeness of the system of eigenvectors, but they satisfy the completeness condition:
\begin{equation}
\frac{1}{\pi}\int \left|\alpha\right>\left<\alpha\right| d^2 \alpha =
\hat{I},
\label{eqCh1_full4coh}
\end{equation}
Here $d^2 \alpha = d\, Re \alpha\, d\, Im \alpha$, $\hat{I}$ is the identity operator, and the integration is carried out over the entire complex plane. We will prove this using polar coordinates, shown in \autoref{figCh1_Coh}.

\input ./part1/quantel/fig5.tex

Using the decomposition of $\left|\alpha\right>$ by $\ket{n}$ in polar coordinates, we get 
\begin{eqnarray}
\int d^2 \alpha \left|\alpha\right>\left<\alpha\right| =
\nonumber \\
=
\int_0^{\infty} d \left|\alpha\right|\int_0^{2
  \pi}\left|\alpha\right|\sum_{(n)}\sum_{(m)}e^{-i \left(n -
  m\right)\varphi}
e^{-\left|\alpha\right|^2}\ket{m}\bra{n}
\frac{\left|\alpha\right|^m \left|\alpha\right|^n}{\sqrt{n! m!}} d\varphi
= 
\nonumber \\
= \pi \sum_{(n)}\frac{1}{n!}\int_0^{\infty} 2
\left|\alpha\right|\left|\alpha\right|^{2n} e^{-\left|\alpha\right|^2} 
\ket{n}\bra{n}
d\left|\alpha\right| = 
\nonumber \\
=
\pi
\sum_{(n)}\frac{\ket{n}\bra{n}}{n!} \int_0^{\infty}x^n
e^{-x}dx = 
\pi \sum_{(n)}\ket{n}\bra{n} = \pi \hat{I},
\end{eqnarray}
where it is considered that
\[
\int_0^{2\pi} e^{-i \left(n - m\right)\varphi} = 2 \pi \delta_{nm},
\]
a change of variables was made:  
\[
\left|\alpha\right|^2 = x, \quad 2 \left|\alpha\right| d
\left|\alpha\right| = d x,
\]
and the integral representation of the factorial is used:
\[
n! = \int_0^{\infty}x^ne^{-x}dx
\]

Thus, the formula \eqref{eqCh1_full4coh} is proved. With the help of this relationship, one coherent state can be expressed through all the others: 
\[
\left|\alpha\right> = \frac{1}{\pi}\int\left|\beta\right>\left<\beta\right|\left.\alpha\right> d^2 \beta,
\]
but in \eqref{eqCh1_ortog} we had
\(
\left<\alpha\right|\left.\beta\right> = 
e^{
-\frac{1}{2} \left|\alpha\right|^2  -\frac{1}{2} \left|\beta\right|^2
+
\alpha^{*} \beta
},
\)
thus,
\begin{equation}
\left|\alpha\right> = \frac{1}{\pi}\int\left|\beta\right>
e^{
-\frac{1}{2} \left|\alpha\right|^2  -\frac{1}{2} \left|\beta\right|^2
+
\alpha^{*} \beta
} d^2 \beta.
\end{equation}