\section{Density of States}
If the quantization volume $L^3$ is large, the density of states (the number of modes per unit frequency interval) will be quite large. The wave vector of the mode is determined by the relation \eqref{eqCh1_period}
\begin{equation}
\vec{k}_{n_x, n_y, n_z} = \frac{2 \pi}{L}\left(n_x \vec{x}_0
+ n_y \vec{y}_0
+ n_z \vec{z}_0
\right)
\end{equation}
Each set of integers corresponds to two waves differing in polarization. The modes can be visually represented as points in a Cartesian coordinate system, with $n_x$, $n_y$, $n_z$ as axes (\autoref{figCh1_pic3}). The number of oscillations in the volume  
\(
\Delta n_x \Delta n_y \Delta n_z
\)
will obviously be  
\(
\Delta N = 2 \Delta n_x \Delta n_y \Delta n_z,
\)
or, considering the relationship  
between $k$ and $n$ \eqref{eqCh1_period},
\begin{equation}
\Delta N = 2 \left(\frac{L}{2 \pi} \right)^3 \Delta k_x \Delta k_y \Delta k_z.
\label{eqCh1_modenumber}
\end{equation}

\input ./part1/quantel/fig3.tex

\input ./part1/quantel/fig4.tex

The factor of two in formula \eqref{eqCh1_modenumber} appears because two modes with different polarizations correspond to one value of $k$. For large $L$ ($L\rightarrow \infty$), the distribution becomes quasi-continuous, and summation over modes, which is almost always required in solving quantum optics problems, can be replaced by integrating over $k$: 
\begin{equation}
2 \sum_{(n)} \left( \dots \right) = 2 \left(\frac{L}{2 \pi} \right)^3
\iiint\limits_{-\infty}^{+\infty} \left( \dots \right) d k_x d k_y d k_z.
\label{eqCh1_modenumber_kvazy_contig}
\end{equation}
The transition from rectangular coordinates $k_x$, $k_y$, $k_z$ to spherical coordinates $k$, $\theta$, $\varphi$ (\autoref{figCh1_pic4}), gives
\begin{eqnarray}
k_x = k \sin \theta \cos \varphi,
\quad 
k_y = k \sin \theta \sin \varphi,
\quad 
k_z  = k \cos \theta,
\nonumber \\
d k_x d k_y d k_z = k^2 \sin \theta d k d \theta d \varphi = k^2 d k
d \Omega,
\end{eqnarray}
where $d \Omega = \sin \theta d \theta d \varphi$ is the solid angle element in the direction of $\vec{k}$. Thus, we obtain
\begin{equation}
d N = 2 \left(\frac{L}{2 \pi} \right)^3 k^2 d k d \Omega
\label{eqCh1_modenumber_1pre}
\end{equation}
or, considering that $k^2 c^2 = \omega^2$, i.e., $k^2 d k = \frac{\omega^2 d \omega}{c^3}$: 
\begin{equation}
d N = 2 \left(\frac{L}{2 \pi c} \right)^3 \omega^2 d \omega d \Omega
\label{eqCh1_modenumber_1}
\end{equation}
i.e., the number of states per unit volume, unit frequency, and unit solid angle
\begin{equation}
g\left(\omega\right)  = \frac{2 \omega^2}{\left(2 \pi c\right)^3}
\end{equation}
(spectral density of eigenstates). 