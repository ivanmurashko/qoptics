\section{Mixed States of the Electromagnetic Field}
So far, we have considered only pure states of the electromagnetic field described by state vectors. One of the most distinctive properties of pure states is the principle of superposition.

Only isolated systems can be pure, whereas mixed systems are not isolated from their environment. To describe mixed states, we use the density matrix apparatus. The density operator can be introduced as follows.

Consider a mixed state, 
for which the probabilities $P_M$ of the field being in each of the states $\ket{M}$ are known. Let's calculate the average value of the operator $\hat{O}$ corresponding to the observable $O$ in this statistically mixed state. The average for the statistical mixture of states  
\begin{equation}
\left<\hat{O}\right> = \sum_{(M)} P_M\bra{M}\hat{O}\ket{M},
\end{equation}
where it is clear that 
\[
\sum_{(M)} P_M = 1.
\]

This expression can be represented differently. Using some complete set of states $\ket{S}$ and the condition of its completeness 
\[
\sum_{(S)}\ket{S}\bra{S} = \hat{I},
\]
we obtain
\begin{equation}
\left<\hat{O}\right> = \sum_{(M)}
P_M\sum_{(S)}\bra{M}\hat{O}\ket{S}\bra{S}\ket{M}
= \sum_{(M)}\sum_{(S)}P_M\bra{S}\ket{M}\bra{M}\hat{O}\ket{S}.
\label{eqCh1_middle}
\end{equation}

Based on \eqref{eqCh1_middle}, the statistical operator $\hat{\rho}$ can be introduced using the expression
\begin{equation}
\hat{\rho} = \sum_{(M)}
P_M\ket{M}\bra{M}.
\end{equation}

Then the equality \eqref{eqCh1_middle} can be presented in the form
\begin{equation}
\left<\hat{O}\right> = \sum_{(S)}
\bra{S}\hat{\rho}\hat{O}\ket{S} = Sp \left(\hat{\rho}\hat{O}\right),
\end{equation}
This relation does not depend on the choice of $\ket{S}$, as the trace of the operator does not depend on the representation. Note that 
\begin{equation}
Sp \left(\hat{\rho}\right) = \sum_{(M)}
\bra{M}\hat{\rho}\ket{M} = \sum_{(M)} P_M = 1
\label{eqCh1_spequal1}
\end{equation}

As an example, consider the thermal excitation of photons in one mode:
\begin{equation}
\hat{\rho} = \sum_{(n)}
P_n\ket{n}\bra{n},
\label{eqCh1_teplovvozb}
\end{equation}
where $P_n$ is determined by the Boltzmann distribution
\[
P_n = e^{-\beta n \hbar \omega}\left(1  -  e^{-\beta \hbar \omega}\right),
\]
here $\beta = \frac{1}{k_b T}$, and $\left(1  -  e^{-\beta \hbar \omega}\right)$ is the normalization factor.

The average number of photons in a mode for thermal excitation:
\begin{eqnarray}
\bar{n} = \left<\hat{n}\right> =  Sp \left(\hat{\rho}\hat{n}\right) = 
Sp \left(\hat{\rho}\hat{a}^{\dag}\hat{a}\right) = 
\nonumber \\
=\sum_{(m)}\sum_{(n)}
P_n\bra{m}\ket{n}\bra{n}\hat{a}^{\dag}\hat{a}\ket{m}
= 
\nonumber \\
= \sum_{(n)}
P_n\bra{n}\hat{n}\ket{n} = \sum_{(n)} n
e^{-\beta n \hbar \omega}\left(1  -  e^{-\beta \hbar \omega}\right) = 
\frac{1}{e^{\beta \hbar \omega} - 1}
\label{eqCh1_plank}
\end{eqnarray}
This is the well-known expression obtained by Planck. Here, we used the following expression
\[
\sum_{n=0}^{\infty} n r^{n -1} = \frac{d}{d r} \sum_{n=0}^{\infty} r^{n} = \frac{1}{\left(1 - r\right)^2},
\]
valid for $\left|r\right| < 1$.

The equalities \eqref{eqCh1_teplovvozb}, \eqref{eqCh1_plank} can be written in another form. From \eqref{eqCh1_plank} we get 
\[
e^{-\beta \hbar \omega} = \frac{\bar{n}}{1 + \bar{n}},
\]
therefore, 
\begin{equation}
P_n = e^{-\beta \hbar \omega n} \left(1  -  e^{-\beta \hbar \omega}\right) = \frac{\bar{n}^n}{\left(1 + \bar{n}\right)^{n+1}}.
\label{eqCh1_plank2}
\end{equation}
In this case, the density matrix acquires the form:
\[
\hat{\rho} = \sum_{(n)}\frac{\bar{n}^n}{\left(1 + \bar{n}\right)^{n+1}}\ket{n}\bra{n}.
\]
This formula describes the density matrix of chaotic light through $\bar{n}$ - the average number of photons in the mode. $\bar{n}$ does not necessarily depend on the frequency according to Planck’s formula \eqref{eqCh1_plank2}. The dependence of $\bar{n}$ on frequency determines the shape of the chaotic light line (different frequencies correspond to different modes). Generalizing the obtained result, we can write the expression for the statistical operator of chaotic light:  
\begin{equation}
\hat{\rho} = \sum_{\left\{n_k\right\}} P_{\left\{n_k\right\}} \left|\left\{n_k\right\}\right>\left<\left\{n_k\right\}\right| = 
\sum_{\left\{n_k\right\}} 
 \left|\left\{n_k\right\}\right>\left<\left\{n_k\right\}\right|
\prod_{\left\{k\right\}} 
\frac{\bar{n}_k^{n_k}}{\left(1 + \bar{n}_k\right)^{n_k+1}},
\label{eqCh1_102}
\end{equation}
where $\bar{n}_k$ depends on the frequency. For example, for chaotic light with a Lorentzian line, 
\[
\bar{n}_k = \frac{I S}{\hbar \omega_{k_0}}
\frac{\gamma}{\left(\omega_{k_0} - \omega_{k}\right)^2 + \gamma^2},
\]
where $\frac{I S}{\hbar \omega_{k_0}}$ is the average number of photons in the beam; $I$ is the intensity of the beam (energy flow); $S$ is the beam cross-section.  