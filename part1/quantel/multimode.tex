\section{Multimode States}
So far we have considered single-mode fields. The results are easily generalized to a multimode state. For the electromagnetic field in some volume, we have the decomposition \eqref{eqCh1_sep1}:
\begin{equation}
\vec{E}\left(r, t\right) = \sum_{(s)}
\frac{q_s\left(t\right) \omega_s}{\sqrt{\varepsilon_0}} \vec{E}_s\left(r\right),
\quad
\vec{H}\left(r, t\right) = \sum_{(s)}
\frac{p_s\left(t\right)}{\sqrt{\mu_0}} \vec{H}_s\left(r\right),
\end{equation}
The Hamiltonian function is equal to the sum of the Hamiltonian functions of all modes, as the modes correspond to independent oscillators: 
\begin{equation}
\mathcal{H} = \frac{1}{2} 
\sum_{(s)} \left(\omega_s^2 q_s^2 + p_s^2 \right)
\end{equation}
Quantization is reduced to replacing $q_s$ and $p_s$ with operators $\hat{q}_s$
and $\hat{p}_s$ with commutation relations: 
\begin{equation}
\left[\hat{q}_s, \hat{p}_{s'}\right] = i\hbar \delta_{ss'},
\quad 
\left[\hat{q}_s, \hat{q}_{s'}\right] = \left[\hat{p}_s,
  \hat{p}_{s'}\right] = 0.
\end{equation}
The Hamiltonian function thus becomes a Hamiltonian operator:
\begin{equation}
\hat{\mathcal{H}} = \frac{1}{2} \sum_{(s)} \left(\omega_s^2 \hat{q}_s^2 + \hat{p}_s^2 \right)
\end{equation}
As in section \ref{Ch1_quantumemf}, we introduce creation and annihilation operators for each mode \eqref{eqCh1_aadef}: 
\begin{equation}
\hat{a}_s = \frac{1}{\sqrt{2 \hbar \omega_s}}
\left( \omega_s \hat{q}_s + i \hat{p}_s\right),
\quad
\hat{a}_s^{\dag} = \frac{1}{\sqrt{2 \hbar \omega_s}}
\left( \omega_s \hat{q}_s - i \hat{p}_s\right),
\end{equation}
With their help, the Hamiltonian can be represented as
\begin{equation}
\hat{\mathcal{H}_s} = \omega_n \hbar 
\left(\hat{a}_s^{\dag} \hat{a}_s + \frac{1}{2}\right)
\nonumber
\end{equation}
The general state, when there are $n_1$ photons in the first mode, $n_2$ in the second, and $n_s$ photons in the $s$-th mode, etc., can be represented as the product of the state vectors of each mode (see \autoref{sec:add:quantum:composite}):  
\begin{equation}
\left|\left\{n_s\right\}\right> = 
\left|n_1, n_2, \dots, n_s, \dots\right> =
\ket{n_1} \otimes
\ket{n_2} \otimes
\dots
\otimes
\ket{n_s} \otimes
\dots, 
\nonumber
\end{equation}
where $\left\{n_s\right\}$ denotes a set of mode occupation numbers.

The action of the operators $\hat{a}_s^{\dag}$ and $\hat{a}_s$, related to the $s$-th mode, on the state vector is described by the equalities 
\begin{eqnarray}
\hat{a}_s \left| n_1, n_2, \dots, n_s, \dots\right> = \sqrt{n_s} \left|
n_1, n_2, \dots, n_s - 1, \dots\right>,
\nonumber \\
\hat{a}_s^{\dag} \left| n_1, n_2, \dots, n_s, \dots\right> = \sqrt{n_s + 1} \left|
n_1, n_2, \dots, n_s + 1, \dots\right>.
\end{eqnarray} 

The state vector in general can be represented as a linear superposition of states $\left| \left\{n_s\right\}\right>$: 
\begin{equation}
\left|\Psi\right> = \sum_{n_1}\sum_{n_2}\dots\sum_{n_s}\dots
C_{n_1, n_2, \dots, n_s, \dots} \left| n_1, n_2, \dots, n_s,
\dots\right>= 
\sum_{\left\{n_s\right\}} C_{\left\{n_s\right\}} \left| \left\{n_s\right\}\right>
\end{equation}

In the decomposition of the field into plane waves, the electric field operator has the form
\begin{equation}
\hat{\vec{E}} = \sum_{(k)} \sqrt{\frac{\hbar \omega_k}{2 \varepsilon_0
V}} \left( 
\hat{a}_k\left(t\right) \vec{e}_k e^{i \left(\vec{k}\vec{r} \right)} +
\hat{a}_k^{\dag}\left(t\right) \vec{e}_k^{*} e^{-i \left(\vec{k}\vec{r} \right)}
\right)
\label{eqCh1_multiE}
\end{equation}
This expression is written in the Heisenberg representation (the operator depends on time). In this representation 
\[
\hat{a}_k\left(t\right) = \hat{a}_k\left(0\right) e^{-i \omega_k t},
\quad
\hat{a}_k^{\dag}\left(t\right) = \hat{a}_k^{\dag}\left(0\right) e^{i \omega_k t},
\]
where $\hat{a}_k\left(0\right)$, $\hat{a}_k^{\dag}\left(0\right)$ are operators in the Schrödinger representation. In the Schrödinger representation in expression \eqref{eqCh1_multiE}, the operators do not depend on time.
For each mode, we have the equalities \eqref{eqCh1_E_middle}, \eqref{eqCh1_E2_middle}: 
\[
\bra{n_k}\hat{a}_k\ket{n_k} = 
\bra{n_k}\hat{a}^{\dag}_k\ket{n_k} = 0
\]
From here for the complete field we get
\begin{equation}
\left<\left\{n_k\right\}\right|\hat{\vec{E}}\left|\left\{n_k\right\}\right>
= 0, \quad
\left<\left\{n_k\right\}\right|\hat{\vec{E}}^2\left|\left\{n_k\right\}\right>
= \sum_{(k)}\frac{\hbar \omega_k}{\varepsilon_0 V}
\left(n + \frac{1}{2} \right)
\end{equation}

Expression \eqref{eqCh1_multiE} can be split into two terms: a positive frequency part, which includes annihilation operators, and a negative frequency part, which includes creation operators: 
\begin{eqnarray}
\hat{\vec{E}} = \hat{\vec{E}}^{(+)} + \hat{\vec{E}}^{(-)},
\nonumber \\
\hat{\vec{E}}^{(+)} = \sum_{(k)} \sqrt{\frac{\hbar \omega_k}{2 \varepsilon_0
V}} \hat{a}_k\left(t\right) \vec{e}_k e^{i \left(\vec{k}\vec{r}
  \right)}, 
\nonumber \\
\hat{\vec{E}}^{(-)} = \sum_{(k)} \sqrt{\frac{\hbar \omega_k}{2 \varepsilon_0
V}}
\hat{a}_k^{\dag}\left(t\right) \vec{e}_k^{*} e^{-i \left(\vec{k}\vec{r} \right)}
\label{eqCh1_79}
\end{eqnarray} 
$\hat{\vec{E}}^{(+)}$ corresponds to the analytic signal in the classical case. 