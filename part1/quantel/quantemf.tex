%% -*- coding:utf-8 -*- 
\section{Quantization of the Free Electromagnetic Field}
\label{Ch1_quantumemf}
\rindex{quantization!free electromagnetic field}
Using the previously noted analogy, the electromagnetic field can be quantized similarly to a simple mechanical harmonic oscillator.

In quantization, $q_n$ and $p_n$ become operators $\hat{q}_n$,
$\hat{p}_n$, satisfying the same commutation relations as position and momentum:
\begin{equation}
\left[\hat{q}_n, \hat{p}_n\right] = \hat{q}_n\hat{p}_n -
\hat{p}_n \hat{q}_n = i\hbar
\label{eqCh1_comut}
\end{equation}
where $\hat{q}_n$ and $\hat{p}_n$ are Hermitian (self-adjoint) operators. In the Schrödinger picture, they are time-independent, while in the Heisenberg picture, they depend on time. It is convenient to introduce new operators via these operators, which are not self-adjoint but are conjugate to each other:
\begin{equation}
\hat{a}_n = \frac{1}{\sqrt{2 \hbar \omega_n}}
\left( \omega_n \hat{q}_n + i \hat{p}_n\right),
\quad
\hat{a}_n^{\dag} = \frac{1}{\sqrt{2 \hbar \omega_n}}
\left( \omega_n \hat{q}_n - i \hat{p}_n\right),
\label{eqCh1_aadef}
\end{equation}
The inverse relation gives
\begin{equation}
\hat{q}_n = \sqrt{\frac{\hbar}{2 \omega_n}}
\left(\hat{a}_n + \hat{a}_n^{\dag}\right),
\quad
\hat{p}_n = i \sqrt{\frac{\hbar \omega_n}{2}}
\left(\hat{a}_n^{\dag} - \hat{a}_n\right),
\label{eqCh1_qpdef}
\end{equation}

Using expressions \eqref{eqCh1_comut}, 
\eqref{eqCh1_aadef}, \eqref{eqCh1_qpdef}, one can obtain the commutation
relations for the operators $\hat{a}_n$ and $\hat{a}_n^{\dag}$:
\begin{eqnarray}
\left[\hat{a}_n, \hat{a}_n^{\dag}\right] = 
\frac{1}{2 \hbar \omega_n}
\left( \omega_n \hat{q}_n + i \hat{p}_n\right) 
\left( \omega_n \hat{q}_n - i \hat{p}_n\right) - 
\nonumber \\
- \frac{1}{2 \hbar \omega_n}
\left( \omega_n \hat{q}_n - i \hat{p}_n\right) 
\left( \omega_n \hat{q}_n + i \hat{p}_n\right) =
\nonumber \\
= \frac{1}{2 \hbar \omega_n}
\left( - 2 i \omega_n 
\left(\hat{q}_n \hat{p}_n - \hat{p}_n \hat{q}_n\right)\right) = 1.
\end{eqnarray}
Thus,
\begin{equation}
\left[\hat{a}_n, \hat{a}_n^{\dag}\right] = 1.
\label{eqCh1_aacomutation}
\end{equation}
Additionally, it is obvious that
\begin{equation}
\left[\hat{a}_n, \hat{a}_n\right] = 0,
\quad
\left[\hat{a}_n^{\dag}, \hat{a}_n^{\dag}\right] = 0.
\end{equation}
\rindex{Hamiltonian}
The Hamiltonian, expressed through the new operators, takes the form
\begin{eqnarray}
\hat{\mathcal{H}_n} = 
\frac{1}{2}\left(\omega_n^2 \hat{q}_n^2 + \hat{p}_n^2\right) = 
\nonumber \\
= \frac{1}{2}\frac{\omega_n \hbar}{2}
\left(
\left(\hat{a}_n + \hat{a}_n^{\dag} \right)
\left(\hat{a}_n + \hat{a}_n^{\dag} \right)
-
\left(\hat{a}_n^{\dag} - \hat{a}_n \right)
\left(\hat{a}_n^{\dag} - \hat{a}_n \right)
\right) =
\nonumber \\
= \frac{\omega_n \hbar}{2} 
\left( \hat{a}_n \hat{a}_n^{\dag} + \hat{a}_n^{\dag} \hat{a}_n\right) =
\frac{\omega_n \hbar}{2} 
\left(1 + \hat{a}_n^{\dag} \hat{a}_n + \hat{a}_n^{\dag} \hat{a}_n\right) =
\nonumber \\
= \omega_n \hbar 
\left(\hat{a}_n^{\dag} \hat{a}_n + \frac{1}{2}\right)
\label{eqCh1_quant_stoyachie_volny}
\end{eqnarray}

Here, the relation of $\hat{q}_n$, $\hat{p}_n$
with $\hat{a}_n^{\dag}$, $\hat{a}_n$ \eqref{eqCh1_aadef}, as well as the
commutation relation \eqref{eqCh1_aacomutation}, are used.

The energy $\frac{1}{2}\omega_n \hbar$ corresponds to zero
oscillations. The total energy of zero oscillations
\[
\frac{1}{2}\sum_{(n)}\omega_n \hbar \to \infty,
\] 
as the number of modes is infinite. This does not lead to 
significant difficulties since we are only interested in energy differences, and the constant part can be discarded. Then the Hamiltonian
takes the form 
\begin{equation}
\hat{\mathcal{H}} = \sum_{(n)}\hat{\mathcal{H}_n} =
\sum_{(n)}\hbar \omega_n \hat{a}_n^{\dag}\hat{a}_n
\end{equation}

Operators of the electric and magnetic fields can be represented as
\begin{eqnarray}
\hat{\vec{E}} = \sum_{(n)}\frac{\hat{q}_n
  \omega_n}{\sqrt{\varepsilon_0}} \vec{E}_n\left(r\right) = 
\sum_{(n)}\sqrt{\frac{\hbar \omega_n}{2 \varepsilon_0}}
\left(\hat{a}_n^{\dag} + \hat{a}_n \right)
\vec{E}_n\left(r\right),
\nonumber \\
\hat{\vec{H}} = \sum_{(n)}\frac{\hat{p}_n}
{\sqrt{\mu_0}} \vec{H}_n\left(r\right) = 
i \sum_{(n)}\sqrt{\frac{\hbar \omega_n}{2 \mu_0}}
\left(\hat{a}_n^{\dag} - \hat{a}_n \right)
\vec{H}_n\left(r\right).
\end{eqnarray}

Let's provide a simple example. In laser theory, a resonator is considered, formed by two parallel mirrors. In this case, an approximate field representation is often used, assuming that the field depends only on the longitudinal coordinate (\autoref{figCh1_Res}). In this approximation, the normalized eigenfunction (mode) can be presented as follows:
\begin{equation}
E_{nx} = \sqrt{\frac{2}{V}} \sin k_n z,
\quad
H_{ny} = \sqrt{\frac{2}{V}} \cos k_n z,
\quad
k_n = \frac{\pi n}{L},
\quad
\omega_n = \frac{c \pi n}{L},
\end{equation}
where $V = SL$, $L$ - length of the resonator, $S$ - cross-section of the light beam.

\input ./part1/quantel/fig1.tex

The operator of the electric field mode will then have the form:
\begin{eqnarray}
\hat{E}_{nx} = 
\sqrt{\frac{\hbar \omega_n}{2 \varepsilon_0}}
\left(\hat{a}_n^{\dag} + \hat{a}_n \right)
\sqrt{\frac{2}{V}} \sin k_n z = 
E_1 \left(\hat{a}_n^{\dag} + \hat{a}_n \right) \sin k_n z, 
\nonumber \\
\hat{H}_{ny} = i E_1 \sqrt{\frac{\varepsilon_0}{\mu_0}}
\left(\hat{a}_n^{\dag} - \hat{a}_n \right) \cos k_n z,
\label{eqCh1_EH_simple}
\end{eqnarray}
where 
$E_1 = \sqrt{\frac{\hbar \omega}{\varepsilon_0 V}}$ - 
electric field corresponding to one photon (quantum) in
the mode. 