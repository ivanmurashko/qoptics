%% -*- coding:utf-8 -*- 
\section{Representation of the Density Operator through Coherent States}
An important aspect of quantum optics is the description of optical phenomena using coherent states. By using the double completeness condition for states with different numbers of photons, for a given statistical operator, we obtain:
\[
\hat{\rho} = \hat{I}\hat{\rho}\hat{I} = 
\sum_{(n)}\sum_{(m)}
\ket{n}\bra{n}\hat{\rho}\ket{m}\bra{m} = 
\sum_{(n)}\sum_{(m)} \rho_{mn}\ket{n}\bra{m}
\]
Similarly for coherent states
\begin{eqnarray}
\hat{\rho} = \hat{I}\hat{\rho}\hat{I} = 
\frac{1}{\pi^2}\int d^2\alpha\int
\left|\alpha\right>\left<\alpha\right|\hat{\rho}\left|\beta\right>\left<\beta\right|
d^2 \beta = 
\nonumber \\
= \frac{1}{\pi^2} \int d^2\alpha \int
R\left(\alpha^{*}, \beta\right) e^{-\frac{1}{2}\left|\alpha\right|^2 -\frac{1}{2}\left|\beta\right|^2}
\left|\alpha\right>\left<\beta\right|d^2 \beta ,
\end{eqnarray}
where
\[
R\left(\alpha^{*}, \beta\right) =
\left<\alpha\right|\hat{\rho}\left|\beta\right>
e^{\frac{1}{2}\left|\alpha\right|^2 +\frac{1}{2}\left|\beta\right|^2}
\]
Expressing $\left|\alpha\right>$ and $\left|\beta\right>$ through states $\ket{n}$, we obtain the formula
\begin{equation}
R\left(\alpha^{*}, \beta\right) = \sum_{(n)}\sum_{(m)}
\frac{\left(\alpha^{*}\right)^n \left(\beta\right)^m}{\sqrt{n!m!}}\rho_{nm}
\end{equation}
i.e., $R\left(\alpha^{*}, \beta\right)$ can be easily found if the density matrix
\rindex{Density Matrix}
is known in the Fock state (photon number) representation. For example, consider thermal excitation of a mode:
\begin{eqnarray}
R\left(\alpha^{*}, \beta\right) = \sum_{(n)}\sum_{(m)}
\frac{\left(\alpha^{*}\right)^n \left(\beta\right)^m}{\sqrt{n!m!}}
e^{-\beta n \hbar \omega}\left(1  -  e^{-\beta \hbar \omega}\right)
\delta_{nm} = 
\nonumber \\
= \sum_{(n)}\frac{\left(\alpha^{*} \beta\right)^n }{n!}
e^{-\beta n \hbar \omega}\left(1  -  e^{-\beta \hbar \omega}\right) =
\left(1  -  e^{-\beta \hbar \omega}\right) e^{\alpha^{*}\beta
  e^{-\beta \hbar \omega}}.
\end{eqnarray}

This representation depends on two parameters $\alpha^{*}$ and $\beta$. A more convenient representation of the density matrix, introduced by R. Glauber \cite{bQuantumOpticsAndRadioPhisicsLecture1966}, called the diagonal representation, depends on one parameter. It has the form 
\begin{equation}
\hat{\rho} = \int
P\left(\alpha\right)\left|\alpha\right>\left<\alpha\right| d^2 \alpha,
\label{eqCh1_rhorepresent}
\end{equation}
where $P\left(\alpha\right)$ is a real function, satisfying the condition 
\[
\int
P\left(\alpha\right) d^2 \alpha = 1.
\]
Furthermore, $P\left(\alpha\right)$ is a real function of a complex argument. This follows from the conditions $Sp\left(\hat{\rho}\right) = 1$ and $\hat{\rho} = \hat{\rho}^{\dag}$.
 
Such decomposition is possible due to the overcompleteness of the coherent state system. If $P\left(\alpha\right)$ is a positive function, it can be interpreted as a probability distribution. This is applicable to some of the most practically interesting field states, such as a completely chaotic state, but it is not true in general. Sometimes $P\left(\alpha\right)$ may be negative in some limited range of values of $\alpha$, in which case it cannot be interpreted as a probability distribution. $P\left(\alpha\right)$ can also be a generalized function (an example is a $\delta$-function). 

A rigorous justification of the possibility of representation \eqref{eqCh1_rhorepresent} is contained in the literature \cite{bQuantumOpticsAndRadioPhisicsLecture1966}, \cite{bKaluderSudershan1970}. The possibility of introducing the diagonal representation \eqref{eqCh1_rhorepresent} can be justified as follows. Assume the statistical operator $\hat{\rho}$, like any other operator acting on the electromagnetic field, can be represented as a function of creation and annihilation operators:
\[
\hat{\rho} = \bar{f}\left(\hat{a}^{\dag}, \hat{a}\right).
\]
The operator can then be represented in ordered form: normal and antinormal. In the first case, creation operators are placed to the right of the annihilation operators, and in the second case, vice versa - to the left. For example,
\[
\left(\hat{a}^{\dag}\right)^m\left(\hat{a}\right)^n
\]
is a normally ordered operator. In an antinormally ordered operator, the order is reversed: creation operators are on the right of the annihilation operator. For example, the operator 
\[
\left(\hat{a}\right)^n\left(\hat{a}^{\dag}\right)^m
\]
is antinormally ordered. Ordering can be achieved by repeatedly applying the commutation relation. For example, the operator 
\[
\left(\hat{a}^{\dag}\hat{a}\right)^2 =
\hat{a}^{\dag}\hat{a}\hat{a}^{\dag}\hat{a} 
\]
is neither normal nor antinormal. Using the condition 
$\left[\hat{a},\hat{a}^{\dag}\right] = 1$,
we bring it to normal form: 
\[
\left(\hat{a}^{\dag}\hat{a}\right)^2 = \hat{a}^{\dag}\left(1 +
\hat{a}^{\dag}\hat{a}\right)\hat{a} = 
\hat{a}^{\dag}\hat{a} + \left(\hat{a}^{\dag}\right)^2\left(\hat{a}\right)^2
\]
this operator can be presented in antinormal form:
\begin{eqnarray}
\left(\hat{a}^{\dag}\hat{a}\right)^2 = 
\left(\hat{a}\hat{a}^{\dag} - 1\right) \left(\hat{a}\hat{a}^{\dag} -
1\right) = \hat{a}\hat{a}^{\dag} \hat{a}\hat{a}^{\dag} - 2
\hat{a}\hat{a}^{\dag} + 1 =
\nonumber \\
= \hat{a}\left(\hat{a}\hat{a}^{\dag} - 1\right)\hat{a}^{\dag} -
2\hat{a}\hat{a}^{\dag} + 1 = 
\left(\hat{a}\right)^2\left(\hat{a}^{\dag}\right)^2 - 3
\hat{a}\hat{a}^{\dag} + 1.
\nonumber
\end{eqnarray}

There are other, more efficient methods for ordering operators \cite{bLuisell1972}.

Let's represent the density operator in antinormal form (considering a single-mode case): 
\begin{equation}
\hat{\rho}^{\left(a\right)}\left(\hat{a},\hat{a}^{\dag}\right) = 
\sum_{(n)}\sum_{(m)}C^{\left(a\right)}_{nm}\left(\hat{a}\right)^n\left(\hat{a}^{\dag}\right)^m
\end{equation}
Now use the decomposition \eqref{eqCh1_full4coh} for the identity operator $\hat{I}$
\[
\hat{I} = \frac{1}{\pi}\int \left|\alpha\right>\left<\alpha\right| d^2 \alpha
\]
as a result, we obtain
\begin{eqnarray}
\hat{\rho}^{\left(a\right)}\left(\hat{a},\hat{a}^{\dag}\right) = 
\sum_{(n)}\sum_{(m)}C^{\left(a\right)}_{nm}\left(\hat{a}\right)^n\hat{I}\left(\hat{a}^{\dag}\right)^m
= 
\nonumber \\
= \frac{1}{\pi}\sum_{(n)}\sum_{(m)}C^{\left(a\right)}_{nm}\int d^2 \alpha
\left(\hat{a}\right)^n
\left|\alpha\right>\left<\alpha\right|
\left(\hat{a}^{\dag}\right)^m = 
\nonumber \\
= \frac{1}{\pi}\sum_{(n)}\sum_{(m)}\int d^2 \alpha
C^{\left(a\right)}_{nm}
\alpha^n
\alpha^{*m}
\left|\alpha\right>\left<\alpha\right|.
\end{eqnarray}
Now denoting 
\[
P\left(\alpha, \alpha^{*}\right) = \frac{1}{\pi}\sum_{(n)}\sum_{(m)}
C^{\left(a\right)}_{nm}
\alpha^n
\alpha^{*m}
\]
we find that $\hat{\rho}$ can be represented as
\begin{equation}
\hat{\rho} = \int d^2 \alpha P\left(\alpha, \alpha^{*}\right) 
\left|\alpha\right>\left<\alpha\right|.
\label{eqCh1_Rho_in_alpha}
\end{equation}

The average value of an operator using the diagonal representation 
\begin{eqnarray}
\left<\hat{O}\right> = Sp \left(\hat{\rho}\hat{O}\right) =
\nonumber \\
= \sum_{(n)}\bra{n}
\int d^2 \alpha P\left(\alpha, \alpha^{*}\right) 
\left|\alpha\right>\left<\alpha\right|
\hat{O}\ket{n} =
\nonumber \\
= \sum_{(n)} \int d^2 \alpha P\left(\alpha, \alpha^{*}\right)
\left<\alpha\right|\hat{O}\ket{n}
\bra{n}\left.\alpha\right> = 
\nonumber \\
=  
\int d^2 \alpha P\left(\alpha, \alpha^{*}\right)
\left<\alpha\right|\hat{O}\left|\alpha\right>,
\label{eqCh1_middleO}
\end{eqnarray}
since $\sum_{(n)}\ket{n}\bra{n} = \hat{I}$. 

The expression $\left<\alpha\right|\hat{O}\left|\alpha\right>$ is easily calculated if operator $\hat{O}$ is presented in normal form:
\begin{equation}
\hat{O} = \hat{O}^{(n)} = \sum_{(n)}\sum_{(m)} d_{nm}
\left(\hat{a}^{\dag}\right)^n
\left(\hat{a}\right)^m,
\label{eqCh1_normalO}
\end{equation}
then
\begin{equation}
\left<\alpha\right|\hat{O}\left|\alpha\right> = 
\sum_{(n)}\sum_{(m)} d_{nm}
\left<\alpha\right|
\left(\hat{a}^{\dag}\right)^n
\left(\hat{a}\right)^m
\left|\alpha\right> = 
\sum_{(n)}\sum_{(m)} d_{nm}
\alpha^{*n}\alpha^{m}.
\end{equation}
From here follows a simple rule: to obtain the necessary matrix element, we replace operators in normal ordered form by $\hat{a}\rightarrow\alpha$, 
$\hat{a}^{\dag}\rightarrow\alpha^{*}$.

As an example, we show that for thermal excitation of a mode 
\begin{equation}
P\left(\alpha, \alpha^{*}\right) = \frac{1}{\pi \bar{n}}
e^{- \frac{\left|\alpha\right|^2}{\bar{n}}}
\label{eqCh1_123a}
\end{equation}
and, therefore
\[
\hat{\rho} = \frac{1}{\pi \bar{n}}
\int d^2 \alpha 
e^{- \frac{\left|\alpha\right|^2}{\bar{n}}}
\left|\alpha\right>
\left<\alpha\right|.
\]
Now find matrix elements of operator $\hat{\rho}$ in the Fock state (photon number) representation
\begin{eqnarray}
\bra{n}
\hat{\rho}
\ket{n} = 
\frac{1}{\pi \bar{n}}
\int d^2 \alpha 
e^{- \frac{\left|\alpha\right|^2}{\bar{n}}}
\bra{n}\left.\alpha\right>
\left<\alpha\right|
\ket{n} = 
\nonumber \\
= 
\frac{1}{\pi \bar{n}}
\int d^2 \alpha 
e^{- \frac{\left|\alpha\right|^2}{\bar{n}}}
\frac{\left|\alpha\right|^{2n}}{n!} 
e^{- \left|\alpha\right|^2} 
\nonumber 
\end{eqnarray}
The resulting integral can be computed using polar coordinates:
\[
\alpha = r e^{i \varphi}, \quad 
d^2 \alpha = r d r d \varphi , \quad 
\mbox{ i.e. } 
\left|\alpha\right| = r, \quad arg\,\alpha = \varphi,
\]
from where
\begin{eqnarray}
\frac{1}{\pi \bar{n}}
\int d^2 \alpha 
e^{- \frac{\left|\alpha\right|^2}{\bar{n}}}
\frac{\left|\alpha\right|^{2n}}{n!} 
e^{- \left|\alpha\right|^2} =
\nonumber \\
=
\frac{1}{\pi \bar{n}}
\int_0^{2 \pi} d \varphi 
\int_0^{\infty}
r dr \frac{exp \left(- r^2\frac{\bar{n} + 1}{\bar{n}}\right)}{n!} r^{2n}= 
\nonumber \\
= 
\frac{1}{\bar{n}}
\int_0^{\infty}
2 r dr \frac{exp \left(- r^2\frac{\bar{n} + 1}{\bar{n}}\right)}{n!}  r^{2n} = 
\nonumber \\
= 
\frac{1}{\bar{n}}
\int_0^{\infty}
\frac{e^{-x}}{n!}\frac{\bar{n}^{n + 1}}
{\left(\bar{n} + 1\right)^{n + 1}}x^n dx = 
\nonumber \\
=
\frac{\bar{n}^{n}}
{\left(\bar{n} + 1\right)^{n + 1}}
\frac{1}{n!}
\int_0^{\infty}
e^{-x}x^n dx = 
\frac{\bar{n}^{n}}
{\left(\bar{n} + 1\right)^{n + 1}},
\label{eqCh1_matrelemRho}
\end{eqnarray}
where a change of variable $x = r^2\frac{\bar{n} + 1}{\bar{n}}$ is made.

Thus, we have obtained the well-known matrix element of the statistical operator of a chaotic field in the Fock state representation. Hence, the expression \eqref{eqCh1_123a} is indeed the statistical operator of a chaotic field in the coherent state representation. Therefore, for thermal photon excitation in a mode we have
\begin{equation}
P\left(\alpha\right) = \frac{1}{\pi \bar{n}}
e^{-\frac{\left|\alpha\right|^2}{\bar{n}}},
\mbox{ where }
\bar{n} = \frac{1}{e^{\hbar \omega \beta} - 1}
\label{eqCh1_task4}
\end{equation}
  
Generalizing the obtained expressions to the multimode case is straightforward: 
\begin{equation}
\left<\hat{O}\right> = \idotsint\limits_{\left\{\alpha_k\right\}}
P^{(a)}
\left(
\left\{\alpha_k\right\}
\right)
O^{(n)}
\left(
\left\{\alpha_k\right\}
\right)
\prod_k d^2 \alpha_k,
\label{eqCh1_113}
\end{equation}
\[
P^{(a)}
\left(
\left\{\alpha_k\right\}
\right)
=
\prod_k \frac{1}{\pi \bar{n}_k}
e^{-\frac{\left|\alpha_k\right|^2}{\bar{n}_k}}. 
\]