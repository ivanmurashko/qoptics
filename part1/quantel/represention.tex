%% -*- coding:utf-8 -*- 
\section{Representation of the Density Operator through Coherent States}
An important aspect of quantum optics is the description of optical phenomena using coherent states. By utilizing the completeness condition twice for states with different numbers of photons, for an arbitrary statistical operator we get: 
\[
\hat{\rho} = \hat{I}\hat{\rho}\hat{I} = 
\sum_{(n)}\sum_{(m)}
\ket{n}\bra{n}\hat{\rho}\ket{m}\bra{m} = 
\sum_{(n)}\sum_{(m)} \rho_{mn}\ket{n}\bra{m}
\]
Similarly for coherent states
\begin{eqnarray}
\hat{\rho} = \hat{I}\hat{\rho}\hat{I} = 
\frac{1}{\pi^2}\int d^2\alpha\int
\left|\alpha\right>\left<\alpha\right|\hat{\rho}\left|\beta\right>\left<\beta\right|
d^2 \beta = 
\nonumber \\
= \frac{1}{\pi^2} \int d^2\alpha \int
R\left(\alpha^{*}, \beta\right) e^{-\frac{1}{2}\left|\alpha\right|^2 -\frac{1}{2}\left|\beta\right|^2}
\left|\alpha\right>\left<\beta\right|d^2 \beta ,
\end{eqnarray}
where
\[
R\left(\alpha^{*}, \beta\right) =
\left<\alpha\right|\hat{\rho}\left|\beta\right>
e^{\frac{1}{2}\left|\alpha\right|^2 +\frac{1}{2}\left|\beta\right|^2}
\]
If we express $\left|\alpha\right>$ and $\left|\beta\right>$ through states $\ket{n}$, we obtain the formula 
\begin{equation}
R\left(\alpha^{*}, \beta\right) = \sum_{(n)}\sum_{(m)}
\frac{\left(\alpha^{*}\right)^n \left(\beta\right)^m}{\sqrt{n!m!}}\rho_{nm}
\end{equation}
i.e., $R\left(\alpha^{*}, \beta\right)$ is easily found if the density matrix is known 
\index{Density Matrix}
in the occupation number representation (photon numbers). As an example, consider the thermal excitation of a mode: 
\begin{eqnarray}
R\left(\alpha^{*}, \beta\right) = \sum_{(n)}\sum_{(m)}
\frac{\left(\alpha^{*}\right)^n \left(\beta\right)^m}{\sqrt{n!m!}}
e^{-\beta n \hbar \omega}\left(1  -  e^{-\beta \hbar \omega}\right)
\delta_{nm} = 
\nonumber \\
= \sum_{(n)}\frac{\left(\alpha^{*} \beta\right)^n }{n!}
e^{-\beta n \hbar \omega}\left(1  -  e^{-\beta \hbar \omega}\right) =
\left(1  -  e^{-\beta \hbar \omega}\right) e^{\alpha^{*}\beta
  e^{-\beta \hbar \omega}}.
\end{eqnarray}

This representation depends on two parameters $\alpha^{*}$ and $\beta$.
A more convenient representation of the density matrix, introduced by R. Glauber \cite{bQuantumOpticsAndRadioPhisicsLecture1966}, called the diagonal representation and depending on one parameter, is of the form  
\begin{equation}
\hat{\rho} = \int
P\left(\alpha\right)\left|\alpha\right>\left<\alpha\right| d^2 \alpha,
\label{eqCh1_rhorepresent}
\end{equation}
where $P\left(\alpha\right)$ is a real function that satisfies the condition 
\[
\int
P\left(\alpha\right) d^2 \alpha = 1.
\]
Moreover, $P\left(\alpha\right)$ is a real function of a complex argument. This all follows from the conditions $Sp\left(\hat{\rho}\right) = 1$ and $\hat{\rho} = \hat{\rho}^{\dag}$.
  
Such decomposition is possible due to the overcompleteness of the coherent state system. If $P\left(\alpha\right)$ is a positive function, it can be interpreted as a probability distribution. This applies to some of the most practically interesting field states, such as the completely chaotic state, but this is not true in the general case. Sometimes $P\left(\alpha\right)$ may be negative over a limited range of $\alpha$, then it cannot be interpreted as a probability distribution. $P\left(\alpha\right)$ may also be a generalized function (an example being the $\delta$-function). 

A rigorous justification for the possibility of the representation \eqref{eqCh1_rhorepresent} is contained in the literature \cite{bQuantumOpticsAndRadioPhisicsLecture1966}, \cite{bKaluderSudershan1970}. 
The possibility of introducing the diagonal representation \eqref{eqCh1_rhorepresent} can be justified as follows. Suppose the statistical operator $\hat{\rho}$, like any other operator acting on the electromagnetic field, can be represented as a function of creation and annihilation operators:
\[
\hat{\rho} = \bar{f}\left(\hat{a}^{\dag}, \hat{a}\right).
\]
Then the operator can be represented in ordered form: normal and antinormal. In the first case, creation operators are positioned to the right of annihilation operators, and in the second case, the opposite - on the left.
For example,
\[
\left(\hat{a}^{\dag}\right)^m\left(\hat{a}\right)^n
\]
is a normally ordered operator. In the antinormally ordered operator, the order is reversed: creation operators are on the right of the annihilation operator. For example, the operator 
\[
\left(\hat{a}\right)^n\left(\hat{a}^{\dag}\right)^m
\]
is antinormally ordered. Ordering can be achieved by multiple applications of the commutation relation. For example, the operator  
\[
\left(\hat{a}^{\dag}\hat{a}\right)^2 =
\hat{a}^{\dag}\hat{a}\hat{a}^{\dag}\hat{a} 
\]
is neither normal nor antinormal. Using the condition 
$\left[\hat{a},\hat{a}^{\dag}\right] = 1$,
bring it to normal form: 
\[
\left(\hat{a}^{\dag}\hat{a}\right)^2 = \hat{a}^{\dag}\left(1 +
\hat{a}^{\dag}\hat{a}\right)\hat{a} = 
\hat{a}^{\dag}\hat{a} + \left(\hat{a}^{\dag}\right)^2\left(\hat{a}\right)^2
\]
and this operator can be presented in antinormal form:
\begin{eqnarray}
\left(\hat{a}^{\dag}\hat{a}\right)^2 = 
\left(\hat{a}\hat{a}^{\dag} - 1\right) \left(\hat{a}\hat{a}^{\dag} -
1\right) = \hat{a}\hat{a}^{\dag} \hat{a}\hat{a}^{\dag} - 2
\hat{a}\hat{a}^{\dag} + 1 =
\nonumber \\
= \hat{a}\left(\hat{a}\hat{a}^{\dag} - 1\right)\hat{a}^{\dag} -
2\hat{a}\hat{a}^{\dag} + 1 = 
\left(\hat{a}\right)^2\left(\hat{a}^{\dag}\right)^2 - 3
\hat{a}\hat{a}^{\dag} + 1.
\nonumber
\end{eqnarray}

There are also other, more efficient methods of operator ordering \cite{bLuisell1972}.

Let's represent the density operator in antinormal form (considering the single-mode case): 
\begin{equation}
\hat{\rho}^{\left(a\right)}\left(\hat{a},\hat{a}^{\dag}\right) = 
\sum_{(n)}\sum_{(m)}C^{\left(a\right)}_{nm}\left(\hat{a}\right)^n\left(\hat{a}^{\dag}\right)^m
\end{equation}
Now use the expansion \eqref{eqCh1_full4coh} for the unit operator $\hat{I}$
\[
\hat{I} = \frac{1}{\pi}\int \left|\alpha\right>\left<\alpha\right| d^2 \alpha
\]
as a result, we get
\begin{eqnarray}
\hat{\rho}^{\left(a\right)}\left(\hat{a},\hat{a}^{\dag}\right) = 
\sum_{(n)}\sum_{(m)}C^{\left(a\right)}_{nm}\left(\hat{a}\right)^n\hat{I}\left(\hat{a}^{\dag}\right)^m
= 
\nonumber \\
= \frac{1}{\pi}\sum_{(n)}\sum_{(m)}C^{\left(a\right)}_{nm}\int d^2 \alpha
\left(\hat{a}\right)^n
\left|\alpha\right>\left<\alpha\right|
\left(\hat{a}^{\dag}\right)^m = 
\nonumber \\
= \frac{1}{\pi}\sum_{(n)}\sum_{(m)}\int d^2 \alpha
C^{\left(a\right)}_{nm}
\alpha^n
\alpha^{*m}
\left|\alpha\right>\left<\alpha\right|.
\end{eqnarray}
Now denote 
\[
P\left(\alpha, \alpha^{*}\right) = \frac{1}{\pi}\sum_{(n)}\sum_{(m)}
C^{\left(a\right)}_{nm}
\alpha^n
\alpha^{*m}
\]
and obtain that $\hat{\rho}$ can be represented in the form
\begin{equation}
\hat{\rho} = \int d^2 \alpha P\left(\alpha, \alpha^{*}\right) 
\left|\alpha\right>\left<\alpha\right|.
\label{eqCh1_Rho_in_alpha}
\end{equation}

The average value of the operator using the diagonal representation 
\begin{eqnarray}
\left<\hat{O}\right> = Sp \left(\hat{\rho}\hat{O}\right) =
\nonumber \\
= \sum_{(n)}\bra{n}
\int d^2 \alpha P\left(\alpha, \alpha^{*}\right) 
\left|\alpha\right>\left<\alpha\right|
\hat{O}\ket{n} =
\nonumber \\
= \sum_{(n)} \int d^2 \alpha P\left(\alpha, \alpha^{*}\right)
\left<\alpha\right|\hat{O}\ket{n}
\bra{n}\left.\alpha\right> = 
\nonumber \\
=  
\int d^2 \alpha P\left(\alpha, \alpha^{*}\right)
\left<\alpha\right|\hat{O}\left|\alpha\right>,
\label{eqCh1_middleO}
\end{eqnarray}
since $\sum_{(n)}\ket{n}\bra{n} = \hat{I}$. 

The expression $\left<\alpha\right|\hat{O}\left|\alpha\right>$ is easily calculated if the operator $\hat{O}$ is represented in normal form:
\begin{equation}
\hat{O} = \hat{O}^{(n)} = \sum_{(n)}\sum_{(m)} d_{nm}
\left(\hat{a}^{\dag}\right)^n
\left(\hat{a}\right)^m,
\label{eqCh1_normalO}
\end{equation}
then
\begin{equation}
\left<\alpha\right|\hat{O}\left|\alpha\right> = 
\sum_{(n)}\sum_{(m)} d_{nm}
\left<\alpha\right|
\left(\hat{a}^{\dag}\right)^n
\left(\hat{a}\right)^m
\left|\alpha\right> = 
\sum_{(n)}\sum_{(m)} d_{nm}
\alpha^{*n}\alpha^{m}.
\end{equation}
From here, we derive a simple rule: to obtain the required matrix element, in the operator represented in normally ordered form, make the replacement $\hat{a}\rightarrow\alpha$, 
$\hat{a}^{\dag}\rightarrow\alpha^{*}$.

As an example, we show that for the thermal excitation of a mode 
\begin{equation}
P\left(\alpha, \alpha^{*}\right) = \frac{1}{\pi \bar{n}}
e^{- \frac{\left|\alpha\right|^2}{\bar{n}}}
\label{eqCh1_123a}
\end{equation}
and, therefore
\[
\hat{\rho} = \frac{1}{\pi \bar{n}}
\int d^2 \alpha 
e^{- \frac{\left|\alpha\right|^2}{\bar{n}}}
\left|\alpha\right>
\left<\alpha\right|.
\]
Now find the matrix elements of the operator $\hat{\rho}$ in the occupation number representation (photon numbers)
\begin{eqnarray}
\bra{n}
\hat{\rho}
\ket{n} = 
\frac{1}{\pi \bar{n}}
\int d^2 \alpha 
e^{- \frac{\left|\alpha\right|^2}{\bar{n}}}
\bra{n}\left.\alpha\right>
\left<\alpha\right|
\ket{n} = 
\nonumber \\
= 
\frac{1}{\pi \bar{n}}
\int d^2 \alpha 
e^{- \frac{\left|\alpha\right|^2}{\bar{n}}}
\frac{\left|\alpha\right|^{2n}}{n!} 
e^{- \left|\alpha\right|^2} 
\nonumber 
\end{eqnarray}
The resulting integral can be computed by writing in polar coordinate system:
\[
\alpha = r e^{i \varphi}, \quad 
d^2 \alpha = r d r d \varphi , \quad 
\mbox{i.e.} 
\left|\alpha\right| = r, \quad arg\,\alpha = \varphi,
\]
from where
\begin{eqnarray}
\frac{1}{\pi \bar{n}}
\int d^2 \alpha 
e^{- \frac{\left|\alpha\right|^2}{\bar{n}}}
\frac{\left|\alpha\right|^{2n}}{n!} 
e^{- \left|\alpha\right|^2} =
\nonumber \\
=
\frac{1}{\pi \bar{n}}
\int_0^{2 \pi} d \varphi 
\int_0^{\infty}
r dr \frac{exp \left(- r^2\frac{\bar{n} + 1}{\bar{n}}\right)}{n!} r^{2n}= 
\nonumber \\
= 
\frac{1}{\bar{n}}
\int_0^{\infty}
2 r dr \frac{exp \left(- r^2\frac{\bar{n} + 1}{\bar{n}}\right)}{n!}  r^{2n} = 
\nonumber \\
= 
\frac{1}{\bar{n}}
\int_0^{\infty}
\frac{e^{-x}}{n!}\frac{\bar{n}^{n + 1}}
{\left(\bar{n} + 1\right)^{n + 1}}x^n dx = 
\nonumber \\
=
\frac{\bar{n}^{n}}
{\left(\bar{n} + 1\right)^{n + 1}}
\frac{1}{n!}
\int_0^{\infty}
e^{-x}x^n dx = 
\frac{\bar{n}^{n}}
{\left(\bar{n} + 1\right)^{n + 1}},
\label{eqCh1_matrelemRho}
\end{eqnarray}
where the substitution $x = r^2\frac{\bar{n} + 1}{\bar{n}}$ was made.

Thus, we obtained the well-known matrix element of the statistical operator of the chaotic field in the occupation number representation. From here, it follows that expression \eqref{eqCh1_123a} is indeed the statistical operator of the chaotic field in the representation of coherent states. Therefore, for the thermal excitation of photons in the mode we have
\begin{equation}
P\left(\alpha\right) = \frac{1}{\pi \bar{n}}
e^{-\frac{\left|\alpha\right|^2}{\bar{n}}},
\mbox{where }
\bar{n} = \frac{1}{e^{\hbar \omega \beta} - 1}
\label{eqCh1_task4}
\end{equation}
  
The generalization of the obtained expressions to the multimode case is obvious: 
\begin{equation}
\left<\hat{O}\right> = \idotsint\limits_{\left\{\alpha_k\right\}}
P^{(a)}
\left(
\left\{\alpha_k\right\}
\right)
O^{(n)}
\left(
\left\{\alpha_k\right\}
\right)
\prod_k d^2 \alpha_k,
\label{eqCh1_113}
\end{equation}
\[
P^{(a)}
\left(
\left\{\alpha_k\right\}
\right)
=
\prod_k \frac{1}{\pi \bar{n}_k}
e^{-\frac{\left|\alpha_k\right|^2}{\bar{n}_k}}. 
\]