%% -*- coding:utf-8 -*- 
\section{Quantization of the Electromagnetic Field by Decomposing it into Plane Waves}
\rindex{quantization!electromagnetic field by decomposing it into plane waves}
For the quantization of the electromagnetic field in this case, we note the analogy between \eqref{eqCh1_separation4hamilton} and \eqref{eqCh1_quant_stoyachie_volny}:
\[
\hat{\mathcal{H}_n} = 
\frac{\omega_n \hbar}{2}
\left(\hat{a}_n \hat{a}_n^{\dag} + \hat{a}_n^{\dag} \hat{a}_n\right)
=
\omega_n \hbar 
\left(\hat{a}_n^{\dag} \hat{a}_n + \frac{1}{2}\right)
\]
and
\[
\mathcal{H} = \varepsilon_0 \sum_{(k)} 
\left(A_k A_k^{*} + A_k^{*} A_k \right).
\]

From this analogy, it follows that the following quantization procedure can be used:
\[
\sqrt{\varepsilon_0}A_k \rightarrow \sqrt{\frac{\omega_k \hbar}{2}}
\hat{a}_k, \quad
\sqrt{\varepsilon_0}A_k^{*} \rightarrow \sqrt{\frac{\omega_k \hbar}{2}}
\hat{a}_k^{\dag}.
\]
\rindex{Hamiltonian}
This substitution leads to the following expression for the Hamiltonian:
\[
\hat{\mathcal{H}_k} = \frac{\omega_k \hbar}{2} 
\left(\hat{a}_k \hat{a}_k^{\dag} + \hat{a}_k^{\dag} \hat{a}_k\right).
\]

Using the commutation relations
\[
\left[\hat{a}, \hat{a}^{\dag} \right] = 1, \quad
\hat{a} \hat{a}^{\dag} - \hat{a}^{\dag}\hat{a} = 1, \quad
\hat{a} \hat{a}^{\dag} = \hat{a}^{\dag}\hat{a} + 1,
\]
we have
\(
\hat{\mathcal{H}_k} = \omega_k \hbar 
\left(\hat{a}_k^{\dag} \hat{a}_k + \frac{1}{2}\right)
\) - 
an expression completely matching \eqref{eqCh1_quant_stoyachie_volny}.
The total Hamiltonian is obtained by summing over all modes:
\begin{equation}
\hat{\mathcal{H}} = \sum_{(k)} \hat{\mathcal{H}}_k = \sum_{(k)} 
\omega_k \hbar \left(\hat{a}_k^{\dag} \hat{a}_k + \frac{1}{2}\right).
\end{equation}

The electromagnetic field momentum can also be represented in Hamiltonian form. The classical momentum of an electromagnetic field in a volume $V$ is defined by the formula
\begin{equation}
\vec{G} = \frac{1}{c^2} \int_{(\nu)}
\left[\vec{E} \vec{H} \right] d \nu.
\label{eqCh1_task3_1}
\end{equation}
Using the decomposition of the field into plane waves and considering orthogonality relations, we obtain
\begin{eqnarray}
  \vec{G} = \frac{1}{c^2} \int_{(\nu)}
  \left[\vec{E} \vec{H} \right] d \nu =
  \nonumber \\
  =
  \frac{1}{c^2} \int_{(\nu)}
  \left[
    \left(\sum_{(k)} A_k \vec{E}_k +
    \sum_{(k)} A_k^\ast \vec{E}_k^\ast \right)
    \left(\sum_{(k')} A_{k'} \vec{H}_{k'} +
    \sum_{(k')} A_{k'}^\ast \vec{H}_{k'}^\ast \right)\right]
  d \nu =
  \nonumber \\
  =
  \sum_{(k)}
  \frac{1}{c^2} \int_{(\nu)}
  \left(
  A_k A_k^\ast \left[\vec{E}_k \vec{H}_k^\ast \right]
  +
  A_k^\ast A_k \left[\vec{E}_k^\ast \vec{H}_k \right]
  \right)
  d \nu =
  \nonumber \\
  =
  \frac{1}{c^2} \sum_{(k)} \frac{\vec{k}}{k}
  \sqrt{\frac{\varepsilon_0}{\mu_0}}
  \left(
  A_k A_k^\ast + A_k^\ast A_k
  \right).
  \nonumber
\end{eqnarray}
Transitioning to operators, we have
\begin{eqnarray}
  \hat{\vec{G}} =
  \frac{1}{c^2} \sum_{(k)} \frac{\vec{k}}{k}
  \sqrt{\frac{\varepsilon_0}{\mu_0}}
  \frac{\hbar \omega_k}{2 \varepsilon_0}
  \left(\hat{a}_k^{\dag} \hat{a}_k + \hat{a}_k \hat{a}_k^{\dag}\right)
  =
  \nonumber \\
  =
  \sum_{(k)}
  \hbar
  \vec{k}
  \frac{\omega_k}{c k}
  \frac{1}{c \sqrt{\varepsilon_0 \mu_0}}
  \frac{\hat{a}_k^{\dag} \hat{a}_k + \hat{a}_k \hat{a}_k^{\dag}}{2}
  =
  \sum_{(k)} \vec{k} \hbar\left( \hat{a}_k^{\dag} \hat{a}_k +
\frac{1}{2} \right).
\nonumber
\end{eqnarray}
From symmetry, it follows that $\sum_{(k)} \vec{k} \hbar = 0$ and, therefore,
\begin{equation}
\hat{\vec{G}} = \sum_{(k)} \vec{k} \hbar\hat{a}_k^{\dag} \hat{a}_k.
\label{eqCh1_task3_2}
\end{equation}

The operator of the electric field is now expressed as follows:
\begin{equation}
\hat{\vec{E}} = \sum_{(k)} \hat{a}_k\sqrt{\frac{\hbar \omega_k}{2 \nu
    \varepsilon_0}} \vec{e}_k e^{i\left(\vec{k}\vec{r}\right)} +
\sum_{(k)} \hat{a}_k^{\dag}\sqrt{\frac{\hbar \omega_k}{2 \nu
    \varepsilon_0}} \vec{e}_k^{*} e^{-i\left(\vec{k}\vec{r}\right)}.
\end{equation}

In the future, we will use the Dirac formulation of quantum mechanics equations, so in \autoref{AddDirac} a brief exposition of the Dirac formalism is given: the concept of state vector is introduced and ways of operating with it are outlined.  