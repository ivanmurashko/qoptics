\section{Quantum state of the electromagnetic field with a definite energy (with a definite number of photons)}

First, let's consider a single mode of the electromagnetic field (a simple harmonic oscillator). The state vector with a definite energy $\ket{E_n}$ satisfies the following equation
\begin{equation}
\hat{\mathcal{H}} \ket{E_n} = E_n \ket{E_n}
\end{equation}
Here and henceforth, we will use Dirac's formalism (see \autoref{AddDirac}). Using relation \eqref{eqCh1_mainpropertyaa}, we obtain
\begin{equation}
\hat{\mathcal{H}} \hat{a}\ket{E_n} = 
\left(\hat{a}\hat{\mathcal{H}} -
\hbar\omega\hat{a}\right)\ket{E_n} =
\left(E_n - \hbar \omega\right)\hat{a}\ket{E_n}
\end{equation}
i.e., $\hat{a}\ket{E_n}$ is also a state vector with energy $E_n - \hbar \omega$. From this, it follows that
\[
\hat{a}\ket{E_n} = \left|E_n - \hbar \omega \right>
\] 
Thus, the operator $\hat{a}$ lowers the energy of the state by $\hbar \omega$, where $\omega$ is the frequency of the mode (oscillator). The operator $\hat{a}$ is often called the lowering operator, or the annihilation operator. The lowest energy must be positive, and reducing the energy cannot continue indefinitely. For an arbitrary state vector, the expected energy value
\begin{equation}
\left< \Phi \right| \hbar \omega \left({a}^{\dag} {a}  +
\frac{1}{2}\right)\left| \Phi \right> = 
\hbar \omega \left< \Phi' \right. \left| \Phi' \right> + \frac{1}{2}
\hbar \omega,
\end{equation}
where $\hat{a} \left| \Phi \right> = \left| \Phi' \right>$,  
$\left< \Phi \right| \hat{a}^{\dag}  = \left< \Phi' \right|$. Since the norm of the state vector must be positive, the minimum energy value will be at
\(
\left< \Phi' \right. \left| \Phi' \right> = 0.
\)
This means $\hat{a}\ket{0} = 0$, where $\left|\Phi\right> = \ket{0}$ is the state vector with the lowest energy. The lowest energy
\begin{equation}
E_0 = \frac{\hbar \omega}{2}
\end{equation}
is called the zero-point energy. To verify this claim, we can write
\begin{eqnarray}
\hat{\mathcal{H}} \ket{0} = 
\hbar \omega \left(\hat{a}^{\dag} \hat{a} +
\frac{1}{2}\right) \ket{0} = 
\nonumber \\
= 
\hbar \omega \hat{a}^{\dag} \hat{a} \ket{0} +
\frac{\hbar \omega}{2}\ket{0} =
\nonumber \\
= \frac{\hbar \omega}{2} \ket{0} = 
E_0 \ket{0}.
\label{eqProper0state}
\end{eqnarray}

Using \eqref{eqCh1_mainpropertyaa}, we can obtain
\begin{eqnarray}
\hat{\mathcal{H}} \hat{a}^{\dag}\ket{0} = 
\hat{\mathcal{H}} \ket{1} =
\left(\hat{a}^{\dag} \hat{\mathcal{H}} + \hbar \omega \hat{a}^{\dag} \right)
\ket{0} = 
\nonumber \\
= \hbar \omega \left(1 + \frac{1}{2}\right)
\hat{a}^{\dag} \ket{0} = 
\hbar \omega \left(1 + \frac{1}{2}\right)
\ket{1}
\end{eqnarray}
where $\ket{1} = \hat{a}^{\dag} \ket{0}$ denotes the state with energy $\hbar \omega \left(1 + \frac{1}{2}\right)$.

By induction, we have
\begin{equation}
\hat{\mathcal{H}} \left(\hat{a}^{\dag}\right)^n\ket{0} = 
\hat{\mathcal{H}} \ket{n} 
= \hbar \omega \left(n + \frac{1}{2}\right)
\left(\hat{a}^{\dag}\right)^n\ket{0} = 
\hbar \omega \left(n + \frac{1}{2}\right)
\ket{n}
\label{eqCh1_aplusinduction}
\end{equation}
where
$\ket{n} = \left(\hat{a}^{\dag}\right)^n\ket{0}$   
without normalization, which we will perform later - the state with energy  
$\hbar \omega \left(n + \frac{1}{2}\right)$, $n$ is a positive integer.

We see that the operator $\hat{a}^{\dag}$ raises the energy of the state by $\hbar \omega$. It can be considered as the creation operator of a photon \rindex{photon} with energy $\hbar \omega$. The photon is better understood as a particle with energy $\hbar \omega$ and momentum $\hbar \vec{k}$ when considering field decomposition into plane waves, as follows from \eqref{eqCh1_task3_2}. 
  
The relations 
$\hat{a} \ket{n} = \ket{n - 1}$
and
$\hat{a}^{\dag} \ket{n} = \ket{n + 1}$
define unnormalized state vectors. Let's define the normalization factor. Assume that  
$\hat{a} \ket{n} = S_n \ket{n - 1}$, where 
$\ket{n}$ and $\ket{n - 1}$ are normalized to 1, and $S_n$
is the normalization factor. From this, we find 
\[
S_n^2\bra{n - 1}\ket{n - 1} =
\bra{n}\hat{a}^{\dag}\hat{a}\ket{n} = 
n  \bra{n}\ket{n}
\]
since the operator    
$\hat{a}^{\dag}\hat{a} = \hat{n}$
is the photon number operator, whose eigenvalue is the photon number. This is evident from the formula
\eqref{eqCh1_aplusinduction}. Indeed, from the equality
\[
\hat{\mathcal{H}} \ket{n} =
\hbar \omega \left(
\hat{a}^{\dag}\hat{a} + \frac{1}{2}
\right)
\ket{n} = 
\hbar \omega \left(n + \frac{1}{2}\right)
\ket{n},
\]
we obtain:
\[
\hat{n}\ket{n} = \hat{a}^{\dag}\hat{a} \ket{n} = n
\ket{n}. 
\]
From the normalization condition, it follows: $\bra{n}\ket{n} = 1$   and
$S_n^2 = n$, thus $S_n = \sqrt{n}$ and hence: 
\begin{equation}
\hat{a}\ket{n} = \sqrt{n}\ket{n - 1}
\end{equation}
Similarly, using the commutation relations
\eqref{eqCh1_aacomutation}, and \ref{eqAddDirac_operator_property1} and
  \ref{eqAddDirac_operator_property2} from \autoref{AddDirac},
  we find 
\begin{eqnarray}
\hat{a}^{\dag}\ket{n} = S_{n+1}\ket{n + 1},
\quad 
\bra{n}\hat{a} = S_{n + 1}\bra{n + 1},
\nonumber \\
\bra{n}\hat{a}\hat{a}^{\dag}\ket{n} = S_{n+1}^2
\bra{n + 1}\ket{n + 1} = 
\bra{n}\hat{a}^{\dag}\hat{a} + 1\ket{n} = 
\left(n + 1\right)\bra{n}\ket{n},
\nonumber \\
S_{n+1}^2 = n + 1.
\end{eqnarray}
Therefore, we have the equality
\begin{equation}
\hat{a}^{\dag}\ket{n} = \sqrt{n + 1}\ket{n + 1},
\end{equation}

The eigenstates of the photon number operator $\hat{n}$ are orthonormal. 
Indeed, from the fact that the operator $\hat{n}$ is Hermitian:
\[
\hat{n}^{\dag} = \left(\hat{a}^{\dag}\hat{a}\right)^{\dag} = 
\hat{a}^{\dag} \left(\hat{a}^{\dag}\right)^{\dag} = 
\hat{a}^{\dag}\hat{a} = \hat{n}
\]
it follows (see \autoref{AddDirac}) that the eigenfunctions of this operator,
corresponding to different eigenvalues, are orthogonal, i.e.
\begin{equation}
\bra{n}\ket{n'} = 0, \mbox{ if } n \ne n'.
\label{eqOrtoN}
\end{equation}

Let's summarize the relations involving the operators $\hat{a}$ and $\hat{a}^{\dag}$:
\begin{eqnarray}
\hat{\mathcal{H}} = \hbar \omega \left(\hat{a}^{\dag}\hat{a} +
\frac{1}{2} \right),
\quad
\hat{a}\ket{0} = 0,
\quad
\hat{a}^{\dag}\hat{a}\ket{n} = \hat{n}\ket{n},
\nonumber \\
\left[\hat{a}, \hat{a}^{\dag}\right] = \hat{a} \hat{a}^{\dag} - \hat{a}^{\dag}
\hat{a} = 1,
\quad
\hat{a}\ket{n} = \sqrt{n}\ket{n - 1}
\nonumber \\
\hat{\mathcal{H}}\ket{n} = \hbar \omega \left(\hat{a}^{\dag}\hat{a} +
\frac{1}{2} \right)\ket{n},
\nonumber \\
\hat{a}^{\dag}\ket{n} = \sqrt{n + 1}\ket{n + 1},
\quad
\ket{n} = \frac{1}{\sqrt{n!}}\left(\hat{a}^{\dag}\right)^n\ket{0}
\end{eqnarray}
and the conjugate equalities
\begin{eqnarray}
\bra{0}\hat{a}^{\dag} = 0,
\quad
\bra{n}\hat{a} = \sqrt{n + 1}\bra{n + 1}
\nonumber \\
\bra{n}\hat{a}^{\dag} = \sqrt{n}\bra{n - 1},
\quad
\bra{n} =  \frac{1}{\sqrt{n!}} \bra{0}\left(\hat{a}^{\dag}\right)^n.
\end{eqnarray}

For the simplest resonator model, we have
\[
\hat{E}\left(z, t\right) = E_1\left( \hat{a} +
\hat{a}^{\dag}\right) \sin k_n z
\]
where $E_1 = \sqrt{\frac{\hbar \omega}{\varepsilon_0 V}}$ is the field,
corresponding to one photon in the mode.  

Consider some properties of energy states, i.e., states
with a definite number of photons. We will show that the average
value of the electric field in this state is zero: 
\begin{eqnarray}
\bra{n}\hat{E}\ket{n} = 
E_1 \sin k_n z \left( \bra{n}\hat{a}\ket{n} +
\bra{n}\hat{a}^{\dag}\ket{n}\right) =
\nonumber \\
= E_1 \sin k_n z \left( \bra{n}\ket{n - 1} \sqrt{n} +
\bra{n}\ket{n + 1} \sqrt{n + 1}
\right) = 0
\label{eqCh1_E_middle}
\end{eqnarray}
which follows from the orthogonality of states \eqref{eqOrtoN}
$\bra{n}\ket{n'} = 0$.

The average value of the square of the electric field operator is non-zero:
\begin{eqnarray}
\bra{n}\hat{E}^2\ket{n} = 
E_1^2 \sin^2 k_n z \bra{n}
\left(
\hat{a}^{\dag} \hat{a}^{\dag} + \hat{a} \hat{a}^{\dag} + \hat{a}^{\dag} \hat{a} +
\hat{a} \hat{a}
\right)
\ket{n} =
\nonumber \\
= 2 E_1^2 \sin^2 k_n z \left( n + \frac{1}{2}
\right).
\label{eqCh1_E2_middle}
\end{eqnarray}

The fields usually considered in quantum optics are not in a
stationary state with a definite energy (with a definite number
of photons). However, an arbitrary state can be represented as
a superposition of states $\ket{n}$: 
\begin{equation}
\left|\psi\right> = \sum_{(n)} C_n \ket{n}
\end{equation}
where $\left|C_n\right|^2$ is the probability of finding $n$ photons in the mode
upon measurement; $\sum_{(n)} \left|C_n\right|^2 = 1$. 
In \autoref{AddDirac}, it is shown that
\[
C_n = \bra{ n }\left| \psi \right>, \quad
\left| \psi \right> = \sum_{(n)} \bra{ n }\left| \psi \right>
\ket{ n } =
\sum_{(n)} \ket{ n }\bra{ n }\left| \psi \right>,
\quad
\sum_{(n)} \ket{ n }\bra{ n } = \hat{I},
\]
where $\hat{I}$ is the identity operator.

\begin{remark}[On states with definite energy in quantum mechanics]
  It is worth noting that states with definite energy do not violate 
  Heisenberg's uncertainty relation for the pair
  energy - time
  \[
  \Delta E \Delta t \ge \frac{\hbar}{2},
  \]
  (see \autoref{AddHeisenbergUncertaintyPrincipleEnergyTime} for more details). 

  Meanwhile, if we look at the energy operator of the harmonic
  oscillator, written as
  \[
  \hat{\mathcal{H}} =\frac{1}{2} \left(\hat{p}^2 +
  \omega^2\hat{q}^2\right)
  \]
  and use expressions \eqref{eqCh1_qpdef}, we find that $\bra{n}\hat{q}\ket{n} =
  \bra{n}\hat{p}\ket{n} = 0$. Meanwhile,
  \begin{eqnarray}
    \bra{n}\hat{q}^2\ket{n} = \frac{\hbar}{2 \omega}
    \left[
      \bra{n}\hat{a}^2\ket{n} +
      \bra{n}\left(\hat{a}^{\dag}\right)^2\ket{n} +
      \right.
      \nonumber \\
      \left.
      +
      \bra{n}\hat{a}^{\dag}\hat{a}\ket{n} +
      \bra{n}\hat{a}\hat{a}^{\dag}\ket{n}
      \right] =
    \nonumber \\
    = \frac{\hbar}{2 \omega}
    \left[n + n + 1\right],
    \nonumber
  \end{eqnarray}
  and also
  \begin{eqnarray}
    \bra{n}\hat{p}^2\ket{n} = - \frac{\hbar \omega}{2}
    \left[
      \bra{n}\hat{a}^2\ket{n} +
      \bra{n}\left(\hat{a}^{\dag}\right)^2\ket{n} - \right.
      \nonumber \\
      \left.
      -
      \bra{n}\hat{a}^{\dag}\hat{a}\ket{n} -
      \bra{n}\hat{a}\hat{a}^{\dag}\ket{n}
      \right] =
    \nonumber \\
    = \frac{\hbar \omega}{2}
    \left[n + n + 1\right].
    \nonumber
  \end{eqnarray}
  Thus,
  \[
  \Delta p = \sqrt{\bra{n}\hat{p}^2\ket{n} -
    \bra{n}\hat{p}\ket{n}^2} =
  \sqrt{\frac{\hbar \omega}{2}\left(2n + 1\right)}
  \]
  and
  \[
  \Delta q = \sqrt{\bra{n}\hat{q}^2\ket{n} -
    \bra{n}\hat{q}\ket{n}^2} =
  \sqrt{\frac{\hbar}{2\omega}\left(2n + 1\right)}
  \]
  or
  \[
  \Delta p \Delta q = \frac{\hbar}{2}\left(2n + 1\right) \ge \frac{\hbar}{2},
  \]
  which is consistent with Heisenberg's uncertainty relations \eqref{eqAddHeisenbergUncertaintyPrinciple}.

  Thus, states with definite energy are states
  in which, despite the impossibility of precisely determining $p$
  and $q$, it is possible to determine the value of 
  \(
  \frac{1}{2} \left(p^2 +
  \omega^2q^2\right)
  \). This reflects the fact that for quantum systems, situations are possible
  where composite events exist in the absence
  of elementary ones (see \autoref{sec:add:quantprobability}). 
  In particular, such a situation occurs for the photon
  - states with definite energy $\ket{1}$.

  This may lead us to consider that a photon is
  a virtual particle, which does not exist in the real physical world
  \cite{Lamb1995}.\rindex{photon} Meanwhile, the mathematical apparatus related to
  the concept of the photon such as creation operators $\hat{a}^\dag$ and
  annihilation $\hat{a}$, electromagnetic field states with
  definite energy $\{\ket{n}\}$ (which can be
  used as basis states), is convenient for theoretical descriptions.

  It is also worth noting that from the perspective of the formal definition of
  a non-classical state \eqref{eqPart3_Nonclass_Nonclass7},
  $\ket{n}$ is a non-classical 
  state of light since from \eqref{eqCh4_26} it follows $G^{(2)} <
  1$.

  On the other hand, if we consider the minimum possible energy
  of the electromagnetic field mode, from the relation
  \eqref{eqCh1_hamilton_one_mode} 
  \[
  \mathcal{H} = \frac{1}{2}\left(\omega^2 q^2 + p^2\right)
  \]
  it follows that in the classical case, the minimum possible zero
  energy is reached at $p = 0, q=0$, but due to
  \eqref{eqAddHeisenbergUncertaintyPrinciple}, zero values
  are impossible in the quantum case, and considering measurement uncertainties
  we find
  \begin{eqnarray}
    \frac{1}{2}\left(\omega^2 (\Delta q)^2 + (\Delta p)^2\right) \ge
    \nonumber \\
    \ge \omega \Delta q \Delta p \ge \frac{\hbar \omega}{2}
    \nonumber
  \end{eqnarray}
  that is, the minimum possible energy (vacuum energy) is determined
  by Heisenberg's inequalities for the coordinate - momentum pair.  
  \label{rem:antiphoton}
\end{remark}