\section{Expansion of the Field into Plane Waves in Free Space}
So far, we have considered the electromagnetic field in a physically
distinguished volume—in a resonator. If we are dealing with free
space, we can artificially distinguish a sufficiently large volume
containing the region of space of interest, define modes (types of oscillations) for it with appropriate boundary conditions, and then
proceed in the manner already considered. If necessary, the volume in
the final result can be extended to infinity. Usually, a sufficiently large cubic volume with a side $L$ 
(\autoref{figCh1_Vfree}) is used. In this case, it is customary to use
periodic boundary conditions:  
\begin{eqnarray}
\vec{E}\left(0, y, z \right) = \vec{E}\left(L, y, z \right),
\nonumber \\
\vec{E}\left(x, 0, z \right) = \vec{E}\left(x, L, z \right),
\nonumber \\
\vec{E}\left(x, y, 0 \right) = \vec{E}\left(x, y, L \right).
\label{eqCh1_period_def}
\end{eqnarray}

\input ./part1/quantel/fig2.tex

It is convenient to decompose into plane waves. As known from the course
of electromagnetic oscillations, the solution corresponding to a plane wave
has the form 
\begin{equation}
\vec{E}_k\left(r, t\right) = 
A_k\left(t\right)\vec{e}_k e^{i\left(\vec{k}\vec{r}\right)} +
\text{(c. c.)},
\label{eqCh1_Emode}
\end{equation}
where $\vec{e}_k$ is the unit vector of wave polarization;  
$k$ is the wave number; $\vec{k}$ is the wave vector;  
$A_k\left(t\right) = A_k\left(0\right) e^{-i \omega_k t}$.

The magnetic field is related to the electric field by the relation
\begin{equation}
\vec{H}_k\left(r, t\right) =
\sqrt{\frac{\varepsilon_0}{\mu_0}}
\frac{1}{k}\left[\vec{k} \vec{e}_k\right] A_k\left(t\right) 
e^{i\left(\vec{k}\vec{r}\right)} + \text{(c. c.)}
.
\label{eqCh1_Hmode}
\end{equation}

The following equalities are valid:
\begin{equation}
\left(\vec{k}\vec{E}_k\right) = 
\left(\vec{k}\vec{H}_k\right) = 
\left(\vec{E}_k\vec{H}_k\right) = 0,
\quad
k^2 = \left(\vec{k}\vec{k}\right) = 
\frac{\omega_k^2}{c^2} 
\end{equation}
indicating that 
$\vec{E}_k$
and $\vec{H}_k$ are perpendicular to the direction of wave propagation and to each
other (the wave is transverse). 

The periodicity condition \eqref{eqCh1_period_def} will be satisfied
if 
\begin{equation}
\vec{k} = \frac{2 \pi}{L}\left(n_x \vec{x}_0
+ n_y \vec{y}_0
+ n_z \vec{z}_0
\right),
\quad
n_x, n_y, n_z = 0, \pm 1, \pm 2, \dots .
\label{eqCh1_period}
\end{equation}
Then we get
\begin{eqnarray}
\left(\vec{k}\vec{r}\right) = \frac{2 \pi}{L}\left(n_x x
+ n_y y
+ n_z z
\right),
\nonumber \\
\left.\left(\vec{k}\vec{r}\right)\right|_{x = L} = 2 \pi n_x + \frac{2 \pi}{L}\left(n_y y
+ n_z z
\right) = 
2 \pi n_x + \left.\left(\vec{k}\vec{r}\right)\right|_{x = 0}.
\end{eqnarray}
Consequently,    
$e^{i\left(\vec{k}\vec{r}\right)}$
is periodic in $x$. In the same way, periodicity in $y$ and $z$ is shown.

In the future, instead of the real functions \eqref{eqCh1_Emode}, 
\eqref{eqCh1_Hmode} we will use
complex eigenfunctions 
\begin{equation}
\vec{E}_k\left(r\right) = \vec{e}_k e^{i \left( \vec{k}\vec{r}\right)},
\quad
\vec{H}_k\left(r\right) = \sqrt{\frac{\varepsilon_0}{\mu_0}}\frac{1}{k}
\left[\vec{k}\vec{E}_k\left(r\right)\right].
\label{eqCh1_EHmode}
\end{equation}
The relations are valid:
\[
\left(\vec{k}\vec{E}_k\right) = 
\left(\vec{k}\vec{H}_k\right) = 
\left(\vec{E}_k\vec{H}_k\right) = 0,
\quad
k^2 = \left(\vec{k}\vec{k}\right) = 
\frac{\omega_k^2}{c^2}.
\]
The eigenfrequencies $\omega_k$ are determined by the equation
\eqref{eqCh1_period}. Indeed, from \eqref{eqCh1_period} taking into account
$k^2 = \frac{\omega_k^2}{c^2}$ we have 
\begin{equation}
\omega_k = c \sqrt{k_x^2 + k_y^2 + k_z^2} = 
\frac{2 \pi c}{L} \sqrt{n_x^2 + n_y^2 + n_z^2}.
\end{equation}

With linear polarization, the polarization vectors $\vec{e}_k$ are:
\[
\vec{e}_{k_1} = \vec{\xi}_0,
\quad
\vec{e}_{k_2} = \vec{\eta}_0,
\]
where $\vec{\xi}_0$, $\vec{\eta}_0$, are the unit vectors of directions, forming a 
rectangular system of directions with the vector $\vec{k}$, consequently, 
\[
\left(\vec{\xi}_0\vec{k}\right) =
\left(\vec{\eta}_0\vec{k}\right) =
\left(\vec{e}_{k_1}\vec{e}_{k_2}\right) = 0.
\]

For circular polarization
\[
\vec{e}_{k_1} = \frac{\vec{\xi}_0 + i \vec{\eta}_0}{\sqrt{2}},
\quad
\vec{e}_{k_2} = \frac{\vec{\xi}_0 - i \vec{\eta}_0}{\sqrt{2}}.
\]

In this case, the orthogonality condition holds:
$\left(\vec{e}_{k_1} \vec{e}_{k_2}^{*}\right)$
and normalization conditions:
$\left(\vec{e}_{k_1} \vec{e}_{k_1}^{*}\right) = \left(\vec{e}_{k_2}
\vec{e}_{k_2}^{*}\right) = 1$.

An arbitrary electromagnetic field using complex functions 
\eqref{eqCh1_EHmode} can be represented by decompositions 
\begin{eqnarray}
\vec{E}\left(r, t\right) = 
\sum_{(k)} 
A_k\left(t\right) \vec{E}_k\left(r\right) + \text{(c. c.)},
\nonumber \\
\vec{H}\left(r, t\right) = 
\sum_{(k)} 
A_k\left(t\right) \vec{H}_k\left(r\right) +
\text{(c. c.)}
\end{eqnarray}