%% -*- coding:utf-8 -*- 
\input preamble.tex

\begin{document}
\Russian
\pagestyle{empty}
\input title.tex

\section*{Introduction}
Presented for your consideration is an application for the publication of a textbook entitled ``Quantum Optics''. This textbook is based on a course of lectures on quantum optics, taught over a long period at SPbPU (Saint Petersburg State Polytechnic University) at the Faculty of Radiophysics for students specializing in quantum radiophysics and quantum electronics.

The textbook consists of three main sections
\begin{itemize}
\item Quantum Electrodynamics. Quantum electrodynamics serves as
  the foundation of quantum optics. The section contains a brief presentation of the basics of quantum electrodynamics,
sufficient for describing optical quantum phenomena in the frequency range from infrared to X-ray radiation.
\item Quantum Optics Part 1. In this section, based on quantum electrodynamics,
  a number of optical phenomena requiring quantum description are considered.
Established results of quantum optics are discussed.
\item Quantum Optics Part 2. This section covers recent results
  in quantum optics related to currently developing directions.
\end{itemize}
Altogether, it amounts to approximately \(15\) printed pages. Formulas: \(\approx 300\), figures: \(\approx 30\). The text is set using the \LaTeX\ package. Figures are done in \LaTeX\ and PostScript formats.

Contact information for prompt communication: email:
ivan.murashko@gmail.com (Ivan Viktorovich Murashko). 

\newpage

\section{Abstract}
The textbook corresponds to the author's course of the discipline ``Quantum Optics''. It provides information on quantum electrodynamics
minimally necessary for the exposition of a range of optics topics requiring consideration of the quantum properties of light. Main attention is given to
the interaction of light with atoms, quantum theory of relaxation,
quantum theory of the laser, and quantum theory of coherence. Additionally,
recent results such as quantum information theory and nonclassical properties of light are considered.

The textbook is intended for senior students specializing
in quantum electronics.

\section{Contents}
\subsection{Quantum Electrodynamics}
\subsubsection{Quantum properties of the electromagnetic field}
Decomposition of the electromagnetic field into modes (types of oscillations).
Hamiltonian form of the electromagnetic field equations.
Quantization of the electromagnetic field.
Decomposition of the field into plane waves in free space.
Density of states.
Hamiltonian form of the field equations when decomposed into plane waves.
Quantization of the electromagnetic field decomposed into
plane waves.
Properties of the operators \( \hat a \) and \( \hat a^\dag \).
Quantum state of the electromagnetic field with a definite
  energy.
Multimode states.
Coherent states.
Mixed states of the electromagnetic field.
Representation of the density operator via coherent
states.
\subsubsection{Interaction of the quantized electromagnetic field with an atom}
Emission and absorption of light by an atom.
Hamiltonian of the atom-field system.
Interaction of an atom with a mode of the electromagnetic field.
Interaction of an atom with a multimode field. Induced and
spontaneous transitions.
Spontaneous emission. Weisskopf-Wigner approximation.
Relaxation of the dynamical system. Density matrix method.
Interaction of the electromagnetic field of a resonator
(harmonic oscillator) with a reservoir of atoms at
temperature \(T\).
Equation for the density matrix of the field in the number
representation.
Equation of motion of the statistical operator of the field mode in
the coherent states representation.
General theory of the interaction of a dynamical system with
a thermostat (dissipative system, reservoir).
Damping (relaxation) of the field and atom in the case of the simplest
reservoir consisting of harmonic oscillators.
Damping of the resonator mode. Heisenberg-Langevin approximation.

\subsection{Quantum Optics Part 1}
\subsubsection{Quantum theory of the laser. Density matrix method}
Laser model. Theory of laser generation. Statistics of laser
photons. Theory of the laser in the representation of coherent states.
\subsubsection{Quantum theory of the laser. Heisenberg-Langevin method}
Quantum Langevin equations. Simple calculation of the laser emission linewidth by the Langevin method.
\subsubsection{Photon optics (quantum phenomena in optics)}
Photoelectric effect.
Coherent properties of light.
Second-order coherence.
Higher-order coherence.
Counting and statistics of photons.
Connection between photon statistics and photoelectron count statistics.
Distribution of photoelectron counts for coherent and chaotic
light.
Determination of photon statistics via photoelectron count
distribution.
Quantum expression for the photoelectron count distribution.
Experiments on photon counting. Application of photon counting
techniques for spectral measurements.

\subsection{Quantum Optics Part 2}
\subsubsection{Nonclassical light}
Criterion of classicality of a quantum state. Photon statistics.
Poissonian, sub-Poissonian, and super-Poissonian
statistics. Photon bunching and antibunching. Experimental
determination of photon statistics. Quasiprobability and its relation to
classical or nonclassical states. Nonclassicality of pure quantum states.
\subsubsection{Quantum description of optical interferometric
  experiments}
Classical description of the device (interferometer).
Quantum description of the light field and Heisenberg representation.
Light scattering matrix and its properties.
Examples. Michelson interferometer. Mach-Zehnder interferometer.
Quantum description. Interferometer output. Balanced detector. Phase measurement error.
\subsubsection{Squeezed states}
Heisenberg uncertainty relation. Squeezed state. Ideally
squeezed state. Operators of quadrature components
of the electromagnetic field. Condition of a squeezed state. Examples:
coherent and energy states are not squeezed. Unitary
squeezing operator and its properties. Action of the squeezing operator on
a coherent state. Quadrature-squeezed coherent
state. Squeezing parameter. Squeezed vacuum. Generation of quadrature
squeezed states via parametric interaction. Degenerate
parametric scattering. Observation of squeezed states and measurement
of the squeezing degree. Homodyne detector (synchronous detector) and its
application for selecting the ``squeezed'' quadrature. Application of squeezed
states to improve the precision of interferometric measurements. Precision improvement limit. Heisenberg limit.
\subsubsection{Entangled states}
Polarization properties of light. Stokes parameters. EPR paradox for Stokes
  parameters and entangled states. Bell inequality for Stokes
  parameters. Bell basis states. Generation and detection of Bell
  states. Quantum teleportation.
\subsubsection{Quantum information theory}
Information and entropy. Information transmission. Classical and quantum
communication channels. Information encoding. Shannon coding
theorem. Quantum coding theorem. Cryptography. Problems
of classical cryptography. Quantum cryptography.
\newpage
\section{Information about the authors}
\subsection{Vsevolod Yurievich Petrunkin}
Vsevolod Yurievich Petrunkin (5.09.1923 - 7.11.2008). Graduated from
the Faculty of Physics and Mechanics at Leningrad Polytechnic Institute (LPI) in 1949 with a specialty in ``Research engineer in radiophysics''. Candidate of Technical Sciences (1953). Doctor of Technical Sciences (1965). His work and scientific activity were connected with LPI. Graduate student (1949-1952), assistant (1952-1956). Associate professor of the ``Radiophysics'' department (1956-1966), professor of the ``Radiophysics''
department (since 1966). In 1968, he founded the Quantum Electronics department
at the Radiophysics faculty of LPI, which he headed until 1988. After that, he remained a professor of this department. His scientific activities were connected with research in electromagnetic wave radiation and propagation and antenna technology. Since 1965, involved in quantum radiophysics (lasers, laser applications). For many years, he was a member of the editorial board of the journal ``Izvestiya VUZov. Radiophysics''. Recently, he was a member of the editorial boards of JTP (Journal of Technical Physics) and ``Letters to JTP''. Laureate of the State Prize for work in radioelectronics (1984). Honored Scientist and Technician of the Russian Federation (1994).
\subsection{Ivan Viktorovich Murashko}
Ivan Viktorovich Murashko (born 09.04.1975). Graduated from the Radiophysics Faculty of Saint Petersburg State Technical University (1998). Candidate of Physical and Mathematical Sciences (2001). Associate Professor of the Quantum Electronics Department of Saint Petersburg State Polytechnic University.

Contact information (I. V. Murashko)
\begin{itemize}
\item home address: St. Petersburg, Shlisselburgsky Ave. 17 bld. 2 apt. 172
\item email: ivan.murashko@gmail.com,
\item mobile phone: +7 (921) 974 2848
\item home phone: +7 (812) 707 1464
\end{itemize}
\end{document}