%% -*- coding:utf-8 -*- 
\input preamble.tex

\begin{document}
\Russian
\input title.tex

\section*{Introduction}
We present an application for the publication of a textbook entitled ``Quantum Optics''. This textbook is based on a course of lectures on quantum optics, taught over a long period at SPbPU (Saint Petersburg State Polytechnic University) in the Radiophysics Faculty for students specializing in quantum radiophysics and quantum electronics.

The textbook consists of three main sections
\begin{itemize}
\item Quantum electrodynamics. Quantum electrodynamics serves as the foundation of quantum optics. This section contains a brief exposition of the basics of quantum electrodynamics sufficient for describing optical quantum phenomena in the frequency range from infrared to X-ray radiation.
\item Quantum optics Part 1. Based on quantum electrodynamics, this section considers a number of optical phenomena requiring a quantum description. Established results of quantum optics are examined.
\item Quantum optics Part 2. This section deals with recent results in quantum optics related to currently developing areas.
\end{itemize}
All together this comprises about \(\approx 15\) printed pages. Formulas \(\approx 300\), figures \(\approx 30\). The text is set using the \LaTeX\ package. Figures are produced in {\LaTeX} and PostScript formats.

Contact information for prompt communication: email: ivan.murashko@gmail.com (Murashko Ivan Viktorovich).

\section{Abstract}
The book corresponds to the author's course on the subject ``Quantum Optics''. It provides information on quantum electrodynamics minimally necessary for addressing a number of optics questions requiring accounting for the quantum properties of light. Main attention is given to issues of light interaction with atoms, quantum relaxation theory, quantum laser theory, and quantum coherence theory. Additionally, recent results, such as quantum information theory and nonclassical properties of light, are considered.

The textbook is intended for senior students specializing in quantum electronics.

\section{Contents}
\subsection{Quantum electrodynamics}
\subsubsection{Quantum properties of the electromagnetic field}
Decomposition of the electromagnetic field into modes (types of oscillations).
Hamiltonian form of the electromagnetic field equations.
Quantization of the electromagnetic field.
Expansion of the field into plane waves in free space.
Density of states.
Hamiltonian form of field equations when expanded into plane waves.
Quantization of the electromagnetic field by expansion into plane waves.
Properties of operators \( \hat a \) and \( \hat a ^\dag \).
Quantum state of the electromagnetic field with a definite energy.
Multimode states.
Coherent states.
Mixed states of the electromagnetic field.
Representation of the density operator via coherent states.
\subsubsection{Interaction of the quantized electromagnetic field with an atom}
Atom's emission and absorption of light.
Hamiltonian of the atom-field system.
Interaction of an atom with a single mode of the electromagnetic field.
Interaction of an atom with a multimode field. Stimulated and spontaneous transitions.
Spontaneous emission. Weisskopf-Wigner approximation.
Relaxation of the dynamical system. Density matrix method.
Interaction of the electromagnetic field of a resonator (harmonic oscillator) with a reservoir of atoms at temperature \(T\).
Equation for the field density matrix in the number representation.
Equation of motion for the mode field statistical operator in the coherent state representation.
General theory of the interaction of a dynamical system with a thermostat (dissipative system, reservoir).
Damping (relaxation) of the field and atom in the case of the simplest reservoir consisting of harmonic oscillators.
Damping of the resonator mode. Heisenberg-Langevin approximation.

\subsection{Quantum Optics Part 1}
\subsubsection{Quantum laser theory. Density matrix method}
Laser model. Theory of laser generation. Statistics of laser photons. Laser theory in the coherent state representation.
\subsubsection{Quantum laser theory. Heisenberg-Langevin method}
Quantum Langevin equations. Simple calculation of laser emission linewidth by the Langevin method.
\subsubsection{Photon optics (quantum phenomena in optics)}
Photoelectric effect.
Coherent properties of light.
Second-order coherence.
Higher-order coherence.
Counting and statistics of photons.
Relation of photon statistics to photocount statistics.
Photocount distribution for coherent and chaotic light.
Determination of photon statistics through photocount distribution.
Quantum expression for photocount distribution.
Experiments on photon counting. Applications of photon counting techniques to spectral measurements.

\subsection{Quantum Optics Part 2}
\subsubsection{Nonclassical light}
Criterion of classicality of a quantum state. Photon statistics. Poissonian, sub-Poissonian, and super-Poissonian statistics. Photon bunching and antibunching. Experimental determination of photon statistics. Quasiprobability and its relation to classicality or nonclassicality of the state. Nonclassicality of pure quantum states.
\subsubsection{Quantum description of optical interference experiments}
Classical description of the device (interferometer).
Quantum description of the light field and Heisenberg representation.
Light scattering matrix and its properties.
Examples. Michelson interferometer. Mach-Zehnder interferometer.
Quantum description. Interferometer output. Balanced detector. Phase measurement error.
\subsubsection{Squeezed states}
Heisenberg uncertainty relation. Squeezed state. Ideally squeezed state. Operators of quadrature components of the electromagnetic field. Condition for squeezed state. Examples: coherent and number states are not squeezed. Unitary squeezing operator and its properties. Action of the squeezing operator on a coherent state. Quadrature-squeezed coherent state. Squeezing parameter. Squeezed vacuum. Generation of quadrature-squeezed states in parametric interaction. Degenerate parametric scattering. Observation of squeezed state and measurement of squeezing degree. Homodyne detector (synchronous detector) and its application for extracting the ``squeezed'' quadrature. Application of squeezed states for improving the accuracy of interference measurements. Limit of accuracy improvement. Heisenberg limit.
\subsubsection{Entangled states}
Polarization properties of light. Stokes parameters. EPR paradox for Stokes parameters and entangled states. Bell's inequality for Stokes parameters. Bell basis states. Generation and registration of Bell states. Quantum teleportation.
\subsubsection{Quantum information theory}
Information and entropy. Information transfer. Classical and quantum communication channels. Information encoding. Shannon coding theorem. Quantum coding theorem. Cryptography. Problems of classical cryptography. Quantum cryptography.

\section{Information about the authors}
\subsection{Petrunkin Vsevolod Yuryevich}
Petrunkin Vsevolod Yuryevich (05.09.1923 - 07.11.2008). Graduated from the Physics-Mechanical Faculty of the Leningrad Polytechnic Institute (LPI) (1949) with the specialty ``Engineer-Researcher in Radiophysics''. Candidate of Technical Sciences (1953). Doctor of Technical Sciences (1965). His work and scientific activity were linked to LPI. Graduate student (1949–1952), assistant (1952–1956). Associate professor of the Department of Radiophysics (1956–1966), professor of the Department of Radiophysics (1966). In 1968, he founded the Department of Quantum Electronics at the Radiophysics Faculty of LPI, which he headed until 1988. Afterward, he was a professor at this department. His scientific activity was connected with work in the field of electromagnetic wave radiation and propagation and antenna technology. Since 1965, he was involved with quantum radiophysics (lasers, laser applications). For many years, he was a member of the editorial board of the journal ``Izvestiya VUZ. Radiophysics.'' Recently, he was a member of the editorial boards of journals JTF (``Journal of Technical Physics'') and ``Letters to JTF.'' Laureate of the State Prize for work in radio electronics (1984). Honored Scientist and Technician of the Russian Federation (1994). 
\subsection{Murashko Ivan Viktorovich} 
Murashko Ivan Viktorovich (born 09.04.1975). Graduated from the Radiophysics Faculty of Saint Petersburg State Technical University (1998). Candidate of Physical and Mathematical Sciences (2001). Associate professor of the Department of Quantum Electronics at Saint Petersburg State Polytechnic University.

Contact information (Murashko I.V.) 
\begin{itemize}
\item Home address: St. Petersburg, Shlisselburgsky pr. 17, bldg. 2, apt. 172
\item email: ivan.murashko@gmail.com,
\item mobile phone: +7 (921) 974 2848
\item home phone: +7 (812) 707 1464
\end{itemize}
\end{document}