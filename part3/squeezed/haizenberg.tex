%% -*- coding:utf-8 -*- 
\section{Heisenberg Uncertainty Relation}
In quantum mechanics, two conjugate observables $A$ and $B$
correspond to non-commuting operators $\hat{A}$ and $\hat{B}$,
satisfying the commutation relation
\begin{equation}
\left[
\hat{A}, \hat{B}
\right] = 
\hat{A}\hat{B} - \hat{B}\hat{A} = i \hat{C},
\nonumber
\end{equation}
where $\hat{C}$ is some Hermitian operator. In this case, the observables
$A$ and $B$ can only be measured with some uncertainty,
expressed by the Heisenberg uncertainty relation
(see in more detail \autoref{AddHeisenbergUncertaintyPrinciple})
\begin{equation}
\left(
\Delta A \Delta B
\right) \ge \frac{1}{2} \left|\left<\hat{C}\right>\right|,
\label{eqPart3Squeezed2}
\end{equation}
where
\[
\left(\Delta A\right)^2 = \left<\hat{A}^2\right> - \left<\hat{A}\right>^2
\]
and
\[
\left(\Delta B\right)^2 = \left<\hat{B}^2\right> - \left<\hat{B}\right>^2
\]
are the mean square deviations from the mean value.
The average $\left<\dots\right>$ corresponds to the quantum state in
which the quantum object is located.

Relation \eqref{eqPart3Squeezed2} limits only the product
of uncertainties, allowing states for which the uncertainty
of one of the observables is significantly less than the uncertainty of the other.
If
\begin{equation}
\left(\Delta A\right)^2 < \frac{1}{2} \left|\left<\hat{C}\right>\right|,
\nonumber
\end{equation}
such a state is called a squeezed state for the observable
$A$. If, in addition, the minimal uncertainty condition
\begin{equation}
\left(
\Delta A \Delta B
\right) = \frac{1}{2} \left|\left<\hat{C}\right>\right|,
\nonumber
\end{equation}
is observed, such a state is called an ideally squeezed state.
