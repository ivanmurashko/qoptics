\section{Generation of Bell States}
\label{subsecPart3NonclassEntanglBelGener}

\rindex{Bell Basis States!generation}
To obtain entangled states, the process of parametric light scattering can be used, which we previously considered for describing the process of obtaining squeezed states of the electromagnetic field (see \ref{pNonClassGenerSqueezed}). At that time, we considered phase synchronism of the first kind, where the polarizations of the two photons are the same. In second-kind phase synchronism, the polarizations of the photons are different. In this case, the photons propagate along two cones, as shown in \autoref{figEntangGen}.

\input ./part3/entang/figgen1.tex

Along one cone, the radiation is polarized as ordinary waves, and along the second as extraordinary. Thus, at the points of intersection of the cones (see \autoref{figEntangGen2}), the state of the light is recorded as follows:
\begin{equation}
\left|\psi\right> =
  \frac{1}{\sqrt{2}}\left(
  \ket{x}_1\ket{y}_2 + e^{i \alpha}
  \ket{y}_1\ket{x}_2
  \right),
\nonumber
\end{equation}
where the phase difference $\alpha$ arises due to the difference in refractive indices for ordinary and extraordinary photons. By using an additional birefringent phase plate, any phase difference $\alpha$ can be obtained, for example, $0$ or $\pi$. As a result, we will obtain the following Bell states:
\begin{eqnarray}
  \left|\psi^{\dag}\right>_{12} = 
  \frac{1}{\sqrt{2}}\left(
  \ket{x}_1\ket{y}_2 + 
  \ket{y}_1\ket{x}_2
  \right),
  \nonumber \\
  \left|\psi^{-}\right>_{12} = 
  \frac{1}{\sqrt{2}}\left(
  \ket{x}_1\ket{y}_2 - 
  \ket{y}_1\ket{x}_2
  \right).
  \label{eqEntangBellBaseGen1}
\end{eqnarray}

\input ./part3/entang/figgen2.tex

Now, if a half-wave phase plate that changes the polarization to the orthogonal one is placed in front of one of the beams, for example, the first:
\begin{equation}
\ket{x}_2 \rightarrow \ket{y}_2, \, \ket{y}_2 \rightarrow \ket{x}_2,
\nonumber
\end{equation}
then from \eqref{eqEntangBellBaseGen1} we get
\begin{eqnarray}
  \left|\phi^{\dag}\right>_{12} = 
  \frac{1}{\sqrt{2}}\left(
  \ket{x}_1\ket{x}_2 + 
  \ket{y}_1\ket{y}_2
  \right),
  \nonumber \\
  \left|\phi^{-}\right>_{12} = 
  \frac{1}{\sqrt{2}}\left(
  \ket{x}_1\ket{x}_2 - 
  \ket{y}_1\ket{y}_2
  \right).
  \label{eqEntangBellBaseGen2}
\end{eqnarray}

Thus, combining \eqref{eqEntangBellBaseGen1} and \eqref{eqEntangBellBaseGen2}, we obtain the complete basis \eqref{eqEntangBellBase}.