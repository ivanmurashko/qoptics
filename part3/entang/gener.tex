%% -*- coding:utf-8 -*- 
\section{Generation of Bell States}
\label{subsecPart3NonclassEntanglBelGener}

\rindex{Bell basis states!generation}
The process of parametric light scattering, which we considered earlier for describing the generation of squeezed states of the electromagnetic field (see \ref{pNonClassGenerSqueezed}), can be used to obtain entangled states. Then we considered the first kind of phase matching, where the polarizations of the two photons are the same. In the second kind of phase matching, the polarizations of the photons are different. In this case, the photons propagate along two cones, as shown in \autoref{figEntangGen}.

\input ./part3/entang/figgen1.tex

Along one cone, the radiation is polarized as ordinary waves, and along the other as extraordinary waves. Thus, at the intersection points of the cones (see \autoref{figEntangGen2}), the state of light is written as
\begin{equation}
\left|\psi\right> =
  \frac{1}{\sqrt{2}}\left(
  \ket{x}_1\ket{y}_2 + e^{i \alpha}
  \ket{y}_1\ket{x}_2
  \right),
\nonumber
\end{equation}
where the phase difference $\alpha$ arises due to the difference in refractive indices for ordinary and extraordinary photons. Using an additional birefringent phase plate, any value of the phase difference $\alpha$ can be obtained, for example $0$ or $\pi$. As a result, we get the following Bell states:
\begin{eqnarray}
  \left|\psi^{\dag}\right>_{12} = 
  \frac{1}{\sqrt{2}}\left(
  \ket{x}_1\ket{y}_2 + 
  \ket{y}_1\ket{x}_2
  \right),
  \nonumber \\
  \left|\psi^{-}\right>_{12} = 
  \frac{1}{\sqrt{2}}\left(
  \ket{x}_1\ket{y}_2 - 
  \ket{y}_1\ket{x}_2
  \right).
  \label{eqEntangBellBaseGen1}
\end{eqnarray}

\input ./part3/entang/figgen2.tex

Now, if a half-wave phase plate is placed before one of the beams, for example the first one, which changes the polarization to the orthogonal:
\begin{equation}
\ket{x}_2 \rightarrow \ket{y}_2, \, \ket{y}_2 \rightarrow \ket{x}_2,
\nonumber
\end{equation}
then from \eqref{eqEntangBellBaseGen1} we obtain
\begin{eqnarray}
  \left|\phi^{\dag}\right>_{12} = 
  \frac{1}{\sqrt{2}}\left(
  \ket{x}_1\ket{x}_2 + 
  \ket{y}_1\ket{y}_2
  \right),
  \nonumber \\
  \left|\phi^{-}\right>_{12} = 
  \frac{1}{\sqrt{2}}\left(
  \ket{x}_1\ket{x}_2 - 
  \ket{y}_1\ket{y}_2
  \right).
  \label{eqEntangBellBaseGen2}
\end{eqnarray}

Thus, combining \eqref{eqEntangBellBaseGen1} and \eqref{eqEntangBellBaseGen2}, we obtain the complete basis \eqref{eqEntangBellBase}.