%% -*- coding:utf-8 -*- 
\section{EPR Paradox for Stokes Parameters and Entangled States}
\label{sec:part3:epr}
Consider the single-photon state \eqref{eqEntangSimpleState}. 
To explain
the physical meaning of the parameters $\alpha$ and $\beta$, one needs to refer to the experimental setup scheme shown in
\autoref{figPart3EntangJones}. Suppose that the source $S$ emits
photons in the state \eqref{eqEntangSimpleState}. Then the average current
of photodetector $D_x$ will be proportional to
$\left|\alpha\right|^2$, and that of photodetector $D_y$ -
$\left|\beta\right|^2$, i.e., $P_x = \left|\alpha\right|^2$ and $P_y =
\left|\beta\right|^2$ describe the probabilities of detecting a photon in
a state polarized along $x$ - $\ket{x}$ or $y$ -
$\ket{y}$. The following question arises - "what is behind these
probabilities?". Two possible answers exist. The first one
assumes that we actually have incomplete knowledge about the source $S$,
i.e., if we knew everything about this source, we could
predict which photon, polarized along $x$ or along $y$, is created at
an arbitrary moment in time and, correspondingly, the readings of
photodetectors $D_x$ and $D_y$. From this point of view,
quantum mechanics represents some intermediate
theory, which can later be replaced by a more precise one,
describing quantum properties of objects with absolute accuracy. 

The second answer variant assumes that it is impossible to determine
all parameters describing some quantum system with arbitrary
precision because the measurement result is not predetermined in advance (as
in our case by properties of source $S$), but rather defined at the moment
of measurement as a result of the interaction between the quantum system and
the macroscopic measuring apparatus. 

The first attempt to answer this question was made by 
A. Einstein, B. Podolsky, and N. Rosen in 1935 in a paper often called EPR \cite{bEPR}. \rindex{EPR paradox} 
In EPR, a physical theory is considered
complete if every element of physical reality has a counterpart in
the physical theory. The following was adopted as a definition of an element of physical
reality: "if, without disturbing a system, we can predict with certainty (probability equal to unity) the value of some physical quantity, then there exists an element of reality corresponding to this
quantity" \cite{bBelokTimHrus}. Non-commuting operators are of particular interest here. In our case, these could be operators
$\hat{S}_1$ and $\hat{S}_2$. Since these operators do not commute
\eqref{eqEntangStokesOperS12Comm}, the values of the corresponding observables cannot be measured with arbitrary
precision. From this, two assumptions can be made:
\begin{enumerate}
\item The quantum mechanical description of reality is incomplete
\item The quantities defined by operators $\hat{S}_1$ and $\hat{S}_2$ cannot be simultaneously real
\end{enumerate}

As a test system, following EPR, we consider a complex
system consisting of several particles. At the same time, some overall wave function is used to describe the entire
system.

To describe the polarization properties of a complex system consisting of two
photons, one must use a wave function of the following form:
\begin{eqnarray}
\left|\psi\right> = \sum_{i,j = x,y} 
c_{ij}\ket{i}_1\ket{j}_2 = 
\nonumber \\
= c_{xx} \ket{x}_1\ket{x}_2 +
c_{xy} \ket{x}_1\ket{y}_2 +
c_{yx} \ket{y}_1\ket{x}_2 +
c_{yy} \ket{y}_1\ket{y}_2.
\label{eqEntang2Common}
\end{eqnarray}
If two or more coefficients $c_{ij}$ in \eqref{eqEntang2Common}
are nonzero, then the wave function is non-factorizable, i.e., 
\[
\left|\psi\right> \ne \left|\psi\right>_1 \left|\psi\right>_2.
\]
Thus, no single wave function can be assigned to each separate photon. Such states are called entangled.

\begin{remark}
Consider the following pure entangled state
\[
  \ket{\psi_{12}} = \frac{
    \ket{ x }_1\ket{ y }_2 -
    \ket{ y }_1\ket{ x }_2
  }{\sqrt{2}},
\]
whose density matrix has the form
\begin{eqnarray}
\hat{\rho}_{12} = \ket{\psi_{12}}\bra{\psi_{12}} =
\nonumber \\ 
\frac{1}{2}
\left(
\ket{ x }_1\ket{ y }_2 -
\ket{ y }_1\ket{ x }_2
\right)
\left(
\bra{ x }_1\bra{ y }_2 -
\bra{ y }_1\bra{ x }_2
\right) =
\nonumber \\
= 
\frac{
\ket{ x }_1\ket{ y }_2\bra{ y }_2\bra{ x }_1 +
\ket{ y }_1\ket{ x }_2\bra{ x }_2\bra{ y }_1 }
{2} -
\nonumber \\
-
\frac{
\ket{ x }_1\ket{ y }_2\bra{ x }_2\bra{ y }_1 +
\ket{ y }_1\ket{ x }_2\bra{ y }_2\bra{ x }_1 
}
{2}.
\nonumber
\end{eqnarray}
At the same time, the density matrix of the second photon has the form
(see \autoref{sec:add:quantum:composite})
\begin{eqnarray}
\hat{\rho}_2 = \mathrm{Tr}_1 \hat{\rho}_{12} = 
\bra{x}_1 \hat{\rho}_{12} \ket{x}_1 +
\bra{y}_1 \hat{\rho}_{12} \ket{y}_1 =
\nonumber \\
=
\frac{
\ket{ y }_2\bra{ y }_2 +
\ket{ x }_2\bra{ x }_2 }
{2}.
\nonumber
\end{eqnarray}
Thus, the state of the second photon is mixed.

This astonishing fact greatly distinguishes
quantum systems from classical ones: if we have a pure state
\rindex{Pure state}
of two particles, then the corresponding states of individual particles are not
necessarily pure. In particular,
if the state of the Universe is a pure quantum state
\cite{PhysRevD.28.2960}, then 
the states of individual parts of this Universe (stars, planets, etc.) are not
necessarily pure and can be mixed, i.e.,
classical. 

\end{remark}

As an example, we study the following wave function:
\begin{equation}
  \ket{\psi} = \frac{
    \ket{ + }_1\ket{ - }_2 -
    \ket{ - }_1\ket{ + }_2
  }{\sqrt{2}},
\label{eqEntang2Test}
\end{equation}
i.e., all we know about the photons is that they have different
circular polarizations. With 50\% probability the first photon has left
circular polarization, and in this case the second photon has right
circular polarization. Conversely, with 50\% probability the first photon
has right circular polarization, and the second photon has
left circular polarization. 

\input ./part3/entang/figent.tex

To measure the polarization properties of the state \eqref{eqEntang2Test},
the scheme shown in \autoref{figEntangMes} can be used. 
In this scheme, entangled photons from the source $S$ are fed to two detectors
of Stokes parameters $D^{(1,2)}$. Detector $D^{(1)}$ measures the average
values of the Stokes operators for the first photon - $\hat{S}_k^{(1)}$, and
$D^{(2)}$ measures the average values of the Stokes operators
$\hat{S}_k^{(2)}$ for the second photon. The setup of each detector
is identical to the scheme shown in \autoref{figPart3EntangStokes}.

Since the eigenvalues of the Stokes operators
\eqref{eqEntangStokesOper} are $s = \pm 1$, at the detector outputs we
will obtain either $+1$ or $-1$. Suppose we measure the Stokes parameter
$\hat{S}_1^{(1)}$ for the first photon. Suppose the measurement outcome is $+1$. At this point, the wave function
\eqref{eqEntang2Test} undergoes reduction. The reduction can be described by the projection operator (see \autoref{AddDiracProjector}) onto the state
$\ket{x}_1$:
\[
\hat{P}_{\ket{x}_1} = \ket{x}_1\bra{x}_1,
\]
and the wave function
\eqref{eqEntang2Test} transforms as follows:
\begin{eqnarray}
  \hat{P}_{\ket{x}_1}\left|\psi\right> =
  \ket{x}_1\bra{x}_1 \left|\psi\right> =
  \nonumber \\
    =
  \ket{x}_1\bra{x}_1
  \frac{
    \left( \ket{x}_1 + i \ket{y}_1 \right)\ket{ - }_2 -
    \left( \ket{x}_1 - i \ket{y}_1 \right)\ket{ + }_2
  }{2} =
  \nonumber \\
  =
  \ket{x}_1
  \frac{\ket{ - }_2 - \ket{ + }_2}{2} = -
  \frac{1}{2\sqrt{2}}\ket{x}_1 \left(2 i\right)
  \ket{y}_2 = -
  \frac{i}{\sqrt{2}}\ket{x}_1\ket{y}_2.
\nonumber
\end{eqnarray}
Therefore, after measuring $\hat{S}_1^{(1)}$, the wave function of the entire
system will be written as
\[
\left|\psi\right>_{red} = \ket{x}_1\ket{y}_2.
\]
This means that the second photon will be in the state $\ket{y}_2$,
i.e., the reading of detector $D^{(2)}$ will be $-1$.

Similarly, if the reading of the first detector was $-1$, the
wave function after measurement will be equal to
\begin{eqnarray}
  P_{\ket{y}_1}\left|\psi\right> =
  \ket{y}_1\bra{y}_1 \left|\psi\right> =
  \nonumber \\
  =
  \ket{y}_1\bra{y}_1
  \frac{
    \left( \ket{x}_1 + i \ket{y}_1 \right)\ket{ - }_2 -
    \left( \ket{x}_1 - i \ket{y}_1 \right)\ket{ + }_2
  }{2} =
  \nonumber \\
  =
  \ket{y}_1 i 
  \frac{\ket{ - }_2 + \ket{ + }_2}{2} =
  \frac{i}{2\sqrt{2}}\ket{y}_1 \left(2\right)
  \ket{x}_2 =
  \frac{i}{\sqrt{2}}\ket{y}_1\ket{x}_2,
  \nonumber
\end{eqnarray}
i.e.,
\[
\left|\psi\right>_{red} = \ket{x}_2\ket{y}_1,
\]
the state of the second photon will be described by the state
$\ket{x}_2$ and the reading of detector $D^{(2)}$ will be $+1$.
Thus, for any measurement of $\hat{S}_1^{(1)}$, we can
reliably (with probability equal to 1) predict the measurement result of
$\hat{S}_1^{(2)}$, without directly affecting the second photon, i.e.,
the quantity $\hat{S}_1^{(2)}$ should be considered an element
of physical reality.

At the same time, the Stokes parameters for the second particle,  $\hat{S}_1^{(2)}$ and
$\hat{S}_2^{(2)}$ have 
different eigenvectors:
$\ket{x}_2, \ket{y}_2$ (\ref{eq:part2:pol:stocks_s1_1},
  \ref{eq:part2:pol:stocks_s1_2}) and
$\frac{1}{\sqrt{2}}\left(\ket{x}_2 \pm
\ket{y}_2\right)$ (\ref{eq:part2:pol:stocks_s2_1},
  \ref{eq:part2:pol:stocks_s2_2}), respectively.
Thus, the first and second Stokes parameters for the second particle
cannot be measured simultaneously; indeed, otherwise,
the measured value would correspond to  
a certain state vector that would be an eigenvector of both
operators $\hat{S}_1^{(2)}$ and
$\hat{S}_2^{(2)}$ (see also
\autoref{AddHeisenbergUncertaintyPrincipleMesuranmet}). 


%% In the considered state, the average value of Stokes parameter
%% $\left<\hat{S}_3^{(2)}\right>$ has the form
%% \begin{eqnarray}
%%   \left<\hat{S}_3^{(2)}\right> =
%%   \left<\psi\right|\hat{S}_3^{(2)}\left|\psi\right> =
%%   \left<\psi\right|\hat{S}_3^{(2)}\frac{
%%     \ket{ + }_1\ket{ - }_2 -
%%     \ket{ - }_1\ket{ + }_2
%%   }{\sqrt{2}} =
%%   \nonumber \\
%%   = \left<\psi\right|\frac{
%%     \ket{ + }_1\ket{ - }_2 +
%%     \ket{ - }_1\ket{ + }_2
%%   }{\sqrt{2}} =
%%   \nonumber \\
%%   = \frac{1}{2} \left(
%%   \bra{ + }_1\bra{ - }_2 -
%%   \bra{ - }_1\bra{ + }_2
%%   \right)
%%   \left(
%%   \ket{ + }_1\ket{ - }_2 -
%%   \ket{ - }_1\ket{ + }_2
%%   \right) = 1.
%%   \label{eqEntangS3Mean}
%% \end{eqnarray}

%% The derivation of \eqref{eqEntangS3Mean} employed expressions
%% \eqref{eqEntangS3Eigenvec}. Thus, from the Heisenberg inequality
%% \eqref{eqAddHeisenbergUncertaintyPrinciple} one obtains
%% \begin{eqnarray}
%%   \Delta s_1^{(2)} \Delta s_2^{(2)} \ge
%%   \frac{\left|\left<\left[\hat{S}_1^{(2)},
%%       \hat{S}_2^{(2)}\right]/i\right>\right|}{2} =
%%   \nonumber \\
%%   = \frac{2 \left|\left< \hat{S}_3^{(2)} \right> \right|}{2} = 1,
%%   \nonumber
%% \end{eqnarray}
%% i.e., quantum mechanics implies that the first and second Stokes
%% parameters for the second particle cannot be measured simultaneously. 


%% On the other hand, we could perform a measurement of
%% $\hat{S}_2^{(1)}$ in detector $D^{(1)}$. In this case, we would observe two values at the detector outputs: $+1$ or $-1$. For the first of these, the wave function \eqref{eqEntang2Test}
%% would transform as follows:
%% \begin{eqnarray}
%% \hat{P}_{\frac{\ket{x}_1 +
%%     \ket{y}_1}{\sqrt{2}}}\left|\psi\right> = 
%% \nonumber \\
%% =
%% \frac{1}{2}\left[
%% \left(\ket{x}_1 + \ket{y}_1\right)
%% \left(\bra{x}_1 + \bra{y}_1\right)
%% \left(
%% \ket{x}_1\ket{y}_2 -
%% \ket{y}_1\ket{x}_2
%% \right)
%% \right] = 
%% \nonumber \\
%% = \frac{1}{2}\left[
%% \left(\ket{x}_1 + \ket{y}_1\right)
%% \left(
%% \ket{y}_2 -
%% \ket{x}_2
%% \right)
%% \right] =
%% \nonumber \\
%% = \frac{\ket{x}_1 + \ket{y}_1}{\sqrt{2}}
%%  \frac{\ket{y}_2 - \ket{x}_2}{\sqrt{2}}.
%% \nonumber
%% \end{eqnarray}
%% Thus, when measuring $\hat{S}_2^{(2)}$ we obtain $-1$ as the result on detector $D^{(2)}$. If the reading of detector $D^{(1)}$ is $-1$, then the system state transforms to 
%% \begin{eqnarray}
%% \hat{P}_{\frac{\ket{x}_1 -
%%     \ket{y}_1}{\sqrt{2}}}\left|\psi\right> = 
%% \nonumber \\
%% =
%% \frac{1}{2}\left[
%% \left(\ket{x}_1 - \ket{y}_1\right)
%% \left(\bra{x}_1 - \bra{y}_1\right)
%% \left(
%% \ket{x}_1\ket{y}_2 -
%% \ket{y}_1\ket{x}_2
%% \right)
%% \right] = 
%% \nonumber \\
%% = \frac{1}{2}\left[
%% \left(\ket{x}_1 - \ket{y}_1\right)
%% \left(
%% \ket{y}_2 +
%% \ket{x}_2
%% \right)
%% \right] =
%% \nonumber \\
%% = \frac{\ket{x}_1 - \ket{y}_1}{\sqrt{2}}
%%  \frac{\ket{x}_2 + \ket{y}_2}{\sqrt{2}}
%% \nonumber
%% \end{eqnarray}
%% and the reading of detector $D^{(2)}$ will be $+1$. Thus, for
%% any measurement of $\hat{S}_2^{(1)}$ we can 
%% reliably (with probability equal to 1) predict the measurement result of
%% $\hat{S}_2^{(2)}$, so $\hat{S}_2^{(2)}$ should be considered an element of physical reality.

Now, if we assume that the results of experiments
are predetermined in advance, i.e., do not depend on which
quantity ($S_1$ or $S_2$) was measured, and since measurements on
the first photon are made in arbitrary order, it follows that for the second photon
$\hat{S}^{(2)}_1$ and
$\hat{S}^{(2)}_2$ must simultaneously be elements of physical
reality. This entails the thesis about the incompleteness of quantum mechanics.

We have a different situation when we accept the assumption that
the result of some measurement is determined at the moment the
experiment is performed. In this case, we can no longer claim that 
$\hat{S}_1^{(2)}$ and $\hat{S}_2^{(2)}$ should be {\bf simultaneously}
considered as elements of physical reality, so there are no
contradictions with the basic principles of quantum mechanics.

% If one measures the averaged values
% of the Stokes operators, i.e., averages the readings of detectors over some
% period of time, then we obtain
% \begin{eqnarray}
% \left<\hat{S}_k^{(1)}\right> =
% \left<\psi\right|\hat{S}_k^{(1)}\left|\psi\right> = 0,
% \nonumber \\
% \left<\hat{S}_k^{(2)}\right> =
% \left<\psi\right|\hat{S}_k^{(2)}\left|\psi\right> = 0.
% \nonumber
% \end{eqnarray}

% On the other hand, the wave function \eqref{eqEntang2Test} is an eigenstate of the operator $\hat{S}_k^{(1)}\hat{S}_k^{(2)}$:
% \begin{equation}
% \hat{S}_k^{(1)}\hat{S}_k^{(2)}\left|\psi\right> = -\left|\psi\right>
% \label{eqEntangCorrel}
% \end{equation}

% From expression \eqref{eqEntangCorrel} it follows that when measuring the mean value of the operator $\hat{S}_k^{(1)}\hat{S}_k^{(2)}$ one gets
% \begin{equation}
% \left<\hat{S}_k^{(1)}\hat{S}_k^{(2)}\right> =
% \left<\psi\right|\hat{S}_k^{(1)}\hat{S}_k^{(2)}\left|\psi\right> = 
% - \left<\psi\right|\left.\psi\right> = -1,
% \nonumber
% \end{equation}
% i.e., there is some correlation between the readings of detectors
% $D^{(1)}$ and $D^{(2)}$: when the output of the first detector is $+1$,
% the output of the second is always $-1$. Such correlations are also possible in
% the classical case. The difference between quantum and classical
% treatments lies in the quantitative magnitude of these correlations.
% To demonstrate this difference, one uses the test of the so-called
% Bell inequalities.