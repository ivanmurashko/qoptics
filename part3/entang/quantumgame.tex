%% -*- coding:utf-8 -*-
\section{Quantum Pseudo-Telepathic Games}

The nonclassicality of entangled states can be vividly
demonstrated in the so-called quantum pseudo-telepathic games
\cite{bPseudoTelepathy2003,bPseudoTelepathy2004}

\subsection{Game Description}
Consider the Mermin-Peres game in which two players,
Alice and Bob, play against a casino. 
Alice and Bob fill a $3 \times 3$ square with numbers $\pm 1$.
The casino tells Alice the number of the row that Alice must fill, and
Bob the number of the column. Alice's product of all digits must be
equal to $+1$, while Bob's must be $-1$. The players win if the number
at the intersection of the chosen row and column matches. They lose otherwise.

Alice and Bob are isolated from each other until they receive
from the casino the row and column numbers, respectively. Thus, they can
only agree on a strategy before this moment.

\begin{example}[Mermin-Peres Game]
Suppose Alice ($A$) received row number 1, and Bob ($B$) column number 3. In this case, a winning combination may look like  
\begin{eqnarray}
A = \left(
\begin{array}{ccc}
+1 & -1 & -1 \\
\ast & \ast & \ast \\
\ast & \ast & \ast  
\end{array}
\right),
\nonumber \\
B = \left(
\begin{array}{ccc}
\ast & \ast & -1 \\
\ast & \ast & +1 \\
\ast & \ast & +1  
\end{array}
\right).
\nonumber 
\end{eqnarray}
This combination is winning because both Alice and Bob have the same number $-1$ 
at the intersection of the selected row and column.


In the opposite case (row 1, column 1) 
\begin{eqnarray}
A = \left(
\begin{array}{ccc}
+1 & -1 & -1 \\
\ast & \ast & \ast \\
\ast & \ast & \ast  
\end{array}
\right),
\nonumber \\
B = \left(
\begin{array}{ccc}
-1 & \ast & \ast \\
+1 & \ast & \ast \\
+1 & \ast & \ast  
\end{array}
\right),
\nonumber 
\end{eqnarray}
the players lose, since Alice obtains $+1$ at the intersection, while Bob has $-1$. 
\end{example}


\subsection{Classical Strategy}
In the classical strategy, Alice and Bob cannot agree on all
elements of the matrix because Alice's and Bob's conditions conflict:
according to Alice, the product of all elements of the matrix
equals $(+1)^3 = 1$, while according to Bob it equals $(-1)^3 = -1$.
Thus, sometimes Alice and Bob win, but there will always be games in which
they lose, since the probability of winning $p < 1$.

\subsection{Quantum Strategy}
As we have found, the classical strategy does not exist, but on the other
hand, if Alice and Bob can create in advance some
entangled state and agree on the measurements they
will perform on this state
\footnote{
naturally, the measurements are performed by Alice and Bob separately, each in their own laboratory
}, then there is a way to ensure a winning probability $p=1$.

The entangled state used by Alice and Bob:
\begin{equation}
\ket{\psi} = \ket{\psi}_{A_1,B_1} \otimes \ket{\psi}_{A_2,B_2}
\nonumber
\end{equation}

The matrix used for measurements
\begin{equation}
X =
\left(
\begin{array}{ccc}
\hat{S}_0 \otimes \hat{S}_3 & \hat{S}_3 \otimes \hat{S}_0 & \hat{S}_3 \otimes \hat{S}_3 \\
\hat{S}_1 \otimes \hat{S}_0 & \hat{S}_0 \otimes \hat{S}_1 & \hat{S}_1 \otimes \hat{S}_1 \\
- \hat{S}_1 \otimes \hat{S}_3 & - \hat{S}_3 \otimes \hat{S}_1 & \hat{S}_2 \otimes \hat{S}_2  
\end{array}
\right)
\label{eq:quantumgames:matrix}
\end{equation}

Since 
\[
\hat{S}_1^2 = \hat{S}_2^2 = \hat{S}_3^2 = \hat{S}_0^2 = \hat{I}
\]
for each element $\hat{x}_{ij}$ of the matrix
\eqref{eq:quantumgames:matrix} we have $\hat{x}_{ij}^2 = \hat{I}$, and
therefore upon measurement we get eigenvalues $\pm 1$.

Moreover, Alice and Bob can perform measurements since the operators in
the same row or the same column commute, i.e., can be measured
simultaneously. The values Alice obtains satisfy
the requirements imposed on the row, and Bob's on the column.

Finally, the measurements performed by Alice and Bob at the matching
position also coincide. This follows from the relation
\[
\left(\hat{x}_{ij}^A \otimes \hat{x}_{ij}^B\right) \ket{\psi} = \ket{\psi},
\]
where $\hat{x}_{ij}^A$ is Alice's measurement, and $\hat{x}_{ij}^B$ is Bob's measurement. Thus the product of the results obtained by Alice and Bob
is equal to $1$, or, since the results can only be $\pm 1$, Alice's and Bob's results coincide. 


TBD
