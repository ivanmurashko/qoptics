%% -*- coding:utf-8 -*- 
\section{Bell Inequality for Stokes Parameters}
\label{pPart3EntangleBell}
The answer to the question about the completeness of quantum mechanics can be given by the following experiment \cite{bBell, bPhisQuantInfo, PhysRevLett.28.938}.

\rindex{photon}Suppose that the system under consideration consists of two photons with the joint wave function of the form
\begin{equation}
\left|\psi\right> = \frac{
\ket{x}_1\ket{y}_2 -
\ket{y}_1\ket{x}_2
}{\sqrt{2}}.
\label{eqBellInequalityState}
\end{equation}

Let detector $D^{(1)}$ (\autoref{figEntangMes}) measure the quantity 
\[
\hat{A} = \hat{S}_1^{(1)}, \, (\xi = 0)
\] 
or
\[
\hat{A}' = \hat{S}_2^{(1)},\,(\xi =\frac{\pi}{4}).
\]
As noted earlier, the eigenvalues of operators $\hat{A}$ and
$\hat{A}'$ will be $\pm 1$, i.e., the readings of detector $D^{(1)}$ can
only be two numbers: $\pm 1$.

Receiver $D^{(2)}$ will measure the following quantities 
\[
\hat{B} = \frac{1}{\sqrt{2}}\left(\hat{S}_1^{(2)} + \hat{S}_2^{(2)}\right),\,(\xi =
\frac{\pi}{8}) 
\]
or 
\[
\hat{B}' = \frac{1}{\sqrt{2}}\left(\hat{S}_1^{(2)} - \hat{S}_2^{(2)}\right),\,(\xi =
- \frac{\pi}{8}).
\]
To find the possible readings of $D^{(2)}$, it is necessary to find
the eigenvalues of operators $\hat{B}$ and $\hat{B}'$. Using
the matrix representations of operators $\hat{S}_1$
\eqref{eqEntangS1Matrix} and $\hat{S}_2$ \eqref{eqEntangS2Matrix},
we obtain for operator $\hat{B}$ the following matrix representation:
\begin{equation}
\hat{B} = 
\left(
\begin{array}{cc}
\frac{1}{\sqrt{2}} & \frac{1}{\sqrt{2}} \\
\frac{1}{\sqrt{2}} & -\frac{1}{\sqrt{2}} 
\end{array}
\right).
\label{eqEntangBMatrix}
\end{equation}
From \eqref{eqEntangBMatrix} we can obtain the characteristic
equation:
\begin{equation}
\left(1 -\sqrt{2} b\right)\left(- 1 -\sqrt{2} b\right) -1 = 0,
\label{eqEntangBCharakterEq}
\end{equation}
from which the eigenvalues are: $b = \pm 1$. In the case of operator
$\hat{B}'$, we can obtain from \eqref{eqEntangS1Matrix} and
\eqref{eqEntangS2Matrix} the following matrix representation:
\begin{equation}
\hat{B}' = 
\left(
\begin{array}{cc}
\frac{1}{\sqrt{2}} & -\frac{1}{\sqrt{2}} \\
-\frac{1}{\sqrt{2}} & -\frac{1}{\sqrt{2}} 
\end{array}
\right).
\label{eqEntangBaddMatrix}
\end{equation}
and the corresponding characteristic equation
\begin{equation}
\left(1 -\sqrt{2} b\right)\left(- 1 -\sqrt{2} b\right) -1 = 0.
\label{eqEntangBaddCharakterEq}
\end{equation}
As is easy to see, equation \eqref{eqEntangBaddCharakterEq} has
the same solutions as \eqref{eqEntangBCharakterEq}, thus
the readings of $D^{(2)}$, just like for $D^{(1)}$, will be only two
values $\pm 1$.

Four series of experiments are conducted, each with $N$ trials, in which
pairs of operators $\left(\hat{A},\hat{B}\right)$,
$\left(\hat{A}',\hat{B}\right)$, $\left(\hat{A},\hat{B}'\right)$, and
$\left(\hat{A}',\hat{B}'\right)$ are measured. As a result, the following
sets of numbers $\left(a_i, b_i\right)$, $\left(a_i', b_i\right)$, $\left(a_i, b_i'\right)$, and
$\left(a_i', b_i'\right)$ are obtained.
As was just shown, each
of the obtained numbers can be either $+1$ or $-1$.

Next, from these pairs, the following value is calculated
\begin{equation}
f_i = \frac{1}{2}\left(
a_i b_i + a_i' b_i + a_i b_i' - a_i' b_i'
\right)
\nonumber
\end{equation}
for which the average is computed as
\begin{equation}
\left<F\right>_N = \frac{1}{N}\sum_i f_i.
\label{eqEntangFmain}
\end{equation}
As $N \rightarrow \infty$ one can assume
\begin{equation}
\left<F\right>_N \rightarrow \left<F\right>.
\nonumber
\end{equation}

\begin{remark}[About the Bell Experiment]
As noted above, in the experiment we perform measurements of the following
pairs of physical quantities $\left(\hat{A},\hat{B}\right)$,
$\left(\hat{A}',\hat{B}\right)$, $\left(\hat{A},\hat{B}'\right)$, and
$\left(\hat{A}',\hat{B}'\right)$, and for computing
\eqref{eqEntangFmain} it is necessary to assume that all these measurements
are made in one go, i.e., simultaneously. In the classical case, unlike the quantum case, this is allowed. In addition,
from the classical point of view, $\hat{A}$ and $\hat{A}'$ are two independent
observables,
as are $\hat{B}$ and $\hat{B}'$. At the same time, from the quantum perspective,
they are no longer independent, for example, for $\hat{A}, \hat{A}'$ we have \eqref{eqEntangStokesOperS12Comm}:
\[
\left[\hat{A}, \hat{A}'\right] = 2 i \hat{S}_3.
\] 
Furthermore, from the classical point of view, the measurements of quantities
$\hat{A}, \hat{A}'$ on one hand and $\hat{B}, \hat{B}'$ on the other are also
independent because they are performed at different points in space.
Thus, in the classical case the four pairs $\left(\hat{A},\hat{B}\right)$,
$\left(\hat{A}',\hat{B}\right)$, $\left(\hat{A},\hat{B}'\right)$, and
$\left(\hat{A}',\hat{B}'\right)$
are independent, and it does not matter whether they are measured in the same experiment 
or in several.

It should be noted that although $\hat{A}$ and $\hat{A}'$
are dependent from the quantum point of view, we cannot measure them
in a single experiment simultaneously (see
\autoref{AddHeisenbergUncertaintyPrincipleMesuranmet}); at the same time,
measurements at different points in space are connected through the joint wave
function, which leads to correlation between the quantities $\hat{A}, \hat{A}'$ on the one hand and
$\hat{B}, \hat{B}'$ on the other, and consequently to differing results of the averaging for the quantity \eqref{eqEntangFmain}.
\end{remark}

The quantum mechanical approach gives the following expression for the average in
the considered state \eqref{eqBellInequalityState} 
\begin{eqnarray}
 \left<F\right> \sim \left<F\right>_{quant} 
=\frac{1}{2}
\left<\psi\right|
\hat{A}\hat{B} + \hat{A}'\hat{B} + \hat{A}\hat{B}' - \hat{A}'\hat{B}'
\left|\psi\right> = 
\nonumber \\
=\frac{1}{2}
\left<\psi\right|
\hat{A}\left(\hat{B} + \hat{B}'\right) + \hat{A}' \left(\hat{B}  -
\hat{B}' \right)
\left|\psi\right> = 
\nonumber \\
= \frac{1}{\sqrt{2}}
\left<\psi\right|
\hat{S}_1^{(1)}\hat{S}_1^{(2)} + \hat{S}_2^{(1)}\hat{S}_2^{(2)}
\left|\psi\right> =
\nonumber \\
= \frac{1}{\sqrt{2}}
\left(-1 - 1\right) = - \sqrt{2}.
\label{eqEntangQuant}
\end{eqnarray}

If we accept the thesis of the incompleteness of quantum mechanics, then we
assume that the polarization of photons, and hence the Stokes parameters,
are defined immediately after the two photons leave the source
$S$.  
Thus, there exist a priori values $a$, $a'$, $b$, and $b'$,
and the properties of the source can be described using
classical probability theory.
Thus, there exist 16 elementary
probabilities $p\left(a,b,a',b'\right)$ such that the average value can
be written as
\begin{equation}
 \left<F\right> \sim \left<F\right>_{class} 
=\sum_{a,b,a',b'=\pm 1} 
p\left(a,b,a',b'\right) f\left(a,b,a',b'\right),
\label{eqEntangClassFuncPre}
\end{equation}
where
\begin{eqnarray}
 f\left(a,b,a',b'\right) = \frac{1}{2} 
\left(
ab + a'b + ab' - a'b'
\right) = 
\nonumber \\
=
\frac{1}{2} 
\left(
a \left(b + b'\right) + a' \left(b - b'\right)
\right).
\label{eqEntangClassFunc}
\end{eqnarray}

Function \eqref{eqEntangClassFunc} can only take two values:
$f = \pm 1$. Indeed, there are two possibilities: $b = b'$
or $b = - b'$. In the first case,
\[
f = \frac{1}{2}\left(2ab\right) = \pm 1.
\]
In the second case,
\[
f = \frac{1}{2}\left(2a'b\right) = \pm 1.
\]
Thus, function $f$ can take values within the following
interval $f_{min} = -1 \le f \le f_{max} = +1$ (unlike the measurable quantity
$f_i = 0, \pm 1, \pm 2$). It is clear that in the classical case
\begin{equation}
\left|\left<F\right>_{class} \right| 
\le 1.
\label{eqEntangClass}
\end{equation}

Comparing expressions \eqref{eqEntangQuant} and \eqref{eqEntangClass},
which are called Bell inequalities,
one can see that quantum correlations have a greater quantitative
value. This quantitative difference can be tested in
an experiment. The first experiment was conducted in 1972
\cite{PhysRevLett.28.938} and its results confirm
the completeness of the quantum description. The experiment verifying the completeness of quantum
mechanics can have not only theoretical importance but also practical
applications, for instance, in quantum cryptography, which we will
consider below (see \ref{subsecPart3QuantInfoQuantCrypto}). 

It should be noted that expression \eqref{eqEntangQuant} can be
obtained using formula \eqref{eqEntangClassFuncPre}, provided
it is accepted that some of the probabilities
$p\left(a,b,a',b'\right) < 0$. Thus we can speak about
the non-classicality of the entangled state.  
