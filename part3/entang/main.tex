\chapter{Entangled States}
\label{ch:entangl}

Since the inception of quantum mechanics, the question of the completeness of this theory has arisen. The accuracy of quantum mechanics' predictions is very high, but it only predicts the probabilities of various events. In particular, there is no possibility to measure both the coordinate and momentum of a particle with arbitrary precision. It seems that the probabilities inherent in the quantum mechanical description reflect its incompleteness, and possibly, there exists another theory that would have the accuracy of quantum mechanics without using a probabilistic approach.

It turned out that the probabilities underlying quantum mechanics have profound physical meaning and there are no theories where these can be discarded or where, for example, one could measure coordinates and momentum with arbitrary precision. A special role is played by what are known as entangled states, which describe systems consisting of several particles, with the behavior of such a composite system described by a general wave function.

Since entangled states are purely quantum, i.e., they have no classical analogs, they offer the possibility to observe phenomena that seem utterly impossible from a classical perspective, such as quantum teleportation. Moreover, in recent times, practical applications of entangled states have emerged, such as quantum dense coding (see \ref{subsecPart3QuantInfoBigCoding}) and quantum cryptography (see \ref{subsecPart3QuantInfoQuantCrypto}).

There are several methods for obtaining entangled photons \cite{bPhisQuantInfo}, related to quantum optics, among which polarization entangled states are noteworthy. This is because there exists a large number of methods for controlling polarization characteristics as well as methods for measuring the polarization properties of light.

%\input ./part3/entang/epr.tex

\input ./part3/entang/entang.tex

\input ./part3/entang/bell.tex

\input ./part3/entang/bellbase.tex

\input ./part3/entang/gener.tex

\input ./part3/entang/reg.tex

\input ./part3/entang/teleport.tex

\input ./part3/entang/quantumgame.tex

\section{Exercises}
\begin{enumerate}
\item Prove the orthonormality of the Bell basis states \eqref{eqEntangBellBase}.
\item Which Bell state should Alice register in the scheme shown in \autoref{figTeleport}, to confirm the fact of teleportation in the case when the source of entangled photon pairs $S$ produces photons in the state
\begin{equation}
  \left|\psi\right>_{23} = \left|\psi^{\dag}\right>_{23} = \frac{1}{\sqrt{2}}\left(
  \ket{x}_2\ket{y}_3 +
  \ket{y}_2\ket{x}_3
  \right).
  \nonumber
\end{equation}
\item Derive expressions for the coefficients 
$c_{\left|\psi^{\dag}\right>_{12}}$, 
$c_{\left|\phi^{\dag}\right>_{12}}$ and 
$c_{\left|\phi^{-}\right>_{12}}$
in decomposition \eqref{eqPart3EntangTeleportsepar}
\end{enumerate}