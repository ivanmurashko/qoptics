%% -*- coding:utf-8 -*- 
\section{Entangled states}
Definition. EPR paradox. Bell inequalities. Experimental
verification. Quantum teleportation.

\section{EPR paradox.}

For a long time, quantum mechanics seemed to have two parts. The first was purely practical and allowed, for example, to calculate
the emission spectra of atoms and molecules. 

The second part of quantum mechanics was more philosophical than
practical. The main question to be answered here was 
“What is the quantum mechanical description of the world?”

Many distinguished scientists such as Albert Einstein found it difficult to accept
quantum mechanics with its probabilistic nature. 

At the dawn of quantum mechanics (early 20th century)
there were two main interpretations of quantum mechanics. The first was 
the so-called statistical interpretation whose main proponent was Albert Einstein. According to the statistical interpretation,
the state of an individual quantum particle is meaningless (or in other words, there is not yet a theory that fully explains this
state). What matters is some statistical ensemble
of quantum particles for which probabilities can be calculated almost
in the same way as in statistical physics. 

The second interpretation of quantum mechanics is called the Copenhagen
interpretation. The author of this interpretation was Niels Bohr. According to this interpretation, the quantum mechanical
description is complete and can be used to describe
individual particles.

Obviously, the second interpretation is more general and includes
the first one (the converse is not true). One of the most famous attempts
by Albert Einstein to defend his viewpoint and demonstrate the incompleteness
of quantum mechanics was the article \cite{bEPR}. This article describes the so-called
Einstein-Podolsky-Rosen (EPR) paradox. 

Later we will consider the EPR paradox using the example of a pair of photons in an entangled 
polarization state. The wave function of an individual photon can be 
represented as follows:
\begin{equation}
\left|\psi\right> = \alpha \ket{V} +
\beta \ket{H}, 
\label{eqEntaglementPSI}
\end{equation}
where $\ket{V}$ denotes vertical polarization and 
$\ket{H}$ horizontal polarization. In a single
measurement, the measuring device can only show one of two 
results: the photon is either vertically or horizontally polarized. At the same time,
there occurs the so-called wave function collapse:
\begin{eqnarray}
\left|\psi\right> \rightarrow \ket{V} \mbox{ upon detection of vertical polarization},
\nonumber \\
\left|\psi\right> \rightarrow \ket{H} \mbox{ upon detection of horizontal polarization}.
\nonumber
\end{eqnarray}
Mathematically, the collapse process is expressed as the action on the wave function of the following 
projection operators.\footnote{For more details about projection
  operators see \autoref{AddDiracProjector}}
\begin{eqnarray}
\ket{V}\bra{V} \mbox{ for vertical polarization},
\nonumber \\
 \ket{H}\bra{H} \mbox{ for horizontal polarization}.
\label{eqEntaglementProjector}
\end{eqnarray}
This gives us $\ket{V}\bra{V}\left|\psi\right> = \alpha \ket{V}$ 
upon detection of vertical polarization and
$\ket{H}\bra{H}\left|\psi\right> = \beta \ket{H}$ 
upon detection of horizontal polarization.

Now suppose we have a particle source that emits two photons whose polarizations are mutually orthogonal. Such a system can be based on the spontaneous 
parametric down-conversion effect \cite{bKlishko}. The wave function of such a composite system 
can be represented as
\begin{equation}
\left|\psi\right> = \frac{1}{\sqrt{2}}\left(
\ket{V}_1\ket{H}_2 - \ket{H}_1\ket{V}_2
\right).
\label{eqEntanglementEntaglement}
\end{equation}
Having originated from one source, these two photons fly off in different directions to two distinct
observers whom we shall, following tradition, call Alice and Bob. Suppose Alice detects photon 1 and 
Bob photon 2. On Alice's side, the effect of the measuring device in a single polarization measurement of 
photon reduces to the action of the projection operators \eqref{eqEntaglementProjector} on 
the wave function \eqref{eqEntanglementEntaglement}. If on Alice's side photon 1 is detected in the
vertically polarized state $\ket{V}_1$, then the wave function \eqref{eqEntanglementEntaglement}
collapses to
\begin{equation}
\ket{V}_1\bra{V}_1 
\frac{1}{\sqrt{2}} \left(
\ket{V}_1\ket{H}_2 - \ket{H}_1\ket{V}_2
\right) = \frac{1}{\sqrt{2}}
\ket{V}_1\ket{H}_2.
\nonumber
\end{equation}
Thus, the state of photon 2, detected by Bob, becomes definite and equal to $\ket{H}_2$.
If Alice had detected photon 1 in the horizontally polarized state, then Bob would with 100\% probability 
detect photon 2 with vertical polarization. Thus these two photons, even having separated by a large 
distance, remain somehow linked to each other, and the results of experiments
on one can affect the results of experiments on the other. 

It might seem that such an assumption contradicts the theory of relativity, which prohibits signaling faster than the speed of light. 
In this case, however, the disturbance would have to propagate instantaneously, since the photons can be anywhere 
at the time of measurement.

Nevertheless, there is no contradiction. According to quantum mechanics, the disturbance caused by measurement is random. In this case,
instantaneous propagation of this disturbance is not signaling, since it cannot carry information.

Indeed, imagine that on two planets at opposite ends of the Galaxy there are two coins that always land the same way. If you record all the outcomes
and then compare them, they will match. However, the outcomes themselves are random and cannot be influenced. One cannot, for example, agree
that heads is a 1 and tails a 0 and send a binary code. Because the sequence of zeros and ones will be random at both "ends of the line"
and will carry no meaning.

Thus, the paradox has an explanation that is logically consistent both with the theory of relativity and with quantum mechanics.

\section{Hidden variable theories. Bell inequalities}
There are several alternative ways to resolve the EPR paradox. The first, proposed by the group of authors in
\cite{bEPR}, is that the quantum-mechanical description of reality is incomplete and there exist some hidden variables,
consideration of which would eliminate the EPR paradox. According to this viewpoint,
quantum mechanics can be applied to describe ensembles of particles (the statistical interpretation of quantum mechanics),
but the description of the behavior of individual particles via the quantum mechanical formalism is not accurate. 

The effect of hidden variables can be illustrated using the example of a die. If we only know that the die is made of homogeneous material (without a shifted center of mass),
all we can say about any particular throw result is the probability of each face — $\frac{1}{6}$. However, if we take into account
the hidden parameters of the problem such as the force and direction of the throw, the initial orientation of the die, and others, then the problem becomes
an ordinary dynamical problem whose solution gives a well-defined answer as to which face will appear.

In 1964 John Bell showed \cite{bBell} that the predictions of quantum mechanics for the EPR paradox differ slightly from the
predictions of a fairly wide class of theories that operate with hidden variables. Roughly speaking, quantum mechanics
predicts stronger statistical correlations\footnote{About 70\% versus 50\% in the hidden variable theories case}
between results performed by Alice and Bob
than hidden variable theories. These differences take the form of inequalities known as Bell inequalities and can be experimentally tested.
Such tests have been performed \cite{bBellTest} and showed that the predictions of quantum mechanics are correct. 

FIX ME!!! derivation of Bell inequalities and experimental test scheme


%% \section{Quantum teleportation.}

%% Quantum mechanics has the so-called no-cloning theorem [FIX ME!!! add citation]. 
%% The meaning of this theorem can be explained by our example of a photon with two mutually orthogonal polarizations,
%% whose wave function is given by relation \eqref{eqEntaglementPSI}. The no-cloning theorem
%% states that it is impossible to create a device that on input receives a particle in state \eqref{eqEntaglementPSI}
%% and outputs two particles in the same state. 

%% Indeed, suppose we have a device that performs state cloning. The action of this device
%% on a photon in vertical polarization is described by the cloning operator $\hat{D}$ as follows:
%% \begin{equation}
%% \hat{D} \ket{D_I}\ket{V} = \ket{D_{FV}}\ket{V}\ket{V},
%% \nonumber
%% \end{equation}
%% where $\ket{D_I}$ is the wave function describing the initial state of the cloning device, and 
%% $\ket{D_{FV}}$ is the wave function describing the state of the device after cloning the photon 
%% with vertical polarization. For a photon in horizontal polarization, we have
%% \begin{equation}
%% \hat{D} \ket{D_I}\ket{H} = \ket{D_{FH}}\ket{H}\ket{H},
%% \nonumber
%% \end{equation}
%% where $\ket{D_{FH}}$ is the wave function describing the state of the device after cloning the photon 
%% with horizontal polarization.

%% If the cloning operator $\hat{D}$ acts on a photon in an arbitrary state \eqref{eqEntaglementPSI}, we get
%% \begin{equation}
%% \hat{D} \ket{D_I}\left|\psi\right> = 
%% \hat{D} \ket{D_I} \left(\alpha \ket{V} +
%% \beta \ket{H}\right) = 
%% \alpha \ket{D_{FV}}\ket{V}\ket{V} +
%% \beta \ket{D_{FH}}\ket{H}\ket{H},
%% \nonumber
%% \end{equation}
%% from which one cannot obtain the expected result
%% \begin{equation}
%% \hat{D} \ket{D_I}\left|\psi\right> = 
%% \ket{D_{FVH}}  \left(\alpha \ket{V} +
%% \beta \ket{H}\right)
%% \left(\alpha \ket{V} +
%% \beta \ket{H}\right).
%% \nonumber
%% \end{equation}

%% But if the state of the photon cannot be cloned, it turns out that it can be transferred from one point in space to another 
%% \footnote{naturally with destruction of the original state}, which is demonstrated by experiments on quantum teleportation.

%% \input ./part3/figteleport.tex

%% The scheme of the quantum teleportation protocol is shown in Fig. \ref{figTeleport}. In this figure, Alice wants to transmit to Bob 
%% the state of photon 1 described by the wave function \eqref{eqEntaglementPSI}.
%% \[
%% \left|\psi\right>_1 = \alpha \ket{V}_1 +
%% \beta \ket{H}_1, 
%% \]
%% There is also a source of EPR entangled photons emitting pairs 2 and 3 in the state described by the following wave function:
%% \begin{equation}
%% \left|\psi\right>_{23} = \frac{1}{\sqrt{2}}\left(
%% \ket{V}_2\ket{H}_3 - 
%% \ket{H}_2\ket{V}_3
%% \right).
%% \nonumber
%% \end{equation}

%% Alice mixes photons 1 and 2 on a beam splitter LD. The combined state of the 3 particles is described by
%% \begin{equation}
%% \left|\psi\right>_{123} = \left|\psi\right>_1 \left|\psi\right>_{23}.
%% \nonumber
%% \end{equation}
%% The wave function $\left|\psi\right>_{123}$ can be decomposed into the following basis called
%% the Bell states:
%% \begin{eqnarray}
%% \left|\psi^{\dag}\right>_{12} = 
%% \frac{1}{\sqrt{2}}\left(
%% \ket{V}_1\ket{H}_2 + 
%% \ket{H}_1\ket{V}_2
%% \right),
%% \nonumber \\
%% \left|\psi^{-}\right>_{12} = 
%% \frac{1}{\sqrt{2}}\left(
%% \ket{V}_1\ket{H}_2 - 
%% \ket{H}_1\ket{V}_2
%% \right),
%% \nonumber \\
%% \left|\phi^{\dag}\right>_{12} = 
%% \frac{1}{\sqrt{2}}\left(
%% \ket{V}_1\ket{V}_2 + 
%% \ket{H}_1\ket{H}_2
%% \right),
%% \nonumber \\
%% \left|\phi^{-}\right>_{12} = 
%% \frac{1}{\sqrt{2}}\left(
%% \ket{V}_1\ket{V}_2 - 
%% \ket{H}_1\ket{H}_2
%% \right).
%% \nonumber
%% \end{eqnarray}
%% The result of the decomposition is written as follows (FIX ME!!! check it)
%% \begin{eqnarray}
%% \left|\psi\right>_{123} = \left|\psi\right>_1 \left|\psi\right>_{23} = 
%% \nonumber \\
%% = \left|\psi^{-}\right>_{12} \frac{\alpha\ket{V}_3 + \beta\ket{H}_3}{2} + 
%% \nonumber \\
%% + 
%% \left|\psi^{\dag}\right>_{12} \frac{- \alpha\ket{V}_3 + \beta\ket{H}_3}{2} +
%% \nonumber \\
%% + 
%% \left|\phi^{\dag}\right>_{12} \frac{- \beta\ket{V}_3 + \alpha\ket{H}_3}{2} +
%% \nonumber \\
%% +
%% \left|\phi^{-}\right>_{12} \frac{\beta\ket{V}_3 + \alpha\ket{H}_3}{2}.
%% \nonumber
%% \end{eqnarray}

%% Thus, each time Alice detects photons 1 and 2 in the Bell state
%% $\left|\psi^{-}\right>_{12}$, photon 3 on Bob's side will be in the state
%% $\alpha\ket{V}_3 + \beta\ket{H}_3$, i.e., photon 1 is “teleported” to Bob.

%% The Bell state $\left|\psi^{-}\right>_{12}$ can be detected by Alice via simultaneous triggering 
%% of photodetectors PH1 and PH2 (FIX ME!!! add calculations). In the case of other Bell states, both photons will go
%% either to detector PH1 or PH2.

%% Two facts are worth noting:
%% \begin{itemize}
%% \item In this teleportation scheme, matter is not transmitted but some information about the quantum object.
%% \item Transmission of this information is instantaneous, but for Bob to know about the teleportation event, a classical communication channel 
%% between Alice and Bob is necessary \footnote{Through this channel, Alice tells Bob when both photodetectors fired, thus informing him about the teleportation}.
%% \end{itemize}