%% -*- coding:utf-8 -*- 
\input preamble.tex

\begin{document}
\Russian
\input title.tex

\section{Goals of the Course}
The goal of this course is to introduce students to theoretical and
experimental studies in a range of areas of modern quantum optics. Based
on the obtained knowledge, students will be able to study the sections
of quantum optics interesting to them in more depth and engage in
research work.

\section{Position of the Course in the Curriculum}
Quantum optics studies optical phenomena in which the quantum nature
of light manifests itself. It can be said that quantum optics considers
optical phenomena where light and the system interacting with it must
be described by quantum equations.

Quantum optics covers the frequency range approximately from
\(f_1 \simeq 10^{13} \mbox{Hz}\) to \(f_2 \simeq 10^{18}
\mbox{Hz}\), i.e., from the infrared range to the
X-ray range. The lower limit is determined by the condition that the
quantum energy exceeds the thermal motion energy:
\(\omega_1 \hbar > k T\). The upper limit
is set based on the fact that in quantum optics,
the electron energies are usually non-relativistic, and therefore,
the quantum energy should be significantly less than the rest energy of
the electron: \(\omega_2 \hbar \ll m c^2\).

\section{Course Load by Types of Academic Work and Forms of Assessment}
\subsection{Time Distribution by Types of Classes}
\begin{longtable}{|l|c|c|}
\hline
Program Sections & Semester & Lectures (hours) \\ \hline
Quantum Electrodynamics & 1 & 16 \\ 
Interaction of Light with Atom & 1 & 16\\ 
Quantum Laser Theory & 2 & 8 \\ 
Photon Optics & 2  & 24 \\ \hline
Total & & 64 \\ \hline
\end{longtable}

\subsection{Types of Classes and Forms of Assessment}
\begin{longtable}{|l|c|}
\hline
Lectures (per week) & 2 \\ 
Exams (per semester) & 1 \\
Credit (per semester) & None \\ \hline
\end{longtable}

\section{Course Content}
The course consists of four parts:
\begin{itemize}
\item Quantum Electrodynamics
\item Interaction of Light with Atom
\item Quantum Laser Theory
\item Photon Optics (Quantum Phenomena in Optics)
\end{itemize}

Below are brief annotations and contents for each of
the four parts of the course.

\subsection{Quantum Electrodynamics}
Quantum electrodynamics serves as the foundation of quantum optics. In the first
part of the course, the principles of the quantum theory of the electromagnetic field
necessary for studying and understanding quantum optics are presented. Detailed
coverage includes the following topics:
\begin{itemize}
\item Decomposition of electromagnetic field into modes (types of oscillations)
\item Hamiltonian form of electromagnetic field equations
\item Quantization of the electromagnetic field
\item Decomposition of the field into plane waves in free space
\item Density of states
\item Hamiltonian form of field equations upon decomposition into plane waves
\item Quantization of the electromagnetic field when decomposed into plane waves
\item Properties of operators \( \hat a \) and \( \hat a^\dag \)
\item Quantum states of the electromagnetic field with a definite energy
\item Multimode states
\item Coherent states
\item Mixed states of the electromagnetic field
\item Representation of the density operator via coherent states
\end{itemize}

\subsection{Interaction of Light with Atom}
This section addresses the interaction of the quantized electromagnetic
field with an atom. A simplified model of a two-level atom is used.
Such simplification is justified in resonance interactions and is widely
used in quantum electronics and quantum optics problems. Special attention
is given to the consideration of interaction between the atom, the resonator mode
(dynamical system), and the thermostat (dissipative system),
which is responsible for relaxation of the dynamical system.

This part of the course covers in detail the following issues:
\begin{itemize}
\item Emission and absorption of light by the atom
\item Hamiltonian of the atom-field system
\item Interaction of the atom with the mode of the electromagnetic field
\item Interaction of the atom with multimode field. Induced and spontaneous transitions
\item Relaxation of the dynamical system. Density matrix method
\item Interaction of the resonator electromagnetic field (harmonic oscillator)
  with a reservoir of atoms at temperature \(T\)
\item Equation for the field's density matrix in the number occupation representation
\item Equation of motion for the statistical operator of the mode field in
  the representation of coherent states
\item General theory of the interaction of a dynamical system with
  a thermostat (dissipative system, reservoir)
\item Damping (relaxation) of the field and atom in the simplest reservoir case,
  consisting of harmonic oscillators
\end{itemize}

\subsection{Quantum Laser Theory}
The semiclassical theory of the laser cannot answer all questions
arising in connection with its operation. According to this theory,
the laser does not generate at all before reaching threshold, and when
exceeding the threshold, it begins to generate classical electromagnetic field (light).

In reality, however, the laser generates chaotic light significantly below threshold,
and its radiation approaches classical above threshold. The threshold and vicinity
form a transitional region from chaotic light to ordered radiation. Only
a fully quantum theory can adequately describe this.

Another problem also requiring quantization of the electromagnetic field
is the determination of the ultimate (natural) linewidth of the laser
emission, where the linewidth is defined by quantum fluctuations of the field,
and various external influences, which are fundamentally eliminable, are not considered.

This part of the course covers in detail the following questions:
\begin{itemize}
\item Laser model
\item Theory of laser generation
\item Statistics of laser photons
\item Laser theory. Representation of coherent states
\end{itemize}

\subsection{Photon Optics (Quantum Phenomena in Optics)}
This section of the course examines optical phenomena in which quantum
properties of light manifest to some degree.

Although many optical phenomena can be considered from classical perspectives,
many can only be fully understood and described within a fully quantum framework.

Quantum consideration allows a deeper understanding of the essence
of interference experiments and thereby the connection between
classical and quantum descriptions. Moreover, the quantum approach
allows the study of new types of experiments investigating
photon statistics in light beams and its relation to the spectral
properties of light.

Detailed topics include:
\begin{itemize}
\item Photoelectric effect
\item Coherent properties of light
\item Second-order coherence
\item Higher-order coherence
\item Photon counting and statistics
\item Relationship between photon statistics and photoelectron count statistics
\item Distribution of photoelectron counts for coherent and chaotic light
\item Determination of photon statistics through the distribution
  of photoelectron counts
\item Quantum expression for photoelectron count distribution
\item Photon counting experiments. Application of photon counting
  techniques for spectral measurements
\end{itemize} 

\section{Educational and Methodical Support}
\begin{enumerate}
\item V. Yu. Petrunkin, O. I. Kotov Quantum Optics -
  St. Petersburg: St. Petersburg State Polytechnical University Publishing, 2003.  
\item L. Mandel, E. Wolf. Optical Coherence and Quantum Optics. Translated from English / ed. by V.V. Samartsev - Moscow:
  Nauka. PHIZMATLIT, 2000. - 896 pages.
\item V.V. Belokurov, O.D. Timofeevskaya,
  O.A. Khrustalev. Quantum Teleportation - Ordinary Miracle. - Izhevsk:
  NRC "Regular and Chaotic Dynamics". 2000. - 256 pages.
\item S.Ya. Kilin. Quantum Information. - Physics-Uspekhi (PU), 1999, vol. 169,
  No. 5, pp. 507-526.
\item D.N. Klyshko. Quantum optics: quantum,
  classical, and metaphysical aspects. - PU, 1994, vol.164, No. 11,
  pp. 1187-1214.
\item D.N. Klyshko. Nonclassical light. - PU, 1996,
  vol.166, No. 6, pp. 613-638.
\item Yu.I. Vorontsov. Phase of the oscillator in quantum
  theory. What is it really? - PU, 2002, vol.172, No. 8,
  pp. 907-929.
\item M.O. Scully, M.S. Zubairy. Quantum Optics. 1997,
  Cambridge University Press, UK, 635 pages.
\item R. Loudon. The Quantum Theory of Light. Third
  Edition. Oxford University Press, 2000, 438 pages.
\item Y. Yamamoto, A. Imamoglu. Mesoscopic quantum
  optics. 1999, USA, J. Wiley \& Sons, 301 pages.
\end{enumerate}

Program developed by: Professor, Doctor of Technical Sciences V. Yu. Petrunkin and
Candidate of Physical and Mathematical Sciences I. V. Murashko.

\end{document}