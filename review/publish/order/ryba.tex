In the study of the outermost atmosphere of bright normal stars of F--G spectral classes, a special role belongs to the analysis of emission lines on the shortwave side of the hydrogen H and K lines and the infrared triplet of calcium Сa II. They characterize the state of the upper chromosphere and are an important indicator of stellar activity. The article describes the previously proposed semi-empirical method for determining the chromospheric activity of stars from low- and medium-resolution spectra and presents the first results of an SiFAP spectral survey of stars with known physical parameters.

The goal of the work was to develop a technique for determining chromospheric activity from low-resolution spectroscopic observations suitable for implementation on telescopes with mirror diameters of 1--1.5 m and CCD detectors, as well as to test this technique on a system of stars with known activity levels.

The choice of the spectral range and resolution (about 3--4 Å) was based on the necessity of registration of the most indicative chromospheric activity indicators -- emission features in the cores of the calcium H and K lines (the fluxes in these lines) and in the Сa II triplet lines (the flux in the infrared triplet). Low resolution reduces the influence of rotational broadening, instrumental and atmospheric dispersion distortions, and significantly reduces requirements to the slit width and conjunctional effects on the signal. This study provides a semi-empirical approach for estimating chromospheric fluxes from these measurements.

The technique was tested on a set of stars with a wide range of known chromospheric activity parameters and effective temperatures, including the Sun at different activity levels and stars from the list by Duncan et al. (1991). The relation of the obtained chromospheric fluxes to the Mount Wilson S-index and other established activity indices was analyzed, making this method usable for comparative studies of stellar activity cycles.

This new semi-empirical method simplifies the use of chromospheric activity diagnostics in various astrophysical contexts, such as studies of stellar rotation, formation and evolution of stellar magnetic fields, and star-planet magnetic interactions. The proposed approach and the first results of the SiFAP spectral survey demonstrate its practical effectiveness and prospects for further expansion and refinement.