%% -*- coding:utf-8 -*- 
\section{Statistics of Laser Photons}
For equation \eqref{eqCh3_6}, one can visually represent the probability flows shown in \autoref{figPart2Ch1_2}. 

\input ./part2/laser/fig2.tex

The term $\frac{\omega}{Q} n \rho_{nn}$ represents the flow from the state $\ket{n}$ to the state $\ket{n - 1}$ due to photon absorption by atoms of the first reservoir. The same can be said for the term $\frac{\omega}{Q} \left(n + 1\right) \rho_{n + 1, n + 1}$ — the probability flow from the state $\ket{n + 1}$ to the state $\ket{n}$ for the same reasons. 

The term $\left[A - \left(n + 1\right) B\right]\left(n + 1\right) \rho_{nn}$ represents the flow from the state $\ket{n}$ to the state $\ket{n + 1}$ and characterizes the birth of photons due to interaction with active atoms, taking saturation effects into account. The same applies to the term $- \left(A - n B\right)n \rho_{n - 1, n - 1}$ — the probability flow from the state $\ket{n - 1}$ to the state $\ket{n}$.  

$\rho_{nn}$ has the meaning of the probability of detecting $n$ photons in the laser mode. In the transient regime, $\rho_{nn}\left(t\right)$ depends on time. In the steady-state regime, $\dot{\rho}_{nn} = 0$, and from equation \autoref{eqCh3_6} one can establish the photon statistics in the steady generation regime \cite{bScally1974}. 

Equating, for example, the probability flows between the states $\ket{n + 1}$ and $\ket{n}$ (using the principle of detailed balance), we obtain: 
\begin{equation}
\frac{\omega}{Q}\left(n + 1\right)\rho_{n + 1, n + 1} =
\left[A - \left(n + 1\right)B\right]\left(n + 1\right)\rho_{nn}
\label{eqCh3_9}
\end{equation}
Equality \eqref{eqCh3_9} gives an iterative relation
\begin{equation}
\rho_{n + 1, n + 1} = 
\frac{A - \left(n + 1\right)B}{\omega/Q} \rho_{nn}
\label{eqCh3_10}
\end{equation}
which allows expressing $\rho_{nn}$ through $\rho_{00}$. We have
\begin{equation}
\rho_{nn} = \rho_{00}\prod_{k = 1}^n
\frac{A - k B}{\omega/Q} 
\label{eqCh3_11}
\end{equation}
$\rho_{00}$ can be found from the normalization condition
$\sum_{(n)} \rho_{nn} = 1$.

Let us qualitatively consider using \eqref{eqCh3_11} $\rho_{nn}$ as a function of $n$, when the laser operates above threshold, i.e., when $A > \frac{\omega}{Q}$. In the steady-state regime, the gain equals the losses (on average), that is 
\[
A - \bar{n}_{st} B = \frac{\omega}{Q},
\]
whence the average number of photons in the steady-state regime equals
\[
\bar{n}_{st} = \frac{A - \omega/Q}{B}.
\] 

For $n \ll \bar{n}_{st}$, $B n \ll A$ the formula \eqref{eqCh3_11} can be represented as 
\[
\rho_{nn} \approx \left(\frac{A Q}{\omega}\right)^n.
\]
This quantity grows exponentially with increasing $n$, since $\frac{A Q}{\omega} > 1$, meaning that for small $n$ the probability increases with $n$, but the growth slows down as $n$ increases. For large $n$ the factor
$\frac{A - nB}{\frac{\omega}{Q}}$
approaches unity when $n$ approaches $\bar{n}_{st}$. 
At $n \approx \bar{n}_{st}$ the factor becomes equal to 1, and the distribution reaches a maximum: 
\[
\left.\frac{A - B k}{\omega/Q}\right|_{k = \bar{n}_{st}}
\rightarrow \frac{A - B \bar{n}_{st}}{\omega/Q} = 
\frac{A - A + \omega/Q}{\omega/Q} = 1.
\] 

For $n > \bar{n}_{st}$ and $k > \bar{n}_{st}$, the factors decrease, reaching zero at $k = \frac{A}{B}$. Note that for $k = \frac{A}{B}$ the factor becomes approximately zero, and terms with larger $k$ should be discarded, since $\rho_{nn}$ cannot be negative. The arising difficulty is related to the fact that we used perturbation theory and therefore assumed that the number of photons is not too large. Our theory is valid while $n < \frac{A}{B}$. These difficulties disappear if one uses the large signal theory developed in \cite{bScally1974}. From the above, it follows that $\rho_{nn}$ as a function of $n$ (the photon distribution in the laser field) first increases, reaching a maximum near $n = \bar{n}_{st}$, and then decreases to zero.   

\input ./part2/laser/fig3.tex

In \autoref{figPart2Ch1_3} a qualitative curve of the photon distribution is shown for the case of threshold excess. At threshold, $A = \frac{\omega}{Q}$. Then the factor $\frac{A - B k}{\omega/Q} = 1 - \left(\frac{B}{A}\right)k$ is less than unity for any $k$. The distribution curve will be monotonically decreasing. Below threshold, $A < \frac{\omega}{Q}$, then $\frac{A - B k}{\omega/Q} \approx \frac{A Q}{\omega} \ll 1$ and $\rho_{nn}$ decays exponentially, as $\left(\frac{A Q}{\omega}\right)^n$. The dependence of $\rho_{nn}$ on $n$ for all three cases can be numerically calculated using formula \eqref{eqCh3_11}. The results of such calculation are shown in \autoref{figPart2Ch1_4}. 

\input ./part2/laser/fig4.tex

\input ./part2/laser/fig5.tex

From the large signal theory \cite{bScally1974}, which we do not consider here, it follows that for very large threshold excess, 
$\rho_{nn} \rightarrow e^{-\bar{n}}\frac{\bar{n}^n}{n!}$, i.e., tends to the distribution characteristic of a coherent state. For moderate threshold excess, the photon distribution in laser radiation noticeably differs from the photon distribution in a coherent state. The difference between the photon distribution in the laser field and the Poisson distribution 
\rindex{Poisson distribution}
(corresponding to a coherent state) under moderate pumping is clearly shown in \autoref{figPart2Ch1_5}.