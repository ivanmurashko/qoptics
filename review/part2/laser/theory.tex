%% -*- coding:utf-8 -*- 
\section{Theory of Laser Generation}
Using equation \eqref{eqCh3_6}, one can obtain the equation for
the average number of photons in the mode 
\[
\left<n\left(t\right)\right> =\sum_{(n)}n\rho_{nn}.
\]
To do this, multiply equation \eqref{eqCh3_6} by $n$ and
sum over all $n$. We have
\[
\frac{d}{dt}\left<n\right> = \frac{d}{dt}\sum_n\rho_{nn}n = 
\sum_n\dot{\rho}_{nn}n.
\]
Thus from \eqref{eqCh3_6}
\begin{eqnarray}
\frac{d}{dt}\left<n\right> = 
-\sum_n n \left[A - \left(n + 1\right)B\right]\left(n + 1\right)\rho_{nn} +
\nonumber \\
+ \sum_n n\left(A - n B\right)n \rho_{n - 1, n - 1} 
- \sum_n n\frac{\omega}{Q}n \rho_{nn} + 
\nonumber \\
+ \sum_n n \frac{\omega}{Q} \left(n + 1\right)\rho_{n + 1, n + 1}.
\label{eqCh3_add78_2_1}
\end{eqnarray}
Replace the summation indices in \eqref{eqCh3_add78_2_1}. In the second
sum, let $n - 1 = m$, i.e. $n = m + 1$. In the last sum 
let $n + 1 = m$, i.e. $n = m - 1$. Substituting this into
\eqref{eqCh3_add78_2_1} gives
\begin{eqnarray}
\frac{d}{dt}\left<n\right>  = 
-\sum_n n \left[A - \left(n + 1\right)B\right]\left(n + 1\right)\rho_{nn} +
\nonumber \\
+ \sum_m \left(m + 1\right)\left(A - \left(m + 1\right)B\right)\left(m + 1\right) \rho_{m, m} 
- \sum_n n\frac{\omega}{Q}n \rho_{nn} + 
\nonumber \\
+ \sum_m \left(m - 1\right) \frac{\omega}{Q} m\rho_{m, m} = 
\sum_m \left(A - \left(m + 1\right) B\right)\left(m + 1\right) \rho_{m, m} - 
\sum_m \frac{\omega}{Q} m\rho_{m, m} = 
\nonumber \\
= \sum_n \left(A - \left(n + 1\right) B\right)\left(n + 1\right) \rho_{n, n} - 
\sum_n \frac{\omega}{Q} n\rho_{n, n}.
\nonumber
\end{eqnarray}
Here, we changed the summation index back from $m$ to $n$ and combined the sums.
From this, we have
\begin{equation}
\frac{d}{d t}\left<n\left(t\right)\right> = 
\left(A - \frac{\omega}{Q}\right)\left<n\left(t\right)\right>
+ A - B \left<\left(n + 1\right)^2\right>
\label{eqCh3_8}
\end{equation}

Here, the first term $\left(A - \frac{\omega}{Q}\right)$ corresponds to
unsaturated gain minus 
losses. The term 
$B\left<\left(n + 1\right)^2\right> = B \left[\left<n^2\right> + 2
  \left<n\right> + 1\right]$
characterizes gain reduction due to saturation. The term $A$
accounts for spontaneous emission, which is absent in the semiclassical
treatment. Apart from that, \eqref{eqCh3_8} resembles the classical
laser equation \cite{bQuantumOpticsAndRadioPhisicsLecture1966}:
%FIXME!!! check reference 
% 6 Feb 2011 Needs verification - book found page 281
% formula not found :(
\[
\dot{I}_n = 2 I_n\left(\alpha_n - \beta_n I_n\right).
\]
