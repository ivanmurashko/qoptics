%% -*- coding:utf-8 -*- 
\section{Laser theory. Representation of coherent states}
It is necessary to use the equation for the quasiprobability function
$P\left(\alpha, t\right)$ \eqref{eqCh3_7}:
\begin{equation}
\frac{\partial}{\partial t} P\left(\alpha, t\right) = 
- \frac{1}{2}\left\{ 
\frac{\partial}{\partial \alpha}
\left[
A - \frac{\omega}{Q} - B \left|\alpha\right|^2
\right] \alpha P + \mbox{c.c.}
\right\} + 
A \frac{\partial^2 P}{\partial \alpha \partial \alpha^{*}}.
\nonumber
\end{equation} 

For our purposes, it is convenient to represent this equation in polar
coordinates $\alpha = r e^{i\theta}$. We have
\begin{eqnarray}
\frac{\partial}{\partial t} P\left(r, \theta, t\right) =
- \frac{1}{2 r} \cdot \frac{\partial}{\partial r}\left[
r^2\left(A - \frac{\omega}{Q} - B r^2\right)P\left(r, \theta,
t\right) 
\right] +
\nonumber \\
+ \frac{A}{4 r^2}
\left(
r \frac{\partial}{\partial r} r \frac{\partial}{\partial r} + 
\frac{\partial^2}{\partial \theta^2}
\right)
P\left(r, \theta, t\right)
\label{eqCh3_12}
\end{eqnarray}

Suppose we are interested in the steady state. Then $\frac{\partial
  P}{\partial t} = 0$, and the phase $\theta$ is uniformly distributed from $0$ to $2
\pi$. It follows that $\frac{\partial^2 P}{\partial
  \theta^2} = 0$ and the quasiprobability $P\left(r, \theta, t\right)$ does not depend on $t$ or
$\theta$. Equation \eqref{eqCh3_12} then takes the form   
\begin{equation}
0 = - \frac{\partial}{\partial r}
\left[
r^2\left(A - \frac{\omega}{Q} - B r^2\right)P\left(r\right) 
\right] + 
\frac{1}{2} A \frac{\partial}{\partial r} r \frac{\partial}{\partial
  r} P\left(r\right).
\label{eqCh3_13}
\end{equation}

This equation is easily integrated. The first integration gives
\begin{equation}
r\left(A - \frac{\omega}{Q} - B r^2\right)P\left(r\right) 
=
\frac{1}{2} A \frac{\partial}{\partial
  r} P\left(r\right) + C.
\label{eqCh3_14}
\end{equation}
The constant $C = 0$, since $P$ and $\frac{\partial P}{\partial r}$
must tend sufficiently quickly to zero as $r \rightarrow
\infty$. 
 
From here follows the equation
\begin{equation}
\frac{d P}{P} = 
\frac{2}{A} r \left(A - \frac{\omega}{Q} - B r^2\right)d r. 
\label{eqCh3_15}
\end{equation}

Its solution is
\begin{eqnarray}
P = N \exp \left(\frac{1}{A}
\left[
r^2 \left(A - \frac{\omega}{Q}\right)
 - 
\frac{B r^4}{2}
\right]
\right) = 
\nonumber \\
= 
N \exp \left(\frac{1}{A}
\left[
r^2 G
 - 
\frac{B r^4}{2}
\right]
\right),
\label{eqCh3_16}
\end{eqnarray}
where $N$ is the normalization factor, equal to
\[
\frac{1}{N} = 2 \pi \int_0^{\infty}
\exp\left(
\frac{1}{A}\left[
r^2 G - \frac{B}{2}r^4
\right]
\right)
r dr.
\]
Distribution \eqref{eqCh3_16}
is a function of $r^2 = \left|\alpha\right|^2 = \bar{n}$. When
threshold is exceeded $A > \frac{\omega}{Q}$, i.e. when $G > 0$,
$P\left(r\right)$ first increases with increasing $r^2$, reaching a maximum at
$r^2 = \frac{A - \omega/Q}{B} = \frac{G}{B}$, and then decreases.  

This same result was obtained earlier. The obtained value $r =
\sqrt{\frac{G}{B}}$ corresponds, at $G > 0$, to the most probable
value of $\left|\alpha\right|$.
  
For $A < \frac{\omega}{Q}$, $G < 0$, the laser is below threshold. With
increasing $r^2$ the curve monotonically decreases. At threshold $A =
\frac{\omega}{Q}$, $G = 0$, the curve also decreases, but at $r^2 = 0$
we have an extremum.   

\input ./part2/laser/fig6.tex

On \autoref{figPart2Ch1_6} are shown dependencies of the quasiprobability on $r^2 =
\left|\alpha\right|^2$, for different values of parameter $G$,
corresponding to below-threshold, threshold, and above-threshold regimes. 

On the complex plane $\alpha = r e^{i \theta}$, when the threshold is substantially
exceeded, the region occupied by the generated field can be visually
represented as shown on
\autoref{figPart2Ch1_7}. The broken curve characterizes the
distribution of the oscillation amplitude, i.e., corresponds to amplitude
noise. The phase $\theta$ is uniformly distributed in the interval $0 - 2\pi$,
therefore the broken curve lies close to the circle $r =
\sqrt{\bar{n}_{st}}$ with the radius corresponding to the most probable
value of $\left|\alpha\right|$. 

\input ./part2/laser/fig7.tex

From \autoref{figPart2Ch1_7} it is seen that the amplitude fluctuates in a narrow region around the circle of radius $r = \sqrt{\bar{n}_{st}}$, while the phase
freely diffuses (wanders) along this circle. 

Phase diffusion in the laser occurs slowly, therefore, if
we consider not too long time intervals, it can be assumed
that the laser generates radiation with a well-defined phase. When
averaged over a long time interval, during which the phase can assume any value
in the range $0 - 2\pi$ with equal probability, the mean value of the laser field over this interval
will be zero.  

Let us consider the phase diffusion process more carefully. Suppose
the amplitude distribution is established and $P\left(r\right)$
corresponds to the stationary amplitude distribution. Then
\[
P\left(r, \theta, t\right) = P\left(r\right) P\left(\theta, t\right).
\]
In this case, equation \eqref{eqCh3_12} takes the form 
\begin{equation}
\frac{\partial P \left(\theta, t\right)}{\partial t} = 
\frac{A}{4 r^2}
\frac{\partial^2}{\partial \theta^2}
P \left(\theta, t\right)
\label{eqCh3_17}
\end{equation}
The function $P\left(r\right)$ cancels out because it satisfies the stationary equation \eqref{eqCh3_15}. 

Equation \eqref{eqCh3_17} can be used to obtain the equation satisfied by the average field. The average electric field of the laser mode,
expressed through $P$, has the form  
\[
\left<\hat{E}\right> = E_1 \int P\left(\alpha, t\right) \alpha d^2
\alpha.
\]
In polar coordinates this looks like:
\[
\left<\hat{E}\right> = E_1 \int_{0}^{\infty}r dr \int_0^{2 \pi}r e^{i
  \theta} P\left(r, \theta, t\right) d \theta.
\]
Using now equation \eqref{eqCh3_17}:
\begin{eqnarray}
\frac{\partial \left<\hat{E}\right>}{\partial t}
= E_1 \frac{\partial}{\partial t} \int_{0}^{\infty}r^2 dr \int_0^{2
  \pi} e^{i
  \theta} P\left(r, \theta, t\right) d \theta = 
\nonumber \\
= E_1 \int_0^{\infty}r^2 dr \int_0^{2\pi}e^{i\theta}\frac{\partial
  P}{\partial t} d \theta
=
\nonumber \\
= \frac{E_1}{\bar{n}_{st}} \frac{A}{4}
\int_{0}^{\infty}r^2 dr \int_0^{2 \pi}e^{i
  \theta} \frac{\partial^2}{\partial \theta^2} P\left(r, \theta,
t\right) d \theta. 
\label{eqCh3_18}
\end{eqnarray}
Here we approximately replaced $\frac{1}{r^2} \approx
\frac{1}{\bar{n}_{st}}$, assuming the amplitude distribution is sufficiently
narrow. 

We integrate the inner integral twice by parts. We have
\begin{eqnarray}
\int_0^{2 \pi} e^{i \theta}
\frac{\partial^2 P}{\partial \theta^2} d \theta = 
e^{i \theta} \left.\frac{\partial P}{\partial \theta}\right|_0^{2 \pi}
- i \int_0^{2 \pi} e^{i \theta}
\frac{\partial P}{\partial \theta} d \theta = 
\nonumber \\
= -i e^{i \theta} \left. P \right|_0^{2 \pi} - 
\int_0^{2 \pi} e^{i \theta} P d \theta.
\label{eqCh3_19}
\end{eqnarray}

The first terms in \eqref{eqCh3_19} equal zero due to periodicity of $P$ and $\frac{\partial P}{\partial \theta}$. Using \eqref{eqCh3_19},
equation \eqref{eqCh3_18} can be written as 
\begin{eqnarray}
  \frac{\partial \left<E\right>}{\partial t} =
  E_1 \int_0^{\infty}r^2 dr \int_0^{2\pi}e^{i\theta}\frac{\partial
    P}{\partial t} d \theta
  =
  \nonumber \\
  = 
- \frac{A E_1}{4 \bar{n}_{st}}
\int_0^{\infty}r dr\int_0^{2 \pi}r e^{i \theta} P d \theta = 
- \frac{A}{4 \bar{n}_{st}} \left<E\right>.
\label{eqCh3_20}
\end{eqnarray}
The solution of this equation is 
\begin{equation}
\left<E\left(t\right)\right> = 
\left<E\left(0\right)\right> e^{- \frac{D}{2}t},
\label{eqCh3_21}
\end{equation}
where 
\begin{equation}
D = \frac{A}{2 \bar{n}_{st}}.
\label{eqCh3_21a}
\end{equation}

Thus, we have obtained that the mean field indeed tends to
zero with the characteristic time $\tau = \frac{1}{D}$. In lasers this
interval can be quite long compared to the oscillation period.

To determine the natural linewidth of the laser emission
it is necessary to find the correlation function of laser radiation; its spectrum by
\myref{thm:khinchin_wiener}{the Wiener-Khinchin theorem} will be
the energy spectrum of laser radiation. We have
\begin{equation}
\left<\hat{E}^{(-)}\left(0\right)\hat{E}^{(+)}\left(t\right)\right>
= E_1^2\int d^2\alpha P\left(\alpha\right) \alpha^{*}\left(0\right)\alpha\left(t\right) e^{-i
  \omega_0 t},
\nonumber
\end{equation}
where $P\left(\alpha\right)$ is the quasiprobability of the laser state,
defined by formula \eqref{eqCh3_12}. 
When switching to polar coordinates
(amplitude-phase) $\alpha = r e^{i \theta}$
we assume that the amplitude is already
established and does not change, while the phase changes slowly, so that 
\begin{eqnarray}
\alpha^{*}\left(0\right) = r e^{i\theta\left(0\right)} = r e^{i 0} = r,
\nonumber \\
\alpha\left(t\right) = r e^{i\theta\left(t\right)} = r e^{i \theta},
\label{eqCh3_addon1}
\end{eqnarray}
In \eqref{eqCh3_addon1} it is assumed that at the initial time 
$\theta\left(0\right) = 0$ and at time $t$ - $\theta\left(t\right) = \theta$.
Thus we have:
\begin{equation}
\left<\hat{E}^{(-)}\left(0\right)\hat{E}^{(+)}\left(t\right)\right>
= E_1^2 e^{-i
  \omega_0 t}\int_0^{\infty}r^3 d r \int_0^{2 \pi}d \theta P\left(r
e^{i \theta}, t\right) e^{i \theta}.
\nonumber
\end{equation}
Let us derive an equation satisfied by 
$\left<\hat{E}^{(-)}\left(0\right)\hat{E}^{(+)}\left(t\right)\right>$. 
From equation \eqref{eqCh3_17}
\(
\frac{\partial P \left(\theta, t\right)}{\partial t} = 
\frac{A}{4 \bar{n}_{st}}
\frac{\partial^2}{\partial \theta^2}
P \left(\theta, t\right)
\),
therefore
\begin{eqnarray}
\frac{d}{dt}\left<\hat{E}^{(-)}\left(0\right)\hat{E}^{(+)}\left(t\right)\right>
= -i \omega_0
\left<\hat{E}^{(-)}\left(0\right)\hat{E}^{(+)}\left(t\right)\right>+
\nonumber \\
+ E_1^2 e^{-i
  \omega_0 t}\int_0^{\infty}r^3 d r \int_0^{2 \pi}d \theta
\frac{\partial}{\partial t}P\left(r
e^{i \theta}, t\right) e^{i \theta}
=
\nonumber \\
=
-i \omega_0
\left<\hat{E}^{(-)}\left(0\right)\hat{E}^{(+)}\left(t\right)\right>+
\nonumber \\ 
+
\frac{E_1^2 A}{4 \bar{n}_{st}} 
\int_0^{\infty}r^3 d r 
\int_0^{2 \pi}
d \theta
\frac{\partial^2}{\partial \theta^2}P\left(r
e^{i \theta}, t\right) e^{i \theta}.
\label{eqPart2Ch1_add84_1}
\end{eqnarray}
Performing two integrations by parts by $\theta$ in \eqref{eqPart2Ch1_add84_1}, we obtain:
\begin{eqnarray}
\int_0^{2 \pi}
d \theta
\frac{\partial^2}{\partial \theta^2}P e^{i \theta} = 
\left.\frac{\partial}{\partial \theta}P e^{i \theta}\right|_0^{2 \pi}
- 
i \int_0^{2 \pi}
d \theta
\frac{\partial}{\partial \theta}P e^{i \theta} = 
\nonumber \\
= - i \int_0^{2 \pi}
d \theta
\frac{\partial}{\partial \theta}P e^{i \theta}  
= 
- i \left.P e^{i \theta}\right|_0^{2 \pi}
-
\int_0^{2 \pi}
d \theta
P e^{i \theta} = 
- \int_0^{2 \pi}
d \theta
P e^{i \theta}.
\nonumber
\end{eqnarray}
Substituting the obtained result into the original expression
\eqref{eqPart2Ch1_add84_1}, we get:
\begin{eqnarray}
\frac{d}{dt}\left<\hat{E}^{(-)}\left(0\right)\hat{E}^{(+)}\left(t\right)\right>
=
-i \omega_0
\left<\hat{E}^{(-)}\left(0\right)\hat{E}^{(+)}\left(t\right)\right>-
\nonumber \\
-
\frac{E_1^2 A}{4 \bar{n}_{st}} 
\int_0^{\infty}r^3 d r 
\int_0^{2 \pi}
d \theta
P\left(r
e^{i \theta}, t\right) e^{i \theta} = 
\nonumber \\
=-i \omega_0
\left<\hat{E}^{(-)}\left(0\right)\hat{E}^{(+)}\left(t\right)\right> -
\frac{A}{4 \bar{n}_{st}}\left<\hat{E}^{(-)}\left(0\right)\hat{E}^{(+)}\left(t\right)\right>.
\nonumber
\end{eqnarray}
The solution of this equation is
\begin{equation}
r_{+}\left(t\right) = \left<\hat{E}^{(-)}\left(0\right)\hat{E}^{(+)}\left(t\right)\right> =
\left<\hat{E}^{(-)}\left(0\right)\hat{E}^{(+)}\left(0\right)\right>
e^{-i \omega_0 t - \frac{D}{2}t},
\label{eqPart2LaserCorr1}
\end{equation}
where
\[
\frac{D}{2} = \frac{A}{4 \bar{n}_{st}}.
\]
Expression \eqref{eqPart2LaserCorr1} defines the correlation function
$r\left(t\right)$ in the domain $t \ge 0$. For $t \le 0$, taking into account
\eqref{eqAddHinchinStatWide3}, the correlation function can be written as  
\begin{equation}
r_{-}\left(t\right) = r_{-}\left(-\left|t\right|\right) =
r_{+}^{*}\left(\left|t\right|\right). 
\label{eqPart2LaserCorr2}
\end{equation}
Using expressions \eqref{eqPart2LaserCorr1} and
\eqref{eqPart2LaserCorr2}, one can write the following expression for the Fourier transform of the correlation function, which according to 
\myref{thm:khinchin_wiener}{the Wiener-Khinchin theorem}
defines the energy spectrum $S\left(\omega\right)$:  
\begin{eqnarray}
S\left(\omega\right) = \tilde{r}\left(\omega\right) = 
\frac{1}{2\pi}
\int_{-\infty}^{+\infty}e^{i \omega t} r\left(t\right) dt =
\nonumber \\
=
\frac{1}{2\pi}
\int_{-\infty}^0e^{i \omega t} r_{-}\left(t\right) dt +
\frac{1}{2\pi}
\int_0^{+\infty}e^{i \omega t} r_{+}\left(t\right) dt = 
\nonumber \\
=
\frac{1}{2\pi}
\int_0^{+\infty}e^{- i \omega t} r_{+}^{*}\left(t\right) dt +
\frac{1}{2\pi}
\int_0^{+\infty}e^{i \omega t} r_{+}\left(t\right) dt = 
\nonumber \\
=
\frac{1}{\pi} \mathrm{Re}
\int_0^{+\infty}e^{i \omega t} r_{+}\left(t\right) dt = 
\nonumber \\
= \frac{1}{\pi}\mathrm{Re}
\int_0^{+\infty}e^{i \omega t}
\left<\hat{E}^{(-)}\left(0\right)\hat{E}^{(+)}\left(t\right)\right> dt
=
\nonumber \\
= \frac{\left<\hat{E}^{(-)}\left(0\right)\hat{E}^{(+)}\left(0\right)\right>}{\pi} \mathrm{Re}
\int_0^{+\infty}e^{\left(i \left(\omega - \omega_0\right) - \frac{D}{2} \right) t} dt
=
\nonumber \\
=
\frac{\left<\hat{E}^{(-)}\left(0\right)\hat{E}^{(+)}\left(0\right)\right>}{\pi} \mathrm{Re}
\frac{1}{\frac{D}{2} - i \left(\omega - \omega_0\right)}
=
\nonumber \\
=
\frac{\left<\hat{E}^{(-)}\left(0\right)\hat{E}^{(+)}\left(0\right)\right>}{\pi} \mathrm{Re}
\frac{\frac{D}{2} + i \left(\omega - \omega_0\right)}
{\left(\omega - \omega_0\right)^2 + \left(D/2\right)^2}
=
\nonumber \\
=
\frac{\left<\hat{E}^{(-)}\left(0\right)\hat{E}^{(+)}\left(0\right)\right>}{\pi}
\frac{D/2}{\left(\omega - \omega_0\right)^2 + \left(D/2\right)^2}. 
\label{eqCh3_22}
\end{eqnarray}
In the derivation of \eqref{eqCh3_22}, the definition of the Fourier transform \eqref{eq:direct_fourier} from \autoref{sec:cont_fourier} was used.

%% From \eqref{eqCh3_21} we have
%% \begin{eqnarray}
%% \left|E\left(\omega\right)\right|^2 = 
%% \left|\int_{-\infty}^{+\infty}e^{i \omega t}\left<E\left(0\right)\right>
%% e^{i \omega_0 t} e^{-\frac{1}{2}D t}
%% dt\right|^2  = 
%% \nonumber \\
%% = 
%% \left|\left<E\left(0\right)\right>\right|^2 
%% \frac{D}{\left(\omega - \omega_0\right)^2 + \left(D/2\right)^2}. 
%% \label{eqCh3_22}
%% \end{eqnarray}

\input ./part2/laser/fig8.tex

It follows that the generation line is Lorentzian, and the linewidth equals $D$ 
(\autoref{figPart2Ch1_8}). Since we considered only the "natural",
fundamentally unavoidable cause of line broadening,
associated with quantum fluctuations, and neglected fundamentally
removable technical causes, the obtained linewidth is
extremely narrow.  

Here we considered the simplest laser model. A more
realistic model of a three-level system is considered in
\cite{bHaken1988}. Results obtained for various laser models
are in good agreement with each other and correspond to the results 
obtained with the simple model used here. 
