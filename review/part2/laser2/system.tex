%% -*- coding:utf-8 -*- 
\section{Heisenberg-Langevin Equation System Describing the Laser}
%% To obtain the system of equations describing the laser operation, one should
%% use the Heisenberg equation \eqref{eqLaserHaizenberg1} by substituting into it
%% the Hamiltonian of the laser system \eqref{eqLaserHaizenberg2}. Since
%% atomic and field operators commute, when deriving equations for
%% field operators, atomic operators included in the Hamiltonian can be omitted,
%% and when deriving equations for atomic operators, field operators can be omitted.
%% This somewhat simplifies the transformations.

%% The derivation of the equations in concept is no different from what we
%% have already done when deriving the quantum equation describing cavity mode damping.
%% However, the laser system is more complex than a simple
%% harmonic oscillator (cavity mode). For this reason, the derivation
%% of the equations is more cumbersome, and the related calculations are more
%% tedious. To avoid cluttering the presentation, we write the corresponding
%% expressions without derivations, referring to an analogy with the problem considered
%% earlier. Necessary additional explanations are given in appendix \ref{AddLanzheven}.

%% We start with the equation for the field mode operator (annihilation operator):
%% \begin{equation}
%% \frac{\hat{a}\left(t\right)}{dt} =
%% - 
%% \left\{
%% \frac{1}{2}\frac{\omega}{Q} + \left(\omega_r - \omega\right)
%% \right\}\hat{a} + i g N \hat{\sigma} + \hat{F}_f\left(t\right).
%% \label{eqLaserHaizenberg11}
%% \end{equation}
%% The equation is derived similarly to that describing
%% mode damping. The first term on the right-hand side has the same form as previously obtained.
%% A new (second) term has appeared describing amplification related to
%% stimulated transitions. The last term is a Langevin noise
%% operator describing the reservoir's influence. The correlation function,
%% characterizing the statistical properties of $\hat{F}_f\left(t\right)$ as
%% before is: 
%% \begin{eqnarray}
%% \left<\hat{F}^{\dag}_f\left(t\right)\hat{F}_f\left(t'\right)\right> = 
%% \frac{\omega}{Q} \bar{n}_{T} \delta\left(t - t'\right),
%% \nonumber \\
%% \left<\hat{F}_f\left(t\right)\hat{F}^{\dag}_f\left(t'\right)\right> = 
%% \frac{\omega}{Q}  \left(\bar{n}_{T} + 1\right) \delta\left(t - t'\right).
%% \label{eqLaserHaizenberg12}
%% \end{eqnarray}

%% In \eqref{eqLaserHaizenberg12} the expression is given through the cavity quality
%% factor $Q$. $N$ is the total number of active atoms. $\bar{n}_T$ is the average number of
%% atoms in the cavity mode at temperature $T$. $\omega$ is the generation frequency,
%% and $\omega_r$ is the resonator eigenfrequency.

%% For other operators, the Heisenberg-Langevin equations have
%% the following form: for the transition operator $\sigma =
%% \ket{b}\bra{a}$ connected with the atomic dipole moment:
%% \begin{eqnarray}
%% \frac{d \hat{\sigma}}{d t} = - \left[\gamma+i\left(\omega_{ab} -
%%   \omega\right)\right]\hat{\sigma} + 
%% \nonumber \\
%% + i g \left(\hat{\sigma}_a - \hat{\sigma}_b \right) + 
%% \hat{F}_{\sigma}\left(t\right),
%% \label{eqLaserHaizenberg13}
%% \end{eqnarray}
%% where $\gamma$ is the dipole moment relaxation rate, $\omega$ is the generation frequency,
%% $\omega_{ab}$ is the transition frequency between atoms $a$
%% and $b$, $g$ is the interaction constant,
%% $\hat{F}_{\sigma}\left(t\right)$ is a Langevin noise operator,
%% having the following correlation functions:
%% \begin{eqnarray}
%% \left<\hat{F}^{\dag}_{\sigma}\left(t\right)\hat{F}_{\sigma}\left(t'\right)\right>
%% =  
%% \delta\left(t - t'\right) \left[\left(2\gamma -
%%   \gamma_z\right)\left<\hat{\sigma_a}\right> + r_a\right],
%% \nonumber \\
%% \left<\hat{F}_{\sigma}\left(t\right)\hat{F}^{\dag}_{\sigma}\left(t'\right)\right>
%% =  
%% \delta\left(t - t'\right) \left[\left(2\gamma -
%%   \gamma_z\right)\left<\hat{\sigma_b}\right>\right],
%% \label{eqLaserHaizenberg14}
%% \end{eqnarray}
%% where $\gamma_z$ characterizes the population relaxation rate,
%% $\left<\hat{\sigma_a}\right>$ is the average value of the upper level population operator,
%% $\left<\hat{\sigma_b}\right>$ is the average value of the lower level population operator.
%% For simplicity, we assume
%% $\gamma_a = \gamma_b = \gamma_z$.

%% The equation for the operator describing the difference in populations of levels $a$ and $b$ 
%% \[
%% \ket{a}\bra{a} - \ket{b}\bra{b} = 
%% \hat{\sigma_a} - \hat{\sigma_b} = \hat{\sigma_z}
%% \]
%% has the form:
%% \begin{eqnarray}
%% \frac{d}{d t}\hat{\sigma_z} = r_a - \gamma_z \hat{\sigma_z} +
%% \nonumber \\
%% + 2 i g \left(\hat{a}^{\dag}\hat{\sigma} - \hat{a}\hat{\sigma}^{\dag}\right)
%% + \hat{F}_z,
%% \label{eqLaserHaizenberg15}
%% \end{eqnarray}
%% where $\hat{F}_z$ is a Langevin noise operator for which the
%% correlation function has the form
%% \begin{equation}
%% \left<\hat{F}_z\left(t\right)\hat{F}_z\left(t'\right)\right>
%% =  
%% \delta\left(t - t'\right) \left[r_a +
%%   \gamma_z\left<\hat{\sigma}_a + \hat{\sigma}_b\right>\right],
%% \label{eqLaserHaizenberg16}
%% \end{equation}

Combining the equations we obtained \eqref{eqLaserHaizenbergA}, 
\eqref{eqLaserHaizenberNB_AB}, and
\eqref{eqLaserHaizenberNZ} we can write the following
system:
\begin{eqnarray}
\frac{d \hat{a}}{d t} = 
- \frac{1}{2}\frac{\omega}{Q}\hat{a}
-i g \hat{N}_{ab} + 
\hat{F}_F,
\nonumber \\
\frac{d \hat{N}_{ab}}{d t} = 
- \gamma \hat{N}_{ab} 
+ i g \hat{N}_{z} \hat{a} 
 + \hat{F}_{ab},
\nonumber \\
\frac{d \hat{N}_z}{d t} = r_a
- \gamma \hat{N}_{z} +
2 i g 
 \left(
\hat{a}^{\dag}\hat{N}_{ab} -
\hat{N}_{ab}^{\dag}\hat{a}
\right) + \hat{F}_{z},
\label{eqLaserHaizenberMain}
\end{eqnarray}
which we will call the Heisenberg-Langevin equation system
describing the laser operation.

By their form, these equations \eqref{eqLaserHaizenberMain} are very similar
to the laser equations obtained in the semiclassical
approximation. The difference is that 
instead of classical quantities, the equations use operators, which
may not commute with each other, so the order of factors must be preserved
during transformations. Moreover, the equations include Langevin terms describing quantum
noise present in the laser system.

The same approximations applied in the
semiclassical case can be applied to equations \eqref{eqLaserHaizenberMain}.

If, as is usually the case, 
%% the dipole moment relaxation rate
%% $\gamma$ is much greater than the population relaxation rate $\gamma_z$ and
%% the mode relaxation rate $\frac{\omega}{Q}$, i.e. $\gamma \gg
%% \gamma_z,\frac{\omega}{Q}$,
the adiabatic approximation can be applied,
neglecting in the second equation \eqref{eqLaserHaizenberMain} the derivative
$\frac{d\hat{N}_{ab}}{d t}$ compared to the term $\gamma
\hat{N}_{ab}$, we obtain
\begin{equation}
\hat{N}_{ab} = \frac{1}{\gamma} \left(i g \hat{N}_z\hat{a} +
\hat{F}_{ab}\right).
\nonumber
\end{equation}
Substituting this expression for $\hat{N}_{ab}$ into the remaining two equations
of system \eqref{eqLaserHaizenberMain} yields a new
system of equations:
\begin{eqnarray}
\frac{d}{dt} \hat{a} = -
\frac{1}{2}\left(\frac{\omega}{Q}\right)\hat{a} + \frac{g^2}{\gamma}
\hat{N}_z\hat{a} + \hat{F}_F - i\frac{g}{\gamma}\hat{F}_{ab},
\nonumber \\
\frac{d}{dt}\hat{N}_z = r_a -
\gamma\hat{N}_z - \frac{4g^2}{\gamma}
\hat{N}_z\hat{a}^{\dag}\hat{a} +
\nonumber \\
+ \hat{F}_z + \frac{2ig}{\gamma} \left(\hat{a}^{\dag}\hat{F}_{ab} -
\hat{F}_{ab}^{\dag}\hat{a}\right).
\label{eqLaserHaizenberg16add}
\end{eqnarray}

Introducing a new notation 
\begin{equation}
\hat{F}_{\sum} = - \left(\frac{g}{\gamma}\hat{F}_{ab} + i \hat{F}_F\right),
\label{eqLaserHaizenbergFSum}
\end{equation}
the system 
\eqref{eqLaserHaizenberg16add} can be rewritten as
\begin{eqnarray}
\frac{d}{dt} \hat{a} = -
\frac{1}{2}\left(\frac{\omega}{Q}\right)\hat{a} + \frac{g^2}{\gamma}
\hat{N}_z\hat{a} + i \hat{F}_{\sum},
\nonumber \\
\frac{d}{dt}\hat{N}_z = r_a -
\gamma\hat{N}_z - \frac{4g^2}{\gamma}
\hat{N}_z\hat{a}^{\dag}\hat{a} +
\nonumber \\
+ \hat{F}_z + \frac{2ig}{\gamma} \left(\hat{a}^{\dag}\hat{F}_{ab} -
\hat{F}_{ab}^{\dag}\hat{a}\right).
\label{eqLaserHaizenberg17}
\end{eqnarray}

For the noise operator \eqref{eqLaserHaizenbergFSum}, using
\eqref{eqLaserHaizenbergFABCorrel_1},
\eqref{eqLaserHaizenbergFABCorrel_2}, and
\eqref{eqLaserHaizenbergFFCorrel}, 
the following correlation relations can be written:
\begin{eqnarray}
\left<\hat{F}^{\dag}_{\sum}\left(t_1\right)\hat{F}_{\sum}\left(t_2\right)\right>
=
\frac{g^2}{\gamma^2}\left<\hat{F}^{\dag}_{ab}\left(t_1\right)\hat{F}_{ab}\left(t_2\right)\right>
+
\left<\hat{F}^{\dag}_{F}\left(t_1\right)\hat{F}_{F}\left(t_2\right)\right>
= 
\nonumber \\
= 
\frac{1}{2}
\left(
\frac{g^2}{\gamma}\bar{N}_a + 2 \frac{g^2}{\gamma^2}r_a + 
\frac{\omega}{Q}\bar{n}_T
\right)
\delta\left(t_1 - t_2\right),
\nonumber \\
\left<\hat{F}_{\sum}\left(t_1\right)\hat{F}^{\dag}_{\sum}\left(t_2\right)\right>
= \frac{1}{2}
\left(
\frac{g^2}{\gamma}\bar{N}_b + 
\frac{\omega}{Q}\left(\bar{n}_T + 1\right)
\right)
\delta\left(t_1 - t_2\right)
\label{eqLaserHaizenbergFSumCorrel}
\end{eqnarray}


Equations \eqref{eqLaserHaizenberg17} resemble the semiclassical
laser equation and can, to a large extent,
be solved using the same approximations and techniques as those
used in the semiclassical problem.