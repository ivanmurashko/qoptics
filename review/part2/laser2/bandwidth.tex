%% -*- coding:utf-8 -*- 
\section{Natural linewidth of emission}
Equations \eqref{eqLaserHaizenberg17} are suitable for solving many
problems related to the quantum nature of the laser: determining the statistics
of laser photons in various operating modes, determining the linewidth of
emission, and with some additions for solving more subtle
issues, for example, the study of direct laser generation of
squeezed states (for more on squeezed states see Ch. \ref{chSqueezed}). 

\input ./part2/laser2/fig3.tex

Here we will consider only the problem of the natural linewidth of
laser emission. We will consider the steady-state laser operation
mode that is sufficiently far from the generation threshold. In this case, the problem
can be greatly simplified since the linewidth depends little on fluctuations
of the population difference and amplitude fluctuations of the generated
radiation, as we established earlier by considering this problem using the density matrix method.
The main influence on the linewidth is due to
phase fluctuations (drift). This can be visually represented on the complex amplitude plane,
as shown in \autoref{figPart2Laser2_3}.

If there were no noise sources (Langevin forces),
the generated field would have a definite amplitude and phase. In
reality, with their presence, the amplitude uncertainty occupies
a narrow region near the circle $\left|\alpha\right| = const$, while the phase
freely diffuses (wanders) along this circle. 

In light of this, the simplification of the equations reduces to
the following. The equation for the population will be replaced by the averaged equation 
\begin{equation}
r_a - \gamma \bar{N}_z - \frac{4 g^2}{\gamma}\bar{N}_z\bar{n} = 0,
\label{eqLaserHaizenberg19}
\end{equation}
where $\bar{N}_z$ is the averaged population value, $\bar{n} =
\left<\hat{a}^{\dag}\hat{a}\right>$ is the average photon number in
the generated mode. Zero on the right means that a
steady-state regime is considered. 

From \eqref{eqLaserHaizenberg19} we find $\bar{N}_z$
\begin{equation}
\bar{N}_z = \frac{r_a}{\gamma\left(1+ \left(\frac{2
    g}{\gamma}\right)^2 \bar{n}\right)} 
\approx
\frac{r_a}{\gamma}
\left(1 - \left(\frac{2 g}{\gamma}\right)^2
\bar{n}\right).
\label{eqLaserHaizenberg20}
\end{equation}
The final value in \eqref{eqLaserHaizenberg20} is obtained under the approximation that the second term in parentheses is small compared to $1$ (laser is not
at threshold). Substituting this result into the field equation, we get
\begin{eqnarray}
\frac{d}{dt}\hat{a} = -\frac{1}{2}\left(\frac{\omega}{Q}\right)\hat{a}
+ r_a \frac{g^2}{\gamma^2}\hat{a} - 
\nonumber \\
- 4 r_a\frac{g^4}{\gamma^4} \bar{n}\hat{a} + i \hat{F}_{\sum},
\label{eqLaserHaizenberg21}
\end{eqnarray}
where the first term on the right corresponds to losses in the resonator, the second -
gain by the active medium, the third characterizes the level saturation degree. 

The field correlation function we need to determine is expressed by the equality
\begin{equation}
I\left(t, 0\right) = \left<\hat{a}^{\dag}\left(t\right)\hat{a}\left(0\right)\right>,
\nonumber
\end{equation}
where $\hat{a}\left(t\right)$ satisfies equation
\eqref{eqLaserHaizenberg21}. 

In our approximation, when we neglect amplitude fluctuations,
but do not neglect phase diffusion, the operator $\hat{a}$ can be
represented in the form
\begin{eqnarray}
\hat{a}\left(t\right) = A e^{i\hat{\varphi}\left(t\right)},
\nonumber \\
\hat{a}^{\dag}\left(t\right) = A e^{-i\hat{\varphi}\left(t\right)},
\label{eqLaserHaizenbergAConstPhi}
\end{eqnarray}
where $A$ is the average amplitude, and $\hat{\varphi}\left(t\right)$ is the phase operator. Then the correlation function can be presented as
\begin{eqnarray}
\left<\hat{a}^{\dag}\left(t\right)\hat{a}\left(0\right)\right> =
\left|A\right|^2 \left<e^{-i\left(\hat{\varphi}\left(t\right) -
  \hat{\varphi}\left(0\right)\right)}\right> = 
\nonumber \\
= 
\left|A\right|^2 \left<e^{-\frac{1}{2}\left(\hat{\varphi}\left(t\right) -
  \hat{\varphi}\left(0\right)\right)^2}\right>.
\label{eqLaserHaizenbergTaskMiddle}
\end{eqnarray}
The last follows from the fact that 
\begin{equation}
\left<e^{-i \left(\Delta \varphi\right)}\right> = 
\left<e^{-\frac{1}{2} \left(\Delta \varphi\right)^2}\right>,
\nonumber
\end{equation}
where $\Delta \varphi = \hat{\varphi}\left(t\right) -
\hat{\varphi}\left(0\right)$, and it can be proved by expanding
the exponential in a series, averaging each series term, and then summing
the result.

We have
\[
\Delta \varphi = \int_0^t \frac{d \varphi}{d t}dt,
\]
then
\[
\left<\left(\Delta \varphi\right)^2\right> = \int_0^t d t_1 \left<\frac{d \varphi}{d t}\right>
\int_0^{t} d t_2 \left<\frac{d \varphi}{d t}\right>,
\]
where $\frac{d \varphi}{d t}$ can be obtained from equation
\eqref{eqLaserHaizenberg21} through the relations
\begin{eqnarray}
i \frac{d \varphi}{d t} = \frac{1}{\hat{a}}\frac{d
  \hat{a}}{d t} = 
-\frac{1}{2}\left(\frac{\omega}{Q}\right)
+ r_a \frac{g^2}{\gamma^2} - 
\nonumber \\
- 4 r_a\frac{g^4}{\gamma^4} \bar{n} + i \frac{\hat{F}_{\sum}}{\hat{a}},
\nonumber \\
- i \frac{d \varphi}{d t} = \frac{1}{\hat{a}^{\dag}}\frac{d
  \hat{a}^{\dag}}{d t} = 
-\frac{1}{2}\left(\frac{\omega}{Q}\right)
+ r_a \frac{g^2}{\gamma^2} - 
\nonumber \\
- 4 r_a\frac{g^4}{\gamma^4} \bar{n} - i \frac{\hat{F}^{\dag}_{\sum}}{\hat{a}^{\dag}},
\label{eqLaserHaizenberg25_pre1}
\end{eqnarray}
thus subtracting the second equation \eqref{eqLaserHaizenberg25_pre1} from
the first we obtain
\begin{equation}
2 \frac{d \varphi}{d t} = 
\frac{\hat{F}_{\sum}}{\hat{a}} + \frac{\hat{F}^{\dag}_{\sum}}{\hat{a}^{\dag}}.
\label{eqLaserHaizenberg25}
\end{equation}
Substituting now \eqref{eqLaserHaizenbergAConstPhi} into
\eqref{eqLaserHaizenberg25} 
we have:
\begin{equation}
\frac{d \varphi}{d t} = \frac{1}{2 A}
\left\{
e^{i\hat{\varphi}\left(t\right)}\hat{F}^{\dag}_{\sum}\left(t\right) +
e^{- i\hat{\varphi}\left(t\right)}\hat{F}_{\sum}\left(t\right)
\right\}.
\nonumber
\end{equation}
It remains to calculate  
\(
\left<\left(\Delta \varphi\right)^2\right>.
\)
This leads to a double integral
\begin{eqnarray}
\left<\left(\Delta \varphi\right)^2\right> = 
\frac{1}{4 A^2}
\int_0^t d t_1 
\int_0^t d t_2
\left<
\left(
e^{i\hat{\varphi}\left(t_1\right)}\hat{F}^{\dag}_{\sum}\left(t_1\right) + 
e^{- i\hat{\varphi}\left(t_1\right)}\hat{F}_{\sum}\left(t_1\right)
\right)\right.
\nonumber \\
\left.
\left(
e^{i\hat{\varphi}\left(t_2\right)}\hat{F}^{\dag}_{\sum}\left(t_2\right) +
e^{- i\hat{\varphi}\left(t_2\right)}\hat{F}_{\sum}\left(t_2\right)
\right)
\right> = 
\nonumber \\
=
\frac{1}{4}\frac{1}{\bar{n}}
\int_0^t d t_1 
\int_0^t d t_2
\left<
\hat{F}^{\dag}_{\sum}\left(t_1\right)\hat{F}_{\sum}\left(t_2\right)e^{-i\left(
\hat{\varphi}\left(t_1\right) - \hat{\varphi}\left(t_2\right)
\right)} + \mbox{h.c.}\right>.
\nonumber
\end{eqnarray}
From here, using \eqref{eqLaserHaizenbergFSumCorrel} we obtain
\begin{eqnarray}
\left<\left(\Delta \varphi\right)^2\right> = 
\frac{1}{4}\frac{1}{\bar{n}}
\left(
\frac{g^2}{2\gamma}\left(\bar{N}_a + \bar{N}_b\right) + \frac{g^2}{\gamma^2}r_a + 
\frac{\omega}{Q}\left(\bar{n}_T + \frac{1}{2}\right)
\right) t.
\label{eqLaserHaizenbergTaskDelta}
\end{eqnarray}

From equations \eqref{eqLaserHaizenbergNA} and 
\eqref{eqLaserHaizenberNB_AB} one can write the equation for
$\bar{N}_a+\bar{N}_b$: 
\begin{equation}
\frac{d\left(\bar{N}_a+\bar{N}_b\right)}{d t} = r_a - \gamma
\left(\bar{N}_a+\bar{N}_b\right), 
\nonumber
\end{equation}
hence in the steady-state regime:
\begin{equation}
r_a = \gamma
\left(\bar{N}_a+\bar{N}_b\right).
\nonumber
\end{equation}
Substituting this relation into \eqref{eqLaserHaizenbergTaskDelta}
we get: 
\begin{eqnarray}
\left<\left(\Delta \varphi\right)^2\right> = 
\frac{1}{4}\frac{1}{\bar{n}}
\left(
\frac{3 g^2}{2\gamma}\left(\bar{N}_a + \bar{N}_b\right) + 
\frac{\omega}{Q}\left(\bar{n}_T + \frac{1}{2}\right)
\right) t.
\label{eqLaserHaizenbergTaskDelta2}
\end{eqnarray}

Consider the laser at threshold:
\begin{equation}
A=R_a\approx \frac{\omega}{Q},
\nonumber
\end{equation}
substituting here the expression for $R_a$ \eqref{eqCh2_RaRbDefenition} in
which we take $\tau \approx \frac{1}{\gamma}$:
\begin{equation}
r_a \frac{g^2}{\gamma^2} \approx \frac{\omega}{Q}.
\nonumber 
\end{equation}
Using this relation from \eqref{eqLaserHaizenberg20} we have:
\begin{equation}
\bar{N}_z \approx
\frac{r_a}{\gamma} \approx \frac{\omega}{Q} \frac{\gamma}{g^2},
\nonumber
\end{equation}
hence
\begin{equation}
\frac{g^2}{\gamma} \approx
\frac{\omega}{Q}\frac{1}{\bar{N}_z}.
\nonumber
\end{equation}
Thus \eqref{eqLaserHaizenbergTaskDelta2} can be rewritten as
\begin{eqnarray}
\left<\left(\Delta \varphi\right)^2\right> = 
\frac{1}{4}\frac{1}{\bar{n}}\frac{\omega}{Q}
\left(
\frac{3}{2}\frac{\bar{N}_a + \bar{N}_b}{\bar{N}_z} + 
\bar{n}_T + \frac{1}{2}
\right) t.
\label{eqLaserHaizenbergTaskDelta3}
\end{eqnarray}
If now in \eqref{eqLaserHaizenbergTaskDelta3} we assume
complete inversion: $\bar{N}_b = 0$, and also neglect $\bar{n}_T$ compared to 1, then we obtain
\begin{eqnarray}
\left<\left(\Delta \varphi\right)^2\right> \approx
\frac{1}{2}\frac{1}{\bar{n}}\frac{\omega}{Q} t.
\label{eqLaserHaizenbergTaskDelta4}
\end{eqnarray}

Substituting this into \eqref{eqLaserHaizenbergTaskMiddle} we get
\begin{eqnarray}
\left<\hat{a}^{\dag}\left(t\right)\hat{a}\left(0\right)\right> =
\left|A\right|^2 e^{-\frac{1}{4}\frac{1}{\bar{n}}\frac{\omega}{Q} t},
\label{eqLaserHaizenbergTaskMiddleFinal}
\end{eqnarray}
which coincides with expression \eqref{eqPart2LaserCorr1} previously obtained,
if we accept the threshold generation mode: 
\[
A \approx \frac{\omega}{Q}.
\]
Thus the expression for the natural linewidth will be the same:
\[
D \approx \frac{1}{2}\frac{1}{\bar{n}}\frac{\omega}{Q}.
\]

