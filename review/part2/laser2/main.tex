%% -*- coding:utf-8 -*- 
\chapter{Quantum Theory of the Laser in the Heisenberg Representation}
\label{chLaser2}

So far we have considered the quantum theory of the laser using
the Schrödinger representation, in which the density matrix depends on time,
while operators do not depend on time. As we have seen from the example
of the problem of the decay of the resonator mode, another approach is possible,
using the Heisenberg representation
\cite{bScullyQuantumOptics2003}, 
in which
operators depend on time, while the density matrix \rindex{Density matrix}
does not depend on time. In some cases,
this approach can be more convenient for investigating subtle
questions of the quantum theory of lasers, such as laser generation
of the field in a squeezed state.

\input ./part2/laser2/model.tex
\input ./part2/laser2/equation.tex
\input ./part2/laser2/system.tex
\input ./part2/laser2/bandwidth.tex

\section{Exercises}
\begin{enumerate}
\item Prove the commutation relations
  \eqref{eqLaserHaizenbergTaskKommutator} and \eqref{eqLaserHaizenbergTaskKommutator2}.
\item Derive the equations of motion for the operator $\hat{\sigma}_b^j$: \eqref{eqLaserHaizenbergSigmaBJ} and 
\eqref{eqLaserHaizenbergFBJ}.
\item Derive the relations \eqref{eqLaserHaizenberNB_AB} for
  the operators $\hat{N}_b$ and $\hat{N}_{ab}$.
\item Prove the second relation \eqref{eqLaserHaizenbergFABJCorrel}.
\item Prove \eqref{eqLaserHaizenbergTaskMiddle}.
\item Prove \eqref{eqLaserHaizenbergTaskDelta}.
\end{enumerate}

%% \begin{thebibliography}{99}
%% \bibitem{bCh1LaserLangevinSkalliZubari} M. O. Scully,
%%   M. S. Zubairy. Quantum 
%%   Optics. Moscow, Fizmatlit, 2003.
%% \bibitem{bCh1LaserLangevinYamamoto} Y. Yamamoto, A. Imamoglu.  Mesoscopic quantum
%%   optics. 1999, USA, J.Wiley \& Son.
%% \bibitem{bCh1LaserLangevinHaken} G. Haken. Laser light dynamics. Moscow: Mir,
%%   1988.
%% \end{thebibliography}