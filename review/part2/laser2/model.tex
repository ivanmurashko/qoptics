%% -*- coding:utf-8 -*- 
\section{Laser Model}

The laser level scheme is shown in
\autoref{figPart2Laser2_1}. Level $c$ is the ground level, level $b$ is the lower working level, and $a$ is the upper working level. In fact,
the scheme is four-level. The pumping is done by incoherent light through
a sufficiently broad absorption line, depicted in the figure by
a dashed line. But since a high rate of non-radiative transition to the upper working level is assumed, it can be considered
that the pumping occurs directly to level $a$. The presence of the fourth level
greatly reduces the reverse transition from level $a$ to the ground
level $c$. In \autoref{figPart2Laser2_1}, the transitions considered in the theory are indicated. $\gamma_a$ and $\gamma_b$
characterize the population relaxation from levels $a$ and $b$ due to
coupling with the dissipative system. $r_a$ is the pumping rate of the upper
working level $a$ due to incoherent optical pumping. The transition
$a \rightarrow c$ is a stimulated transition caused by the laser field
being generated.

\input ./part2/laser2/fig1.tex

The laser scheme is presented in \autoref{figPart2Laser2_2}. The scheme
contains a resonator $F$ in which the generated mode is excited,
interacting with the active atoms of the working medium, whose scheme
is shown in \autoref{figPart2Laser2_1}. In addition, there are two
reservoirs at temperature $T$: $R_{a}$ (associated with
active atoms) and $R_{F}$ (associated with the mode field), which cause relaxation
of the atoms and the mode field.

\input ./part2/laser2/fig2.tex

