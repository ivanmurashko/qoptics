%% -*- coding:utf-8 -*- 
\section{Quantum description of the polarization properties of light}
When describing the polarization properties of a single-photon state, it is convenient
\rindex{Jones vector}
to use the Jones vector, in which case the wave function will have the form:
\begin{equation}
\left|\psi\right> = 
\alpha \ket{x} + 
\beta \ket{y},
\label{eqEntangSimpleState}
\end{equation}
where $\left|\alpha\right|^2 + \left|\beta\right|^2 = 1$, and
$\ket{x} = \ket{1}_x\otimes\ket{0}_y$ denotes
a single-photon state polarized along $x$. Similarly,
$\ket{y} = \ket{0}_x\otimes\ket{1}_y$ is a
single-photon state polarized along $y$. 

\rindex{Stokes parameters!quantum case}
The measurable quantities for us will be the Stokes parameters, so the variables
\eqref{eqEntangStokes} must be replaced by operators. For this, we
replace the electric field operators $E_{x,y}$ with annihilation operators $\hat{a}_{x,y}$, resulting in
\begin{eqnarray}
\hat{S}_0 = \hat{a}_x^{\dag} \hat{a}_x + \hat{a}_y^{\dag} \hat{a}_y,
\nonumber \\
\hat{S}_1 = \hat{a}_x^{\dag} \hat{a}_x - \hat{a}_y^{\dag} \hat{a}_y,
\nonumber \\
\hat{S}_2 = \hat{a}_x^{\dag} \hat{a}_y + \hat{a}_x \hat{a}_y^{\dag},
\nonumber \\
\hat{S}_3 = \frac{\hat{a}_x^{\dag} \hat{a}_y - \hat{a}_x \hat{a}_y^{\dag}}{i}.
\label{eqEntangStokesOper}
\end{eqnarray}

Further, we will be primarily interested in the two operators
$\hat{S}_1$ and $\hat{S}_2$, for which we will find the commutator,
eigenvalues, and eigenvectors.

For the commutator $\left[\hat{S}_1,\hat{S}_2\right]$, we have:
\begin{eqnarray}
\left[\hat{S}_1,\hat{S}_2\right] = \hat{S}_1\hat{S}_2 -
\hat{S}_2\hat{S}_1 = 
\nonumber \\
=
\hat{a}_x^{\dag}\hat{a}_x\hat{a}_x^{\dag}\hat{a}_y -
\hat{a}_y^{\dag}\hat{a}_y\hat{a}_y\hat{a}_x^{\dag} + 
\nonumber \\
+ \hat{a}_x^{\dag}\hat{a}_x\hat{a}_x\hat{a}_y^{\dag} -
\hat{a}_y^{\dag}\hat{a}_y\hat{a}_y^{\dag}\hat{a}_x -
\nonumber \\
- \hat{a}_x^{\dag}\hat{a}_x^{\dag}\hat{a}_x\hat{a}_y +
\hat{a}_y\hat{a}_y^{\dag}\hat{a}_y\hat{a}_x^{\dag} -
\nonumber \\
- \hat{a}_x\hat{a}_x^{\dag}\hat{a}_x\hat{a}_y^{\dag} +
\hat{a}_y^{\dag}\hat{a}_y^{\dag}\hat{a}_y\hat{a}_x =
\nonumber \\
= \hat{a}_x^{\dag}\left[\hat{a}_x\hat{a}_x^{\dag}\right]\hat{a}_y -
\left[\hat{a}_x\hat{a}_x^{\dag}\right]\hat{a}_x\hat{a}_y^{\dag} +
\nonumber \\
+\left[\hat{a}_y\hat{a}_y^{\dag}\right]\hat{a}_y\hat{a}_x^{\dag} -
\hat{a}_y^{\dag}\left[\hat{a}_y\hat{a}_y^{\dag}\right]\hat{a}_x =
\nonumber \\
= \hat{a}_x^{\dag}\hat{a}_y + \hat{a}_y\hat{a}_x^{\dag} -
\hat{a}_x\hat{a}_y^{\dag} - \hat{a}_y^{\dag}\hat{a}_x = 
\nonumber \\
= 2 \left(\hat{a}_x^{\dag}\hat{a}_y - \hat{a}_x\hat{a}_y^{\dag}\right) = 2 i
\hat{S}_3 \ne 0.
\label{eqEntangStokesOperS12Comm}
\end{eqnarray}
In the derivation of \eqref{eqEntangStokesOperS12Comm}, we used
the commutation relations for creation and annihilation operators:
\begin{equation}
\left[\hat{a}_x, \hat{a}^{\dag}_x\right] = \left[\hat{a}_y, \hat{a}^{\dag}_y\right] = 1,
\nonumber
\end{equation}
as well as the fact that operators acting on different polarization components
$x$ and $y$ commute with each other.

To find the eigenvalues and eigenvectors of the operators $\hat{S}_1$ and
$\hat{S}_2$, it is convenient to represent them in matrix form. To do this, we will
use the basis formed by the vectors $\ket{x}$ and
$\ket{y}$:
\begin{eqnarray}
\ket{x} = \left(
\begin{array}{c}
1 \\
0
\end{array}
\right),
\nonumber \\
\ket{y} = \left(
\begin{array}{c}
0 \\
1
\end{array}
\right),
\nonumber
\end{eqnarray}

For the operator $\hat{S}_1$, we have:
\begin{eqnarray}
\hat{S}_1 \ket{x} = \hat{a}_x^{\dag} \hat{a}_x
\ket{1}_x\otimes\ket{0}_y - \hat{a}_y^{\dag}
\hat{a}_y\ket{1}_x\otimes\ket{0}_y =
\nonumber \\
= 
\hat{a}_x^{\dag} \hat{a}_x
\ket{1}_x\otimes\ket{0}_y =
\ket{1}_x\otimes\ket{0}_y = \ket{x},
\nonumber \\
\hat{S}_1 \ket{y} = \hat{a}_x^{\dag} \hat{a}_x
\ket{0}_x\otimes\ket{1}_y - \hat{a}_y^{\dag}
\hat{a}_y\ket{0}_x\otimes\ket{1}_y =
\nonumber \\
=
-\hat{a}_y^{\dag}
\hat{a}_y\ket{0}_x\otimes\ket{1}_y
=-\ket{0}_x\otimes\ket{1}_y = -\ket{y},
\label{eqEntangS1MatrixPre}
\end{eqnarray}
from which we obtain the following matrix representation:
\begin{equation}
\hat{S}_1 = 
\left(
\begin{array}{cc}
1 & 0 \\
0 & -1 
\end{array}
\right).
\label{eqEntangS1Matrix}
\end{equation}
From \eqref{eqEntangS1Matrix} one can write the eigenvalue equation:
\[
\left(1-s\right)\left(1 + s\right) = 0,
\]
from which two eigenvalues can be found, $s_1 = 1$ and
$s_2 = -1$. As can be easily verified, the eigenvector for $s_1 = 1$
will be $\ket{s_1} = \ket{x}$. Indeed, from \eqref{eqEntangS1MatrixPre}:
\begin{equation}
\hat{S}_1 \ket{s_1} = \hat{S}_1 \ket{x} = 1 \cdot \ket{s_1}.
\label{eq:part2:pol:stocks_s1_1}
\end{equation}
For the second eigenvalue, the eigenvector will be
$\ket{s_2} = \ket{y}$:
\begin{equation}
\hat{S}_1 \ket{s_2}  = - \ket{y} = -1 \cdot \ket{s_2}.
\label{eq:part2:pol:stocks_s1_2}
\end{equation}


For the operator $\hat{S}_2$, we have:
\begin{eqnarray}
\hat{S}_2 \ket{x} = \hat{a}_x^{\dag} \hat{a}_y
\ket{1}_x\otimes\ket{0}_y + \hat{a}_y^{\dag}
\hat{a}_x\ket{1}_x\otimes\ket{0}_y =
\nonumber \\
= 
\hat{a}_y^{\dag}
\hat{a}_x\ket{1}_x\otimes\ket{0}_y =
\ket{0}_x\otimes\ket{1}_y = \ket{y},
\nonumber \\
\hat{S}_2 \ket{y} = \hat{a}_x^{\dag} \hat{a}_y
\ket{0}_x\otimes\ket{1}_y + \hat{a}_y^{\dag}
\hat{a}_x\ket{0}_x\otimes\ket{1}_y =
\nonumber \\
=
\hat{a}_x^{\dag} \hat{a}_y
\ket{0}_x\otimes\ket{1}_y
=\ket{1}_x\otimes\ket{0}_y = \ket{x}.
\label{eqEntangS2MatrixPre}
\end{eqnarray}
From \eqref{eqEntangS2MatrixPre} we obtain the following matrix representation
of the operator $\hat{S}_2$:
\begin{equation}
\hat{S}_2 = 
\left(
\begin{array}{cc}
0 & 1 \\
1 & 0 
\end{array}
\right).
\label{eqEntangS2Matrix}
\end{equation}
Analogous to the operator $\hat{S}_1$, from \eqref{eqEntangS2MatrixPre} and
\eqref{eqEntangS2Matrix} one can find two eigenvalues $s_1 =
1$ and $s_2 = -1$.
For the first eigenvalue, the eigenvector is
\begin{equation}
\ket{s_1} = \frac{1}{\sqrt{2}}\left(\ket{x} +
\ket{y}\right),
\label{eq:part2:pol:stocks_s2_1}
\end{equation}
and for the second
\begin{equation}
\ket{s_2} = \frac{1}{\sqrt{2}}\left(\ket{x} - \ket{y}\right).
\label{eq:part2:pol:stocks_s2_2}
\end{equation}

For the operator $\hat{S}_3$, we have:
\begin{eqnarray}
  \hat{S}_3 \ket{x} = \frac{1}{i}\left(\hat{a}_x^{\dag} \hat{a}_y
\ket{1}_x\otimes\ket{0}_y - \hat{a}_y^{\dag}
\hat{a}_x\ket{1}_x\otimes\ket{0}_y\right) =
\nonumber \\
= 
-\frac{1}{i}\hat{a}_y^{\dag}
\hat{a}_x\ket{1}_x\otimes\ket{0}_y =
-\frac{1}{i}\ket{0}_x\otimes\ket{1}_y =
-\frac{\ket{y}}{i} = i \ket{y},
\nonumber \\
\hat{S}_2 \ket{y} = \frac{1}{i}\left(\hat{a}_x^{\dag} \hat{a}_y
\ket{0}_x\otimes\ket{1}_y + \hat{a}_y^{\dag}
\hat{a}_x\ket{0}_x\otimes\ket{1}_y\right) =
\nonumber \\
=
\frac{1}{i}\hat{a}_x^{\dag} \hat{a}_y
\ket{0}_x\otimes\ket{1}_y
=\frac{1}{i}\ket{1}_x\otimes\ket{0}_y =
-i \ket{x}.
\label{eqEntangS3MatrixPre}
\end{eqnarray}
Thus, from \eqref{eqEntangS3MatrixPre} we obtain the following
matrix representation of the operator $\hat{S}_3$:
\begin{equation}
\hat{S}_3 = 
\left(
\begin{array}{cc}
0 & i \\
-i & 0 
\end{array}
\right).
\label{eqEntangS3Matrix}
\end{equation}

%% (%i1) A: matrix([0, i],[-i, 0]);  
%%                                   [  0   i ]
%% (%o1)                             [        ]
%%                                   [ - i  0 ]
%% (%i2) [vals, vecs] : eigenvectors (A);
%% (%o2)        [[[- %i i, %i i], [1, 1]], [[[1, - %i]], [[1, %i]]]]
%% (%i3) vecs;
%% (%o3)                      [[[1, - %i]], [[1, %i]]]
%% (%i4) 

From \eqref{eqEntangS3Matrix}, one can find that the eigenvalues
are again $s_1 = 1$ and $s_2 = -1$. At the same time, the eigenstates of the operator
$\hat{S}_3$ are the states with left and right circular polarizations:
\begin{eqnarray}
  \ket{ s_1 } = \ket{ - } = \frac{1}{\sqrt{2}}
  \left(
  \ket{x} - i \ket{y}
  \right),
  \nonumber \\
  \ket{ s_2 } = \ket{ + } = \frac{1}{\sqrt{2}}
  \left(
  \ket{x} + i \ket{y}
  \right).
  \label{eqEntangS3Eigenvec}
\end{eqnarray}

TBD: Add the connection with Pauli matrices

% Acting with the operators \eqref{eqEntangStokesOper} on the state 
% \eqref{eqEntangSimpleState}:
% \begin{eqnarray}
% \hat{S}_0\left|\psi\right> = 
% \alpha \ket{x} + 
% \beta \ket{y},
% \nonumber \\
% \hat{S}_1\left|\psi\right> = 
% \alpha \ket{x} - 
% \beta \ket{y},
% \nonumber \\
% \hat{S}_2\left|\psi\right> = 
% \beta \ket{x} + 
% \alpha \ket{y},
% \nonumber \\
% \hat{S}_3\left|\psi\right> = 
% -i \beta \ket{x} + 
% i \alpha \ket{y}.
% \label{eqEntangStokesPsi}
% \end{eqnarray}

% Using \eqref{eqEntangStokesPsi} 
% one can write the values of the averaged Stokes parameters for the state
% given by \eqref{eqEntangSimpleState}:
% \begin{eqnarray}
% \left<\hat{S}_0\right> = 
% \left<\psi\right|\hat{S}_0\left|\psi\right> = 1,
% \nonumber \\
% \left<\hat{S}_1\right> = 
% \left<\psi\right|\hat{S}_1\left|\psi\right> = 
% \left|\alpha\right|^2 - \left|\beta\right|^2,
% \nonumber \\
% \left<\hat{S}_2\right> = 
% \left<\psi\right|\hat{S}_2\left|\psi\right> =
% \alpha^{*}\beta + \alpha\beta^{*} = 
% 2 Re\left(\alpha^{*}\beta\right),
% \nonumber \\
% \left<\hat{S}_3\right> = 
% \left<\psi\right|\hat{S}_3\left|\psi\right> = 
% \frac{\alpha^{*}\beta - \alpha\beta^{*}}{i} =
% 2 Im\left(\alpha^{*}\beta\right).
% \label{eqEntangStokesMid}
% \end{eqnarray}

% From \eqref{eqEntangStokesMid}, it follows that the degree of polarization of the state
% $\left|\psi\right>$ equals 1.