%% -*- coding:utf-8 -*- 
\section{Determination of photon statistics through the distribution of photocounts}
Usually, the main interest is the photon statistics characterizing the light beam. To determine it from the measured
$P_m\left(T\right)$, one should invert \eqref{eqCh4_49}, expressing
$P\left(\bar{I}\right)$ through $P_m\left(T\right)$.

There are several methods to find $P\left(\bar{I}\right)$. (See \cite{bDvait1973} for inversion methods.) We will focus on one of them. Consider the expression \eqref{eqCh4_49}
\[
P_m\left(T\right) = 
\int_0^{\infty}
P\left(\bar{I}\right)
\frac{\left(\xi \bar{I}T\right)^m}{m!} e^{-
  \xi \bar{I} T} 
d \bar{I}.
\]
From this equality, given the measured $P_m\left(T\right)$, one needs to determine
the probability density $P\left(\bar{I}\right)$ that characterizes
the statistics of the incident light.

Let us change the variable to $u = \xi \bar{I} T$ and define a new
probability density 
\[
P_T\left(u\right) = \frac{1}{\xi T} P\left(\bar{I}\right).
\]
We have: 
\[
P_T\left(u\right) du = \frac{1}{\xi T} P\left(\bar{I}\right) \xi T d
\bar{I} = P\left(\bar{I}\right) d \bar{I}.
\]
Hence: 
\[
\int_0^{\infty}P_T\left(u\right) du = 
\int_0^{\infty}P\left(\bar{I}\right) d\bar{I} = 1, 
\]
as it should be. Rewrite equation \eqref{eqCh4_49} using these variables:
\begin{equation}
P_m\left(T\right) = 
\int_0^{\infty}
P_T\left(u\right)
\frac{u^m}{m!} e^{-u} 
d u.
\label{eqCh4_53}
\end{equation}
Next, we use the orthogonal Laguerre polynomials
\begin{eqnarray}
L_n\left(y\right) = \sum_{k = 0}^{n}{n\choose k}
\frac{\left(-y\right)^k}{k!},
\nonumber \\
{n\choose k} = C_k^n = \frac{n!}{k!\left(n - k\right)!}
\label{eqCh4_54}
\end{eqnarray}
which satisfy the orthogonality conditions
\begin{equation}
\int_0^{\infty}L_p\left(y\right)L_q\left(y\right)e^{-y} dy =
\delta_{pq},
\label{eqCh4_TaskLager1}
\end{equation}
which can be put into the form by substituting $y \rightarrow 2y$:
\begin{equation}
2 \int_0^{\infty}L_p\left(2 y\right)L_q\left(2 y\right)e^{-2 y} dy =
\delta_{pq}.
\label{eqCh4_TaskLager2}
\end{equation}
Now, expand $P_T\left(u\right)$ in a series of Laguerre polynomials:
\begin{equation}
P_T\left(u\right) = 
\sum_{n = 0}^{\infty}
A_n L_n\left(2u \right)
e^{-u} 
\label{eqCh4_55}
\end{equation}
Using the orthogonality conditions \eqref{eqCh4_TaskLager2}, we get for
the expansion coefficients $A_n$ the expression  
\begin{equation}
A_n = 2 \int_0^{\infty}
L_n\left(2 u\right)P_T\left(u\right)e^{-u}du.
\label{eqCh4_56}
\end{equation}
Substituting $L_n\left(2 u\right)$ in the form of the series
\eqref{eqCh4_54} into \eqref{eqCh4_56}, we obtain: 
\begin{eqnarray}
A_n = 2 \sum_{k = 0}^{k = n}
{n\choose k}\left(- 2\right)^k\int_0^{\infty}
P_T\left(y\right)\frac{y^k}{k!}e^{-y} dy = 
\nonumber \\
= 2 \sum_{k = 0}^{k = n}
{n\choose k}
\left(- 2\right)^k
P_k\left(T\right),
\label{eqCh4_57}
\end{eqnarray}
where the expression
\[
\int_0^{\infty}P\left(y\right)\frac{y^k}{k!}e^{-y}dy =
P_k\left(T\right)
\]
was used. 

Assume that from measurements we know a sufficient number of values $P_k\left(T\right)$. This will allow us to compute a sufficient number of expansion coefficients $A_n$ using formula \eqref{eqCh4_57}. Substituting them into the series \eqref{eqCh4_55} for
$P_T\left(u\right)$ and reverting to the original variables, we get
\begin{equation}
P\left(\bar{I}\right) = \xi T\sum_{k = 0}^{\infty}
A_n L_n\left(2 \xi T \bar{I}\right)e^{- \xi T \bar{I}}.
\label{eqCh4_58}
\end{equation}
Using this expression, it is in principle possible to invert Mandel's formula. The difficulty here lies in the fact that only a finite number of $P_n\left(T\right)$ are known with limited accuracy.