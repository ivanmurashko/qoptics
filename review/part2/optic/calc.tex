%% -*- coding:utf-8 -*- 
\section{Photon Counting and Statistics}
We have confirmed that the quantization of the optical field and the existence of quanta of light (photons) preserve the overall pattern of interference phenomena, provided that the observation is conducted long enough for an averaged pattern to emerge. In our formulas, this corresponds to averaging using the density matrix. However, the existence of photons enables experiments of a new kind, based on photon counting and the investigation of their statistical regularities. These methods are called photon counting methods. Their essence is as follows: the studied light is directed onto a photodetector connected to a counter that counts the number of photoelectrons recorded over a certain time interval. A shutter before the photodetector (or a circuit lock) controls the counting duration. With the counter reset to zero, the shutter opens for a time $T$, and the number of photoelectrons is recorded. After a time longer than the correlation time $\tau_c$, the process repeats many times. Based on the measurement results, one can determine $P_m\left(T\right)$ — the probability of registering $m$ photoelectron counts during time $T$: 
\begin{equation}
P_m\left(T\right) = \frac{N_m}{N},
\label{eqCh4_40}
\end{equation}
where $N_m$ is the number of measurements in which $m$ photoelectrons were recorded, and $N$ is the total number of measurements, which must be large. It is assumed, of course, that the light flux is stationary. The obtained distribution contains information about the spectral properties of light beams. The first task here is how, based on the statistics of photo counts that we can measure, to find the photon statistics needed to obtain information about the properties of light beams.  
