%% -*- coding:utf-8 -*- 
\section{Higher-Order Coherence}
First- and second-order coherence functions are special,
although fundamental cases of coherence functions. The
$n$th-order coherence function in the quantum case is defined
as follows 
\begin{eqnarray}
G^{(n)}\left(x_1, x_2, \dots , x_n\right) =  
\nonumber \\
=
\frac{\left<
\hat{E}^{(-)}\left(x_1\right)
\hat{E}^{(-)}\left(x_2\right)
\dots
\hat{E}^{(-)}\left(x_n\right)
\hat{E}^{(+)}\left(x_{n}\right)
\hat{E}^{(+)}\left(x_{n - 1} \right)
\dots
\hat{E}^{(+)}\left(x_{1}\right)
\right>}
{
\left<
\hat{E}^{(-)}\left(x_1\right)
\hat{E}^{(+)}\left(x_1\right)
\right>
\dotsc
\left<
\hat{E}^{(-)}\left(x_{n}\right)
\hat{E}^{(+)}\left(x_{n}\right)
\right>
}.
\label{eqCh4_39}
\end{eqnarray}
Angle brackets denote quantum-mechanical averaging using
the statistical operator (density matrix), $x$ denotes the set
of variables $t$, $\vec{r}$. The degree of $n$th-order coherence
determines the counting rate in an experiment where $n$ photons
are registered in some way. 
