%% -*- coding:utf-8 -*- 
\section{Quantum expression for the photocount distribution}
The expression for the photocount distribution obtained above is based on
the semiclassical approach. A fully quantum expression is given by \cite{bLoudon1976}:
\begin{equation}
P_m\left(T\right) = Sp \left\{
\hat{\rho}\hat{N}
\left[
\frac{\left(\beta \hat{\bar{I}}\left(t\right) T\right)^m}{m!}
e^{- \beta \hat{\bar{I}}\left(t\right) T}
\right]
\right\} 
\label{eqCh4_59}
\end{equation}
where
\begin{eqnarray}
\hat{\bar{I}}\left(t\right) = \frac{1}{T}\int_t^{t + T}2
\varepsilon_0 c \hat{\bar{E}}^{(-)}\left(t'\right)
\hat{\bar{E}}^{(+)}\left(t'\right)dt' =
\nonumber \\
= \frac{2}{T}\int_t^{t + T}
\sqrt{\frac{\varepsilon_0}{\mu_0}} \hat{\bar{E}}^{(-)}\left(t'\right)
\hat{\bar{E}}^{(+)}\left(t'\right)dt'
\nonumber
\end{eqnarray}
corresponds to the operator of the energy flux averaged over the counting period,
$\beta$ is the quantum efficiency of the photodetector. Formula
\eqref{eqCh4_59} looks very 
similar to Mandel's formula. The difference is that here the classical
fields are replaced by their operators, averaging is performed quantum mechanically
using the statistical operator, and there is the normal ordering operator
$\hat{N}$. The action of this operator reduces to rearranging the operators
so that annihilation operators stand to the right of creation operators.
When expanding
the operator inside \eqref{eqCh4_59} in a power series, terms of the form
$\left(\hat{\bar{E}}^{(-)}\hat{\bar{E}}^{(+)}\right)^n$ appear.  
The action of $\hat{N}$ on them is given by the expression 
\[
\hat{N}\left(\hat{\bar{E}}^{(-)}\hat{\bar{E}}^{(+)}\right)^n = 
\left(\hat{\bar{E}}^{(-)}\right)^n\left(\hat{\bar{E}}^{(+)}\right)^n.
\]
Such terms correspond to higher-order coherences, meaning
the photocount distribution depends on coherences of all orders. Formula \eqref{eqCh4_59} can be justified as follows:
Assume that the field is in a coherent state
$\left|\left\{\alpha_k\right\}\right>$. Then 
the probability of electron emission in the time interval $dt$ is 
\[
Pdt = \sigma
\left<\left\{\alpha_k\right\}\right|
\hat{E}^{(-)} \hat{E}^{(+)}
\left|\left\{\alpha_k\right\}\right> dt = 
\sigma \left(E^{*} E\right)dt,
\]
where $\sigma$ characterizes the efficiency of the photocathode, and $E$ is the classical field corresponding to the state $\left|\left\{\alpha_k\right\}\right>$
\begin{eqnarray}
\hat{E}^{(+)} \left|\left\{\alpha_k\right\}\right> = E \left|\left\{\alpha_k\right\}\right>,
\nonumber \\
\left<\left\{\alpha_k\right\}\right|\hat{E}^{(-)} = E^{*} \left<\left\{\alpha_k\right\}\right|.
\nonumber
\end{eqnarray}

Further consideration proceeds in the same way as in
the semiclassical approach. One obtains a similar formula:
\begin{equation}
P_m\left(\left.T\right|\left\{\alpha_k\right\}\right) = 
\frac{\left(\sigma \overline{E^\ast E} T\right)^m}{m!}
e^{- \sigma T \overline{E^\ast E}}
\label{eqCh4_60}
\end{equation}
where
\[
\overline{E^\ast E} = \frac{1}{T} \int_t^{t + T}E^\ast\left(t'\right)
E\left(t'\right)dt'.
\]
This expression can be written in operator form:
\begin{equation}
P_m\left(\left.T\right|\left\{\alpha_k\right\}\right) = 
\left<\left\{\alpha_k\right\}\right|
\hat{N}
\left\{
\frac{\left(\sigma \overline{\hat{E}^{(-)} \hat{E}^{(+)}} T\right)^m}{m!}
e^{- \sigma T \overline{\hat{E}^{(-)} \hat{E}^{(+)}}}
\right\}
\left|\left\{\alpha_k\right\}\right>,
\label{eqCh4_61}
\end{equation}
where $P_m\left(\left.T\right|\left\{\alpha_k\right\}\right)$ is the probability of counting $m$ photoelectrons in time $T$ if the field
is in the state $\left|\left\{\alpha_k\right\}\right>$.
The operator inside the curly braces can  
be considered as the operator counting $m$ photoelectrons in time $T$. If
the field is in a statistically mixed state with statistical
operator, defined in the diagonal representation by the function 
$P_m\left(\left.T\right|\left\{\alpha_k\right\}\right)$,
\[
\hat{\rho} = \int_{\left\{\alpha_k\right\}}
P\left(\left\{\alpha_k\right\}\right)
\left|\left\{\alpha_k\right\}\right>
\left<\left\{\alpha_k\right\}\right|
\prod_k d^2\alpha_k,
\]
the average value of the operator in expression \eqref{eqCh4_61},
according to \eqref{eqCh1_middleO}, will be 
\begin{eqnarray}
P_m\left(T\right) =
\int_{\left\{\alpha_k\right\}}
P\left(\left\{\alpha_k\right\}\right)
\left<\left\{\alpha_k\right\}\right|
\cdot
\nonumber \\
\cdot 
\hat{N}
\left\{
\frac{\left(\sigma \overline{\hat{E}^{(-)} \hat{E}^{(+)}} T\right)^m}{m!}
e^{- \sigma T \overline{\hat{E}^{(-)} \hat{E}^{(+)}}}
\left|\left\{\alpha_k\right\}\right>
\right\}
\prod_k d^2\alpha_k
\label{eqCh4_62}
\end{eqnarray}
The obtained formula can be written in any representation, since the operation $Sp$ does not depend on the representation. We have
\begin{equation}
P_m = Sp\left(
\hat{\rho}
\hat{N}
\left\{
\frac{\left(\sigma \overline{\hat{E}^{(-)} \hat{E}^{(+)}} T\right)^m}{m!}
e^{- \sigma T \overline{\hat{E}^{(-)} \hat{E}^{(+)}}}
\right\}
\right),
\label{eqCh4_63}
\end{equation}
where
\[
\overline{\hat{E}^{(-)} \hat{E}^{(+)}} = \frac{1}{T} \int_t^{t + T}\hat{E}^{(-)}\left(t'\right)
\hat{E}^{(+)}\left(t'\right)dt'.
\]

Formula \eqref{eqCh4_63} fully corresponds to formula
\eqref{eqCh4_59} taken from \cite{bLoudon1976}. Calculations using
formula \eqref{eqCh4_63} are somewhat 
more complicated than using Mandel's formula. As an example, let's consider a few
simple cases. First, consider single-mode states. Then
\eqref{eqCh4_63} can be written as 
\[
P_m = Sp\left(
\hat{\rho}
\hat{N}
\left\{
\frac{\left(\gamma \overline{\hat{a}^{\dag} \hat{a}} T\right)^m}{m!}
e^{- \gamma T \overline{\hat{a}^{\dag} \hat{a}}}
\right\}
\right)
\]
where $\gamma$ is a coefficient characterizing the photocathode
efficiency in this notation. The operator 
\[
\hat{N}
\left\{
\frac{\left(\gamma \overline{\hat{a}^{\dag} \hat{a}} T\right)^m}{m!}
e^{- \gamma T \overline{\hat{a}^{\dag} \hat{a}}}
\right\}
\]
has only diagonal elements in the number representation (photon number). Then $Sp\left(\dots\right)$ will be determined
only by the diagonal elements of the density matrix, which we denote as $P_n = \bra{n}\hat{\rho}\ket{n}$.

Then \eqref{eqCh4_63} can be represented as:
\begin{eqnarray}
P_m = \sum_n P_n 
\bra{n}
\hat{N}
\frac{\left(\gamma \hat{a}^{\dag} \hat{a} T\right)^m}{m!}
e^{- \gamma T \hat{a}^{\dag} \hat{a}}
\ket{n} = 
\nonumber \\
=
\sum_n P_n 
\frac{\left(\gamma T\right)^m}{m!}
\bra{n}
\sum_l\left(-1\right)^l
\frac{\left(\gamma T\right)^l}{l!}
\left(\hat{a}^{\dag}\right)^{l + m}
\left(\hat{a}\right)^{l + m}
\ket{n} = 
\nonumber \\
=
\sum_{n = m}
P_n 
\frac{\left(\gamma T\right)^m}{m!}
\sum_{l = 0}^{n - m}\left(-1\right)^l
\frac{\left(\gamma T\right)^l}{l!}
\frac{n!}{\left(n - m - l\right)!},
\label{eqCh4_64}
\end{eqnarray}
which follows from the relation:
\begin{eqnarray}
\bra{n}
\left(\hat{a}^{\dag}\right)^{l + m}
\left(\hat{a}\right)^{l + m}
\ket{n} = 
\nonumber \\
= \left\{
n \left(n - 1\right)\left(n - 2\right) \dotsc
\left(n - m - l + 1\right)
\right\} = 
\frac{n!}{\left(n - m - l\right)!}.
\nonumber
\end{eqnarray}
Expression \eqref{eqCh4_64} can be further transformed. The known expansion (Newton's binomial) is:
\[
\left(1 - \gamma T\right)^{n - m} = 
\sum_{l = 0}^{n -m}
\left(-1\right)^l
\left(\gamma T\right)^l
\frac{\left(n - m\right)!}{l!\left(n - m - l\right)!}.
\]
From here we have:
\begin{equation}
P_m\left(T\right) = 
\sum_{n = m}^{\infty}
P_n 
\frac{n!}{m!\left(n - m\right)!}
\left(\gamma T\right)^m
\left(1 - \gamma T\right)^{n - m}.
\label{eqCh4_64a}
\end{equation}
Note that, as known from probability theory (see Bernoulli distribution), the general term of the sum gives the probability that out of $n$
photons $m$ are registered, and $n - m$ remain
unregistered. Let's apply the general formula \eqref{eqCh4_64} to the case of
chaotic light. Then  
\[
P_n = \frac{\bar{n}^n}{\left(\bar{n} + 1\right)^{n + 1}},
\]
and from \eqref{eqCh4_64} we get 
\begin{eqnarray}
P_m\left(T\right) = 
\sum_{n = m}^{\infty}
\frac{\bar{n}^n}{\left(\bar{n} + 1\right)^{n + 1}}
\left(\gamma T\right)^m
\left(1 - \gamma T\right)^{n - m} 
\frac{n!}{m!\left(n - m\right)!}
=
\nonumber \\
=
\left(\gamma T\right)^m
\sum_{l = 0}^{\infty}
\frac{\bar{n}^{l + m}}{\left(\bar{n} + 1\right)^{l + m + 1}}
\left(1 - \gamma T\right)^{l}
\frac{\left(l + m\right)!}{m! l!}. 
\label{eqCh4_65}
\end{eqnarray}

In the transformation, the summation index substitution is made: $l = n - m$,
$n = l + m$.

The known expansion \cite{bDvait1973} is:
\[
\left(1 - x\right)^{-\left(m + 1\right)} = 
\sum_{l = 0}^\infty
\frac{x^l \left(m + l\right)!}{m! l!}.
\]
Using it, we get:
\begin{eqnarray}
P_m\left(T\right) = 
\frac{\left(\gamma T \bar{n}\right)^m}{\left(1 + \bar{n}\right)^{m +
    1}}
\left(
1 - \frac{\bar{n}}{1 + \bar{n}}\left(1 - \gamma T\right)
\right)^{-\left(m + 1\right)} = 
\nonumber \\
=
\frac{\left(\gamma T \bar{n}\right)^m}{\left(1 + \bar{n}\right)^{m +
    1}} 
\left\{
\frac{1 - \gamma T \bar{n}}{1 + \bar{n}} 
\right\}^{-\left(m + 1\right)} = 
\frac{\left(\gamma T \bar{n}\right)^m}{\left(1 + \gamma T \bar{n}\right)^{m +
    1}} =
\frac{\bar{m}^m}{\left(\bar{m} + 1\right)^{m + 1}},
\label{eqCh4_66}
\end{eqnarray}
where $\bar{m} = \gamma T \bar{n}$.  We obtained the same result as
in the semiclassical 
consideration. The distribution $P_m\left(T\right)$ mirrors $P_n$, but at a different 
scale. Instead of $\bar{n}$, there is $\bar{m} = \gamma T \bar{n}$. Now consider another limiting case: the field is in a
coherent state. It is convenient to consider it in the representation
of coherent states. We have  
\begin{eqnarray}
P_m\left(T\right) = 
\left<\alpha\right|
\hat{N}
\frac{\left(\gamma \hat{a}^{\dag} \hat{a} T\right)^m}{m!}
e^{- \gamma T \hat{a}^{\dag} \hat{a}}
\left|\alpha\right> = 
\nonumber \\
=
\frac{\left(\gamma \left|\alpha\right|^2 T\right)^m e^{-\gamma
    \left|\alpha\right|^2 T}}{m!} = 
\frac{\bar{m}^m}{m!}e^{-\bar{m}},
\label{eqCh4_67}
\end{eqnarray}
where $\bar{m} = \gamma \left|\alpha\right|^2 T = \gamma \bar{n} T$.

The photon number distribution for a coherent state is known
\eqref{eqCh1_PuassonCoh}:
\[
P_n = \frac{\bar{n}^n e^{-\bar{n}}}{n!}.
\]
Thus, in this case as well, the photocount distribution
mirrors the photon distribution but at a different scale. Instead of
$\bar{n}$, there is $\bar{m}$.  In the general case, such a simple relation does not hold. From the derivation of the quantum photocount formula
it follows that as long as $P\left(\left\{\alpha_k\right\}\right)$ in the expression 
for the statistical operator can be interpreted as a probability distribution, there is no discrepancy between semiclassical results
(Mandel's formula) and the fully quantum treatment. However, if
$P\left(\left\{\alpha_k\right\}\right)$ is negative in some regions, the results will differ.  
