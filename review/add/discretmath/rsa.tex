%% -*- coding:utf-8 -*- 
\section{RSA Algorithm}
\label{AddRSA}
\rindex{RSA algorithm}
The RSA algorithm (an acronym for the surnames Rivest, Shamir, and Adleman) is a 
public-key encryption algorithm
\footnote{A public-key (asymmetric) encryption algorithm is one in which 
  two different keys are used: one for encryption and the other for decryption},  
based on the difficulty of factoring a number into prime factors.  

\subsection{Key Generation}
\rindex{RSA algorithm!key generation}
Consists of several steps
\input ./add/discretmath/rsagenalgo.tex

The original numbers $p$ and $q$ are kept secret, since they make
computing $\phi(n)$ trivial.

It should be noted that to obtain the private key from the public key,
one must compute $\phi(n)$ given $n$. This is a difficult problem (if 
$p$ and $q$ are unknown), as noted in remark
\ref{rem:add:discretmath:eulerfuncomplex}. 

\input ./add/discretmath/rsagenex.tex

\subsection{Encryption}
\rindex{RSA algorithm!encryption}
\input ./add/discretmath/rsaencrypt.tex

\subsection{Decryption}
\rindex{RSA algorithm!decryption}
\input ./add/discretmath/rsadecrypt.tex

\subsection{Proof}
We want to prove that 
\[
\left(m^e\right)^d \equiv m \mod{p \cdot q}
\]
for any positive number $m$ when $p$ and $q$ are prime numbers, and $e$
and $d$ satisfy the relation
\[
d \cdot e \equiv 1 \mod{\phi\left(p \cdot q\right)},
\]
which can be rewritten as
\[
d \cdot e - 1 = h \left(p - 1\right)\left(q - 1\right).
\]

Thus,
\[
m^{e \cdot d} = m \cdot m^{h \left(p - 1\right)\left(q - 1\right)}.
\]
Next, two cases are possible: when $m$ is divisible by $p$, and when $m$ and
$p$ are coprime.

In the first case,
\[
m^{e \cdot d} \equiv m \equiv 0 \mod{p}.
\]
In the second case, we use
\myref{addDiscretSmallFerma}{Fermat's little theorem}:
\[
m \cdot m^{h \left(p - 1\right)\left(q - 1\right)} 
= m \left(m^{p - 1}\right)^{h \left(q - 1\right)} \equiv m \cdot 1^{h
  \left(q - 1\right)} \equiv m \mod{p}.
\]
Similarly, we have either
\[
m^{e \cdot d} \equiv m \equiv 0 \mod{q}
\]
or, by Fermat's little theorem,
\[
m \cdot m^{h \left(p - 1\right)\left(q - 1\right)} 
= m \left(m^{q - 1}\right)^{h \left(p - 1\right)} \equiv m \cdot 1^{h
  \left(p - 1\right)} \equiv m \mod{q}.
\]
Thus, we have the following two types of congruences:
the trivial equalities
\begin{eqnarray}
x_1 = m \equiv m \mod p,
\nonumber \\
x_1 = m \equiv m \mod q,
\nonumber
\end{eqnarray}
and the just obtained congruences
\begin{eqnarray}
x_2 = m^{ed} \equiv m \mod p,
\nonumber \\
x_2 = m^{ed} \equiv m \mod q,
\nonumber
\end{eqnarray}
from which, by \myref{thm:chineseremainder}{the Chinese remainder theorem}, 
we get
\[
x_1 \equiv x_2 \mod p \cdot q,
\]
i.e.
\[
m^{e \cdot d} \equiv m \mod n.
\]