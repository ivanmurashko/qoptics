%% -*- coding:utf-8 -*-
\chapter{Density Matrix and Operator}
\label{AddState}

\subsection{Pure and Mixed States}
\rindex{Mixed state}
\rindex{Pure state}
In general, the state of a quantum system is considered defined if
the wave function describing this state is known. The wave function
can be obtained as follows. We perform measurements of the eigenvalues
of operators corresponding to a complete set. The number of these measured
quantities (quantum numbers) equals the number of degrees of freedom of the system. Using the obtained quantum numbers,
we find the wave function which
is an eigenfunction for each of the operators in the complete set,
and the eigenvalues must correspond to the measured quantum numbers.
In general (up to a constant factor) such
a function is unique. It can be accepted that this function describes the state
at the initial moment of time. The state at subsequent moments in time
can be found using the Schrödinger equation. Such a state with
a precisely defined wave function is called
a \textbf{pure state}.

In some cases, the wave function of the system cannot be uniquely
defined, for example in a system with a large number of degrees of freedom.
In this case, we consider a statistical mixture of states in which
each wave function enters with its statistical weight. That is,
the system may be in states described by wave functions
\(\left\{\left|\psi_n\right>\right\}\). At the same time, the probability
of the system being in a particular state \(\left|\psi_n\right>\) from
this set equals \(p_n\). Obviously,
\[
\sum_n p_n = 1.
\]
A state described by a mixture of pure states is called
a \textbf{mixed state}.

\subsection{Density Matrix}
For a system in a mixed state, to calculate the expectation values of a certain operator, it is convenient to use the formalism of the density matrix, which was proposed by John von Neumann and
independently by Landau and Bloch in 1927.
\rindex{Density matrix!definition}

In \ref{eqAddDiracMidViaRho}, it was shown that the expectation value
of some operator \(\hat{L}\) in a pure state
\(\left|\psi_n\right>\) can be written as
\[
\left< \hat{L} \right>_{\psi_n} = \mathrm{Sp} \left(\hat{\rho_n} \hat{L} \right),
\]
where
\[
\hat{\rho_n} = \hat{P}_n = \left|\psi_n\right>\left<\psi_n\right|.
\]

For a mixed state, the formula for calculating the expectation values can be written
as the sum of the expectation values in pure states with given weights:
\[
\left< \hat{L} \right>_{mix} = \sum_n p_n \left< \hat{L}
\right>_{\psi_n}.
\]
Thus, the expectation value in a mixed state can be written in the following
form
\begin{eqnarray}
\left< \hat{L} \right>_{mix} = \mathrm{Sp} \left(\hat{\rho} \hat{L} \right),
\end{eqnarray}
where
\begin{eqnarray}
\hat{\rho} = \sum_n p_n \hat{\rho_n} = 
\sum_n p_n \left|\psi_n\right>\left<\psi_n\right|.
\end{eqnarray}