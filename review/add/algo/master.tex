%% -*- coding:utf-8 -*-
\section{Master Theorem for Recurrence Relations}

\begin{theorem}[Master Theorem for Recurrence Relations]
\label{addAlgoMasterTheorem}
If the following recurrence relation holds for the complexity
of some algorithm
\[
T\left( n \right ) = a T \left( \frac{n}{b} \right) + f\left( n \right ),
\]
then it is possible to determine the asymptotic behavior of the function 
$T\left( n\right ) $ in the following cases
\begin{enumerate}
\item If $f\left(n\right) = O\left( n^{\log_b a - \epsilon}\right)$,
  for some $\epsilon > 0$, then 
$T\left(n\right) = \Theta\left(n^{\log_b a}\right)$
\item If 
$f\left(n\right) = \Theta\left( n^{\log_b a}\log^{k} n\right)$, then 
$T\left(n\right) = \Theta\left(n^{\log_b a}\log^{k + 1} n\right)$
\item If $f\left(n\right) = \Omega\left( n^{\log_b a + \epsilon}\right)$,
  for some $\epsilon > 0$ and $a f\left(\frac{n}{b}\right) \le c f
  \left( n \right)$ for some constant $c < 1$ and large $n$, then 
$T\left(n\right) = \Theta\left(f\left(n\right)\right)$
\end{enumerate}
\end{theorem}