%% -*- coding:utf-8 -*-
\section{Computability Theory}

A Turing machine $M$ (see \autoref{figAddAlgoDecidability}) can be interpreted as a certain function $m$ whose argument is a set of symbols representing the initial state of the tape $x$. As a result of processing $x$, the machine $M$ can finish its work, i.e., end up in some final state $f \in F \subset Q$. In this case, it is said that $M$ accepts $x$. Then the corresponding function value is $m(x) = 1$. 

At the same time, the computation of the function $m(x)$ can be interpreted as a certain problem.  
\begin{definition}[Problem]
We call a problem with respect to a Turing machine a yes/no answer to a question about some set of possible objects.
\end{definition}

\begin{example}[Hamiltonian Graph]
The Hamiltonian graph problem - for a given graph, whether there exists a path (chain) containing each vertex of this graph exactly once.
\nonumber
\end{example}

\input ./add/algo/figdecidability.tex

%% \begin{definition}[Language $L\left(M\right)$ of a Turing Machine $M$]
%% The set of inputs $x$ for which the Turing machine $M$ halts is called the language $L\left(M\right)$. That is,  
%% $\forall x \in L\left(M\right): m(x) = 1$.
%% \end{definition}

In the case of halting (halt) of $M$, i.e., when $M$ reaches a state from which no further transition is defined, it is said that $M$ rejects $x$. In this case, the corresponding function value is $m(x) = 0$. 

Also, the machine $M$ on input $x$ may never terminate its operation. In this case, the value of the function $m(x)$ is undefined.    

\begin{definition}[Recursive Language]
The set of strings $L$
\footnote{the set of initial tape states of some
Turing machine}
is called a recursive language if there exists a Turing machine $M$
such that $\forall x \in L$ the corresponding function $m(x) = 1$, and 
$\forall x \notin L$ the corresponding function $m(x) = 0$. 
\end{definition}

Thus, $M$ corresponding to a recursive language $L\left(M\right)$ 
halts on any input. The set of recursive languages
is denoted $R$.

\begin{definition}[Recursively Enumerable Language] 
The set of strings $L$
is called a recursively enumerable language if there exists a Turing machine $M$
such that $\forall x \in L$, $M$ halts in some
final state, i.e., the corresponding function $m(x) = 1$. 
\end{definition}

Thus, $M$ corresponding to a recursively enumerable language $L\left(M\right)$ 
halts on every input belonging to the language, but may never halt on input 
not belonging to the language $L\left(M\right)$. The set of recursively enumerable languages is denoted $RE$. Obviously $R \subset RE$.

A Turing machine defines a class of functions for which the notion of computability can be defined.

\begin{definition}[Computable Function]
\footnote{
In fact, the notion of computability cannot be defined strictly mathematically and is given here in the formulation of the Church-Turing thesis.
}
A function $m$ is called computable, and the corresponding problem -
decidable if there exists some
Turing machine $M$ for it, and the language corresponding to this machine is
recursive, $L\left(M\right) \in R$. Otherwise, the corresponding
problem is called undecidable.
\end{definition}

\begin{theorem}[On the cardinality of the set of Turing Machines]
The set of Turing machines is countable.
\label{theoremAddAlgoTuringCountability}
\end{theorem}

\begin{proof}
A set is called countable \cite{bShenSet2012} if there exists a natural number assigned to each element. 
A Turing machine is uniquely determined by the following group of elements:
\begin{itemize}
\item Set of states $Q$
\item Start state $q_0$
\item Set of final states $F \subset Q$
\item Alphabet $\Sigma$
\item Transition table $\Delta$
\end{itemize}

Since each of the listed elements of a Turing machine
takes on a finite number of values, a unique (for the corresponding machine) number can be assigned to each element, and therefore to the entire Turing machine,
\footnote{
For example, one can encode each machine by a unique
sequence of numbers $0$ and $1$. The resulting sequence
is finite due to the finiteness of the set of elements comprising
the Turing machine. Any finite binary sequence corresponds to a unique natural number for which
the considered sequence is the binary representation.
}
which proves the countability
of the set of all Turing machines.
\end{proof}

After introducing the notion of a computable function, the question arises about the computability of an arbitrary function. Suppose we have some input data (function argument) $x$. Does there exist a Turing machine $M$ that accepts this data, i.e., $M$ halts in some final state? The following theorem answers this question.

\begin{theorem}[On Undecidability]
There exists a language $L_d$ that is not accepted by any Turing machine, i.e., the corresponding problem $L_d$ is fundamentally undecidable (not computable).
\label{theoremAddAlgoTuringLdUndecidability}
\end{theorem}

Language $L_d$ is often called the diagonalization language because if one constructs a matrix
\cite{bUllman2006} whose row indices denote the numbers of input data of Turing machines, and the column numbers correspond to the Turing machine number, then the value of this matrix $a_{ij}$ is equal to $1$ if $M_j$ accepts $x_i$ and $0$ otherwise. The inverted \footnote{$0$ replaced by $1$ and $1$ by $0$} numbers on the diagonal define the language $L_d$.

\begin{proof}
By theorem \myref{theoremAddAlgoTuringCountability}{}, one can talk about the $i$-th Turing machine $M_i$, where $i$ is some natural number. Suppose $x_i$ is a string not recognized by $M_i$, i.e., 
$x_i \notin L\left(M_i\right)$. The strings $x_i$ form some language $L_d$. By the definition of $L_d$, no Turing machine exists that accepts all strings of this language.
\end{proof}

\begin{definition}[Universal Turing Machine]
A universal Turing machine $M_u$ is a machine that receives on input a string $x$ and a code for some Turing machine $M$ and computes the result of applying $x$ to $M$.
\end{definition}

\input ./add/algo/figuturing.tex

The set of pairs of strings $\left(x, M\right)$ form some language $L_U$ of the machine
$M_U$. 

\input ./add/algo/figuturingproof.tex

\begin{theorem}[On the Language of the Universal Turing Machine]

The language of the universal Turing machine is recursively enumerable but not recursive, i.e., $L_U \in RE$ and $L_U \notin R$. Thus, the problem of recognizing the language of the universal Turing machine is undecidable.
\label{theoremAddAlgoTuringUniversalMachine}
\end{theorem}

\begin{proof}
The statement $L_U \in RE$ follows directly from the definition of the universal
Turing machine: if $M$ accepts $x$, then $M_U$ accepts the pair of
strings $\left(x, M\right)$.

We prove $L_U \notin R$ by contradiction. Suppose $L_U \in R$. Using such a machine $M_U$, construct a Turing machine $M_d$ that will accept all strings of the language $L_d$. For each string
$x_i \in L_d$ there is a corresponding Turing machine $M_i$ which does not accept this string, i.e., $x_i \notin L\left(M_i\right)$. The pair $\left(x_i, M_i\right)$ can be input to the universal Turing machine as shown in 
\autoref{figAddAlgoUTuringProof}. If $L_U \in R$, then machine $M_U$ can decide whether $M_i$ accepts input $x_i$ or not. If $M_i$ does not accept $x_i$, then machine $M_d$ should accept this data. Thus, we have constructed a Turing machine $M_d$ that accepts all strings $x_i \in L_d$ and rejects all strings $x_i \notin L_d$, which is impossible due to theorem \myref{theoremAddAlgoTuringLdUndecidability}{}.
\end{proof}

Every Turing machine $M$ has some code which can be interpreted as input data for another Turing machine $M_P$. The machine can either accept or reject this code. Due to the one-to-one correspondence between a Turing machine $M$ and its language $L\left(M\right)$, one can talk about a Turing machine that accepts or rejects languages $L$. If machine $M_P$ accepts some set of languages, it is said that these languages possess some property $P$.

\begin{definition}[Nontrivial Property of a Turing Machine Language]
A property of a language is called nontrivial if among all languages one can distinguish both those which satisfy the property and those which do not.
\end{definition}

\begin{theorem}[Rice's Theorem]
The problem of determining any nontrivial property of the language 
$L\left(M\right)$ of a Turing machine $M$ is undecidable.
\end{theorem}

\begin{proof}
Suppose we have a machine $M_P$ that can determine
some nontrivial language property $P$, i.e., it can
classify the languages of all 
Turing machines into two sets (each of which is nonempty). 
The first set contains all languages satisfying $P$ (the machine $M_P$
accepts languages from this set), and the second set contains all languages that do not
satisfy property $P$. 

\input ./add/algo/figrice.tex

Let us try to construct on the basis of $M_P$ a universal Turing machine $M_U$
that accepts (or rejects) pairs $\left(x, M\right)$ 
(see \autoref{figAddAlgoRiceTheorem}).

Suppose we have a machine $M_L$ that accepts strings $l$ from
some language $L$
\footnote{There is no relation between language $L$ and property $P$ at this step}.

As the first step, construct a machine $M'$ that accepts strings
$l$. At the same time, the pair $\left(x, M\right)$ is fixed and embedded in the code of $M'$. The key property of $M'$ is the following: $M'$ accepts strings $l \in L$ if and only if $M$ accepts $x$. Thus, two cases are possible:
\begin{itemize}
\item $L(M') = L(M_L)$ if $M'$ accepts $l$ as a result of accepting
  the pair $\left(x, M\right)$ 
\item $L(M') = \emptyset$ if $M'$ rejects any input
  $l$,
i.e., in the case where $M$ never halts on input $x$. 
\end{itemize}

Next, recall the property $P$ and suppose that the language $L$
(accepted by machine $M'$) has property $P$, that is, $L \in P$. Also assume that $\emptyset \notin P$,
\footnote{
If $\emptyset \in P$, then one can consider the property
$\bar{P} = \left\{\forall L: L \notin P\right\}$. In this case,
$\emptyset \notin \bar{P}$.
}. 
Thus, if the code for $M'$ is given as input to machine $M_p$, then $M_p$ accepts the input (code $M'$) if $L(M') \in P$ and rejects it if $L(M') = \emptyset$.

On the other hand, the pair $\left(x, M\right)$ is embedded in the original code of $M'$, so we can assume that $M_P$ receives exactly the pair 
$\left(x, M\right)$ as input, accepting it if $M$ accepts $x$
($L\left(M'\right) \in P$) and rejecting
this pair if $M$ does not accept $x$ 
($L\left(M'\right) = \emptyset \notin P$). Thus, using
$M_P$ we have constructed a universal Turing machine and shown that its
language is recursive, which is impossible by theorem 
\myref{theoremAddAlgoTuringUniversalMachine}{}.
\end{proof}