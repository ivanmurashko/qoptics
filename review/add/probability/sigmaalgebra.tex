%% -*- coding:utf-8 -*-
\section{Set algebra and basic concepts of probability theory}

\begin{definition}[$\sigma$-algebra]
\label{def:sigma_algebra}
Suppose that $\Omega$ is some set, then a family
$\mathcal{F}$ of subsets of the set $\Omega$ is called
a $\sigma$-algebra if the following conditions are satisfied:
\begin{itemize}
\item $\mathcal{F}$ contains $\Omega$: $\Omega \in \mathcal{F}$
\item if $\Upsilon  \in \mathcal{F}$ then $\Omega \setminus \Upsilon
  \in \mathcal{F}$
\item the union or intersection of a countable subfamily of $\mathcal{F}$
  belongs to $\mathcal{F}$
\end{itemize}
\end{definition}

\begin{definition}[Measure]
\label{def:measure} 
If $\mathcal{F}$ is some $\sigma$-algebra then
the following function 
\[
\mu: \mathcal{F} \to \left[0, \infty\right]
\]
is called a measure if 
\begin{itemize}
\item $\mu\left(\emptyset\right) = 0$
\item $\forall A, B \in \mathcal{F}, A \cap B = \emptyset$ we have
$\mu\left(A \cup B\right) = \mu\left(A\right) +
  \mu\left(B\right)$ 
\end{itemize}
\end{definition}

