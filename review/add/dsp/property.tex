%% -*- coding:utf-8 -*-
\subsection{Properties of the Discrete Fourier Transform}
The following properties of the Fourier transform should be noted, which
play a fundamental role in Shor's algorithm:

\begin{lemma}
\emph{(Shift)}
\label{lemmaAddFourierDiscretFourierShiftTime}
If $\left\{x_n\right\} \longleftrightarrow \left\{\tilde{X}_k\right\}$ then
$\left\{x_{\left(n - m\right) \mod M}\right\} \longleftrightarrow
\left\{e^{-i \omega m k}\tilde{X}_k\right\}$ 
\end{lemma}

\begin{proof}
For the sequence $\left\{x_{\left(n - m\right) \mod M}\right\}$
one can write
\begin{eqnarray}
\left\{x_{\left(n - m\right) \mod M}\right\} = 
\begin{pmatrix}
x_{-m \mod M} \\
x_{-m+1 \mod M}\\ 
\vdots \\
x_{-1 \mod M}\\
x_0 \\
x_1 \\
\vdots \\
x_{M - m - 1}
\end{pmatrix} = 
\begin{pmatrix}
x_{M - m} \\
x_{M - m + 1}\\ 
\vdots \\
x_{M - 1}\\
x_0 \\
x_1 \\
\vdots \\
x_{M - m - 1}
\end{pmatrix},
\nonumber
\end{eqnarray}
and
\begin{eqnarray}
\hat{F}\left\{x_{\left(n - m\right) \mod M}\right\} = 
\nonumber \\
= \frac{1}{\sqrt{M}}
\begin{pmatrix}
1 & 1 & 1 & \cdots & 1 \\
1 & e^{-i \omega} & e^{-2 i \omega} & \cdots & 
e^{-\left( M - 1 \right) i \omega} \\
1 & e^{-2 i \omega} & e^{-4 i \omega} & \cdots & 
e^{-2 \left( M - 1 \right) i \omega} \\
1 & e^{-3 i \omega} & e^{-6 i \omega} & \cdots & 
e^{-3 \left( M - 1 \right) i \omega} \\
\vdots & \vdots & \vdots & \ddots & \vdots \\
1 & e^{-\left( M - 1 \right) i \omega} & e^{-2\left( M - 1 \right) i \omega} & \cdots & 
e^{- \left( M - 1 \right)\left( M - 1 \right) i \omega} \\
\end{pmatrix}
\begin{pmatrix}
x_{M - m} \\
x_{M - m + 1}\\ 
\vdots \\
x_0 \\
\vdots \\
x_{M - m - 1}
\end{pmatrix} = 
\nonumber \\
= 
\begin{pmatrix}
\frac{x_{M - m}}{\sqrt{M}} + \frac{x_{M - m + 1}}{\sqrt{M}} + \cdots + 
\frac{x_0}{\sqrt{M}} + \cdots + \frac{x_{M - m - 1}}{\sqrt{M}}\\
\frac{x_{M - m}}{\sqrt{M}} + 
\frac{e^{-i \omega} x_{M - m + 1}}{\sqrt{M}} + 
\cdots + 
\frac{e^{-i \omega m  } x_0}{\sqrt{M}} + 
\cdots +
\frac{e^{-i \omega \left( M - 1 \right) } x_{M - m - 1}}{\sqrt{M}}\\ 
\frac{x_{M - m}}{\sqrt{M}} + 
\frac{e^{-2 i \omega} x_{M - m + 1}}{\sqrt{M}} + 
\cdots + 
\frac{e^{-2 i \omega m  } x_0}{\sqrt{M}} + 
\cdots +
\frac{e^{-2 i \omega \left( M - 1 \right) } x_{M - m - 1}}{\sqrt{M}}\\ 
\vdots \\
\frac{x_{M - m}}{\sqrt{M}} + 
\frac{e^{-m i \omega} x_{M - m + 1}}{\sqrt{M}} + 
\cdots + 
\frac{e^{-m i \omega m  } x_0}{\sqrt{M}} + 
\cdots +
\frac{e^{-m i \omega \left( M - 1 \right) } x_{M - m - 1}}{\sqrt{M}}\\ 
\vdots 
\end{pmatrix}.
\label{eqlemmaAddFourierDiscretFourierShiftTimeMatrix}
\end{eqnarray}
Taking into account the relation
\[
e^{-i \omega k M} = 1, k \in \left\{0, 1, \dots \right\},
\]
the expression
\eqref{eqlemmaAddFourierDiscretFourierShiftTimeMatrix} can
be rewritten in the following form
\begin{eqnarray}
\hat{F}\left\{x_{\left(n - m\right) \mod M}\right\} = 
\nonumber \\
=
\frac{1}{\sqrt{M}}
\begin{pmatrix}
x_{M - m} + \cdots +
x_{M - m - 1}\\
e^{-i \omega m } e^{-i \omega \left( M - m \right) } x_{M - m} + 
\cdots + 
e^{-i \omega 2 m } e^{-i 2 \omega \left( M - m - 1\right) }\\ 
e^{-i \omega 2 m } e^{-i 2 \omega \left( M - m \right) } x_{M - m} + 
\cdots + 
e^{-i \omega 2 m } e^{-i 2 \omega \left( M - 1\right) }\\ 
\vdots 
\end{pmatrix} = 
\nonumber \\
= 
\begin{pmatrix}
\tilde{X}_0 \\
e^{-i \omega m} \tilde{X}_1 \\
e^{-i \omega 2 m} \tilde{X}_2 \\
\vdots  
\end{pmatrix}.
\nonumber
\end{eqnarray}
\end{proof}


\begin{lemma}
\emph{(Periodicity)}
\label{lemmaAddFourierDiscretFourierPeriod}
If the sequence $\left\{x_n\right\}$ has period $r$: $x_n =
x_{n + r}$, and the number of samples $M$ is divisible by $r$, then the nonzero terms
of the Fourier transform appear with period $\frac{M}{r}$.
\begin{proof}
Indeed, if $M \mod r = 0$ and $k r \mod M \ne 0$,
then
\[
1 - e^{-i \omega k r} \ne 0,
\]
whence
\begin{eqnarray}
\tilde{X}_k = \frac{1}{\sqrt{M}}\sum_{n = 0}^{M - 1}e^{-i \omega k n}x_n = 
\nonumber \\
= \frac{1}{\sqrt{M}} \left(
\sum_{n = 0}^{r - 1}e^{-i \omega k n} x_n + 
\sum_{n = 0}^{r - 1}e^{-i \omega k \left(n + r \right) } x_{n+r} +
\right.
\nonumber \\
\left. +
\sum_{n = 0}^{r - 1}e^{-i \omega k \left(n + 2r \right) } x_{n+2r} + 
\dots  \right)=
\nonumber \\
= \frac{1}{\sqrt{M}} \left(
\sum_{n = 0}^{r - 1}e^{-i \omega k n} x_n + 
\sum_{n = 0}^{r - 1}e^{-i \omega k \left(n + r \right) } x_n + 
\sum_{n = 0}^{r - 1}e^{-i \omega k \left(n + 2r \right) } x_n + 
\dots \right) =
\nonumber \\
= \frac{1}{\sqrt{M}} \left( \sum_{n = 0}^{r - 1} x_n e^{-i \omega k n} 
\sum_{l = 0}^{\frac{M}{r}- 1} e^{-i \omega k l r } = 
\sum_{n = 0}^{r - 1} x_n e^{-i \omega k n} 
\frac{1 - e^{-i \omega k \frac{M}{r} r }}{1 - e^{-i \omega k r }}
\right) = 
\nonumber \\
=
\frac{1}{\sqrt{M}}
\frac{1 - e^{-i \omega k M }}{1 - e^{-i \omega k r}}
\sum_{n = 0}^{r - 1} x_n e^{-i \omega k n} = 0.
\label{eqAddFourierDiscretFourierPeriodCase1}
\end{eqnarray}
If $M \mod r = 0$ and $k r \mod M = 0$, then
\[
e^{-i \omega k r } = e^{-i \frac{2 \pi }{M} k r } = 1,
\]
whence
\begin{eqnarray}
\tilde{X}_k = \frac{1}{\sqrt{M}} \sum_{n = 0}^{M - 1}e^{-i \omega k n}x_n = 
\nonumber \\
= \frac{1}{\sqrt{M}} \left(
\sum_{n = 0}^{r - 1}e^{-i \omega k n} x_n + 
\sum_{n = 0}^{r - 1}e^{-i \omega k n } x_{n+r} + 
\sum_{n = 0}^{r - 1}e^{-i \omega k n } x_{n+2r} + 
\dots \right)=
\nonumber \\
= \frac{1}{\sqrt{M}} \left(
\sum_{n = 0}^{r - 1}e^{-i \omega k n} x_n + 
\sum_{n = 0}^{r - 1}e^{-i \omega k n  } x_n + 
\sum_{n = 0}^{r - 1}e^{-i \omega k n  } x_n + 
\dots \right) = 
\nonumber \\ 
= \frac{1}{\sqrt{M}} \frac{M}{r} \sum_{n = 0}^{r - 1}e^{-i \omega k n  } x_n \ne 0.
\label{eqAddFourierDiscretFourierPeriodCase2}
\end{eqnarray}
Thus, from expressions 
\eqref{eqAddFourierDiscretFourierPeriodCase1} and 
\eqref{eqAddFourierDiscretFourierPeriodCase2} it follows that
the nonzero terms appear with a period $T = \frac{M}{r}$.
\end{proof}
\end{lemma}

\begin{remark}
\emph{(Periodicity of maxima)}
\label{rem:dsp:fourier:periodprop}
It is worth noting that expression 
\eqref{eqAddFourierDiscretFourierPeriodCase1} (in the case when
the period is not a divisor of the number of samples: $M \mod r \ne 0$) will be approximately
equal to 0 for those values of $k$ that are far from multiples of $\frac{M}{r}$, i.e. local maxima of the Fourier transform
will repeat with period $\frac{M}{r}$.
\end{remark}