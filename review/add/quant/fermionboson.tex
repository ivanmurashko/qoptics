%% -*- coding:utf-8 -*- 
\section{Bosons and Fermions}
\label{AddFermionBoson}

Consider an ensemble of identical particles. The state of a system consisting
of $n$ identical particles can be written in the following form
\begin{equation}
  \left|\psi\right> = \left|x_1, x_2, \dots, x_k, \dots, x_j, \dots,
  x_n\right>.
  \label{eqAddFBAnsamble}
\end{equation}
For each of the particles in expression \eqref{eqAddFBAnsamble} we
are interested in some physical characteristic (for example, coordinate)
which is denoted by $x$.

Obviously, from the classical point of view, there should be no difference if we swap two
particles. At the same time, the wave function, which is defined up to
a phase, may change. Thus, denoting by
$\hat{S}_{k,j}$ the operator that swaps two particles with
indices $k$ and $j$, we get 
\begin{eqnarray}
  \hat{S}_{k, j} \left|\psi\right> = \left|x_1, x_2, \dots, x_j,
  \dots, x_k, \dots,  x_n\right> =
  \nonumber \\
  = e^{i \phi} \left|x_1, x_2, \dots, x_k, \dots, x_j, \dots,
  x_n\right> = e^{i \phi} \left|\psi\right>
  \nonumber
\end{eqnarray}

If the operator $\hat{S}_{k,j}$ is applied twice, it is obvious that
the system will be brought completely back to the original state, i.e.,
\begin{eqnarray}
  \hat{S}_{k, j}^2 \left|\psi\right> =
  e^{2 i \phi} \left|\psi\right> = \left|\psi\right>
  \nonumber
\end{eqnarray}
from which it follows that $\phi$ can take two values $0$ and $\pi$.
Thus, we can divide all particles into two classes. For
the first of them the wave function does not change when swapping two
particles ($\psi = 0$), and for the second it changes sign ($\psi = \pi$).

\begin{example}
  Consider a system consisting of two particles. Suppose the first particle
  can be in one of two states $\left|\psi^{(1)}_1\right>$
  or $\left|\psi^{(1)}_2\right>$. Similarly, the second particle can
  be in one of two states $\left|\psi^{(2)}_1\right>$
  or $\left|\psi^{(2)}_2\right>$.

  Assume the total wave function $\left|\psi^{(12)}\right>$ of the two
  particles can be represented as a superposition of the single-particle wave functions, i.e. 
  \begin{equation}
    \left|\psi^{(12)}\right> = \left|\psi^{(1)}_{1,2}\right> \pm
    \left|\psi^{(2)}_{1,2}\right>
    \nonumber
  \end{equation}

  The case of a symmetric (under particle exchange) wave function will
  have the form
  \begin{equation}
    \left|\psi^{(12)}\right> = \left|\psi^{(1)}_{1,2}\right> +
    \left|\psi^{(2)}_{1,2}\right>
    \nonumber
  \end{equation}
  and both particles can be in the same state,
  i.e., have the same wave function
  \begin{equation}
  \left|\psi\right> = \left|\psi^{(1)}_{1}\right> =
  \left|\psi^{(2)}_{1}\right>.
  \label{eqAddFermionBosonSameState}
  \end{equation}
  In this case
  \[
  \left|\psi^{(12)}\right> = 2 \left|\psi\right>, 
  \]
  i.e., the particles can simultaneously be in the same
  state.

  For the antisymmetric wave function case we have
  \begin{equation}
    \left|\psi^{(12)}\right> = \left|\psi^{(1)}_{1,2}\right> -
    \left|\psi^{(2)}_{1,2}\right>
    \nonumber
  \end{equation}
  and if both particles are in the same state
  \eqref{eqAddFermionBosonSameState}, then for $\psi^{(12)}$ we get
  \[
  \left|\psi^{(12)}\right> = \left|\psi\right> - \left|\psi\right> = 0,
  \]
  i.e., in this case particles cannot be in the same
  state simultaneously.
  \label{exAddFermionBoson}
\end{example}
Bearing in mind Example \ref{exAddFermionBoson}, we can consider the general
case of an ensemble of identical particles, each of which is in
the same state or equivalently has the same wave function. In this case, under particle exchange, the actual wave
function of the ensemble does not change:
\[
\left|\psi^{(12)}\right> = \left|\psi^{(21)}\right>.
\]
That is, in the antisymmetric case
\[
\left|\psi^{(12)}\right> = - \left|\psi^{(21)}\right> =
-\left|\psi^{(12)}\right> = 0.
\] 

Thus, we can say that particles
whose ensemble states are symmetric can be simultaneously in
the same state. Such particles are called bosons.\rindex{boson}

For the antisymmetric case, we 
assume that such particles cannot simultaneously be in
the same state. Such particles are called fermions.
\rindex{fermion}

Consider a state with a definite number of particles $n$
$\left|\psi\right> = \ket{n}$. In the case of bosons \rindex{boson}
we will have 
\begin{equation}
\left|\psi_b\right> = \ket{n}_b.
\label{eqAddQuantBoson}
\end{equation}
The creation operators $\hat{a}_b^{\dag}$ and annihilation operators $\hat{a}_b$ act on
the state \eqref{eqAddQuantBoson} as follows:
\begin{eqnarray}
\hat{a}_b^{\dag}\ket{n}_b = \sqrt{n+1}\ket{n+1}_b, 
\nonumber \\
\hat{a}_b\ket{n}_b = \sqrt{n}\ket{n-1}_b
\nonumber
\end{eqnarray}

For fermions \rindex{fermion} there are two possible states: $\ket{0}_f$ and
$\ket{1}_f$, with the corresponding creation
operators $\hat{a}_f^{\dag}$ and annihilation operators $\hat{a}_f$ acting as follows:
\begin{eqnarray}
\hat{a}_f^{\dag}\ket{0}_f = \ket{1}_f, 
\nonumber \\
\hat{a}_f^{\dag}\ket{1}_f = 0, 
\nonumber \\
\hat{a}_f\ket{0}_f = 0, 
\nonumber \\
\hat{a}_f\ket{1}_f = \ket{0}_f.
\nonumber
\end{eqnarray}