%% -*- coding:utf-8 -*- 
\section{Measurability of Physical Quantities}
\subsection{Heisenberg Uncertainty Principle}
\label{AddHeisenbergUncertaintyPrinciple}
Suppose two operators $\hat{A}$ and $\hat{B}$ do not commute with each other, i.e. 
\begin{equation}
\left[
\hat{A}\hat{B}
\right] = 
\hat{A}\hat{B} - \hat{B}\hat{A} = i \hat{C} \ne 0,
\nonumber
\end{equation}
where $\hat{C}$ is some Hermitian operator.

Suppose the system is in the state $\left|\psi\right>$. Then the average values of the operators are expressed by the following relations:
\begin{eqnarray}
\left<\psi\right|\hat{A}\left|\psi\right> = \left<\hat{A}\right>,
\nonumber \\
\left<\psi\right|\hat{B}\left|\psi\right> = \left<\hat{B}\right>.
\nonumber
\end{eqnarray}

Define the uncertainty of measuring the quantities $A$ and $B$ as their variances as follows:
\begin{eqnarray}
\Delta A = \sqrt{\left<\psi\right|
\hat{\mathcal{A}}^2\left|\psi\right>}, 
\nonumber \\
\Delta B = \sqrt{\left<\psi\right|
\hat{\mathcal{B}}^2\left|\psi\right>}, 
\nonumber
\end{eqnarray}
where
\begin{eqnarray}
\hat{\mathcal{A}} = \hat{A}-\left<\hat{A}\right>, 
\nonumber \\
\hat{\mathcal{B}} = \hat{B}-\left<\hat{B}\right>.
\nonumber
\end{eqnarray}

Introduce the operator $\hat{D}$ as follows:
\begin{equation}
\hat{D} = \hat{\mathcal{A}} + i \lambda \hat{\mathcal{B}}.
\nonumber
\end{equation}
Consider the operator $\hat{D}^{\dag}\hat{D}$, which is Hermitian. Its expectation in the state $\left|\psi\right>$ is:
\begin{eqnarray}
\left<\psi\right|\hat{D}^{\dag}\hat{D}\left|\psi\right> = 
\left<\phi\right|\left.\phi\right> \ge 0,
\nonumber
\end{eqnarray}
where
$\left|\phi\right> = \hat{D}\left|\psi\right>$. On the other hand,
\begin{eqnarray}
\left<\psi\right|\hat{D}^{\dag}\hat{D}\left|\psi\right> = 
\left<\psi\right|\left(\hat{\mathcal{A}} - i \lambda \hat{\mathcal{B}}\right)
\left(\hat{\mathcal{A}} + i \lambda \hat{\mathcal{B}}\right)\left|\psi\right> =
\nonumber \\
=
\left<\psi\right|\hat{\mathcal{A}}^2\left|\psi\right> +
\lambda^2\left<\psi\right|\hat{\mathcal{B}}^2\left|\psi\right> +
i \lambda 
\left<\psi\right|
\left[ \mathcal{\hat{A}}, \mathcal{\hat{B}}\right]
\left|\psi\right>
 = 
\nonumber \\
=
\left(\Delta A\right)^2 + \lambda^2 \left(\Delta B\right)^2 +
i \lambda 
\left<\psi\right|
\left[ \hat{A}, \hat{B}\right]
\left|\psi\right> = 
\nonumber \\
=
\lambda^2 \left(\Delta B\right)^2 - 
\lambda \left<C\right> + \left(\Delta A\right)^2 \ge 0.
\nonumber
\end{eqnarray}
Consider the polynomial 
\[
f\left(\lambda\right) = \lambda^2 \left(\Delta B\right)^2 - 
\lambda \left<C\right> + \left(\Delta A\right)^2.
\]
We have $f\left( \pm \infty \right) > 0$, so 
$f\left(\lambda\right) \ge 0$ if this polynomial has at most one real root, i.e.
\[
\left<C\right>^2 - 4 \left(\Delta A\right)^2 \left(\Delta B\right)^2
\le 0
\]
or
\begin{equation}
  \Delta A \Delta B \ge \frac{\left|\left< C \right>\right|}{2},
  \label{eqAddHeisenbergUncertaintyPrinciple}
\end{equation}
which is the Heisenberg inequality. 

\subsection{Energy-Time Uncertainty Relation}
\label{AddHeisenbergUncertaintyPrincipleEnergyTime}
Time in quantum mechanics does not have a corresponding operator, and therefore to estimate the time one should use some separate observable $\hat{O}$. Using
\eqref{eqAddHeisenbergUncertaintyPrinciple}, the following relation can be obtained:
\begin{eqnarray}
  \Delta E \Delta O \ge \frac{\left|\left< C \right>\right|}{2},
  \nonumber
\end{eqnarray}
where, taking into account \eqref{eqAddWaveFunc_HeizenbergT},
\[
\frac{d \hat{O}}{d t} = \frac{i}{\hbar}
\left[\hat{\mathcal{H}}, \hat{O}\right]
\]
and therefore
\[
C = \frac{1}{i}\left[\hat{\mathcal{H}}, \hat{O}\right] =
- \hbar \frac{d \hat{O}}{d t}.
\]
Thus,
\begin{eqnarray}
  \Delta E \Delta O \ge \frac{\left|\left< C \right>\right|}{2} =
  \frac{\hbar}{2}\left|\left<\frac{d \hat{O}}{d t}\right>\right|.
  \nonumber
\end{eqnarray}
Denoting $\Delta t = \frac{\Delta O}{\left|\left<\frac{d \hat{O}}{d
    t}\right>\right|}$
\footnote{
  We can assume that at small times the observable $\hat{O}$
  changes linearly, i.e. $\left<\frac{d \hat{O}}{dt}\right> =
  \frac{\Delta O}{\Delta t}$. Thus, $\Delta t$ is the time of significant
  change of the observable $O$. 
}
finally we get
\begin{eqnarray}
  \Delta E \Delta t \ge \frac{\hbar}{2}.
  \label{eqAddHeisenbergUncertaintyPrincipleET}
\end{eqnarray}

It is worth noting that states with definite energy do not contradict
relation \eqref{eqAddHeisenbergUncertaintyPrincipleET} because
although $\Delta E = 0$, for any observable we have
$\left<\frac{d \hat{O}}{dt}\right> = 0$ and consequently, $\Delta t =
\infty$. 

\subsection{Simultaneous Measurability of Physical Quantities}
\label{AddHeisenbergUncertaintyPrincipleMesuranmet}
The question of simultaneous measurability of two physical quantities is very meaningful, especially when studying quantum mechanical paradoxes such as the EPR paradox (see \autoref{sec:part3:epr}).
\rindex{EPR paradox}

Suppose we have two observables $\hat{A}, \hat{B}$, each corresponding to a set of eigenfunctions $\ket{a_i},
\ket{b_i}$ and eigenvalues $\{a_i\}, \{b_i\}$, i.e.
\begin{eqnarray}
  \hat{A}\ket{a_i} = a_i \ket{a_i},
  \nonumber \\
  \hat{B}\ket{b_i} = b_i \ket{b_i}.
  \nonumber
\end{eqnarray}
If at some moment we can measure the values of these quantities simultaneously, then the obtained values can be described by numbers $a$ and $b$, where obviously $a \in \{a_i\}, b \in \{b_i\}$ and
\begin{eqnarray}
  \hat{A}\ket{a} = a \ket{a},
  \nonumber \\
  \hat{B}\ket{b} = b \ket{b}.
  \nonumber
\end{eqnarray}
Since the measurement of the quantities was performed simultaneously, due to the collapse of the wavefunction (see \autoref{sec:add:reduction}), we have
\[
\ket{a} = \ket{b},
\]
that is, the operators $\hat{A}$ and $\hat{B}$ have common eigenfunctions. It is not necessary that all eigenvectors of these operators coincide, it is sufficient that a single vector matches.

Note that if the operators commute, then their sets of eigenvectors coincide, which is consistent with the Heisenberg inequality, which in this case takes the form
\[
\Delta a \Delta b \ge 0.
\]

Indeed, if
\[
\hat{A} \ket{a} = a \ket{a}, 
\]
then, considering $\hat{A}\hat{B} = \hat{B}\hat{A}$,
\[
\hat{A}\hat{B}\ket{a} = 
\hat{B}\hat{A}\ket{a} =
a \hat{B}\ket{a}.
\]
Denoting $\ket{b} = \hat{B}\ket{a}$, we have
\[
\hat{A}\ket{b} = 
a \ket{b},
\]
i.e.,
$\ket{b} = \ket{a}$ and this vector is an eigenvector 
of both operators $\hat{A}$ and  $\hat{B}$. Thus, if the operators
commute, they have a common basis (set of eigenvectors). The converse is also true, see \cite{bHolevo2016}.