%% -*- coding:utf-8 -*-
\section{Fermi's Golden Rule}
Fermi's golden rule allows one to use time-dependent perturbation theory
to calculate the transition probability between two states of a quantum
system.

\begin{theorem}[Fermi's Golden Rule]
  \label{addQuantGoldenRuleFermi}
  Suppose that a quantum system is initially in the state
  $\ket{i}$ and we want to estimate the probability of transition to the state
  $\ket{f}$. The Hamiltonian of the considered system consists of
  \rindex{Hamiltonian}
  two parts: a stationary Hamiltonian $\hat{H_0}$ and a weak
  perturbation $\hat{V}$:
  \begin{equation}
    \hat{H} = \hat{H_0} + \hat{V}
    \nonumber
  \end{equation}
  In this case, the desired probability is given by
  \begin{equation}
    W_{i \rightarrow f} = \frac{2 \pi}{\hbar}
    \left|
    \left<
    f
    \right|
    \hat{V}
    \left|
    i
    \right>
    \right|^2 \rho
    \nonumber,
  \end{equation}
  where $\rho$ is the density of final states.
\end{theorem}

\begin{proof}
  Suppose there exists a set of states
  $\left\{\ket{n}\right\}$ forming a complete set of eigenvectors
  of the Hamiltonian $\hat{H_0}$, i.e.,
  \begin{equation}
    \hat{H_0} \ket{ n } = E_n \ket{ n },
    \nonumber
  \end{equation}
  while the considered initial state $\ket{i}$ and
  final state $\ket{f}$ belong to this vector system, i.e.,
  \begin{equation}
    \ket{i}, \ket{f} \in \left\{\ket{n}\right\}.
    \nonumber
  \end{equation}
  At some moment in time, the system is in the state
  $\left|\psi\right>$, which satisfies the Schrödinger equation:
  \begin{equation}
    i \hbar \frac{\partial \left|\psi\right>}{\partial t} =
    \hat{H} \left|\psi\right>.
    \label{eqAddQuantGoldenRuleFermiShred}
  \end{equation}
  Our task is to calculate the probability that upon observation
  the system will be found in the state $\ket{f}$, given that
  the initial condition for \eqref{eqAddQuantGoldenRuleFermiShred}
  is
  \begin{equation}
    \left.\left|\psi\right>\right|_{t=0} = \ket{i}.
    \label{eqAddQuantGoldenRuleFermiInitialCond}
  \end{equation}

  Due to the completeness of the system $\left\{\ket{n}\right\}$,
  the state $\left|\psi\right>$ can be represented in the form
  \begin{equation}
    \left|\psi\right> = \sum_n a_n\left(t\right) \ket{n}
    e^{\frac{-i E_n t}{\hbar}}
    \label{eqAddQuantGoldenRuleFermiExp}
  \end{equation}
  Substituting \eqref{eqAddQuantGoldenRuleFermiExp} into the Schrödinger equation
  \eqref{eqAddQuantGoldenRuleFermiShred} yields
  \begin{eqnarray}
    i \hbar \sum_n \frac{ \partial a_n\left(t\right)}{\partial t }
    \ket{n} e^{\frac{-i E_n t}{\hbar}} +
    \sum_n a_n\left(t\right) \ket{n}
    i \hbar \frac{\partial e^{\frac{-i E_n t}{\hbar}}}{ \partial t} =
    \\ \nonumber
    = i \hbar \sum_n  e^{\frac{-i E_n t}{\hbar}} \ket{n} \left(
    \frac{ \partial a_n\left(t\right)}{\partial t } - \frac{i E_n}{\hbar} 
    \right) =
    \\ \nonumber
    =  i \hbar \sum_n  e^{\frac{-i E_n t}{\hbar}}
    \frac{ \partial a_n\left(t\right)}{\partial t } \ket{n}+
    \sum_n E_n a_n\left(t\right) \ket{n}
    e^{\frac{-i E_n t}{\hbar}} =
    \nonumber \\
    =
     i \hbar \sum_n  e^{\frac{-i E_n t}{\hbar}}
     \frac{ \partial a_n\left(t\right)}{\partial t } \ket{n}+
      \sum_n E_n a_n\left(t\right) \ket{n}
    e^{\frac{-i E_n t}{\hbar}}.
    \label{eqAddQuantGoldenRuleFermiExp2}
  \end{eqnarray}
  On the other hand, expression \eqref{eqAddQuantGoldenRuleFermiExp2}
  must equal
  \begin{equation}
    \hat{H} \left|\psi\right> =
    \sum_n E_n a_n\left(t\right) \ket{n}
    e^{\frac{-i E_n t}{\hbar}} +
    \sum_n  a_n\left(t\right) 
    e^{\frac{-i E_n t}{\hbar}} \hat{V} \ket{n},
    \nonumber
  \end{equation}
  i.e.,
  \begin{eqnarray}
    i \hbar \sum_n  e^{\frac{-i E_n t}{\hbar}}
    \frac{ \partial a_n\left(t\right)}{\partial t } \ket{n} =
     \sum_n  a_n\left(t\right) 
    e^{\frac{-i E_n t}{\hbar}} \hat{V} \ket{n}.
    \label{eqAddQuantGoldenRuleFermiExp3}
  \end{eqnarray}
  We will solve expression \eqref{eqAddQuantGoldenRuleFermiExp3}
  by perturbation theory, i.e.,
  \begin{equation}
    a_n\left(t\right) = a_n^{(0)}\left(t\right) +
    a_n^{(1)}\left(t\right) + \dots,    
    \nonumber
  \end{equation}
  and suppose that
  \[
  a_n^{(0)}\left(t\right) = \left.const\right|_t = a_n^{(0)}\left(0\right),
  \]
  from which, given the initial condition
  \eqref{eqAddQuantGoldenRuleFermiInitialCond},
  \[
  a_n^{(0)}\left(t\right) = \delta_{ni}.
  \]
  Thus, for the first order of perturbation theory from
  \eqref{eqAddQuantGoldenRuleFermiExp3} we get
  \begin{eqnarray}
    i \hbar \sum_n  e^{\frac{-i E_n t}{\hbar}}
    \frac{ \partial a_n^{(1)}\left(t\right)}{\partial t } \ket{n} =
     \sum_n  \delta_{ni} 
     e^{\frac{-i E_n t}{\hbar}} \hat{V} \ket{n} =
     e^{\frac{-i E_i t}{\hbar}} \hat{V} \ket{i}.
    \label{eqAddQuantGoldenRuleFermiExp4}
  \end{eqnarray}
  Therefore, from \eqref{eqAddQuantGoldenRuleFermiExp4} we have
  \begin{equation}
    \frac{ \partial a_n^{(1)}\left(t\right)}{\partial t }  =
    - \frac{i}{\hbar} e^{\frac{-i \left(E_i - E_n\right) t}{\hbar}}
    \bra{n}\hat{V}\ket{i},
    \nonumber
  \end{equation}
  denoting $E_i - E_n = \hbar \omega_{in}$, where $\omega_{in}$
  is the transition frequency between the states $\ket{i}$ and
  $\ket{n}$, after integrating we get
  \begin{equation}
  a_n^{(1)}\left(t\right)  =
    - \frac{i}{\hbar} \int_0^t e^{-i \omega_{in} t'}
    \bra{n}\hat{V}\ket{i} dt',
    \nonumber
  \end{equation}
  i.e.,
  \begin{eqnarray}
  a_n^{(1)}\left(t\right)  =
  \frac{1}{\hbar \omega_{in}} \bra{n}\hat{V}\ket{i}
  \left(e^{-i \omega_{in} t} -  1\right) =
  \\ \nonumber =
  \frac{1}{\hbar \omega_{in}} \bra{n}\hat{V}\ket{i}
  e^{\frac{-i \omega_{in} t}{2}}
  \left(e^{\frac{-i \omega_{in} t}{2}} -  e^{\frac{i \omega_{in}
      t}{2}}\right) =
  \\ \nonumber
  =
  - \frac{2 i}{\hbar \omega_{in}} \bra{n}\hat{V}\ket{i}
  e^{\frac{-i \omega_{in} t}{2}}
  \sin\left(\frac{ \omega_{in} t}{2}\right)
  \nonumber
  \end{eqnarray}
  
  
  We are interested in the rate $W_{i \rightarrow f}$ of transition to
  the state $\ket{f}$, which is given by the expression
  \[
  W_{i \rightarrow f} = \frac{\left|a_f^{(1)}\left(t\right)\right|^2}{t}.
  \]
  Or
  \begin{eqnarray}
    W_{i \rightarrow f} =
    \frac{t}{\hbar^2}
    \left|\bra{f}\hat{V}\ket{i}\right|^2
    \frac{\sin^2\left(\frac{ \omega_{if} t}{2}\right)}
         {\left(\frac{ \omega_{if} t}{2}\right)^2}.
    \label{eqAddQuantGoldenRuleFermiExp5}
  \end{eqnarray}

  In the case where the final state $\ket{f}$ represents
  some continuous spectrum near the energy $E_f$, the expression
  \eqref{eqAddQuantGoldenRuleFermiExp5} must be summed over all final states
  for which $\omega_{nf} \approx 0$. We introduce the concept of the density of states
  $\rho(E) = \frac{dE}{dn}$, so that the number of states $dn$ corresponding
  to an infinitesimal energy interval $dE$ can be represented as
  $dn = \rho dE$. Therefore, the transition rate can be rearranged as:
  \begin{equation}
    W_{i \rightarrow f} = \sum_{n: E_n \approx E_f}
    \rho\left(E_n\right) W_{i \rightarrow f}\left(E_n\right), 
    \nonumber
  \end{equation}
  i.e., assuming for the considered energy interval
  $\rho\left(E_n\right) \approx \rho\left(E_f\right)$,
  \[
  \left|\bra{n}\hat{V}\ket{i}\right|^2 \approx
  \left|\bra{f}\hat{V}\ket{i}\right|^2
  \]
  and replacing summation by integration, we get
  \begin{eqnarray}
    W_{i \rightarrow f} \approx
    \int_{E_n \approx E_f} \rho\left(E_n\right) d E_n      \frac{t}{\hbar^2}
    \left|\bra{n}\hat{V}\ket{i}\right|^2
    \frac{\sin^2\left(\frac{ \omega_{in} t}{2}\right)}
         {\left(\frac{ \omega_{in} t}{2}\right)^2} =
         \\ \nonumber
         = \int_{\omega_{in} \approx \omega_{if}} \rho\left(E_n\right)
         d E_n      \frac{t}{\hbar^2}
    \left|\bra{n}\hat{V}\ket{i}\right|^2
    \frac{\sin^2\left(\frac{ \omega_{in} t}{2}\right)}
         {\left(\frac{ \omega_{in} t}{2}\right)^2} \approx
         \\ \nonumber
    \approx
    \frac{2}{\hbar}
    \left|\bra{f}\hat{V}\ket{i}\right|^2 \rho
    \int_{-\infty}^{\infty}
     d \left(\frac{ \omega t}{2}\right)
    \frac{\sin^2\left(\frac{ \omega t}{2}\right)}
         {\left(\frac{ \omega t}{2}\right)^2} =
    \frac{2 \pi}{\hbar} 
    \left|\bra{f}\hat{V}\ket{i}\right|^2 \rho,
    \label{eqAddQuantGoldenRuleFermiExp6}
  \end{eqnarray}
  which coincides with the expression to be proved.
  
  
\end{proof}