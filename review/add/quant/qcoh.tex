%% -*- coding:utf-8 -*- 
\section{Coordinate representation of a coherent state}
\label{AddQCoh}
Let's start with the vacuum state, which satisfies the equation
\begin{equation}
\hat{a}\ket{0} = 0.
\label{eqAddQCoh1}
\end{equation}
In the coordinate representation $\hat{q} = q$, $\hat{p} = -i
\frac{\partial}{\partial q}$, then
\begin{eqnarray}
\hat{a} = \frac{1}{\sqrt{2 \hbar \omega}}
\left(
\omega q + \hbar \frac{\partial}{\partial q}
\right),
\nonumber \\
\hat{a}^{\dag} = \frac{1}{\sqrt{2 \hbar \omega}}
\left(
\omega q - \hbar \frac{\partial}{\partial q}
\right).
\label{eqAddQCoh2}
\end{eqnarray}

Substituting \eqref{eqAddQCoh2} into \eqref{eqAddQCoh1}, we obtain the equation
for the wave function of the vacuum state of the mode in the coherent
representation:
\begin{equation}
\left(
\omega q + \hbar \frac{\partial}{\partial q}
\right) \Phi_0\left(q\right) = 0,
\label{eqAddQCoh3}
\end{equation}
where $\Phi_0\left(q\right) = \bra{q}\ket{0}$.
The normalized solution of this equation is
\begin{equation}
\Phi_0\left(q\right) = \left(\frac{\omega}{\pi
  \hbar}\right)^{\frac{1}{4}} e^{-\frac{\omega q^2}{2 \hbar}}.
\nonumber
\end{equation}
The uncertainty of this solution $\Delta q \Delta p =
\frac{\hbar}{2}$ is minimal. The probability density to detect
a certain coordinate $q$ (in our case a certain electric field strength) is:
\begin{equation}
\Phi_0\left(q\right)\Phi_0^{*}\left(q\right) = \left(\frac{\omega}{\pi
  \hbar}\right)^{\frac{1}{2}} e^{-\frac{\omega q^2}{\hbar}}.
\label{eqAddQCoh4}
\end{equation}
For the coherent state, the following equation holds
\[
\hat{a}\left|\alpha\right> = \alpha \left|\alpha\right>,
\]
which in the coordinate representation has the form
\begin{equation}
\left(
\omega q + \hbar \frac{\partial}{\partial q}
\right) \Phi_{\alpha}\left(q\right) = 
\alpha \Phi_{\alpha}\left(q\right),
\label{eqAddQCoh5}
\end{equation}
where 
\[
\Phi_{\alpha}\left(q\right) = \bra{q}\left.\alpha\right>.
\]

Comparing \eqref{eqAddQCoh3} and \eqref{eqAddQCoh5}, it is easy to see that
the solution of equation \eqref{eqAddQCoh5} differs from \eqref{eqAddQCoh4}
by a shift of 
$\alpha\sqrt{\frac{2\hbar}{\omega}}$. Indeed, equation
\eqref{eqAddQCoh5} can be rewritten as
\begin{equation}
\left(
\omega \left(q - \frac{\alpha}{\omega}\right) + \hbar \frac{\partial}{\partial q}
\right) \Phi_{\alpha}\left(q\right) = 0.
\nonumber
\end{equation}
The solution of this equation is:
\begin{equation}
\Phi_{\alpha}\left(q\right) = \left(\frac{\omega}{\pi
  \hbar}\right)^{\frac{1}{4}} e^{-\frac{\omega}{2
    \hbar}\left(q-\sqrt{\frac{2\hbar}{\omega}}\alpha\right)^2}. 
\label{eqAddQCoh7}
\end{equation}

The coherent state is not a stationary state. It must satisfy the non-stationary Schrödinger equation. The parameter $\alpha = \alpha\left(t\right)$ depends on time. Substituting \eqref{eqAddQCoh7} into the Schrödinger equation, this dependence can be determined, but even without that it is clear that $\alpha\left(t\right)$ must oscillate harmonically at the mode frequency. We assume 
\[
\alpha\left(t\right) = \left|\alpha\right| e^{-i \omega t}.
\]
Then the probability density takes the form
\begin{equation}
\Phi_{\alpha}\left(q\right)\Phi_{\alpha}^{*}\left(q\right) = N e^{-\frac{\omega}{2
    \hbar}
\left[\left(q-\sqrt{\frac{2\hbar}{\omega}}\alpha\right)^2
+
\left(q-\sqrt{\frac{2\hbar}{\omega}}\alpha^{*}\right)^2
\right]
}.
\nonumber
\end{equation}
Let's transform the exponent's argument
\begin{eqnarray}
\left[\left(q-\sqrt{\frac{2\hbar}{\omega}}\alpha\right)^2
+
\left(q-\sqrt{\frac{2\hbar}{\omega}}\alpha^{*}\right)^2
\right] = 
\nonumber \\
=
q^2 + \frac{2\hbar}{\omega}\alpha^2 - 2 q \alpha
  \sqrt{\frac{2\hbar}{\omega}} +
q^2 + \frac{2\hbar}{\omega}\alpha^{2 *} - 2 q \alpha^{*}
  \sqrt{\frac{2\hbar}{\omega}} =
\nonumber \\
= 2
\left[
q^2 + \frac{2\hbar}{\omega}\frac{\alpha^2 + \alpha^{*2}}{2} -
2 q \sqrt{\frac{2\hbar}{\omega}} \frac{\alpha + \alpha^{*}}{2} +
\frac{2 \hbar}{\omega}\left(\alpha\alpha^{*}\right) -
\frac{2 \hbar}{\omega}\left(\alpha\alpha^{*}\right)
\right] = 
\nonumber \\
= 2
\left[
q^2 + \frac{4\hbar}{\omega}\left(\frac{\alpha + \alpha^{*}}{2}\right)^2 -
2 q \sqrt{\frac{2\hbar}{\omega}} \frac{\alpha + \alpha^{*}}{2}  -
\frac{2 \hbar}{\omega}\left(\alpha\alpha^{*}\right)
\right] = 
\nonumber \\
= 2
\left[
q - \sqrt{\frac{2\hbar}{\omega}} \frac{\alpha + \alpha^{*}}{2}
\right]^2 + const = 
\nonumber \\
= 
2
\left[
q - \sqrt{\frac{2\hbar}{\omega}} \left|\alpha\right|\cos\,\omega t
\right]^2 + const.
\nonumber
\end{eqnarray}
The constant will go into the normalization factor, so we obtain:
\begin{equation}
\Phi_{\alpha}\left(q,t\right)\Phi_{\alpha}^{*}\left(q,t\right) = N
e^{-\frac{\omega}{\hbar}
\left[
q - \sqrt{\frac{2\hbar}{\omega}} \left|\alpha\right|\cos\,\omega t
\right]^2
}.
\label{eqAddQCoh9}
\end{equation}
From expression \eqref{eqAddQCoh9} it is seen that the probability density for
$q$ (in our case for the electric field strength) at a fixed time is the same as for the vacuum state, but shifted by 
$\sqrt{\frac{2\hbar}{\omega}}\alpha\left(t\right)$. Over time, the center
of the distribution moves harmonically, as depicted in \autoref{figQCoh_1} 

\input ./add/figqhoc.tex
% Maxima
%% load(draw);
%% f(q, t):=exp(-1*(q-cos(t))^2);
%% draw3d(explicit(f(q,t), q,-5,5, t,-10,10));

Since the width of the distribution does not depend on the amplitude, the relative
uncertainty decreases with increasing amplitude. Therefore, the state will tend to become classical.