%% -*- coding:utf-8 -*- 
\section{Equation for the field density matrix in the number representation}
\label{ch2_7}
In the number representation (photon number representation), the matrix elements are defined by the equality
\[
\rho_{nm} = \bra{n}\hat{\rho}\ket{m}.
\]
To obtain the required equation, multiply
\eqref{eqCh2_rho_final2} from the left by $\bra{n}$ and from the right by  
$\ket{m}$. Using
the properties of the operators $\hat{a}$ and $\hat{a}^{\dag}$, we obtain a system
of equations relating the matrix elements 
\begin{eqnarray}
\dot{\rho}_{nm} = - \frac{1}{2}
\left(
R_a\left(n + 1 + m + 1\right) + 
R_b\left(n + m\right)
\right)\rho_{nm} +
\nonumber \\
+ 
R_a\sqrt{nm}\rho_{n - 1, m - 1} +
R_b\sqrt{\left(n + 1\right)\left(m + 1\right)}\rho_{n + 1, m + 1}.
\label{eqCh2_task5}
\end{eqnarray}
For diagonal elements $n = m$, hence
\begin{eqnarray}
\dot{\rho}_{nn} = - 
\left(
R_a\left(n + 1\right) + 
R_b n
\right)\rho_{nn} +
\nonumber \\
+ 
R_a n \rho_{n - 1, n - 1} +
R_b\left(n + 1\right)\rho_{n + 1, n + 1}.
\label{eqCh2_51}
\end{eqnarray}

\input ./part1/interaction/fig8.tex

This equation can be considered as a balance of probability flows
of photon numbers. This is schematically shown in \autoref{figPart1Ch2_8}. At
thermal equilibrium, the flows must be equal. Using
the principle of detailed balance, we obtain equalities: 
\begin{eqnarray}
R_a n \rho_{n - 1, n - 1} = R_b n \rho_{nn},
\nonumber \\
R_a \left(n + 1\right) \rho_{n n} = R_b 
\left(n + 1\right) \rho_{n + 1, n + 1}
\label{eqCh2_52}
\end{eqnarray}
From \eqref{eqCh2_52} we have
\begin{equation}
\rho_{n + 1, n + 1} = \frac{R_a}{R_b}\rho_{nn} = 
e^{-\frac{\hbar \omega}{k_B T}}\rho_{nn},
\label{eqCh2_53}
\end{equation}
since  
\(
\frac{R_a}{R_b} = 
\exp \left(-\frac{\hbar \omega}{k_B T}\right)
\)
(the reservoir is in equilibrium at temperature
$T$). Applying \eqref{eqCh2_53} successively starting at $n=0$, 
we get:  
\begin{equation}
\rho_{nn} = \rho_{00} 
e^{-\frac{n \hbar \omega}{k_B T}} =
\left(1 - e^{-\frac{\hbar \omega}{k_B T}}\right) 
e^{-\frac{n \hbar \omega}{k_B T}},
\end{equation}
where $\rho_{00}$ is found from the normalization condition $\sum_{(n)}\rho_{nn} = 1$.

The average photon number in the mode, as expected, is given by the Planck formula 
\begin{equation}
\bar{n} = \left<n\right> = 
\sum_{(n)} n \rho_{nn} = 
\frac{1}{ e^{\frac{\hbar \omega}{k_B T}} - 1}.
\end{equation}

It follows that over time the radiation temperature 
approaches the temperature of the atomic beam (reservoir).  
The time evolution of the average photon number can be obtained from equation 
\eqref{eqCh2_51}
\begin{eqnarray}
\frac{d}{d t}\left<n\left(t\right)\right> = \sum_{(n)}n \dot{\rho} = 
\nonumber \\
= \sum_{(n)}\left(-R_a\left(n^2 + n\right)\rho_{nn} - R_b n^2
\rho_{nn} + \right.
\nonumber \\
+ \left.
R_b \left(n^2 + n\right) \rho_{n +1, n+ 1} +
R_a n^2 \rho_{n - 1, n - 1}
\right).
\end{eqnarray}
Change the summation variables by $m= n + 1$ in the third sum and $m = n - 1$ in the fourth sum. We get:
\begin{eqnarray}
\frac{d}{d t}\left<n\right> = 
-R_a \sum_{(n)}\left(n^2 + n\right)\rho_{nn}
 - R_b \sum_{(n)} n^2 \rho_{nn} +
\nonumber \\
+ R_b\sum_{(m)}\left(m^2 - m\right)\rho_{m,m} 
+ R_a\sum_{(m)}\left(m^2 +2 m + 1\right)\rho_{m,m} = 
\nonumber \\
= R_a \sum_{(m)}\left( m + 1\right)\rho_{m,m} - R_b\sum_{(m)} m
\rho_{m,m} =
\nonumber \\
= \left(R_a - R_b\right) \left<n\right> + R_a.
\label{eqCh2_57}
\end{eqnarray}
At equilibrium
\[
\frac{d}{d t}\left<n\right> = 0,
\]  
and we obtain the previous relation
\begin{equation}
\left<n_{(\infty)}\right> = \bar{n} = 
\frac{R_a}{R_b - R_a} = \frac{1}{\frac{R_b}{R_a} - 1} = 
\frac{1}{e^{\frac{\hbar \omega}{k_B T}} - 1}.
\end{equation}

\begin{definition}[Quality factor (Q factor)]
\label{def:Qfactor}
The quality factor $Q$ of an oscillatory system is defined as the number of oscillations 
performed by the system during the characteristic decay time $\tau$ \cite{bKarlov2003}. That is, if 
the change in the energy of the system $U$ (complex amplitude) satisfies the equation
\begin{equation}
\label{eq:part1:q:remark}
\frac{d U}{d t} = - \frac{1}{\tau} \left(U - U_0\right),
\end{equation}
then
\[
U = \left(\left.U\right|_{t=0} - U_0 \right)e^{-\frac{t}{\tau}} + U_0.
\]
Thus,
\[
Q = \omega \tau,
\]
and equation \eqref{eq:part1:q:remark} can be rewritten as
\[
\frac{d U}{d t} = - \frac{\omega}{Q} \left(U-U_0\right).
\]
\end{definition}


Equation \eqref{eqCh2_57} describes the change of the photon number (energy)
over time due to relaxation (interaction with a dissipative
system). The classical equation describing this process is
(see def. \ref{def:Qfactor})
\begin{equation}
\frac{d}{d t}\left<n\right> = 
- \frac{\omega}{Q}\left<n\right> + \frac{\omega}{Q} \left<n\right>_{\mathrm{eq}},
\nonumber
\end{equation}
where $\left<n\right>_{\mathrm{eq}} = \bar{n}_T$ is the equilibrium value of
$\left<n\right>$ at temperature $T$, and $Q$ is the quality factor of the resonator
(see def. \ref{def:Qfactor}).
\rindex{Quality factor}
Therefore, one can set
\begin{eqnarray}
R_b - R_a = \frac{\omega}{Q},
\nonumber \\
R_a = \bar{n}_T \frac{\omega}{Q},
\nonumber \\
R_b = \frac{\omega}{Q} \left(1 + \bar{n}_T\right),
\label{eqCh2_RabQw}
\end{eqnarray}
i.e., the introduced quantities $R_a$ and $R_b$ are expressed through
the classical quantity $\frac{\omega}{Q}$ characterizing losses in
the resonator. 

Another quantity of interest is the average electromagnetic field
\begin{eqnarray}
\bar{E}_{(f)} = E_0 \sin k z
\mathrm{Tr}\left(\hat{\rho}\left(\hat{a}^{\dag} + \hat{a}\right)\right) = 
\nonumber \\
= E_0 \sin k z \mathrm{Tr}\left(\hat{\rho}\hat{a}\right) + \mathrm{c.c.} = 
\nonumber \\
= E_1 \mathrm{Tr}\left(\hat{\rho}\hat{a}\right) + \mathrm{c.c.} =
\left<E\right> + \mathrm{c.c.}, 
\nonumber
\end{eqnarray}
where $\left<E\right>$ is the analytic signal of the classical
field. 
The equation satisfied by the field can be obtained using the equation of motion for the density matrix. Write the equation for
the density operator of the mode field  
\eqref{eqCh2_rho_final2} and
using the expressions for $R_a$, $R_b$ through $Q$ and $\bar{n}_T$
 \eqref{eqCh2_RabQw}, we have
\begin{eqnarray}
\dot{\hat{\rho}} =
- \frac{\omega}{2Q}\bar{n}_T
\left(\hat{a}\hat{a}^{\dag}\hat{\rho} - 
2 \hat{a}^{\dag}\hat{\rho}\hat{a} + \hat{\rho}\hat{a}\hat{a}^{\dag}
\right)
- 
\nonumber \\
- \frac{\omega}{2Q}\left(\bar{n}_T + 1\right)
\left(\hat{a}^{\dag}\hat{a}\hat{\rho} - 
2 \hat{a}\hat{\rho}\hat{a}^{\dag}
+ \hat{\rho}\hat{a}^{\dag}\hat{a}
\right)
\label{eqCh2_eq1_add1}
\end{eqnarray}

Since 
\begin{equation}
\left<E\right> = E_1 \mathrm{Tr}\left(\hat{\rho}\hat{a}\right), \quad
\frac{d \left<E\right>}{d t} = E_1 \mathrm{Tr}\left(\frac{d \hat{\rho}}{dt}\hat{a}\right),
\label{eqCh2_eq1_add2}
\end{equation}
we multiply equation \eqref{eqCh2_eq1_add1} by $E_1 \hat{a}$ and take the trace. The result is:
\begin{eqnarray}
\dot{\left<E\right>} =
- \frac{\omega E_1}{2Q}\bar{n}_T
\left\{\mathrm{Tr}\left(\hat{a}\hat{a}^{\dag}\hat{\rho}\hat{a} - 
2 \hat{a}^{\dag}\hat{\rho}\hat{a}\hat{a} +
\hat{\rho}\hat{a}\hat{a}^{\dag}\hat{a}\right) + 
\right.
\nonumber \\
+\left.
\mathrm{Tr}\left(\hat{a}^{\dag}\hat{a}\hat{\rho}\hat{a} - 
2 \hat{a}\hat{\rho}\hat{a}^{\dag}\hat{a}
+ \hat{\rho}\hat{a}^{\dag}\hat{a}\hat{a}
\right)
\right\}
- 
\nonumber \\
- \frac{\omega E_1}{2Q}
\mathrm{Tr}\left(\hat{a}^{\dag}\hat{a}\hat{\rho}\hat{a} - 
2 \hat{a}\hat{\rho}\hat{a}^{\dag}\hat{a}
+ \hat{\rho}\hat{a}^{\dag}\hat{a}\hat{a}
\right).
\label{eqCh2_eq1_add3}
\end{eqnarray}
It is known that under the trace one can perform cyclic permutations
of operators. Apply this to \eqref{eqCh2_eq1_add3}.
For example,
\begin{eqnarray}
\mathrm{Tr}\left(\hat{a}\hat{a}^{\dag}\hat{\rho}\hat{a} - 
2 \hat{a}^{\dag}\hat{\rho}\hat{a}\hat{a} +
\hat{\rho}\hat{a}\hat{a}^{\dag}\hat{a}\right) = 
\nonumber \\
= \mathrm{Tr}\left(\hat{a}\hat{a}^{\dag}\hat{\rho}\hat{a} - 
2 \hat{a}\hat{a}^{\dag}\hat{\rho}\hat{a} +
\hat{\rho}\hat{a}\hat{a}^{\dag}\hat{a}\right) = 
\nonumber \\
= \mathrm{Tr}\left(\hat{a}\hat{a}^{\dag}\hat{\rho}\hat{a} - 
2 \hat{a}\hat{a}^{\dag}\hat{\rho}\hat{a} +
\hat{\rho}\hat{a}\left(\hat{a}\hat{a}^{\dag} - 1\right)\right) = 
\nonumber \\
\mathrm{Tr}\left(\hat{a}\hat{a}^{\dag}\hat{\rho}\hat{a} - 
2 \hat{a}\hat{a}^{\dag}\hat{\rho}\hat{a} +
\left(\hat{a}\hat{a}^{\dag} - 1\right)\hat{\rho}\hat{a}\right) = 
- \mathrm{Tr}\left(\hat{\rho}\hat{a}\right).
\label{eqCh2_eq1_add4}
\end{eqnarray}
Doing the same for the second bracket in \eqref{eqCh2_eq1_add3}, we get
\begin{eqnarray}
\mathrm{Tr}\left(\hat{a}^{\dag}\hat{a}\hat{\rho}\hat{a} - 
2 \hat{a}\hat{\rho}\hat{a}^{\dag}\hat{a}
+ \hat{\rho}\hat{a}^{\dag}\hat{a}\hat{a} \right) = 
\nonumber \\
= 
\mathrm{Tr}\left(\hat{\rho}\hat{a}\hat{a}^{\dag}\hat{a} - 
2 \hat{\rho}\hat{a}^{\dag}\hat{a}\hat{a}
+ \hat{\rho}\hat{a}^{\dag}\hat{a}\hat{a} \right) = 
\nonumber \\
= 
\mathrm{Tr}\left(\hat{\rho}\left(\hat{a}^{\dag}\hat{a} + 1\right)\hat{a} - 
2 \hat{\rho}\hat{a}^{\dag}\hat{a}\hat{a}
+ \hat{\rho}\hat{a}^{\dag}\hat{a}\hat{a} \right) = 
\mathrm{Tr}\left(\hat{\rho}\hat{a}\right).
\label{eqCh2_eq1_add5}
\end{eqnarray}
Substituting \eqref{eqCh2_eq1_add4} and \eqref{eqCh2_eq1_add5} into 
\eqref{eqCh2_eq1_add3}, we have
\begin{eqnarray}
\dot{\left<E\right>} =
- \frac{\omega E_1}{2Q}\bar{n}_T
\left\{\mathrm{Tr}\left(\hat{\rho}\hat{a}\right) -
\mathrm{Tr}\left(\hat{\rho}\hat{a}\right)\right\} -
\nonumber \\
- \frac{\omega E_1}{2Q}
\mathrm{Tr}\left(\hat{\rho}\hat{a}\right) = 
- \frac{\omega}{2Q}\left<E\right>.
\label{eqCh2_61}
\end{eqnarray}

Thus, we have related the reservoir parameters (beam parameters) to the classical quantity $Q$ and to the average photon number in the mode 
at the reservoir temperature. This enables writing the density matrix equation of motion 
for the field in a general form, while the specific reservoir model does not matter 
\begin{eqnarray}
\dot{\rho}_{nm} = - \frac{\omega}{2 Q}
\left(2 \bar{n}_T\left( n + m + 1\right) + n + m \right)\rho_{nm} +
\nonumber \\
+ \frac{\omega \bar{n}_T}{Q}\sqrt{nm}\rho_{n - 1, m - 1} +
\frac{\omega}{Q}\left(\bar{n}_T + 1\right)
\sqrt{\left(n + 1\right)\left(m + 1\right)}
\rho_{n + 1, m + 1}
\label{eqCh2_63}
\end{eqnarray}
In operator form, this equation reads
\begin{eqnarray}
\dot{\rho} = - \frac{\omega}{2 Q}
\left\{
\bar{n}_T\left(\hat{a}\hat{a}^{\dag}\hat{\rho} - 
\hat{a}^{\dag}\hat{\rho}\hat{a}\right)
\right. +
\nonumber \\
+
\left .
\left(\bar{n}_T + 1\right)\left(\hat{a}^{\dag}\hat{a}\hat{\rho} - 
\hat{a}\hat{\rho}\hat{a}^{\dag}\right)
\right\} + \mathrm{h.c.}
\label{eqCh2_64}
\end{eqnarray}
The equations \eqref{eqCh2_rho_final2}, \eqref{eqCh2_64} in
number representation are one of many possible. Often it is convenient to use other representations.  
