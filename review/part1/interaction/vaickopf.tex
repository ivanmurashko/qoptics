%% -*- coding:utf-8 -*- 
\section{Spontaneous Emission. \\
  Weisskopf-Wigner Approximation.}
Above we obtained a formula for the rate of spontaneous emission,
valid for sufficiently short times. At the same time, when solving,
for example, the problem of the lifetime of an excited atom, one cannot
restrict to short times. In this case, a different method is applied,
called the Weisskopf-Wigner approximation.

Let us consider this method. We have established that the atom transitions from the excited
state to the ground state due to interaction with all modes
of the space, even if they are not excited and are in the vacuum
state. The interaction Hamiltonian in this case has the form
\begin{equation}
\hat{V}_I = \hbar \sum_{k} g_k \hat{\sigma}^{\dag}\hat{a}_k e^{i\left(
\omega_{ab} - \omega_k
\right)t} + \mbox{h.c.},
\label{eqCh2VaiskopfV}
\end{equation}
where $g_k$ is the interaction constant with the $k$-th mode. Compared to the Hamiltonian \eqref{eqCh2_task22}
we used earlier,
\eqref{eqCh2VaiskopfV} differs by summation over all quantization modes
of the space.

Assume that at the initial time $t=0$ the atom is excited, and
there are no photons in the field. Then the initial state reads
\begin{equation}
\left|\psi\left(0\right)\right> =
 \left|a, \left\{0\right\}\right>,
\nonumber
\end{equation}
that is, the atom is excited and all modes are in the vacuum state. At
later times, due to atom-field interaction, the state of the system
will be
\begin{equation}
\left|\psi\left(t\right)\right> =
\sum_{k} C_{bk}\left(t\right) \ket{b, 1_k}
+
C_{a}\left(t\right) \left|a, \left\{0\right\}\right>,
\label{eqCh2Vaiskopf3}
\end{equation}
where $\ket{b, 1_k}$ is the state in which the atom is in the ground
state and there is one photon in the $k$-th mode. \rindex{photon}

The Schrödinger equation in the interaction picture reads
\begin{equation}
\frac{d}{dt} \left|\psi\left(t\right)\right> =
- \frac{i}{\hbar} \hat{V}_I \left|\psi\left(t\right)\right>.
\label{eqCh2Vaiskopf4}
\end{equation}
Substituting \eqref{eqCh2Vaiskopf3} here and projecting equation
\eqref{eqCh2Vaiskopf4} sequentially onto
$\left<a, \left\{0\right\}\right|$ and $\bra{b, 1_k}$, we obtain
a system of equations for the probability amplitudes 
$C_{a}\left(t\right)$ and $C_{bk}\left(t\right)$:
\begin{eqnarray}
\dot{C}_{a}\left(t\right) = - i \sum_{k} g_k e^{i \left(\omega_{ab} - 
  \omega_k\right)t} C_{bk}\left(t\right),
\nonumber \\
\dot{C}_{bk}\left(t\right) = - i g_k e^{- i \left(\omega_{ab} -
  \omega_k\right)t} C_{a}\left(t\right).
\label{eqCh2Vaiskopf5}
\end{eqnarray}
Integrate the second equation \eqref{eqCh2Vaiskopf5} over time from $0$ to
$t$: 
\begin{equation}
C_{bk}\left(t\right) = - i g_k \int_0^{t} e^{- i \left(\omega_{ab} -
  \omega_k\right)t'} C_{a}\left(t'\right) dt'.
\label{eqCh2Vaiskopf6}
\end{equation}
Substituting \eqref{eqCh2Vaiskopf6} into the first equation of the system
\eqref{eqCh2Vaiskopf5}, we obtain the following integro-differential
equation:
\begin{equation}
\dot{C}_{a}\left(t\right) = - \sum_{k} g_k^2 
\int_0^t
e^{i \left(\omega_{ab} - \omega_k\right)\left(t - t'\right)}  
C_{a}\left(t'\right) dt'.
\label{eqCh2Vaiskopf7}
\end{equation}

Now, let's make several simplifications (the Weisskopf-Wigner approximation).
We will consider the mode distribution quasi-continuous and replace summation over
$k$ by integration using the relation
\eqref{eqCh1_modenumber_1pre}:
\begin{equation}
d N = 2 \left(\frac{L}{2 \pi} \right)^3 k^2 d k d \Omega = 
2 \frac{V}{\left(2 \pi\right)^3}  k^2 d k d \Omega
\nonumber
\end{equation}
Then we get \eqref{eqCh1_modenumber_kvazy_contig}
\begin{eqnarray}
\sum_{k} g_k^2 
\int_0^t
e^{i \left(\omega_{ab} - \omega_k\right)\left(t - t'\right)}  
C_{a}\left(t'\right) dt' = 
\nonumber \\
= 2 \frac{V}{\left(2 \pi\right)^3} \int_{4\pi}d \Omega \int_0^{\infty}
g_k^2 k^2 dk  \int_0^t dt'
e^{i \left(\omega_{ab} - \omega_k\right)\left(t - t'\right)}  
C_{a}\left(t'\right).
\label{eqCh2Vaiskopf8pre}
\end{eqnarray}
In \eqref{eqCh2Vaiskopf8pre} 
\begin{equation}
g_k^2 = \frac{\omega_k\left|p\right|^2 \sin^2 \theta}{4 \hbar
  \varepsilon_0 V},
\label{eqCh2VaiskopfGk}
\end{equation}
where $\theta$ is the angle between the directions $\vec{k}$ and $\vec{p}$.
Expression \eqref{eqCh2VaiskopfGk} is obtained taking into account averaging over
polarizations \eqref{eqCh2_PolyarMedian}.

Transitioning in \eqref{eqCh2Vaiskopf8pre} from $k$ to $\omega$ using the relations 
\begin{equation}
k = \frac{\omega_k}{c}, \quad k^2 dk = \frac{\omega_k^2 d \omega_k}{c^3}
\nonumber
\end{equation}
and denoting for convenience \(\omega_k = \omega\),
we get
\begin{equation}
\dot{C}_{a}\left(t\right) = - 
2 \frac{V}{\left(2 \pi c\right)^3} \int_{4\pi}d \Omega \int_0^{\infty}
g^2\left(\omega\right) \omega^2 d\omega  \int_0^t dt'
e^{i \left(\omega_{ab} - \omega\right)\left(t - t'\right)}  
C_{a}\left(t'\right).
\label{eqCh2Vaiskopf8}
\end{equation}

For an approximate evaluation of the integral in \eqref{eqCh2Vaiskopf8}
we additionally use a number of simplifying assumptions.
We first integrate \eqref{eqCh2Vaiskopf8} over frequency (i.e., changing
the order of integration, assuming this is possible). From the structure of 
\eqref{eqCh2Vaiskopf8} it is seen that the main contribution to the time integral
comes from the frequency region $\omega \approx \omega_{ab}$. For this reason,
we can take 
\[
\omega^2 g^2\left(\omega\right) \approx 
\omega_{ab}^2 g^2\left(\omega_{ab}\right).
\]
In this approximation, the frequency integral looks like (see
the integral representation of the delta-function
\eqref{eq:delta_from_integral}):  
\begin{eqnarray}
\int_0^{\infty}d \omega e^{i\left(\omega_{ab} - \omega\right)\left(t -
  t'\right)}  = \left|\nu = \omega - \omega_{ab}\right| =
\int_{- \omega_{ab}}^{\infty}d \nu e^{-i \nu\left(t - t'\right)} \approx
\nonumber \\
\approx \int_{- \infty}^{\infty} d \nu e^{-i \nu\left(t - t'\right)} = 
\int_{- \infty}^{\infty} d \nu e^{i \nu\left(t' - t\right)} =
2 \pi \delta\left(t' - t\right).
\label{eqCh2Vaiskopf9}
\end{eqnarray}

Substituting \eqref{eqCh2Vaiskopf9} into equation \eqref{eqCh2Vaiskopf8} and
integrating over time, we get
\begin{equation}
\dot{C}_{a}\left(t\right) = - 
2 \frac{V}{\left(2 \pi c\right)^3} \int_{4\pi}d \Omega 
g^2\left(\omega_{ab}\right) \omega_{ab}^2   
2 \pi C_{a}\left(t\right) = - \frac{\Gamma}{2} C_{a}\left(t\right).
\label{eqCh2Vaiskopf10}
\end{equation}
Substituting here the expression for $g_k^2$ \eqref{eqCh2VaiskopfGk} and
performing the angular integral (over $d \Omega$), we find an expression for
the decay coefficient $\Gamma$:
\begin{equation}
\Gamma = \frac{\omega_{ab}^3 \left|p\right|^2}{3 \pi c^2 \hbar}
\sqrt{\frac{\mu_0}{\varepsilon_0}}. 
\label{eqCh2Vaiskopf11}
\end{equation}
Note that expression \eqref{eqCh2Vaiskopf11} coincides with that obtained
earlier by another method \eqref{eqCh2_Wspon_final}.

From \eqref{eqCh2Vaiskopf10} it follows that $\Gamma$ characterizes
the rate of change of the probability $\left|C_{a}\right|^2$. Indeed,
from \eqref{eqCh2Vaiskopf10} we have:
\begin{eqnarray}
\dot{C}_{a}C_{a}^{*} = - \frac{\Gamma}{2}C_{a}C_{a}^{*},
\nonumber \\
\dot{C}_{a}^{*}C_{a} = - \frac{\Gamma}{2}C_{a}^{*}C_{a},
\nonumber
\end{eqnarray}
from which it follows that
\begin{equation}
\frac{d C_{a}C_{a}^{*}}{dt} = -\Gamma \left(C_{a}C_{a}^{*}\right),
\nonumber
\end{equation}
or equivalently
\begin{equation}
\frac{d \left|C_{a}\right|^2}{dt} = -\Gamma \left|C_{a}\right|^2.
\nonumber
\end{equation}
The solution has the form (see \autoref{fig:part1:vaickopf})
\[
\left|C_{a}\right|^2 = e^{- \Gamma t},
\]
if the initial value $\left.\left|C_{a}\right|^2\right|_{t = 0} =
1$, i.e. the atom was initially excited.

\input ./part1/interaction/figvaickopf.tex 

\begin{remark}[Weisskopf-Wigner Approximation and Rabi Oscillations]
If we compare \autoref{fig:part1:rabi} and
\autoref{fig:part1:vaickopf}, we can see that these two graphs
differ significantly despite apparently describing the same model system.
In this regard, the question arises when we can
apply the single-mode approximation and correspondingly obtain
Rabi oscillations (see \autoref{fig:part1:rabi}), and when it is necessary
to use a multi-mode approximation like
Weisskopf-Wigner (see \autoref{fig:part1:vaickopf}).

Formally, we can consider that the interaction with each mode
of the electromagnetic field gives its contribution to the resulting probabilities,
while according to \eqref{eqCh2_prob_C_bn} the change
of probability (for small $t$)
\[
\Delta \left|C_{a, n}\left(t\right)\right|^2 = -\Delta \left|C_{b, n +
  1}\left(t\right)\right|^2 \sim \left(n + 1\right).
\]
Thus, those modes with a large number of photons give a greater contribution than
those with fewer. Consequently, if we have a mode with
a large number of photons $n \gg 1$, then the influence of vacuum modes with photon number $n = 0$ can be neglected. Otherwise,
especially when all modes are equivalent, it is necessary to take into account all modes and
apply approximations of the Weisskopf-Wigner type.
\end{remark}