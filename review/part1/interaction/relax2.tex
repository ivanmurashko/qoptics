%% -*- coding:utf-8 -*- 
\section{Relaxation of an atom in the case of the simplest
  reservoir consisting of harmonic oscillators}
Consider a multimode electromagnetic field described by creation and annihilation operators $\hat{a}_k^{\dag}$,
$\hat{a}_k$. The frequency of the $k$-th mode is denoted by $\omega_k$. 

This electromagnetic field interacts with a certain two-level
atom. The atom can be in two states: 
excited $\ket{a}$ and non-excited
$\ket{b}$. The transition operators remain 
$\hat{\sigma}$ and $\hat{\sigma}^{\dag}$. The transition frequency is denoted by
\rindex{Hamiltonian}
$\Omega$. The interaction Hamiltonian in this case has the form  
\begin{equation}
\hat{V}\left(t\right) = \hbar
\sum_{(k)} g_k \hat{a}_k^{\dag} \hat{\sigma}e^{-i \left(
\Omega - \omega_k
\right) t} +
\mbox{h.c.}
\label{eqCh2_94}
\end{equation}
where $g_k$ is the interaction constant.

Thus, the system $B$ (the reservoir) is the multimode
electromagnetic field, and the dynamic system $A$ under study is a
two-level atom. The density operator of the reservoir (electromagnetic
field) is denoted by $\hat{\rho}_R$. The density operator of the dynamic
system (atom) under study is denoted by $\hat{\rho}_{at}$.

The following relations hold:
\begin{equation}
Sp_R\left(
\hat{a}_k^{\dag}
\hat{a}_k
\hat{\rho}_R
\right) = \bar{n}_k
\label{eqCh2_96}
\end{equation}
where $\bar{n}_k$ is the
average number of photons in the reservoir mode; 
\begin{eqnarray}
Sp_R\left(\hat{a}_k\hat{a}_k^{\dag}\hat{\rho}_R\right) = \bar{n}_k + 1,
\nonumber \\
Sp_R\left(\hat{a}_k\hat{a}_k\hat{\rho}_R\right) = 
Sp_R\left(\hat{a}_k^{\dag}\hat{a}_k^{\dag}\hat{\rho}_R\right) = 0.
\label{eqCh2_96_add}
\end{eqnarray}
Using these equalities, the general equation
\eqref{eqCh2_93} can be rewritten as:  
%FIXME!!! check it
\begin{eqnarray}
\dot{\hat{\rho}}_{at}\left(t\right) = 
- \int_{t_0}^t\sum_{(k)}g_k^2
\left\{
\hat{\sigma}^{\dag} \hat{\sigma}\hat{\rho}_{at}\left(t'\right)
\left(
\bar{n}_k + 1
\right)
e^{i\left(\Omega - \omega_k\right)
\left(t - t'\right)}
+
\right.
\nonumber \\
\left.
+\hat{\sigma}\hat{\sigma}^{\dag}
\hat{\rho}_{at}\left(t'\right)
\bar{n}_k
e^{-i\left(\Omega - \omega_k\right)
\left(t - t'\right)} -
\hat{\sigma}^{\dag}
\hat{\rho}_{at}\left(t'\right)
\hat{\sigma}
\bar{n}_k
e^{i\left(\Omega - \omega_k\right)
\left(t - t'\right)}
-
\right.
\nonumber \\
-
\left.
\hat{\sigma}
\hat{\rho}_{at}\left(t'\right)
\hat{\sigma}^{\dag}
\left(\bar{n}_k + 1\right)
e^{-i\left(\Omega - \omega_k\right)
\left(t - t'\right)}
\right\}dt'
+ \mbox{h.c.}
\label{eqCh2_97}
\end{eqnarray}

Summation over $k$, taking into account the quasi-continuous spectrum of states
of the dissipative system, can be replaced by integrals
\eqref{eqCh1_modenumber_kvazy_contig} 
\[
\sum_{(k)} \rightarrow \int D\left(\omega\right)d \omega,
\]  
where $D\left(\omega\right)$ is the spectral density of states. We replace
$D\left(\omega\right)$, $\bar{n}\left(\omega\right)$,
$g^2\left(\omega\right)$ with $D$, $\bar{n}$, $g^2$ — their values at
$\omega = \Omega$, assuming that these functions slowly   
depend on frequency. We use the equality (see
the integral representation of the delta function
\eqref{eq:delta_from_integral})
\begin{eqnarray}
\int_{0}^\infty
e^{\pm i\left(\Omega - \omega\right)
\left(t - t'\right)}d \omega = \left|\nu = \omega - \Omega\right| =
\nonumber \\
=
\int_{-\Omega}^\infty
e^{ - i \nu
\left(t \mp t'\right)}d \nu \approx
\int_{-\infty}^\infty
e^{ - i \nu
\left(t \mp t'\right)}d \nu = 
\nonumber \\
= 2 \pi \delta \left(t \mp t'\right) 
\label{eqCh2_98}
\end{eqnarray}
and integrate over time. This leads us to the equation 
\begin{eqnarray}
\dot{\hat{\rho}}_{at} = -\frac{1}{2}\gamma \left\{
\bar{n}\left[
\hat{\sigma}\hat{\sigma}^{\dag}\hat{\rho}_{at} -
\hat{\sigma}^{\dag}\hat{\rho}_{at}\hat{\sigma}
\right] +
\right.
\nonumber \\
+
\left.
\left(\bar{n} + 1\right)\left[
\hat{\sigma}^{\dag}\hat{\sigma}\hat{\rho}_{at} -
\hat{\sigma}\hat{\rho}_{at}\hat{\sigma}^{\dag}
\right]
\right\}
+ \mbox{h.c.}
\label{eqCh2_99}
\end{eqnarray}
where $\gamma =4 \pi D g^2$, with $\bar{n}$ taken at the mode frequency $\Omega$.
Equation \eqref{eqCh2_99} has the same form as equation
\eqref{eqCh2_64} obtained earlier. The equations 
completely coincide if one takes $\gamma = \frac{\omega}{Q}$ and
makes the substitution $\hat{\sigma} \rightarrow \hat{a}$, $\hat{\sigma}^{\dag}
\rightarrow \hat{a}^{\dag}$.  This once again confirms that the final
result does not depend on the nature of the reservoir.   

At reservoir temperature $T = 0$  photons are absent and $\bar{n} =
0$.  Assume that at the initial moment the atom is excited 
\[
\hat{\rho}_{at}\left(t_0\right) = 1.
\]
Using the properties \eqref{eqCh2_8} - \eqref{eqCh2_task1} of the operators 
$\hat{\sigma}$ and $\hat{\sigma}^{\dag}$, we obtain   
\begin{eqnarray}
\dot{\rho}_{aa} = \bra{a}\dot{\hat{\rho}}_{at}\ket{a} =
-\frac{1}{2}\gamma
\bra{a}\hat{\sigma}^{\dag}\hat{\sigma}\hat{\rho}_{at}\ket{a}
+ 
\nonumber \\
+ \frac{1}{2}\gamma
\bra{a}\hat{\sigma}\hat{\rho}_{at}\hat{\sigma}^{\dag}\ket{a}
+ \mbox{c.c.} = 
\nonumber \\
= 
- \frac{1}{2}\gamma
\bra{a}\hat{\rho}_{at}\ket{a} + \mbox{c.c.} = 
- \gamma \rho_{aa}.
\label{eqCh2_101}
\end{eqnarray}
Here we used the relations
\(\hat{\sigma}^{\dag}\ket{a} = 0\),
\(\hat{\sigma}^{\dag}\ket{b} = \ket{a}\),
\(\hat{\sigma}\ket{a} = \ket{b}\),
\(\hat{\sigma}\ket{b} = 0\),
and their conjugates. The final result is 
\begin{equation}
\dot{\hat{\rho}}_{aa} = - \gamma \rho_{aa}.
\label{eqCh2_102}
\end{equation}
This means that as a result of spontaneous emission the probability
of finding the atom in the upper level decreases exponentially. 
