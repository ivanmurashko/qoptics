%% -*- coding:utf-8 -*- 
\section{Decomposition of the field into plane waves in free space}
So far, we have considered the electromagnetic field in a physically
distinct volume - in a resonator. If we are dealing with free
space, we can artificially select a sufficiently large volume
containing the region of space of interest, define for it
modes (types of oscillations) under appropriate boundary conditions,
and then proceed using the previously considered method. If necessary,
the volume can be taken to infinity in the final result. Usually,
a sufficiently large cubic volume with side length $L$ is used
(\autoref{figCh1_Vfree}). In this case, periodic boundary conditions are adopted:
\begin{eqnarray}
\vec{E}\left(0, y, z \right) = \vec{E}\left(L, y, z \right),
\nonumber \\
\vec{E}\left(x, 0, z \right) = \vec{E}\left(x, L, z \right),
\nonumber \\
\vec{E}\left(x, y, 0 \right) = \vec{E}\left(x, y, L \right).
\label{eqCh1_period_def}
\end{eqnarray}

\input ./part1/quantel/fig2.tex

It is convenient to expand in plane waves. As is known from the course
on electromagnetic oscillations, the solution corresponding to a plane wave
has the form 
\begin{equation}
\vec{E}_k\left(r, t\right) = 
A_k\left(t\right)\vec{e}_k e^{i\left(\vec{k}\cdot\vec{r}\right)} +
\mbox{(c. c.)},
\label{eqCh1_Emode}
\end{equation}
where $\vec{e}_k$ is the unit polarization vector of the wave;  
$k$ is the wavenumber; $\vec{k}$ is the wavevector;  
$A_k\left(t\right) = A_k\left(0\right) e^{-i \omega_k t}$.

The magnetic field is related to the electric field by the relation
\begin{equation}
\vec{H}_k\left(r, t\right) =
\sqrt{\frac{\varepsilon_0}{\mu_0}}
\frac{1}{k}\left[\vec{k} \times \vec{e}_k\right] A_k\left(t\right) 
e^{i\left(\vec{k}\cdot\vec{r}\right)} + \mbox{(c. c.)}
.
\label{eqCh1_Hmode}
\end{equation}

The following equalities hold:
\begin{equation}
\left(\vec{k}\cdot\vec{E}_k\right) = 
\left(\vec{k}\cdot\vec{H}_k\right) = 
\left(\vec{E}_k \cdot \vec{H}_k\right) = 0,
\quad
k^2 = \left(\vec{k}\cdot\vec{k}\right) = 
\frac{\omega_k^2}{c^2} 
\end{equation}
indicating that 
$\vec{E}_k$
and $\vec{H}_k$ are perpendicular to the direction of wave propagation and to each other (the wave is transverse). 

The periodicity condition \eqref{eqCh1_period_def} will be satisfied if 
\begin{equation}
\vec{k} = \frac{2 \pi}{L}\left(n_x \vec{x}_0
+ n_y \vec{y}_0
+ n_z \vec{z}_0
\right),
\quad
n_x, n_y, n_z = 0, \pm 1, \pm 2, \dots .
\label{eqCh1_period}
\end{equation}
Then we obtain
\begin{eqnarray}
\left(\vec{k}\cdot\vec{r}\right) = \frac{2 \pi}{L}\left(n_x x
+ n_y y
+ n_z z
\right),
\nonumber \\
\left.\left(\vec{k}\cdot\vec{r}\right)\right|_{x = L} = 2 \pi n_x + \frac{2 \pi}{L}\left(n_y y
+ n_z z
\right) = 
2 \pi n_x + \left.\left(\vec{k}\cdot\vec{r}\right)\right|_{x = 0}.
\end{eqnarray}
Therefore,    
$e^{i\left(\vec{k}\cdot\vec{r}\right)}$
is periodic in $x$. In a similar way, periodicity in $y$ and $z$ is shown.
 
In what follows, instead of the real functions \eqref{eqCh1_Emode}, 
\eqref{eqCh1_Hmode}, we will use
complex eigenfunctions 
\begin{equation}
\vec{E}_k\left(r\right) = \vec{e}_k e^{i \left( \vec{k}\cdot\vec{r}\right)},
\quad
\vec{H}_k\left(r\right) = \sqrt{\frac{\varepsilon_0}{\mu_0}}\frac{1}{k}
\left[\vec{k} \times \vec{E}_k\left(r\right)\right].
\label{eqCh1_EHmode}
\end{equation}
The following relations hold:
\[
\left(\vec{k}\cdot\vec{E}_k\right) = 
\left(\vec{k}\cdot\vec{H}_k\right) = 
\left(\vec{E}_k \cdot \vec{H}_k\right) = 0,
\quad
k^2 = \left(\vec{k}\cdot\vec{k}\right) = 
\frac{\omega_k^2}{c^2}.
\]
The eigenfrequencies $\omega_k$ are determined by the equation
\eqref{eqCh1_period}. Indeed, from \eqref{eqCh1_period} and taking into account
$k^2 = \frac{\omega_k^2}{c^2}$ we have 
\begin{equation}
\omega_k = c \sqrt{k_x^2 + k_y^2 + k_z^2} = 
\frac{2 \pi c}{L} \sqrt{n_x^2 + n_y^2 + n_z^2}.
\end{equation}

For linear polarization, the polarization vectors $\vec{e}_k$ are:
\[
\vec{e}_{k_1} = \vec{\xi}_0,
\quad
\vec{e}_{k_2} = \vec{\eta}_0,
\]
where $\vec{\xi}_0$, $\vec{\eta}_0$ are basis directions forming, together with
the vector $\vec{k}$, a rectangular coordinate system; therefore, 
\[
\left(\vec{\xi}_0 \cdot \vec{k}\right) =
\left(\vec{\eta}_0 \cdot \vec{k}\right) =
\left(\vec{e}_{k_1} \cdot \vec{e}_{k_2}\right) = 0.
\]

For circular polarization
\[
\vec{e}_{k_1} = \frac{\vec{\xi}_0 + i \vec{\eta}_0}{\sqrt{2}},
\quad
\vec{e}_{k_2} = \frac{\vec{\xi}_0 - i \vec{\eta}_0}{\sqrt{2}}.
\]

In this case, the orthogonality condition holds:
$\left(\vec{e}_{k_1} \cdot \vec{e}_{k_2}^{*}\right) = 0$
and the normalization conditions:
$\left(\vec{e}_{k_1} \cdot \vec{e}_{k_1}^{*}\right) = \left(\vec{e}_{k_2}
\cdot \vec{e}_{k_2}^{*}\right) = 1$.

An arbitrary electromagnetic field using the complex functions 
\eqref{eqCh1_EHmode} can be represented by expansions 
\begin{eqnarray}
\vec{E}\left(r, t\right) = 
\sum_{(k)} 
A_k\left(t\right) \vec{E}_k\left(r\right) + \mbox{(c. c.)},
\nonumber \\
\vec{H}\left(r, t\right) = 
\sum_{(k)} 
A_k\left(t\right) \vec{H}_k\left(r\right) +
\mbox{(c. c.)}
\end{eqnarray}