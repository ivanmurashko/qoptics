%% -*- coding:utf-8 -*- 
\section{Quantum state of the electromagnetic field with a definite
  energy (with a definite number of photons)}
First, consider a single mode of the electromagnetic field (a simple
harmonic oscillator). The state vector with a definite energy
$\ket{E_n}$ satisfies the following equation
\begin{equation}
\hat{\mathcal{H}} \ket{E_n} = E_n \ket{E_n}
\end{equation}
Here and below we will use the Dirac formalism (see 
\autoref{AddDirac}). Using the relation \eqref{eqCh1_mainpropertyaa},
we obtain 
\begin{equation}
\hat{\mathcal{H}} \hat{a}\ket{E_n} = 
\left(\hat{a}\hat{\mathcal{H}} -
\hbar\omega\hat{a}\right)\ket{E_n} =
\left(E_n - \hbar \omega\right)\hat{a}\ket{E_n}
\end{equation}
i.e. $\hat{a}\ket{E_n}$ is also a state vector with
energy $E_n - \hbar \omega$. From this it follows that
\[
\hat{a}\ket{E_n} = \left|E_n - \hbar \omega \right>
\] 
Therefore, the operator $\hat{a}$ lowers the energy of the state by $\hbar
\omega$,  where $\omega$ is the frequency of the mode (oscillator). Often $\hat{a}$
is called the lowering operator, or annihilation operator. The lowest
energy must be positive, and the lowering of energy cannot
continue indefinitely. For an arbitrary state vector,
the expected value of energy is
\begin{equation}
\left< \Phi \right| \hbar \omega \left({a}^{\dag} {a}  +
\frac{1}{2}\right)\left| \Phi \right> = 
\hbar \omega \left< \Phi' \right. \left| \Phi' \right> + \frac{1}{2}
\hbar \omega,
\end{equation}
where  $\hat{a} \left| \Phi \right> = \left| \Phi' \right>$,  
$\left< \Phi \right| \hat{a}^{\dag}  = \left< \Phi' \right|$.  Since
the norm of the state vector must be positive, 
the lowest value of energy will be at  
\(
\left< \Phi' \right. \left| \Phi' \right> = 0.
\)
This means that $\hat{a}\ket{0} = 0$,  where
$\left|\Phi\right> = \ket{0}$ is the state vector with the lowest
energy. The lowest energy is  
\begin{equation}
E_0 = \frac{\hbar \omega}{2}
\end{equation}
called the zero-point energy. To verify this claim,
one can write
\begin{eqnarray}
\hat{\mathcal{H}} \ket{0} = 
\hbar \omega \left(\hat{a}^{\dag} \hat{a} +
\frac{1}{2}\right) \ket{0} = 
\nonumber \\
= 
\hbar \omega \hat{a}^{\dag} \hat{a} \ket{0} +
\frac{\hbar \omega}{2}\ket{0} =
\nonumber \\
= \frac{\hbar \omega}{2} \ket{0} = 
E_0 \ket{0}.
\label{eqProper0state}
\end{eqnarray}


Using \eqref{eqCh1_mainpropertyaa}, one can obtain
\begin{eqnarray}
\hat{\mathcal{H}} \hat{a}^{\dag}\ket{0} = 
\hat{\mathcal{H}} \ket{1} =
\left(\hat{a}^{\dag} \hat{\mathcal{H}} + \hbar \omega \hat{a}^{\dag} \right)
\ket{0} = 
\nonumber \\
= \hbar \omega \left(1 + \frac{1}{2}\right)
\hat{a}^{\dag} \ket{0} = 
\hbar \omega \left(1 + \frac{1}{2}\right)
\ket{1}
\end{eqnarray}
where $\ket{1} = \hat{a}^{\dag} \ket{0}$ denotes the
state with energy $\hbar \omega \left(1 + \frac{1}{2}\right)$.

By induction, we have
\begin{equation}
\hat{\mathcal{H}} \left(\hat{a}^{\dag}\right)^n\ket{0} = 
\hat{\mathcal{H}} \ket{n} 
= \hbar \omega \left(n + \frac{1}{2}\right)
\left(\hat{a}^{\dag}\right)^n\ket{0} = 
\hbar \omega \left(n + \frac{1}{2}\right)
\ket{n}
\label{eqCh1_aplusinduction}
\end{equation}
where
$\ket{n} = \left(\hat{a}^{\dag}\right)^n\ket{0}$   
without normalization, which we will perform later - a state with energy  
$\hbar \omega \left(n + \frac{1}{2}\right)$,  $n$  is an integer
positive number.

We see that the operator  $\hat{a}^{\dag}$  raises the energy of the state by
$\hbar \omega$.  It can be considered as the creation operator of a
particle - a photon \rindex{photon} with energy  $\hbar \omega$.  It is better
to talk about a photon as a particle in the case of expansion of the field in plane waves. Then it will be
a particle with energy $\hbar \omega$ and momentum $\hbar \vec{k}$, as
follows from \eqref{eqCh1_task3_2}. 
  
Relations 
$\hat{a} \ket{n} = \ket{n - 1}$
and
$\hat{a}^{\dag} \ket{n} = \ket{n + 1}$
define unnormalized state vectors. Let us define the normalization
factor. Suppose that  
$\hat{a} \ket{n} = S_n \ket{n - 1}$,  where 
$\ket{n}$ and $\ket{n - 1}$ are normalized to  1,  and $S_n$
is the normalization factor. From this we get 
\[
S_n^2\bra{n - 1}\ket{n - 1} =
\bra{n}\hat{a}^{\dag}\hat{a}\ket{n} = 
n  \bra{n}\ket{n}
\]
since the operator    
$\hat{a}^{\dag}\hat{a} = \hat{n}$
is the photon number operator, whose eigenvalue
is the number of photons. This can be seen from the formula
\eqref{eqCh1_aplusinduction}. Indeed, from the equality
\[
\hat{\mathcal{H}} \ket{n} =
\hbar \omega \left(
\hat{a}^{\dag}\hat{a} + \frac{1}{2}
\right)
\ket{n} = 
\hbar \omega \left(n + \frac{1}{2}\right)
\ket{n},
\]
we get:
\[
\hat{n}\ket{n} = \hat{a}^{\dag}\hat{a} \ket{n} = n
\ket{n}. 
\]
From the normalization condition: $\bra{n}\ket{n} = 1$   and
$S_n^2 = n$,  hence $S_n = \sqrt{n}$ and, therefore, we have: 
\begin{equation}
\hat{a}\ket{n} = \sqrt{n}\ket{n - 1}
\end{equation}
Similarly, using commutation relations
\eqref{eqCh1_aacomutation}, as well as \ref{eqAddDirac_operator_property1} and
  \ref{eqAddDirac_operator_property2} from \autoref{AddDirac},
  we obtain 
\begin{eqnarray}
\hat{a}^{\dag}\ket{n} = S_{n+1}\ket{n + 1},
\quad 
\bra{n}\hat{a} = S_{n + 1}\bra{n + 1},
\nonumber \\
\bra{n}\hat{a}\hat{a}^{\dag}\ket{n} = S_{n+1}^2
\bra{n + 1}\ket{n + 1} = 
\bra{n}\hat{a}^{\dag}\hat{a} + 1\ket{n} = 
\left(n + 1\right)\bra{n}\ket{n},
\nonumber \\
S_{n+1}^2 = n + 1.
\end{eqnarray}
Therefore, we have the equality
\begin{equation}
\hat{a}^{\dag}\ket{n} = \sqrt{n + 1}\ket{n + 1},
\end{equation}

The eigenstates of the photon number operator $\hat{n}$ are orthonormal. 
Indeed, from the fact that the operator $\hat{n}$ is Hermitian:
\[
\hat{n}^{\dag} = \left(\hat{a}^{\dag}\hat{a}\right)^{\dag} = 
\hat{a}^{\dag} \left(\hat{a}^{\dag}\right)^{\dag} = 
\hat{a}^{\dag}\hat{a} = \hat{n}
\]
it follows (see \autoref{AddDirac}) that the eigenfunctions of this operator,
corresponding to different eigenvalues, are orthogonal, i.e.
\begin{equation}
\bra{n}\ket{n'} = 0, \mbox{ if } n \ne n'.
\label{eqOrtoN}
\end{equation}

We give a summary of the relations involving the operators $\hat{a}$ and $\hat{a}^{\dag}$:
\begin{eqnarray}
\hat{\mathcal{H}} = \hbar \omega \left(\hat{a}^{\dag}\hat{a} +
\frac{1}{2} \right),
\quad
\hat{a}\ket{0} = 0,
\quad
\hat{a}^{\dag}\hat{a}\ket{n} = \hat{n}\ket{n},
\nonumber \\
\left[\hat{a}, \hat{a}^{\dag}\right] = \hat{a} \hat{a}^{\dag} - \hat{a}^{\dag}
\hat{a} = 1,
\quad
\hat{a}\ket{n} = \sqrt{n}\ket{n - 1}
\nonumber \\
\hat{\mathcal{H}}\ket{n} = \hbar \omega \left(\hat{a}^{\dag}\hat{a} +
\frac{1}{2} \right)\ket{n},
\nonumber \\
\hat{a}^{\dag}\ket{n} = \sqrt{n + 1}\ket{n + 1},
\quad
\ket{n} = \frac{1}{\sqrt{n!}}\left(\hat{a}^{\dag}\right)^n\ket{0}
\end{eqnarray}
and the conjugate equalities
\begin{eqnarray}
\bra{0}\hat{a}^{\dag} = 0,
\quad
\bra{n}\hat{a} = \sqrt{n + 1}\bra{n + 1}
\nonumber \\
\bra{n}\hat{a}^{\dag} = \sqrt{n}\bra{n - 1},
\quad
\bra{n} =  \frac{1}{\sqrt{n!}} \bra{0}\left(\hat{a}^{\dag}\right)^n.
\end{eqnarray}

For the simplest model of a resonator, we have
\[
\hat{E}\left(z, t\right) = E_1\left( \hat{a} +
\hat{a}^{\dag}\right) \sin k_n z
\]
where $E_1 = \sqrt{\frac{\hbar \omega}{\varepsilon_0 V}}$ is the field
corresponding to one photon in the mode.  

Let us consider some properties of energy states, i.e. states
with a definite number of photons. We show that the average value
of the electric field in this state is zero: 
\begin{eqnarray}
\bra{n}\hat{E}\ket{n} = 
E_1 \sin k_n z \left( \bra{n}\hat{a}\ket{n} +
\bra{n}\hat{a}^{\dag}\ket{n}\right) =
\nonumber \\
= E_1 \sin k_n z \left( \bra{n}\ket{n - 1} \sqrt{n} +
\bra{n}\ket{n + 1} \sqrt{n + 1}
\right) = 0
\label{eqCh1_E_middle}
\end{eqnarray}
which follows from the orthogonality of states \eqref{eqOrtoN}
$\bra{n}\ket{n'} = 0$.

The average value of the square of the electric field operator is nonzero:
\begin{eqnarray}
\bra{n}\hat{E}^2\ket{n} = 
E_1^2 \sin^2 k_n z \bra{n}
\left(
\hat{a}^{\dag} \hat{a}^{\dag} + \hat{a} \hat{a}^{\dag} + \hat{a}^{\dag} \hat{a} +
\hat{a} \hat{a}
\right)
\ket{n} =
\nonumber \\
= 2 E_1^2 \sin^2 k_n z \left( n + \frac{1}{2}
\right).
\label{eqCh1_E2_middle}
\end{eqnarray}

Usually, the fields considered in quantum optics are not
in a stationary state with a definite energy (with a definite number of
photons). However, an arbitrary state can be represented as
a superposition of states $\ket{n}$: 
\begin{equation}
\left|\psi\right> = \sum_{(n)} C_n \ket{n}
\end{equation}
where $\left|C_n\right|^2$ is the probability of detecting $n$
photons in the mode upon measurement; $\sum_{(n)} \left|C_n\right|^2 = 1$. 
In \autoref{AddDirac} it is shown that
\[
C_n = \bra{ n }\left| \psi \right>, \quad
\left| \psi \right> = \sum_{(n)} \bra{ n }\left| \psi \right>
\ket{ n } =
\sum_{(n)} \ket{ n }\bra{ n }\left| \psi \right>,
\quad
\sum_{(n)} \ket{ n }\bra{ n } = \hat{I},
\]
where $\hat{I}$ is the identity operator.

\begin{remark}[About the states with definite energy in quantum
    mechanics]
  It is worth noting that states with definite energy do not violate 
  the Heisenberg uncertainty relation for the energy-time pair
  \[
  \Delta E \Delta t \ge \frac{\hbar}{2},
  \]
  (see more details in \autoref{AddHeisenbergUncertaintyPrincipleEnergyTime}). 

  At the same time, if one looks at the energy operator of the harmonic
  oscillator, written as
  \[
  \hat{\mathcal{H}} =\frac{1}{2} \left(\hat{p}^2 +
  \omega^2\hat{q}^2\right)
  \]
  and uses the expressions \eqref{eqCh1_qpdef}, one can obtain
  that $\bra{n}\hat{q}\ket{n} =
  \bra{n}\hat{p}\ket{n} = 0$. Moreover,
  \begin{eqnarray}
    \bra{n}\hat{q}^2\ket{n} = \frac{\hbar}{2 \omega}
    \left[
      \bra{n}\hat{a}^2\ket{n} +
      \bra{n}\left(\hat{a}^{\dag}\right)^2\ket{n} +
      \right.
      \nonumber \\
      \left.
      +
      \bra{n}\hat{a}^{\dag}\hat{a}\ket{n} +
      \bra{n}\hat{a}\hat{a}^{\dag}\ket{n}
      \right] =
    \nonumber \\
    = \frac{\hbar}{2 \omega}
    \left[n + n + 1\right],
    \nonumber
  \end{eqnarray}
  and also
  \begin{eqnarray}
    \bra{n}\hat{p}^2\ket{n} = - \frac{\hbar \omega}{2}
    \left[
      \bra{n}\hat{a}^2\ket{n} +
      \bra{n}\left(\hat{a}^{\dag}\right)^2\ket{n} - \right.
      \nonumber \\
      \left.
      -
      \bra{n}\hat{a}^{\dag}\hat{a}\ket{n} -
      \bra{n}\hat{a}\hat{a}^{\dag}\ket{n}
      \right] =
    \nonumber \\
    = \frac{\hbar \omega}{2}
    \left[n + n + 1\right].
    \nonumber
  \end{eqnarray}
  Thus,
  \[
  \Delta p = \sqrt{\bra{n}\hat{p}^2\ket{n} -
    \bra{n}\hat{p}\ket{n}^2} =
  \sqrt{\frac{\hbar \omega}{2}\left(2n + 1\right)}
  \]
  and
  \[
  \Delta q = \sqrt{\bra{n}\hat{q}^2\ket{n} -
    \bra{n}\hat{q}\ket{n}^2} =
  \sqrt{\frac{\hbar}{2\omega}\left(2n + 1\right)}
  \]
  or
  \[
  \Delta p \Delta q = \frac{\hbar}{2}\left(2n + 1\right) \ge \frac{\hbar}{2},
  \]
  which is consistent with the Heisenberg uncertainty relations
  \eqref{eqAddHeisenbergUncertaintyPrinciple}.

  Thus, states with definite energy are such states,
  where despite the impossibility of precisely determining $p$
  and $q$, it is possible to determine the quantity 
  \(
  \frac{1}{2} \left(p^2 +
  \omega^2q^2\right)
  \). This reflects the fact that for quantum systems there exist
  situations where compound events exist in the absence
  of elementary ones (see \autoref{sec:add:quantprobability}). 
  In particular, such a situation exists for the photon
  - a state with definite energy $\ket{1}$.

  This can lead us to the reasoning that the photon is a
  virtual particle, which does not exist in the real physical world
  \cite{Lamb1995}.\rindex{photon} At the same time, the mathematical apparatus related to
  the concept of the photon, such as creation operators $\hat{a}^\dag$ and
  annihilation operators $\hat{a}$, states of the electromagnetic field with
  definite energy $\{\ket{n}\}$ (which can be
  used as basis states), appears convenient
  for theoretical description.

  It is also worth noting that from the formal definition
  of a nonclassical state \eqref{eqPart3_Nonclass_Nonclass7},
  $\ket{n}$ is a nonclassical 
  state of light since from \eqref{eqCh4_26} it follows that $G^{(2)} <
  1$.

  On the other hand, if one looks at the minimal possible energy
  of a mode of the electromagnetic field, from the relation
  \eqref{eqCh1_hamilton_one_mode}, 
  \[
  \mathcal{H} = \frac{1}{2}\left(\omega^2 q^2 + p^2\right)
  \]
  it follows that in the classical case the minimal possible zero
  energy is achieved at $p = 0, q=0$, but due to
  \eqref{eqAddHeisenbergUncertaintyPrinciple} zero values
  are impossible in the quantum case, and taking into account measurement uncertainties
  we obtain 
  \begin{eqnarray}
    \frac{1}{2}\left(\omega^2 (\Delta q)^2 + (\Delta p)^2\right) \ge
    \nonumber \\
    \ge \omega \Delta q \Delta p \ge \frac{\hbar \omega}{2}
    \nonumber
  \end{eqnarray}
  i.e. the minimal possible energy (vacuum energy) is determined
  by the Heisenberg inequalities for the coordinate-momentum pair.  
  \label{rem:antiphoton}
\end{remark}