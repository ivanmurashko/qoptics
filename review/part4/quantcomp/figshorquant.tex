\begin{figure}
\centering

% draw2d(yrange=[0, 18], points_joined = true, point_type = 6,
% points(map(lambda([x, y], cons(x, [y])),  makelist(x, x, 1,
% 64),makelist(power_mod(2, x, 21), x, 1, 64) )), user_preamble="set
% output 'picshorquantfourier1.tex'; set terminal latex; set xlabel
% '$x$'; set ylabel '$f(x)$'"); 

% draw2d(yrange=[0, 1.2], points_joined = false, point_type = 6,
% points(map(lambda([x, y], cons(x, [y])),  makelist(x, x, 1,
% 64),makelist(1 - signum(mod(x,6)), x, 1, 64) )), user_preamble="set
% output 'picshorquantfourier3.tex'; set terminal latex; set xlabel
% '$x$'; set ylabel '$f(x)$'"); 

% draw2d(yrange=[0, 1.2], points_joined = false, point_type = 6,
% points(map(lambda([x, y], cons(x, [y])),  makelist(x, x,
% 1,64),abs(fft(makelist(1 - signum(mod(x,6)), x, 1, 64))) )), 
% user_preamble="set output 'picshorquantfourier3.tex'; set terminal
% latex; set xlabel '$y$'; set ylabel '$F(y)$'");

\input ./part4/quantcomp/picshorquantfourier1.tex
\input ./part4/quantcomp/picshorquantfourier2.tex
\input ./part4/quantcomp/picshorquantfourier3.tex

\caption{Shor's algorithm. Finding the period of the function 
  $f\left(x, a\right) = a^x \mod{N}$ for $a=2$, $N = 21$ (top graph). 
  The function value 1 repeats with period $r=6$ (middle graph). 
  Local maxima of the Fourier transform of the middle graph 
  occur with period $\frac{M}{r} \approx 10.67$ (bottom graph). See
  example \ref{exPart4QuantCompShorQuantPeriodFinding}}
\label{picPart4QuantCompShorQuantPart}
\end{figure}