%% -*- coding:utf-8 -*- 
\section{Quantum Fourier Transform and Discrete Logarithm}
The discrete logarithm (see \autoref{AddDiscretLog}) is the basis for
a large number of modern cryptographic algorithms (see
\autoref{sec:add:dm:elgamal}, \autoref{sec:add:dm:dh}). At the same time, 
the method proposed by Shor for factoring integers can also be
applied to computing discrete logarithms, making it possible to break
the corresponding cryptographic algorithms.

Let's set the problem as follows: there is an expression
\[
b = a^x \mod p,
\]
in which the numbers $a, b$, and $p$ are given, and the number $x$ is
unknown and needs to be determined. By analogy with the application
of the quantum Fourier transform for factoring numbers (see
\autoref{sec:part4:algoshor:periodfind}), we must
construct some periodic function, the period of which will allow us to
determine the desired number $x$. We choose the analyzed function as
\begin{equation}
f\left(x_1, x_2\right) = b^{x_1}a^{x_2} = a^{x \cdot x_1} a^{x_2} \mod p
\label{eq:part4:quantcomp:discretlogfunc}
\end{equation}

As an example, we will consider the quantum analog of solving the problem from example
\ref{ex:dm:discretlog}:
\begin{example}
\emph{($ind_3{14} \mod{17}$)}
% maxima
% pts:makelist(makelist([x,y,mod(power_mod(17, 13,
% x)*power_mod(17,3,y),17)], x, 1, 25), y, 1, 25); 
%
% scene:points(pts);
\input ./part4/quantcomp/figdiscretlog0.tex
In our example, $p = 17$, $b=14$, and $a=3$. The function
\eqref{eq:part4:quantcomp:discretlogfunc} has the form
\[
f\left(x_1, x_2\right) = b^{x_1}a^{x_2} = 14^{x_1}3^{x_2}.
\]
and is shown in \autoref{fig:part4:quantcomp:dl0}.

Both $b=14$ and $a=3$ are generators of
$\left(\mathbb{Z}/17\mathbb{Z}\right)^\times$. Moreover, $3 \equiv 14^9
\mod 17$. The periods of the depicted function, as seen in
\autoref{fig:part4:quantcomp:dl0}, are the following numbers:
\begin{eqnarray}
t_1 \equiv 1 \mod 16,
\nonumber \\
t_2 \equiv 9 \mod 16
\end{eqnarray} 
\label{ex:part4:quantcomp:discretlog:periodfinding0}
\end{example}


By analogy with the factorization problem solution, a measurement
of this function is made. Suppose the result of the measurement is a number $c \in
\left(\mathbb{Z}/p\mathbb{Z}\right)^\times$. Since $a$
is a generating element (see def.
\ref{def:add:algebra:cyclic_group}) of the multiplicative group   
$\left(\mathbb{Z}/p\mathbb{Z}\right)^\times$ (see def.
\ref{def:add:algebra:mult_group}), $\exists x_0: c = a^{x_0}$. Thus,
taking into account \myref{addDiscretSmallFerma}{Fermat's little theorem} $a^{p-1}
\equiv 1 \mod p$ and therefore
\[
x_0 \equiv x x_1 + x_2 \mod q,
\] 
where $q = p - 1$.
From this expression it follows that
\[
x_2 \equiv x_0 - x x_1 \mod q.
\]
That is, if the function is periodic in the first argument:
\[
f(x_1 + t_1, x_2) = f(x_1,x_2),
\]
then it will also be periodic in the second argument
\[
f(x_1, x_2 + t_2) = f(x_1,x_2),
\]
with
\begin{equation}
t_2 \equiv x t_1 \mod q.
\label{eq:part4:quantcomp:discretlogeq}
\end{equation}

\subsection{Two-dimensional Fourier Transform and Period of the Function $f(x_1,
  x_2)$}
Our function to analyze will be the following:
\begin{equation}
\label{eq:part4:quantcomp:shordiscretlog:fprime}
f'\left(x_1, x_2\right) = 
\begin{cases}
1, & x x_1 + x_2 \equiv x_0 \mod q \\
0, & x x_1 + x_2 \not\equiv x_0 \mod q \\
\end{cases}
\end{equation}
\begin{example}
\emph{($ind_3{14} \mod{17}$)}
% maxima
% pts:makelist(makelist([x,y,mod(power_mod(17, 13,
% x)*power_mod(17,3,y),17)], x, 1, 25), y, 1, 25); 
%
% scene:points(pts);

\input ./part4/quantcomp/figdiscretlog1.tex
Continuing example \ref{ex:part4:quantcomp:discretlog:periodfinding0}
suppose that as a result of measuring the function $f$ we obtained the value
$f = 3$. That is, $f = a^{x_0} = 3^{x_0} \equiv 3 \mod 17$.  
As a result, only those values of $x_1, x_2$ remain that correspond
to the observed function value (see
\autoref{fig:part4:quantcomp:dl1}), i.e. $x x_1 + x_2 = x_0 \equiv 1
\mod 16$. Here, with fixed values $x, x_1$ and the number
of samples $M = q = 16$. 
\label{ex:part4:quantcomp:discretlog:periodfinding1}
\end{example}


For the Fourier image $\tilde{f'}$ we have 
\begin{eqnarray}
\tilde{f'}\left(j_1, j_2\right) = 
\frac{1}{M}\sum_{x_1 = 0}^{M-1}\sum_{x_2 = 0}^{M-1} 
f'\left(x_1, x_2\right)e^{-i \omega\left(x_1 j_1 + x_2j_2\right)},
\label{eq:part4:quantcomp:discretlog:ftq16_pre}
\end{eqnarray}
where $\omega = \frac{2 \pi}{M}$, $M$ is the number of samples. 

Consider first the case when $M = q$. In this case, there are
two options for $x_2$:  
\begin{enumerate}
\item $x_2 = x_0 - x x_1$, if $x_0 \ge x x_1$
\item $x_2 = x_0 + q - x x_1$, if $x_0 < x x_1$
\end{enumerate}
Thus,
\begin{eqnarray}
e^{-i \omega x_2 j_2} = e^{-i \omega\left(x_0 - x x_1 + q\right) j_2} = 
\nonumber \\
= e^{-i \omega\left(x_0 - x x_1\right) j_2 - i \omega q j_2} = 
e^{-i \omega\left(x_0 - x x_1\right) j_2},
\nonumber
\end{eqnarray}
i.e., both cases coincide and can be reduced to the first:
$x_2 = x_0 - x x_1$.

Therefore, continuing \eqref{eq:part4:quantcomp:discretlog:ftq16_pre} we get
\begin{eqnarray}
\tilde{f'}\left(j_1, j_2\right) = 
\frac{1}{M}\sum_{x_1 = 0}^{M-1}\sum_{x_2 = 0}^{M-1} 
f'\left(x_1, x_2\right)e^{-i \omega\left(x_1 j_1 + x_2j_2\right)} =
\nonumber \\
=
 \frac{1}{M}\sum_{x_1 = 0}^{M-1}
e^{-i \omega\left(x_1 j_1 + (x_0 - x
   x_1) j_2\right)} = 
\nonumber \\
= e^{-i \omega x_0 j_2}\frac{1}{M}\sum_{x_1 = 0}^{M-1}
e^{-i  \omega x_1 \left(j_1 - x j_2\right)} =
\frac{1}{M} e^{-i \omega x_0 j_2} 
\sum_{x_1 = 0}^{M-1} e^{-i  \omega x_1 \left(j_1 - x j_2\right)}.
\label{eq:part4:quantcomp:discretlog:ftq16}
\end{eqnarray}
In the expression
\eqref{eq:part4:quantcomp:discretlog:ftq16} $\tilde{f'}(j_1, j_2) =
e^{-i \omega x_0 j_2} \ne 0$, if 
\begin{equation}
j_1 \equiv x j_2 \mod M.
\label{eq:part4:quantcomp:discretlog:j1xj2}
\end{equation} 
Otherwise, using the formula for a geometric series 
\begin{eqnarray}
\tilde{f'}\left(j_1 \ne x j_2, j_2\right) = 
e^{-i \omega x_0 j_2}\frac{1}{M}
\sum_{x_1 = 0}^{M-1}e^{-i
  \omega x_1 \left(j_1 - x j_2\right)} = 
\nonumber \\
=
\frac{e^{-i \omega x_0 j_2}}{M} \frac{e^{-i
  \omega M \left(j_1 - x j_2\right)} - 1}{e^{-i
  \omega \left(j_1 - x j_2\right)} - 1} = 
\nonumber \\
=
 \frac{e^{-i \omega x_0 j_2}}{M} 
\frac{e^{-i \frac{2 \pi}{M} M \left(j_1 - x j_2\right)} - 1}{e^{-i
  \omega \left(j_1 - x j_2\right)} - 1} = 0.
\nonumber
\end{eqnarray} 
Thus, to determine the period it is necessary to find the coordinates $(j_1, j_2)$
of some maximum of the Fourier transform and use the expression 
\begin{equation}
x \equiv j_1 j_2^{-1} \mod M,
\label{eq:part4:quantcomp:discretlog:periodfourier}
\end{equation}
which follows from \eqref{eq:part4:quantcomp:discretlog:j1xj2}.

\begin{remark}[On zero divisors in $\mathbb{Z}/M\mathbb{Z}$]
If there exists a number $y$ such that 
\[
j_2 y \equiv 0 \mod M,
\]
then $j_2$ is called a zero divisor. 
It is also clear that 
\[
\gcd\left(j_2, M\right) \ne 1,
\]
thus from \autoref{sec:add:discretmath:mod:equationsolve}
it follows that $j_2^{-1}$ does not exist. Therefore, for such
$j_2$, the expression
\eqref{eq:part4:quantcomp:discretlog:periodfourier} is undefined. In
this case, other coordinates $(j_1, j_2)$ must be used. 
\end{remark}

\begin{example}
\emph{($ind_3{14} \mod{17}$)}
% maxima
% pts:makelist(makelist([x,y,mod(power_mod(17, 13,
% x)*power_mod(17,3,y),17)], x, 1, 25), y, 1, 25); 
%
% scene:points(pts);

\input ./part4/quantcomp/figdiscretlog2.tex

The Fourier image of the function from \autoref{fig:part4:quantcomp:dl1} is shown in
\autoref{fig:part4:quantcomp:dl2}, from which it is seen that with the highest
probability, samples recorded follow with intervals $T_{j_1} = 9$ along coordinate $j_1$ and with interval $T_{j_2} =
1$ along coordinate $j_2$. Given that the number of samples $M=16$, one can obtain
the maximum coordinates of the Fourier transform $j_1 = 9$
and $j_2 = 1$. The solution of the equation $3^x \equiv 14 \mod 17$
is, according to
\eqref{eq:part4:quantcomp:discretlog:periodfourier}, $x = 9 \cdot 1^{-1}
= 9$, which corresponds to the result of example 
\ref{ex:dm:discretlog}.

A similar result can be obtained by taking the point with coordinates 
$j_1 = 11, j_2 = 3$. Given that $3 \cdot 11 = 33 \equiv 1 \mod 16$,
we have
$j_2^{-1} \equiv 11 \mod 16$, i.e., $x \equiv 11 \cdot 11 \equiv 121
\equiv 9 \mod 16$, which again corresponds to the result of example 
\ref{ex:dm:discretlog}.

It is worth noting that points lying on the diagonal, for example $j_1 = 6, j_2
= 6$ will not give a correct result because $\gcd(6,
16) = 2 \ne 1$

\label{ex:part4:quantcomp:discretlog:periodfinding}
\end{example}

Note that the obtained result
\eqref{eq:part4:quantcomp:discretlog:periodfourier} directly
corresponds to lemma \ref{lemmaAddFourierDiscretFourierPeriod} for
the one-dimensional Fourier transform. At the same time, there is also an analog of
commentary \ref{rem:dsp:fourier:periodprop}, which states that in
case the number of samples in the Fourier transform is not equal to $q$: $M \ne
q$, but $M \approx q$, we can still approximately consider the expression 
\eqref{eq:part4:quantcomp:discretlog:periodfourier}
\cite{Proos:2003:SDL:2011528.2011531}.  


\begin{example}
\emph{($ind_2{14} \mod{59}$)}

As an example, consider $p = 59$ with number of samples $M = 64
\approx q = p - 1 = 58$. The group generator $\mathbb{F}_{59}$ (see
\autoref{sec:add:diskretmath:mod:fp}) 
is $g = 2$, i.e. $\mathbb{F}_{59} = \langle 2 \rangle$. This
means that the equation $2^x \equiv b \mod 59$ has a solution for
any $b$, in particular $x = 19$ is a solution of the equation
%% -- 2 is generator of F_59
%% -- Prelude> filter (\x -> x == 2 ) $ map (\x -> mod (2^x) 59) [1..59]
%% -- [2,2]
%% -- Prelude> mod (2^19) 59
%% -- 14
\[
2^x \equiv 14 \mod 59.
\] 

The function under study is
\[
f(x_1, x_2) = 14^{x_1} 2^{x_2} \mod 59,
\]
%% Prelude> mod (2^50) 59
%% 3
Suppose $x_0 = 50$, i.e., the recorded function value is
$f(x_1, x_2) = 2^{x_0} = 2^{50} \equiv 3 \mod 59$.

\input ./part4/quantcomp/figdiscretlog4.tex

As stated above, for the number of Fourier transform samples we have $M=64$. It should be noted that
$q = p - 1 = 58 \not\equiv 0 \mod 64$.

The Fourier image of the sampled function 
\[
f'(x_1, x_2) = 
\begin{cases}
1, & \text{if } 14^{x_1} 2^{x_2} \equiv 3 \mod 59 \\
0, & \text{if } 14^{x_1} 2^{x_2} \not\equiv 3 \mod 59 
\end{cases}
\]
is shown in \autoref{fig:part4:quantcomp:dl4}. 
The three lowest maxima have coordinates 
\[
(j_1, j_2) \approx (20,1), (41,2.2), (62,3), 
\]
which give the following estimates of $x$: $x \approx 20, 18.6, 20.6$,
which is close to the actual value $x = 19$.
\label{ex:part4:quantcomp:discretlog:periodfinding3}
\end{example}




\begin{example}
\emph{($ind_3{14} \mod{19}$)}
% maxima
% pts:makelist(makelist([x,y,mod(power_mod(17, 13,
% x)*power_mod(17,3,y),17)], x, 1, 25), y, 1, 25); 
%
% scene:points(pts);

As an example, consider the problem of determining $x$ such that 
\[
3^x \equiv 14 \mod 19.
\]

The function under study is
\[
f(x_1, x_2) = 14^{x_1} 3^{x_2} \mod 19,
\]
Suppose $x_0 = 1$, i.e., the recorded function value is
$f(x_1, x_2) = 3$.

\input ./part4/quantcomp/figdiscretlog3.tex

We take the number of Fourier transform samples $M=64$. Note that,
$18 \not\equiv 0 \mod 64$.

The Fourier image of the sampled function 
\[
f'(x_1, x_2) = 
\begin{cases}
1, & \text{if } 14^{x_1} 3^{x_2} \equiv 3 \mod 19 \\
0, & \text{if } 14^{x_1} 3^{x_2} \not\equiv 3 \mod 19 
\end{cases}
\]
is shown in \autoref{fig:part4:quantcomp:dl3}. The lowest maximum
has coordinates $j_1 = 46, j_2 = 3.5$, from which the estimate
\[
x \approx \frac{46}{3.5} \approx 13.14
\]
follows. It is worth noting that the solution of the desired equation $x = 13$ corresponds
to the found approximate solution.
\label{ex:part4:quantcomp:discretlog:periodfinding2}
\end{example}


\subsection{Two-dimensional Quantum Fourier Transform}

\input part4/quantcomp/figquantfourier2d.tex

To determine the periods of a two-argument function one can use
the two-dimensional Fourier transform, which can be constructed from
blocks performing the one-dimensional Fourier transform, as shown in \autoref{figQuantCompQuantFourier2d}. To analyze this
scheme, consider the trivial case when
at the input we have (see also \eqref{eqPart4QuantCompShorQuantFourierSeries})
\begin{eqnarray}
\ket{x} = \ket{x}_1 \otimes \ket{x}_2,
\nonumber \\
\ket{x}_{1,2} = \sum_{k_{1,2} = 0}^{M-1} x_{k_{1,2}}^{(1,2)} \ket{k_{1,2}}.
\nonumber
\end{eqnarray}
Considering that the output is
\[
\ket{\tilde{X}} = \ket{\tilde{X}_1} \otimes \ket{\tilde{X}_2},
\]
where
\[
\ket{\tilde{X}_{1,2}} = \sum_{j_{1,2} = 0}^{M-1} \tilde{X}_{j_{1,2}}^{(1,2)} \ket{j_{1,2}}
\]
and, according to \eqref{eqPart4QuantCompShorQuantFourierEnd}
\[
\tilde{X}_{j_{1,2}}^{(1,2)} = \frac{1}{\sqrt{M}}\sum_{k_{1,2} = 0}^{M - 1}e^{-i \omega_{1,2} k_{1,2} j_{1,2}} x_{k_{1,2}}^{(1,2)}.
\]
we get
\begin{eqnarray}
\ket{\tilde{X}} = \ket{\tilde{X}_1} \otimes \ket{\tilde{X}_2} = 
\nonumber \\
= \sum_{j_1 = 0}^{M-1}\sum_{j_2 = 0}^{M-1}
\tilde{X}_{j_{1}}^{(1)} \tilde{X}_{j_{2}}^{(2)} \ket{j_1} \otimes
\ket{j_2} =
\nonumber \\
= \sum_{j_1 = 0}^{M-1}\sum_{j_2 = 0}^{M-1}
\tilde{X}_{j_{1},j_{2}} \ket{j_1} \otimes
\ket{j_2}, 
\nonumber
\end{eqnarray}
where
\begin{eqnarray}
\tilde{X}_{j_{1},j_{2}} = \frac{1}{\left( \sqrt{M} \right)^2} 
\sum_{k_{1} = 0}^{M - 1}\sum_{k_{2} = 0}^{M - 1}
e^{-i \omega \left( k_{1} j_{1} + k_{2} j_{2}\right)}
x_{k_1}^{(1)}x_{k_2}^{(2)} =
\nonumber \\
= \frac{1}{\left( \sqrt{M} \right)^2}
\sum_{k_{1} = 0}^{M - 1}\sum_{k_{2} = 0}^{M - 1}
e^{-i \omega \left( k_{1} j_{1} + k_{2} j_{2}\right)}
x_{k_1, k_2}
\nonumber
\end{eqnarray}
which, according to definition \ref{def:add:dsp:fourier2d},
\rindex{two-dimensional Fourier transform}
is the two-dimensional Fourier transform of the original two-dimensional 
signal
\[
\ket{x} = 
\sum_{k_1 = 0}^{M-1}\sum_{k_2 = 0}^{M-1}
x_{k_1}^{(1)}x_{k_2}^{(2)} \ket{k_1} \otimes \ket{k_2} =
\sum_{k_1 = 0}^{M-1}\sum_{k_2 = 0}^{M-1}
x_{k_1,k_2} \ket{k_1} \otimes \ket{k_2}.  
\]

\input part4/quantcomp/figquantperiodfinding2.tex

Therefore, using the scheme shown in
\autoref{figQuantCompQuantPeriodFinding2}, one can determine the coordinates
of the maxima of the two-dimensional Fourier transform $j_1, j_2$, and subsequently use
\eqref{eq:part4:quantcomp:discretlog:periodfourier} to find the desired $x$.