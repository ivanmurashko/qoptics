%% -*- coding:utf-8 -*-
\section{Shor's Algorithm and the Discrete Logarithm on Elliptic Curves} 
Consider an elliptic curve 
\[
E\left(\mathbb{F}\right) = \{
(x,y) \in \mathbb{F}_p \times \mathbb{F}_p, y^2 = x^3 +a x + b,
\}
\]
with a given base point $g \in E\left(\mathbb{F}_p\right)$ such that: 
\[
n g = 0.
\]
The task is to solve the following problem: given $q
\in E\left(\mathbb{F}_p\right)$ find such $x$ that
\begin{equation}
x g = q \mod n
\label{eq:part4:quantcomp:discretlogeliptic}
\end{equation}

Consider the following auxiliary function
\begin{equation}
f(x_1, x_2) = x_1 q + x_2 g = \left(x x_1 + x_2\right) g,
\label{eq:part4:quantcomp:discretlogeliptic:f}
\end{equation}
where $q,g \in E\left(\mathbb{F}_p\right)$ and are taken from the conditions of our
problem \eqref{eq:part4:quantcomp:discretlogeliptic}. This function
is analogous to \eqref{eq:part4:quantcomp:discretlogfunc}
used in solving the discrete logarithm problem. Then
a measurement is performed on this function. The result of this measurement
is a certain point $c \in E\left(\mathbb{F}_p\right)$. At the same time,
from \eqref{eq:part4:quantcomp:discretlogeliptic:f} it follows that 
$c \in \left<g\right>$, i.e. $\exists x_0$ such that $c = x_0 g$. 

Thus, by analogy with \eqref{eq:part4:quantcomp:shordiscretlog:fprime}, we
compose the following function 
\begin{equation}
\label{eq:part4:quantcomp:shorelliptic:fprime}
f'\left(x_1, x_2\right) = 
\begin{cases}
1, x x_1 + x_2 \equiv x_0 \mod n \\
0, x x_1 + x_2 \not\equiv x_0 \mod n \\
\end{cases}
\end{equation}

Coordinates $(j_1,j_2)$ of the Fourier transform maximum $\tilde{f'}$ give, in
accordance with formula
\eqref{eq:part4:quantcomp:discretlog:periodfourier}, some value
of the sought number $x$. In our case, practically always $n \ne M$
so we can only use an approximate estimate
\[
x \approx \frac{j_1}{j_2}.
\]

\begin{example}[Discrete logarithm on an elliptic curve]
Consider the task from example \ref{ex:add:discretmath:ecdh}
The curve and base point (see example \ref{ex:add:elliptic:basepoint})
are given by
\[
(p,a,b,g,n,h) = (97, -7, 10, (96,93), 41, 2)
\]
Suppose we know Alice's public key
\[
A = (37, 35)
\]
and we want to find such $x \in \{0,1, \dots 40\}$ that
$x g = A$, as follows from example \ref{ex:add:discretmath:ecdh}
the answer is $x = d_a  = 5$.
Our investigated function will be
\[
f\left(x_1,x_2\right) = x_1 A + x_2 g = x_1 (37,35) + x_2 (96,93)
\]

\input ./part4/quantcomp/figellipticdiscretlog1.tex

As the measurement result, choose $c = g$, i.e. $x_0 = 1$.
The graph of the function $f'(x_1, x_2)$ corresponding to this measurement
is shown in \autoref{fig:part4:quantcomp:dle1}. It is worth noting that
the function $f'(x_1, x_2)$ is periodic, and if we take any two close
points, for example $(8,2)$ and $(16,3)$, we can notice that
the period along the $x_1$ coordinate is $T_1=8$, and along the $x_2$ coordinate it is $T_2
= 1$. Solving the equation $x = T_2 T_1^{-1} \mod n$ gives $x = 8^{-1}
 \equiv 5 \mod 41$, which corresponds to the sought solution.

\input ./part4/quantcomp/figellipticdiscretlog2.tex

The Fourier transform of the function $f'(x_1, x_2)$ is shown in
\autoref{fig:part4:quantcomp:dle2}. From it, the sought $x =
5$ can be found. 

\end{example}