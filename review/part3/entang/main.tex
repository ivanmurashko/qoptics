%% -*- coding:utf-8 -*- 
\chapter{Entangled States}
\label{ch:entangl}

Since the advent of quantum mechanics, the question of the completeness
of this theory has arisen. The accuracy of quantum mechanical predictions is very high, but
only probabilities of certain events are predicted. In particular,
it is impossible to measure the coordinate and
momentum of a particle with arbitrary precision. It seems that
the probabilities inherent in the quantum mechanical description reflect its
incompleteness, and perhaps there exists another theory that would possess
the accuracy of quantum mechanics and at the same time would not use
a probabilistic approach.

It turned out that the probabilities underlying quantum mechanics have
a deep physical meaning and there are no theories in which
they can be discarded, and in which, for example,
it would be possible to measure coordinate and momentum with arbitrary precision.
Entangled states play a special role in this context. They describe systems consisting of several particles,
while the behavior of such a composite system is described by a common wave
function.

Because entangled states are purely quantum,
i.e., have no classical analogues, it is possible using them
to observe phenomena that seem completely
impossible from a classical point of view, such as quantum
teleportation. In addition, recently practical
applications of entangled states have appeared, such as quantum dense
coding 
(see \ref{subsecPart3QuantInfoBigCoding})
and quantum cryptography
(see \ref{subsecPart3QuantInfoQuantCrypto}).

There are several ways to obtain entangled photons
\cite{bPhisQuantInfo}, i.e., related to quantum optics,
among which polarization-entangled
states should be highlighted. This is because there are many
methods for controlling polarization characteristics, as well as
methods for measuring the polarization properties of light.

% It so happened that the simplest methods of obtaining entangled
% states are related to quantum optics. One such method
% is the polarization-entangled state, which will be
% considered further.

%\input ./part3/entang/epr.tex

\input ./part3/entang/entang.tex

\input ./part3/entang/bell.tex

\input ./part3/entang/bellbase.tex

\input ./part3/entang/gener.tex

\input ./part3/entang/reg.tex

\input ./part3/entang/teleport.tex

\input ./part3/entang/quantumgame.tex

\section{Exercises}
\begin{enumerate}
\item Prove the orthonormality of the Bell basis states
  \eqref{eqEntangBellBase}. 
\item Which Bell state should Alice register, in the scheme
  shown in \autoref{figTeleport}, to
  confirm the fact of teleportation in the case when the source
  of entangled photon pairs $S$ produces photons in the state 
\begin{equation}
  \left|\psi\right>_{23} = \left|\psi^{\dag}\right>_{23} = \frac{1}{\sqrt{2}}\left(
  \ket{x}_2\ket{y}_3 +
  \ket{y}_2\ket{x}_3
  \right).
  \nonumber
\end{equation}
\item Obtain expressions for the coefficients 
$c_{\left|\psi^{\dag}\right>_{12}}$, 
$c_{\left|\phi^{\dag}\right>_{12}}$, and 
$c_{\left|\phi^{-}\right>_{12}}$
in the decomposition \eqref{eqPart3EntangTeleportsepar}.
\end{enumerate}