%% -*- coding:utf-8 -*- 
\section{EPR Paradox for Stokes Parameters and Entangled States}

Consider the states with left and right circular polarizations:
\begin{eqnarray}
  \ket{ + } = \frac{1}{\sqrt{2}}
  \left(
  \ket{x} + i \ket{y}
  \right),
  \nonumber \\
  \ket{ - } = \frac{1}{\sqrt{2}}
  \left(
  \ket{x} - i \ket{y}
  \right).
  \nonumber
\end{eqnarray}

Let us take the following entangled state of two photons
\begin{eqnarray}
  \left|\psi\right> = \frac{
    \ket{ + }_1\ket{ - }_2 -
    \ket{ - }_1\ket{ + }_2
  }{\sqrt{2}}.
  \nonumber
\end{eqnarray}

\rindex{photon}If the measurement of the linear polarization of the 1st photon gives
$\ket{x}_1$, i.e., when measuring the Stokes parameters
$\hat{S}_1^{(1)}$ we get the value $+1$ for the first photon, then for
the second photon we have
\begin{eqnarray}
  P_{\ket{x}_1}\left|\psi\right> =
  \ket{x}_1\bra{x}_1 \left|\psi\right> =
  \nonumber \\
  =
  \ket{x}_1\bra{x}_1
  \frac{
    \left( \ket{x}_1 + i \ket{y}_1 \right)\ket{ - }_2 -
    \left( \ket{x}_1 - i \ket{y}_1 \right)\ket{ + }_2
  }{2} =
  \nonumber \\
  =
  \ket{x}_1
  \frac{\ket{ - }_2 - \ket{ + }_2}{2} =
  \frac{1}{2\sqrt{2}}\ket{x}_1 \left(2 i\right)
  \ket{y}_2 =
  \frac{i}{\sqrt{2}}\ket{x}_1\ket{y}_2,
  \nonumber
\end{eqnarray}
i.e., for the second photon the value of the Stokes parameter 
$\hat{S}_1^{(2)}$ is $-1$.

Similarly, if upon measuring the Stokes parameter $\hat{S}_1^{(1)}$
of the first photon we get $-1$, then for the second photon we obtain
\begin{eqnarray}
  P_{\ket{y}_1}\left|\psi\right> =
  \ket{y}_1\bra{y}_1 \left|\psi\right> =
  \nonumber \\
  =
  \ket{y}_1\bra{y}_1
  \frac{
    \left( \ket{x}_1 + i \ket{y}_1 \right)\ket{ - }_2 -
    \left( \ket{x}_1 - i \ket{y}_1 \right)\ket{ + }_2
  }{2} =
  \nonumber \\
  =
  \ket{y}_1 i 
  \frac{\ket{ - }_2 + \ket{ + }_2}{2} =
  \frac{i}{2\sqrt{2}}\ket{y}_1 \left(2\right)
  \ket{x}_2 =
  \frac{i}{\sqrt{2}}\ket{y}_1\ket{x}_2,
  \nonumber
\end{eqnarray}
i.e., for the second photon the value of the Stokes parameter 
$\hat{S}_1^{(2)}$ is $+1$.

At the same time, the Stokes parameters for the second particle  $\hat{S}_1^{(2)}$ and
$\hat{S}_2^{(2)}$ have 
different eigenvectors:
$\ket{x}_2, \ket{y}_2$ (\ref{eq:part2:pol:stocks_s1_1,
  eq:part2:pol:stocks_s1_2}) and
$\frac{1}{\sqrt{2}}\left(\ket{x} \pm
\ket{y}\right)$ (\ref{eq:part2:pol:stocks_s1_2,
  eq:part2:pol:stocks_s2_2}) respectively.
%% Meanwhile, in the considered state, the mean value of the Stokes parameter
%% $\left<\hat{S}_3^{(2)}\right>$ has the form
%% \begin{eqnarray}
%%   \left<\hat{S}_3^{(2)}\right> =
%%   \left<\psi\right|\hat{S}_3^{(2)}\left|\psi\right> =
%%   \left<\psi\right|\hat{S}_3^{(2)}\frac{
%%     \ket{ + }_1\ket{ - }_2 -
%%     \ket{ - }_1\ket{ + }_2
%%   }{\sqrt{2}} =
%%   \nonumber \\
%%   = \left<\psi\right|\frac{
%%     \ket{ + }_1\ket{ - }_2 +
%%     \ket{ - }_1\ket{ + }_2
%%   }{\sqrt{2}} =
%%   \nonumber \\
%%   = \frac{1}{2} \left(
%%   \bra{ + }_1\bra{ - }_2 -
%%   \bra{ - }_1\bra{ + }_2
%%   \right)
%%   \left(
%%   \ket{ + }_1\ket{ - }_2 -
%%   \ket{ - }_1\ket{ + }_2
%%   \right) = 1.
%%   \label{eqEntangS3Mean}
%% \end{eqnarray}

%% In the derivation of \eqref{eqEntangS3Mean}, expressions
%% \eqref{eqEntangS3Eigenvec} were used. Thus, from the Heisenberg inequality
%% \eqref{eqAddHeisenbergUncertaintyPrinciple} one can obtain that
%% \[
%% \Delta s_1^{(2)} \Delta s_2^{(2)} \ge 1,
%% \]
Thus, the first and second Stokes parameters for the second particle cannot
be measured simultaneously; indeed, if they could be
measured simultaneously, then the measured value would correspond to
some state vector that would be an eigenvector for both
operators $\hat{S}_1^{(2)}$ and
$\hat{S}_2^{(2)}$.