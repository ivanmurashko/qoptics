%% -*- coding:utf-8 -*- 
\section{Registration of Bell States}
\label{pPart3EntangleBellReg}
\rindex{Bell basis states!registration}
Of the four basis states \eqref{eqEntangBellBase}, only 
$\left|\psi^{-}\right>_{12}$ is antisymmetric, i.e., it changes
its sign upon particle (photon) permutation. This allows us to
distinguish this state from all others. For this purpose, the scheme
shown in \autoref{figBellReg} can be used, where two photons
will be detected in different photodetectors only for the
antisymmetric state; for all other states, both photons are detected
either in $D^{(1)}$ or $D^{(2)}$.

\input ./part3/entang/figreg.tex
 
To prove this, let us consider how a beamsplitter acts on a single-photon
state. A photon can impinge on the beamsplitter from half-plane
$\left(1\right)$ (\autoref{figBellLS}); in this case, with probability
$0.5$ the photon ends up in half-plane $\left(1\right)$ and with probability
$0.5$ in $\left(2\right)$. If the photon comes from half-plane
$\left(2\right)$, it again can end up in either half-plane with equal probability. 
Thus, to describe the beamsplitter we will use the scattering matrix
expression in the form \eqref{eqPart2Interfero4b}, where
$r=t=\frac{1}{\sqrt{2}}$, i.e., the action of the beamsplitter can be
expressed by the following operator 
\rindex{Hadamard transformation}
(Hadamard\cite{bPhisQuantInfo}):
\begin{eqnarray}
\hat{H} \ket{1} = \frac{1}{\sqrt{2}}
\left(\ket{1'} +
\ket{2'}\right),
\nonumber \\
\hat{H} \ket{2} = \frac{1}{\sqrt{2}}\left(\ket{1'} -
\ket{2'}\right).
\label{eqEntangBellHadamar}
\end{eqnarray}

In our case, two photons fall on the beamsplitter from different sides,
so only two wavefunctions are possible, describing the spatial part of
our initial state:
\begin{eqnarray}
\ket{S}_{12} = \frac{1}{\sqrt{2}}\left(
\ket{1}_1\ket{2}_2 +
\ket{2}_1\ket{1}_2\right),
\nonumber \\
\ket{A}_{12} = \frac{1}{\sqrt{2}}\left(
\ket{1}_1\ket{2}_2 -
\ket{2}_1\ket{1}_2\right).
\label{eqEntangBellSpaceInit}
\end{eqnarray}
As seen from \eqref{eqEntangBellSpaceInit}, the state $\ket{S}_{12}$
is symmetric, i.e., it remains unchanged under photon permutation, while
the state $\ket{A}_{12}$ is antisymmetric (changes sign upon photon permutation).

\input ./part3/entang/figls.tex


Let us consider how the states \eqref{eqEntangBellSpaceInit}
transform upon passing through the beamsplitter. Since the operator
$\hat{H}$ acts independently on each photon, for
$\ket{S}_{12}$ we have 
\begin{equation}
\hat{H}\ket{S}_{12} = \ket{S'}_{12} = 
\frac{1}{\sqrt{2}}
\left(
\hat{H}\ket{1}_1\hat{H}\ket{2}_2 +
\hat{H}\ket{2}_1\hat{H}\ket{1}_2
\right),
\nonumber 
\end{equation}
from which, using \eqref{eqEntangBellHadamar}, we obtain
\begin{eqnarray}
\hat{H}\ket{S}_{12} =
\frac{1}{\sqrt{2}}
\left(
\frac{1}{2}
\left(\ket{1'}_1 +
\ket{2'}_1\right)
\left(\ket{1'}_2 -
\ket{2'}_2\right) +
\right.
\nonumber \\
+ \left.
\frac{1}{2}
\left(\ket{1'}_1 -
\ket{2'}_1\right)
\left(\ket{1'}_2 +
\ket{2'}_2\right)
\right) = 
\nonumber \\
=
\frac{1}{2 \sqrt{2}}
\left(
\ket{1'}_1 \ket{1'}_2 +
\ket{2'}_1 \ket{1'}_2 -
\ket{1'}_1 \ket{2'}_2 -
\ket{2'}_1 \ket{2'}_2 +
\right. 
\nonumber \\
+ \left.
\ket{1'}_1 \ket{1'}_2 -
\ket{2'}_1 \ket{1'}_2 +
\ket{1'}_1 \ket{2'}_2 -
\ket{2'}_1 \ket{2'}_2
\right) =
\nonumber \\
=
\frac{1}{2 \sqrt{2}}
\left(
\ket{1'}_1 \ket{1'}_2 
- \ket{2'}_1 \ket{2'}_2 
+ \ket{1'}_1 \ket{1'}_2 
- \ket{2'}_1 \ket{2'}_2
\right) = 
\nonumber \\
\frac{1}{\sqrt{2}}
\left(
\ket{1'}_1 \ket{1'}_2 
- \ket{2'}_1 \ket{2'}_2 
\right).
\nonumber
\end{eqnarray}
Thus, for the state $\ket{S'}_{12}$ both photons appear
either in half-plane $\left(1\right)$ or in half-plane
$\left(2\right)$. 

For the antisymmetric state, we have
\begin{eqnarray}
\hat{H}\ket{A}_{12} = \ket{A'}_{12} = 
\frac{1}{\sqrt{2}}
\left(
\hat{H}\ket{1}_1\hat{H}\ket{2}_2 -
\hat{H}\ket{2}_1\hat{H}\ket{1}_2
\right) = 
\nonumber \\
=
\frac{1}{\sqrt{2}}
\left(
\frac{1}{2}
\left(\ket{1'}_1 +
\ket{2'}_1\right)
\left(\ket{1'}_2 -
\ket{2'}_2\right) -
\right.
\nonumber \\
- \left.
\frac{1}{2}
\left(\ket{1'}_1 -
\ket{2'}_1\right)
\left(\ket{1'}_2 +
\ket{2'}_2\right)
\right) = 
\nonumber \\
=
\frac{1}{2 \sqrt{2}}
\left(
\ket{1'}_1 \ket{1'}_2 +
\ket{2'}_1 \ket{1'}_2 -
\ket{1'}_1 \ket{2'}_2 -
\ket{2'}_1 \ket{2'}_2 -
\right. 
\nonumber \\
- \left.
\ket{1'}_1 \ket{1'}_2 +
\ket{2'}_1 \ket{1'}_2 -
\ket{1'}_1 \ket{2'}_2 +
\ket{2'}_1 \ket{2'}_2
\right) =
\nonumber \\
=
\frac{1}{2 \sqrt{2}}
\left(
\ket{2'}_1 \ket{1'}_2 
- \ket{1'}_1 \ket{2'}_2 
+ \ket{2'}_1 \ket{1'}_2 
- \ket{1'}_1 \ket{2'}_2
\right) = 
\nonumber \\
- \frac{1}{\sqrt{2}}
\left(
\ket{1'}_1 \ket{2'}_2 
- \ket{2'}_1 \ket{1'}_2 
\right).
\nonumber
\end{eqnarray}
Thus, in the case of the antisymmetric state, after passing
through the beamsplitter, the photons will be in different half-planes.

Since photons are bosons, 
\rindex{boson}
\rindex{photon}
the total wavefunction
describing both polarization and spatial properties
must be symmetric \cite{bFeinman}
(see \autoref{AddFermionBoson}). Thus, from the combination of 
\eqref{eqEntangBellBase} and \eqref{eqEntangBellSpaceInit}, only the
following combinations are possible:
\begin{eqnarray}
\left|\psi^{\dag}\right>_{12}\ket{S}_{12},
\nonumber \\ 
\left|\psi^{-}\right>_{12}\ket{A}_{12}, 
\nonumber \\ 
\left|\phi^{\dag}\right>_{12}\ket{S}_{12}, 
\nonumber \\ 
\left|\phi^{-}\right>_{12}\ket{S}_{12}.
\nonumber
\end{eqnarray}
Consequently, for Bell states 
$\left|\psi^{\dag}\right>_{12}$, $\left|\phi^{\dag}\right>_{12}$, and
$\left|\phi^{-}\right>_{12}$, both photons after passing through
the beamsplitter will hit the same photodetector, while for
$\left|\psi^{-}\right>_{12}$ they will go to different ones, which
allows us to uniquely distinguish the state $\left|\psi^{-}\right>_{12}$
from all others. 

\input ./part3/entang/figreg2.tex

The scheme in \autoref{figBellReg} can be modified so that it
can also register the state
$\left|\psi^{\dag}\right>_{12}$. To do this, note that among all
symmetric Bell states only in this one do the photons have different
polarizations. Thus, this state can be singled out by measuring the photon polarizations.

\autoref{figBellReg2} shows a scheme that allows
registration of $\left|\psi^{-}\right>_{12}$ and
$\left|\psi^{\dag}\right>_{12}$. For the state
$\left|\psi^{\dag}\right>_{12}$, photodetectors 
$D^{(1)}_x$ and $D^{(1)}_y$ or $D^{(2)}_x$ and $D^{(2)}_y$ will fire simultaneously.
For $\left|\psi^{-}\right>_{12}$, it is $D^{(1)}_x$ and $D^{(2)}_y$ or
$D^{(2)}_x$ and $D^{(1)}_y$. For $\left|\phi^{\dag}\right>_{12}$ and
$\left|\phi^{-}\right>_{12}$, both photons will be detected
simultaneously in one of the 4 photodetectors. Recently,
works have appeared \cite{bKulik} where registration of
all four Bell states has been performed.