%% -*- coding:utf-8 -*- 
\section{Squeezed Quadrature State}
First, let's consider the theoretical possibility of creating squeezed
quadrature states. Consider constructing a squeezed state from
a coherent state, in which one of the quadrature
components will be squeezed. The state transformation is performed using
a certain unitary transformation.

Consider the unitary operator known as the squeezing operator
(the reason for this name will become clear later)
\begin{equation}
\hat{S}\left(z\right) = e^{\frac{1}{2}z^{*}\hat{a}^2 -
\frac{1}{2}z\hat{a}^{+2}},
\nonumber
\end{equation}
where $z = r e^{i\theta}$ is an arbitrary complex number.
The conjugate operator has the form
\begin{equation}
\hat{S}^{\dag}\left(z\right) = e^{\frac{1}{2}z\hat{a}^{+2} -
\frac{1}{2}z^{*}\hat{a}^{2}}, 
\nonumber
\end{equation}
from which it follows that
\begin{equation}
\hat{S} \hat{S}^{\dag} = \hat{S}^{\dag} \hat{S} = \hat{I},
\nonumber
\end{equation}
where $\hat{I}$ is the identity operator. Thus, $\hat{S}^{\dag} =
\hat{S}^{-1}$ is indeed a unitary operator.

Consider the action of the operator $\hat{S}\left(z\right)$ on the operators
$\hat{a}$ and $\hat{a}^{\dag}$ and on the coherent state. For this purpose,
we apply the operator decomposition theorem (see detailed derivation
in \myref{thm:addoperatorequality}{the theorem on operator identity}):  
\begin{equation}
e^{\hat{A}}\hat{B}e^{-\hat{A}} = 
\hat{B} + \left[\hat{A},\hat{B}\right] + 
\frac{1}{2!} \left[\hat{A},\left[\hat{A},\hat{B}\right]\right] + \dots
\label{eqPart3Squeezed15}
\end{equation}
Using \eqref{eqPart3Squeezed15}, it is easy to show, by taking
\[
e^{\hat{A}} = \hat{S}\left(z\right), \quad \hat{B} = \hat{a},
\]
that
\begin{eqnarray}
\hat{A} = \hat{S}\left(z\right)\hat{a}\hat{S}^{\dag}\left(z\right) =
\hat{a} + z \hat{a}^{\dag} + \frac{\left|z\right|^2\hat{a}}{2!} +
\frac{z\left|z\right|^2\hat{a}^{\dag}}{3!} + \dots = 
\nonumber \\
=\hat{a} \cosh r + \hat{a}^{\dag} e^{i\theta} \sinh r = 
\mu \hat{a} + \nu \hat{a}^{\dag},
\label{eqPart3Squeezed16}
\end{eqnarray}
where $\mu = \cosh r$, $\nu = e^{i\theta} \sinh r$, $\left|\mu\right|^2 -
\left|\nu\right|^2  = 1$.
Similarly, it can be shown that for the reverse order of operators
$\hat{S}^{\dag} \dots \hat{S}$ the formula differs from \eqref{eqPart3Squeezed16} only by the sign between
the terms:
\begin{eqnarray}
\hat{S}^{\dag}\left(z\right)\hat{a}\hat{S}\left(z\right) 
=\hat{a} \cosh r - \hat{a}^{\dag} e^{i\theta} \sinh r,
\nonumber \\
\hat{S}^{\dag}\left(z\right)\hat{a}^{\dag}\hat{S}\left(z\right) 
=\hat{a}^{\dag} \cosh r - \hat{a} e^{-i\theta} \sinh r.
\label{eqPart3Squeezed16a}
\end{eqnarray}

Let's act with the operator $\hat{S}$ on the coherent state vector. Then
we obtain a new state:
\begin{equation}
\left|\alpha, z\right> = \hat{S}\left(z\right)\left|\alpha\right>.
\label{eqPart3Squeezed17}
\end{equation}
Now show that the state $\left|\alpha, z\right>$ is an eigenstate of the operator $\hat{A}$:
\begin{eqnarray}
\hat{A}\left|\alpha, z\right> = 
\hat{S}\left(z\right)\hat{a}\hat{S}^{\dag}\left(z\right)\hat{S}\left(z\right)\left|\alpha\right>
= 
\hat{S}\left(z\right)\hat{a}\left|\alpha\right> = 
\nonumber \\
= \alpha \hat{S}\left(z\right)\left|\alpha\right> = 
\alpha \left|\alpha, z\right>.
\label{eqPart3Squeezed18}
\end{eqnarray}
Here the definition of $\hat{A}$ and $\hat{S}\hat{S}^{\dag} = \hat{I}$ were used.
From \eqref{eqPart3Squeezed18} it follows that the state $\left|\alpha, z\right>$ is
an eigenstate of the operator $\hat{A}$, and the eigenvalue coincides with
the eigenvalue of the coherent state from which
the state $\left|\alpha, z\right>$ is obtained.
The conjugate equality is
\begin{equation}
\left<\alpha, z\right|\hat{A}^{\dag} = 
\alpha^{*}\left<\alpha, z\right|
\label{eqPart3Squeezed19}
\end{equation}
Based on relations \eqref{eqPart3Squeezed18} and
\eqref{eqPart3Squeezed19}, by analogy with the operators $\hat{a}$ and
$\hat{a}^{\dag}$, $\hat{A}$ and $\hat{A}^{\dag}$ are called quasiparticle creation and quasiparticle annihilation operators.

It remains to determine whether the state $\left|\alpha, z\right>$
is a squeezed state. To this end, the uncertainty relation
for $\Delta X_1$ and $\Delta X_2$ should be written. We have
\begin{equation}
\left(\Delta X_{1,2}\right)^2 = 
\left<\alpha, z\right| \hat{X}_{1,2}^2\left|\alpha, z\right> -
\left<\alpha, z\right| \hat{X}_{1,2}\left|\alpha, z\right>^2.
\nonumber
\end{equation}
This expression, using \eqref{eqPart3Squeezed17}, can be rewritten as
\begin{equation}
\left(\Delta X_{1,2}\right)^2 = 
\left<\alpha\right|\hat{S}^{\dag}\left(z\right) \hat{X}_{1,2}^2\hat{S}\left(z\right)\left|\alpha\right> -
\left<\alpha\right|\hat{S}^{\dag}\left(z\right) \hat{X}_{1,2}\hat{S}\left(z\right)\left|\alpha\right>^2.
\label{eqPart3Squeezed20}
\end{equation}
The operators $\hat{X}_{1,2}$ can be expressed through $\hat{a}$ and $\hat{a}^{\dag}$:
\begin{eqnarray}
\hat{X}_1 = \frac{1}{2}\left(\hat{a} + \hat{a}^{\dag}\right), 
\nonumber \\
\hat{X}_2 = \frac{1}{2 i}\left(\hat{a} - \hat{a}^{\dag}\right).
\nonumber
\end{eqnarray}
Substituting this into \eqref{eqPart3Squeezed20} and transforming the operators
$\hat{a}$ and $\hat{a}^{\dag}$ using \eqref{eqPart3Squeezed16a},
we obtain the final expressions for $\Delta X_1$ and $\Delta X_2$:
(under the condition $z = r$, i.e. $\theta = 0$)
%(see detailed derivation in appendix FIXME!!! add it):
\begin{eqnarray}
\left(\Delta X_1\right)^2 = \frac{1}{4}e^{-2 r},
\nonumber \\
\left(\Delta X_2\right)^2 = \frac{1}{4}e^{2 r},
\nonumber \\
\left(\Delta X_1 \Delta X_2\right) = \frac{1}{4},
\label{eqPart3Squeezed21}
\end{eqnarray}
where $r = \left|z\right|$ is the squeezing parameter. From
\eqref{eqPart3Squeezed21} it follows that the state $\left|\alpha,
z\right>$ is indeed a squeezed state for one of the
quadrature components. Since the uncertainty product has
the minimum value, the state is an ideally squeezed
state. We will call this state the ideally squeezed
quadrature state.

If the operator $\hat{S}\left(z\right)$ acts on the vacuum
state ($\alpha = 0$), we obtain the squeezed vacuum state (squeezed
vacuum)
\begin{equation}
\hat{S}\left(z\right)\ket{0} = \ket{z, 0}.
\nonumber
\end{equation}

The average photon number in the quadrature squeezed coherent state
is given by the expression:
\begin{eqnarray}
\left<\alpha, z\right|\hat{a}^{\dag}\hat{a}\left|\alpha, z\right> =
\left<\alpha\right|\hat{S}^{\dag}\hat{a}^{\dag}\hat{a}\hat{S}\left|\alpha\right>
=
\nonumber \\
=
\left<\alpha\right|\hat{S}^{\dag}\hat{a}^{\dag}\hat{S}\hat{S}^{\dag}\hat{a}\hat{S}\left|\alpha\right>
= 
\nonumber \\
=
\left<\alpha\right|
\left(\hat{a}^{\dag} \cosh r - \hat{a}e^{-i\theta}\sinh r \right)
\left(\hat{a} \cosh r - \hat{a}^{\dag} e^{i\theta}\sinh r \right)
\left|\alpha\right> = 
\nonumber \\
=
\left(
\left|\alpha\right|^2\left(\cosh^2 r + \sinh^2 r\right) -
\left(\alpha^{*}\right)^2
e^{i\theta} \sinh r \cosh r - 
\alpha^2 \sinh r \cosh r e^{- i\theta} + \sinh^2 r
\right).
\label{eqPart3Squeezed23}
\end{eqnarray}
In deriving \eqref{eqPart3Squeezed23} we used relation
\eqref{eqPart3Squeezed16a}.

\rindex{Squeezed vacuum state}
\rindex{squeezed vacuum}
For the squeezed vacuum state, $\alpha = 0$. Then the average
photon number in this state is
\begin{equation}
\bra{0, z}\hat{a}^{\dag}\hat{a}\ket{0, z} =
\sinh^2 r \ne 0,
\label{eqPart3Squeezed24}
\end{equation}
where $r$ is the squeezing parameter.

Obviously, the squeezed vacuum state is not a true
vacuum state, since it can be detected by a photodetector
using the photoelectric effect, which is impossible for a true vacuum state.

\input ./part3/squeezed/fig1.tex
\input ./part3/squeezed/fig2.tex
\input ./part3/squeezed/fig3.tex
%\input ./part3/squeezed/fig4.tex
%\input ./part3/squeezed/fig5.tex

Visually, quadrature squeezed states are shown in Figures
\ref{figPart3Squeezed_1}-\ref{figPart3Squeezed_3}, where initial states and
time evolution of oscillations in various squeezed states are presented.
The graphs on the right depict the uncertainty regions, which have the shape of ellipses.
Such an image could be seen if we had something like an optical
stroboscopic oscilloscope, which in reality does not exist.

Let's examine the obtained states in more detail. The formula
\eqref{eqPart3Squeezed17} can be written somewhat differently by
using the representation of the coherent state via the vacuum state
in the form \eqref{eqCh1_astate4squeezed}:
\begin{equation}
\left|\alpha\right> =  
e^{\alpha \hat{a}^{\dag} - \alpha^{*} \hat{a}}\ket{0} = 
\hat{D}\ket{0},
\nonumber
\end{equation}
where $\hat{D} = e^{\alpha \hat{a}^{\dag} - \alpha^{*} \hat{a}}$ is called
the displacement operator. $\hat{D}$ is a unitary operator, since 
\[
\hat{D}^{\dag} = e^{-\left(\alpha \hat{a}^{\dag} - \alpha^{*} \hat{a}\right)},
\]
from which
\[
\hat{D}^{\dag} \hat{D} = \hat{D} \hat{D}^{\dag} = \hat{I},
\]
i.e. $\hat{D}^{\dag} = \hat{D}^{-1}$.

The name of the operator is related to the fact that the action of
$\hat{D}\left(\alpha\right)$ on the operators $\hat{a}$ and $\hat{a}^{\dag}$
leads to their displacement by the amount $\alpha$ ($\alpha^{*}$):
\begin{eqnarray}
\hat{D}^{\dag}\left(\alpha\right)\hat{a}\hat{D}\left(\alpha\right) =
\hat{a} + \alpha,
\nonumber \\
\hat{D}^{\dag}\left(\alpha\right)\hat{a}^{\dag}\hat{D}\left(\alpha\right) =
\hat{a}^{\dag} + \alpha^{*},
\label{eqPart3SqueezedTaskOffset}
\end{eqnarray}
This result follows from the operator decomposition theorem
\eqref{eqPart3Squeezed15} when taking $\hat{A} = - \alpha \hat{a}^{\dag}
+ \alpha^{*} \hat{a}$, $B=\hat{a}, \hat{a}^{\dag}$.

\input ./part3/squeezed/fig6.tex

Thus, the coherent state $\left|\alpha\right>$ is
a vacuum state displaced by $\alpha$
(see \autoref{figPart3Squeezed_6}).
%where the shaded area shows the coherence area). 
Based on the above, a squeezed
state can be considered as a result of two actions: displacement of the vacuum state and
its subsequent squeezing, as shown in
\autoref{figPart3Squeezed_7}. 

\input ./part3/squeezed/fig7.tex
\input ./part3/squeezed/fig8.tex

The squeezing procedure considered here is not unique. One can
apply another sequence of operations – first squeeze the vacuum
state, then displace it to obtain a new squeezed state
(see \autoref{figPart3Squeezed_8}). In operator form, this operation can be written as:
\begin{equation}
\left|\alpha, z\right> =
\hat{D}\left(\alpha\right)\hat{S}\left(z\right) \ket{0}.
\label{eqPart3Squeezed28}
\end{equation}
To distinguish the states, the parameter order here is reversed compared to 
\[
\left|z,\alpha\right> =
\hat{S}\left(z\right) \hat{D}\left(\alpha\right)\ket{0}.
\]
The state \eqref{eqPart3Squeezed28} also corresponds to a quadrature squeezed state, but they are not entirely identical.
We will not consider this case in detail here.

\input ./part3/squeezed/fig9.tex

The above results and illustrations shown in
\autoref{figPart3Squeezed_1}-\ref{figPart3Squeezed_3} were obtained for the
case when parameters $z = r e^{i\theta}$ and $\alpha =
\left|\alpha\right|e^{i \varphi}$ are real ($\varphi = \theta =
0$). If the parameters are complex, the picture changes somewhat
(see \autoref{figPart3Squeezed_9}): the squeezing axis is rotated by an angle $\frac{\theta}{2}$
relative to the $X_1$, $X_2$ axes, and the position of the uncertainty region center is rotated by an angle $\varphi$ relative to
$X_1$. By changing the phase of the light beams used in the generation and detection process, we can move to new quadratures $Y_1$, $Y_2$ as
shown in \autoref{figPart3Squeezed_9}. If we further compensate the phase shift
$\varphi$, we return to the situation depicted in Figures 
\ref{figPart3Squeezed_1}-\ref{figPart3Squeezed_3}. From
\autoref{figPart3Squeezed_9}, it is clear that by measuring the angles $\theta$ and
$\varphi$ it is possible, for example, to transform a state squeezed in $X_1$ into a
state squeezed in $X_2$, and vice versa.