%% -*- coding:utf-8 -*- 
\section{Interference Measurements Using Squeezed Light}
Consider the Mach-Zehnder interferometer scheme shown in
\autoref{figPart3Squeezed_14}.
This scheme has been previously examined during the study of
interference measurement errors. The difference is that at the zero
input, squeezed vacuum radiation is fed instead of vacuum radiation.
We still assume that the input and output mirrors are partially transparent ($t
=r =\frac{1}{\sqrt{2}}$), and the angular mirrors are opaque ($r = 1$, $t =
0$). The scattering matrix of the mirrors is 
\begin{equation}
\hat{S} = \frac{1}{\sqrt{2}}\left(
\begin{array}{cc}
1 & i \\
i & 1 \\
\end{array}
\right).
\nonumber
\end{equation}
The interferometer is considered as a sensor of some physical quantity
affecting the optical length of one arm. On the scheme,
the phase shifter included in the upper arm can respond to the
action of this quantity.

\input ./part3/squeezed/fig14.tex

The interferometer equations are given by \eqref{eqPart2Interfero11}:
\begin{eqnarray}
\hat{a}_2 = \frac{1}{\sqrt{2}} \left(\hat{a}_0 + i \hat{a}_1\right),
\,
\hat{a}_3 = \frac{1}{\sqrt{2}} \left(i \hat{a}_0 + \hat{a}_1\right),
\nonumber \\
\hat{a}_4 = \frac{1}{\sqrt{2}} \left(i \hat{a}_2 + e^{i \varphi}
\hat{a}_3\right) = 
\frac{1}{2}\left[
i \left(1 + e^{i \varphi}\right)\hat{a}_0 -
\left(1 - e^{i \varphi}\right)\hat{a}_1
\right],
\nonumber \\
\hat{a}_5 = \frac{1}{\sqrt{2}} \left(\hat{a}_2 + i e^{i \varphi}
\hat{a}_3\right) = 
\frac{1}{2}\left[
\left(1 - e^{i \varphi}\right)\hat{a}_0 +
i \left(1 + e^{i \varphi}\right)\hat{a}_1
\right].
\label{eqPart3SqueezedAddAddAdd2}
\end{eqnarray}
Let the input signal field (input 0, operator $\hat{a}_0$)
be in a squeezed vacuum state. The second input field
(input 1, operator $\hat{a}_1$) is in a coherent state with
a large amplitude. So large that the input field is
two-mode 
\[
\left|\psi\right>_{\mbox{in}} =
\left|\alpha\right>
\left|re^{i \theta}, 0\right>,
\]
where $\left|\alpha\right>$ is the coherent state,
$\left|re^{i \theta}, 0\right>$ is the squeezed vacuum state. Channels 4
and 5 are each detected by their own photodetector. The signals from
the photodetectors are subtracted and recorded.

Thus, the output signal is determined by the average value of the operator
\begin{eqnarray}
\hat{n}_{54} = \hat{a}_5^{\dag}\hat{a}_5 - 
\hat{a}_4^{\dag}\hat{a}_4 = 
\nonumber \\
=
\left(
\hat{a}_1^{\dag}\hat{a}_1 - 
\hat{a}_0^{\dag}\hat{a}_0
\right) \cos\,\varphi
- 
\left(
\hat{a}_0^{\dag}\hat{a}_1 - 
\hat{a}_1^{\dag}\hat{a}_0
\right) \sin\,\varphi
\end{eqnarray}
Here the equality \eqref{eqPart3SqueezedAddAddAdd2} was used.

Let us find the average value of the operator
$\left<\psi\right|\hat{n}_{54}\left|\psi\right>$. We take into account that
each operator acts only on its own mode. We have:
\begin{eqnarray}
\left<\psi\right|\hat{n}_{54}\left|\psi\right> = 
\left(
\left<\alpha\right|\hat{a}_1^{\dag}\hat{a}_1\left|\alpha\right>
- 
\left<r e^{i\theta}, 0\right|\hat{a}_0^{\dag}\hat{a}_0\left|r
e^{i\theta}, 0\right> 
\right) \cos\,\varphi -
\nonumber \\
-
\left(
\left<r e^{i\theta}, 0\right|\hat{a}_0^{\dag}\left|r
e^{i\theta}, 0\right>\alpha +
\alpha 
\left<r e^{i\theta}, 0\right|\hat{a}_0\left|r
e^{i\theta}, 0\right>
\right) \sin\,\varphi.
\label{eqPart3SqueezedAddAddAdd4}
\end{eqnarray}
The first term in the first bracket equals the average number of pump photons
$\left|\alpha\right|^2$, and the second term, as we know, equals the number of
photons in the squeezed state, which is $sh^2 r$. The second bracket on the right, as
we found when examining the balanced detector, is zero. Thus we get
\begin{equation}
\left<\psi\right|\hat{n}_{54}\left|\psi\right> = 
\left(
\left|\alpha\right|^2 - sh^2 r
\right)
\cos\,\varphi.
\nonumber
\end{equation}
If we set $\varphi=\frac{\pi}{2}$, then the average value of the output
signal is zero:
\[
\left<\psi\right|\hat{n}_{54}\left|\psi\right> = 0.
\]

The average value of the second term in \eqref{eqPart3SqueezedAddAddAdd4}
is zero, but its mean square is not zero. This means it is a source of noise
which limits measurement accuracy.

Let us calculate the mean square of $\hat{n}_{54}$:
\begin{equation}
\left(\Delta n_{54}\right)^2 = 
\left<\psi\right|\hat{n}_{54}^2\left|\psi\right> -
\left<\psi\right|\hat{n}_{54}\left|\psi\right>^2.
\nonumber
\end{equation}
The mean in our case is zero, so we have
\begin{eqnarray}
\left(\Delta n_{54}\right)^2 = 
\left<\psi\right|\hat{n}_{54}^2\left|\psi\right> =
\nonumber \\
=
\left|\alpha\right|^2
\left<r e^{i\theta}, 0\right|
\left(\hat{a}_0^{\dag} + \hat{a}_0\right)^2
\left|r e^{i\theta}, 0\right> =
\nonumber \\
=
\left|\alpha\right|^2
\left<r e^{i\theta}, 0\right|
\left(\hat{a}_0^{\dag}\right)^2 + 
\left(\hat{a}_0\right)^2 + 
\hat{a}_0^{\dag}\hat{a}_0 +
\hat{a}_0\hat{a}_0^{\dag}
\left|r e^{i\theta}, 0\right>.
\nonumber
\end{eqnarray}

Previously, we had \eqref{eqPart3SqueezedAddAdd7}:
\begin{eqnarray}
\left<\hat{a}_0\right> = 
\bra{0}
\hat{S}^{\dag}\left(r, 0\right)
\hat{a}_0
\hat{S}\left(r, 0\right)
\ket{0} = 
\bra{0}
\hat{a}_0 ch\,r - \hat{a}_0^{\dag} sh\,r
\ket{0} = 0,
\nonumber \\
\left<\hat{a}_0^{\dag}\right> = 
\bra{0}
\hat{a}_0^{\dag} ch\,r - \hat{a}_0 sh\,r
\ket{0} = 0,
\nonumber
\end{eqnarray}
from which we get
\begin{equation}
\left(\Delta n_{54}\right)^2 = 
\left<\psi\right|\hat{n}_{54}^2\left|\psi\right> =
\left|\alpha\right|^2\left(
ch\,2r - sh\,2r
\right) = 
\left|\alpha\right|^2 e^{-2 r}.
\nonumber
\end{equation}
The root mean square noise value will be 
\[
\left|\Delta n_{54}\right| = 
\left|\alpha\right| e^{- r}.
\]
We take this value as the threshold below which we will not be able to
detect or measure the signal. Hence, the signal must be
above the threshold.

If the initial value $\varphi=\frac{\pi}{2}$ changes by an amount $\Delta
\varphi$, a signal equal to
\[
\Delta n_{54} = 
\left|\alpha\right|^2
\Delta \varphi
\]
will appear.
It is necessary that 
\[
\left|\alpha\right|^2
\Delta \varphi > 
\left|\alpha\right|
e^{-r},
\]
from which we obtain
\begin{equation}
\Delta \varphi >
\frac{1}{\left|\alpha\right|} e^{-r},
\label{eqPart3SqueezedAddAddAdd7}
\end{equation}
where $\left|\alpha\right| = \sqrt{\bar{n}}$, and $\bar{n}$ is the average number
\rindex{homodyne}
of photons in the homodyne field.

Expression \eqref{eqPart3SqueezedAddAddAdd7} gives a smaller value than
the one obtained earlier when considering the accuracy of interference
measurements. Using the squeezed state can increase measurement
accuracy. 
