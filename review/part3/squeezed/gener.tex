%% -*- coding:utf-8 -*- 
\section{Generation of Quadrature-Squeezed States}
\label{pNonClassGenerSqueezed}
So far, we have considered the theoretical possibility of the existence and
generation of squeezed states. The question arises — in what real
processes can one obtain light in a squeezed state? Here we consider
the possibility of obtaining a quadrature squeezed state through
parametric interaction occurring when a strong
pump field passes through a medium with quadratic nonlinearity. In such
an interaction, two waves arise, related to the pump frequency by the relations:
\begin{equation}
\omega_p = \omega_s + \omega_i,
\nonumber
\end{equation}
where $\omega_p$ is the pump frequency, $\omega_s$ is the signal frequency,
$\omega_i$ is the idler frequency. This is schematically shown in 
\autoref{figPart3Squeezed_10}.

\input ./part3/squeezed/fig10.tex


\rindex{Hamiltonian}The interaction Hamiltonian between the pump, signal, and idler
waves in the interaction picture is given by:
\begin{equation}
\hat{V} = \hbar \kappa \left(
\hat{a}^{\dag}_s \hat{a}^{\dag}_i \hat{b} + 
\hat{a}_s \hat{a}_i \hat{b}^{\dag}
\right),
\label{eqPart3Squeezed30}
\end{equation}
where $\hat{a}^{\dag}_s$, $\hat{a}^{\dag}_i$, 
$\hat{a}_s$, and $\hat{a}_i$ are creation and annihilation operators for
the signal and idler waves, and $\kappa$ is the interaction constant.

\input ./part3/squeezed/fig11.tex

In the degenerate regime, the signal and idler waves have the same
frequency equal to half the pump frequency (see \autoref{figPart3Squeezed_11}):
\[
\omega_s = \omega_i = \frac{\omega_p}{2} = \omega.
\]

For the degenerate parametric process, the interaction Hamiltonian
\eqref{eqPart3Squeezed30} simplifies to:
\begin{equation}
\hat{V} = \hbar \kappa \left(
\left(\hat{a}^{\dag}\right)^2 \hat{b} + 
\hat{a}^2 \hat{b}^{\dag}
\right).
\nonumber
\end{equation}
If the pump field is in a coherent state with large
amplitude $\alpha_p = A_p e^{i \varphi}$, $\left|\alpha_p\right| \gg
1$, the pump can be replaced by a classical field. Then the interaction
Hamiltonian \rindex{Hamiltonian}
further simplifies to:
\begin{equation}
\hat{V} = \hbar \kappa A_p\left(
\left(\hat{a}^{\dag}\right)^2 e^{- i \varphi} + 
\hat{a}^2 e^{i \varphi}
\right),
\nonumber
\end{equation}
where $A_p$ and $\varphi$ are the real amplitude and phase
of the pump. Obviously, in this approximation, we neglect pump depletion. This is valid
as long as the amplitude of the signal (idler)
wave remains small compared to the pump field amplitude. The Heisenberg equation
for the operator $\hat{a}$ takes the form:
\begin{eqnarray}
\frac{d\hat{a}}{dt} = \frac{i}{\hbar}
\left[\hat{V}, \hat{a}\right] = i \kappa A_p
\left[
\left(\hat{a}^{\dag}\right)^2 e^{- i \varphi} + 
\hat{a}^2 e^{i \varphi},
\hat{a}
\right] = 
\nonumber \\
=
i \kappa A_p e^{- i \varphi}
\left(
\left(
\hat{a}^{\dag}\right)^2  \hat{a} -
\hat{a}
\left(
\hat{a}^{\dag}\right)^2
\right) = 
\nonumber \\
=
i \kappa A_p e^{- i \varphi}
\left(
\left(
\hat{a}^{\dag}\right)^2  \hat{a} -
\left(\hat{a}^{\dag}\hat{a} + 1\right)
\hat{a}^{\dag}
\right) = 
\nonumber \\
=
i \kappa A_p e^{- i \varphi}
\left(
\hat{a}^{\dag}\hat{a}^{\dag}  \hat{a} -
\hat{a}^{\dag}\hat{a}\hat{a}^{\dag} -
\hat{a}^{\dag}
\right) = 
\nonumber \\
=
i \kappa A_p e^{- i \varphi}
\hat{a}^{\dag}
\left(
\hat{a}^{\dag}  \hat{a} -
\hat{a}\hat{a}^{\dag} -
1
\right) = 
i \kappa A_p e^{- i \varphi}
\hat{a}^{\dag}
\left(
\left(-1\right) -
1
\right) =
\nonumber \\
= 
- 2 i \kappa A_p e^{- i \varphi}
\hat{a}^{\dag}.
\nonumber
\end{eqnarray}
Thus we have
\begin{equation}
\frac{d\hat{a}}{dt} = 
- i \Omega_p e^{- i \varphi}
\hat{a}^{\dag},
\label{eqPart3Squeezed34a}
\end{equation}
where 
\[
\Omega_p = 2 \kappa A_p
\]
is the effective Rabi frequency (energy expressed through frequency).
Similarly,
\begin{equation}
\frac{d\hat{a}^{\dag}}{dt} = 
i \Omega_p e^{i \varphi}
\hat{a}.
\label{eqPart3Squeezed34b}
\end{equation}

Eliminating $\hat{a}^{\dag}$ from the system \eqref{eqPart3Squeezed34a} and
\eqref{eqPart3Squeezed34b}, we get
\begin{equation}
\frac{d^2\hat{a}\left(t\right)}{dt^2} = 
- i \Omega_p e^{- i \varphi}
\frac{d\hat{a}^{\dag}\left(t\right)}{dt} = 
\Omega_p^2\hat{a}\left(t\right).
\label{eqPart3Squeezed35}
\end{equation}
The initial conditions for this equation are
\begin{eqnarray}
\hat{a}\left(0\right) = \hat{a}_0, 
\nonumber \\
\hat{a}^{\dag}\left(0\right) = \hat{a}^{\dag}_0, 
\nonumber
\end{eqnarray}
from which
\begin{eqnarray}
\left.\hat{a}\right|_{t=0} = \hat{a}_0, 
\nonumber \\
\left.\frac{d\hat{a}}{dt}\right|_{t=0} = 
- i \Omega_p e^{- i \varphi} \hat{a}^{\dag}_0,
\label{eqPart3Squeezed36}
\end{eqnarray}
where $\hat{a}_0$ and $\hat{a}^{\dag}_0$ are the initial values of the operators at $t
= 0$. The solution of \eqref{eqPart3Squeezed35} with the initial
conditions \eqref{eqPart3Squeezed36} is:
\begin{equation}
\hat{a}\left(t\right) = \hat{a}_0 ch \left(\Omega_p t \right) - 
i \hat{a}^{\dag}_0 sh \left(\Omega_p t\right) e^{-i \varphi}.
\label{eqPart3Squeezed37a}
\end{equation}
Similarly,
\begin{equation}
\hat{a}^{\dag}\left(t\right) = \hat{a}^{\dag}_0 ch \left(\Omega_p t \right) +
i \hat{a}_0 sh \left(\Omega_p t\right) e^{i \varphi}.
\label{eqPart3Squeezed37b}
\end{equation}

%% Check in maxima
%% (%i10) e1: diff(x(t),t) = - %i * Omega * exp(-%i * Phi) * y(t);
%%                     d                        - %i Phi
%% (%o10)              -- (x(t)) = - %i Omega %e         y(t)
%%                     dt
%% (%i11) e2: diff(y(t),t) =  %i * Omega * exp(%i * Phi) * x(t);
%%                       d                      %i Phi
%% (%o11)                -- (y(t)) = %i Omega %e       x(t)
%%                       dt
%% (%i12) desolve ([e1, e2], [x(t), y(t)]);
%%                        %i Phi              Omega t - %i Phi
%%                (x(0) %e       - %i y(0)) %e
%% (%o12) [x(t) = --------------------------------------------
%%                                     2
%%            %i Phi              - Omega t - %i Phi
%%    (x(0) %e       + %i y(0)) %e
%%  + ----------------------------------------------, 
%%                          2
%%                   %i Phi           Omega t
%%        (%i x(0) %e       + y(0)) %e
%% y(t) = -----------------------------------
%%                         2
%%               %i Phi           - Omega t
%%    (%i x(0) %e       - y(0)) %e
%%  - -------------------------------------]
%%                      2
%% (%i13)


Note that at $\varphi = \frac{\pi}{2}$ expressions
\eqref{eqPart3Squeezed37a} and \eqref{eqPart3Squeezed37b} reduce to
the relations
\eqref{eqPart3Squeezed16a}:
\begin{eqnarray}
\hat{a}\left(t\right) = \hat{a}_0 ch \left(\Omega_p t \right) - 
\hat{a}^{\dag}_0 sh \left(\Omega_p t\right),
\nonumber \\
\hat{a}^{\dag}\left(t\right) = \hat{a}^{\dag}_0 ch \left(\Omega_p t \right) -
\hat{a}_0 sh \left(\Omega_p t\right).
\nonumber
\end{eqnarray}

As we see, the operator $\hat{a}\left(t\right)$ results from a unitary
transformation \eqref{eqPart3Squeezed16a} which in our case
has the form:
\begin{equation}
\hat{a}\left(t\right) =
\hat{S}^{\dag}\left(t\right)
\hat{a}\left(0\right)
\hat{S}\left(t\right) =
\hat{a}_0 ch \left(\Omega_p t \right) - 
\hat{a}^{\dag}_0 sh \left(\Omega_p t\right).
\label{eqPart3SqueezedGenerHeizenberg}
\end{equation}
Thus, moving from the Heisenberg picture
\eqref{eqAddWaveFunc_HeizenbergU}
to the Schrödinger picture \eqref{eqAddWaveFunc_ShredingerU}, one can
conclude that the initial coherent state undergoes the same
unitary transformation as the annihilation
operator \eqref{eqPart3SqueezedGenerHeizenberg}:
\begin{equation}
\hat{S}\left(t\right)
\left|\alpha\right> =
\left|t, \alpha\right>,
\nonumber
\end{equation}
i.e., the parametric interaction is equivalent to the action of the squeezing operator. In this case,
for the quadrature components
\[
\hat{X}_1 = \frac{\hat{a} + \hat{a}^{\dag}}{2}
\]
and
\[
\hat{X}_2 = \frac{\hat{a} - \hat{a}^{\dag}}{2i}
\]
a squeezed state is realized with uncertainties
\begin{eqnarray}
\left(\Delta X_1\right)^2 = 
\frac{1}{4}e^{-2 \Omega_p t},
\nonumber \\
\left(\Delta X_2\right)^2 = 
\frac{1}{4}e^{2 \Omega_p t},
\nonumber \\
\left(\Delta X_1 \Delta X_2\right) = 
\frac{1}{4}.
\nonumber
\end{eqnarray}
The squeezing parameter $r = \Omega_p t$ grows unbounded in time and,
therefore, the degree of squeezing increases indefinitely. In
reality, this does not happen. Unlimited squeezing is a consequence
of overly idealized assumptions: pump depletion, deviation of the real pump from
a monochromatic classical light, and several other factors were neglected. All these lead to limitations on the degree
of squeezing. The closer the real conditions to ideal, the higher the degree
of squeezing that can be obtained.

The case considered above of obtaining squeezed states via
parametric processes pertains to the generation of squeezed vacuum,
since the generation of the squeezed state occurs from the always-present
vacuum seed field. This is sometimes called parametric
scattering of the pump field. As we have found, the "squeezed vacuum" is not
a vacuum field in the strict sense. The mean photon number in this field is not
zero but depends on the degree of squeezing according to
formula \eqref{eqPart3Squeezed24} and can be quite large at
high degrees of squeezing.
