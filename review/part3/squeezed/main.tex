%% -*- coding:utf-8 -*- 
\chapter{Squeezed States}
\label{chSqueezed}
In squeezed states, the possibility allowed by the Heisenberg uncertainty relation is realized: to reduce the uncertainty of one of the conjugate observables at the expense of increasing the uncertainty of the other. Such states may prove useful in quantum optics for practical needs. For example, to improve the accuracy of interference experiments.

Next, the theory of squeezed states in parametric processes and the applications of squeezed states in interference experiments are considered.

\input ./part3/squeezed/haizenberg.tex
\input ./part3/squeezed/quad1.tex
\input ./part3/squeezed/quad2.tex
\input ./part3/squeezed/gener.tex
\input ./part3/squeezed/observ.tex
\input ./part3/squeezed/interfero.tex
\input ./part3/squeezed/nonlinear.tex
\input ./part3/squeezed/nonclass.tex
\input ./part3/squeezed/questions.tex