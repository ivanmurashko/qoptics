%% -*- coding:utf-8 -*- 
\chapter{Nonclassical Light}
\label{chNonClass}
Quantum states of light can be divided into two groups: states
that asymptotically approach
a classical state (classical light) as the intensity of the light increases, and states that do not
possess this property. For example, among the states we considered earlier,
the coherent state has a classical limit, while the number (Fock) state does not.

Earlier we defined the concept of the second-order coherence function 
$G^{(2)}$. For single-mode light \eqref{eqCh4_28}
\begin{eqnarray}
G^{(2)} = \frac{\left<\hat{a}^{\dag}\hat{a}^{\dag}\hat{a}\hat{a}\right>}
{\left<\hat{a}^{\dag}\hat{a}\right>^2} = 
\nonumber \\
= 
\frac{\left<\hat{a}^{\dag}\hat{a}\left(\hat{a}^{\dag}\hat{a} - 1\right)\right>}
{\left<\hat{a}^{\dag}\hat{a}\right>^2} = 
\frac{\left<n^2\right> - \left<n\right>}{\left<n\right>^2},
\label{eqPart3_Nonclass_Nonclass1}
\end{eqnarray}
where the commutation relations
$\left[\hat{a}, \hat{a}^{\dag}\right] = 1$, 
$\hat{n} = \hat{a}^{\dag}\hat{a}$ is the particle number operator, have been used.

In the classical case, $\left<\hat{a}^{\dag}\hat{a}\right> =
\left|\alpha\right|^2$, where $\left|\alpha\right|^2 = I$ is the intensity
of the field (the squared modulus of the field amplitude). Since classical quantities
commute, formula \eqref{eqPart3_Nonclass_Nonclass1} in the classical
case can be replaced by
\begin{equation}
G^{(2)}_{\mbox{cl}} = 
\frac{\left<\left|\alpha\right|^4\right>}{\left<\left|\alpha\right|^2\right>^2}
= 
\frac{\left<I^2\right>}{\left<I\right>^2},
\label{eqPart3_Nonclass_Nonclass2}
\end{equation}
where averaging is performed with the help of a positive
definite classical distribution function $P\left(\alpha\right)$
\begin{equation}
\left<\left|\alpha\right|^{2n}\right> = 
\int_0^{\infty}P\left(\left|\alpha\right|^2\right)
\left|\alpha\right|^{2n}d^2\alpha, \quad n=1,2.
\nonumber
\end{equation}
Consider the difference 
$G^{(2)}_{\mbox{cl}} - G^{(1)}_{\mbox{cl}} = G^{(2)}_{\mbox{cl}} -
1$, since for a single-mode field, as we know, $G^{(1)}_{\mbox{cl}} =
1$ always. From \eqref{eqPart3_Nonclass_Nonclass2} it follows
\begin{eqnarray}
G^{(2)}_{\mbox{cl}} - G^{(1)}_{\mbox{cl}} = G^{(2)}_{\mbox{cl}} -
1 = 
\nonumber \\
=
\frac{\left<\left|\alpha\right|^4\right>}{\left<\left|\alpha\right|^2\right>^2}
- 1 = 
\frac{\left<\left|\alpha\right|^4\right> -
  \left<\left|\alpha\right|^2\right>^2}{\left<\left|\alpha\right|^2\right>^2}
= 
\nonumber \\
= \frac{\int_0^{\infty} P \left|\alpha\right|^4 d^2\alpha - 2
  \left<\left|\alpha\right|^2\right>^2 +
  \left<\left|\alpha\right|^2\right>^2}{\left<\left|\alpha\right|^2\right>^2}
= 
\nonumber \\
= 
\frac{\int_0^{\infty} P \left(\left|\alpha\right|^4  - 2
  \left|\alpha\right|^2
  \left<\left|\alpha\right|^2\right> +
  \left<\left|\alpha\right|^2\right>^2\right)d^2\alpha}
     {\left<\left|\alpha\right|^2\right>^2}
=
\nonumber \\
=
\frac{\int_0^{\infty} P \left(\left|\alpha\right|^2  - 
  \left<\left|\alpha\right|^2\right>\right)^2 d^2\alpha}{\left<\left|\alpha\right|^2\right>^2}
\ge 0.
\label{eqPart3_Nonclass_Nonclass4}
\end{eqnarray}
From inequality \eqref{eqPart3_Nonclass_Nonclass4} it follows that
\begin{equation}
G^{(2)}_{\mbox{cl}} \ge 1.
\nonumber
\end{equation}
In the quantum case, due to operator noncommutativity, the situation will be
different. Formula \eqref{eqPart3_Nonclass_Nonclass1} can be rewritten in the following
form:
\begin{eqnarray}
G^{(2)} = \frac{\left<\hat{n}^2\right> - \left<\hat{n}\right>}{\left<\hat{n}\right>^2} = 
\nonumber \\
=
1 + \frac{\left(\left<\hat{n}^2\right> - \left<\hat{n}\right>^2\right) -
  \left<\hat{n}\right>}{\left<\hat{n}\right>^2} = 1 + \frac{\sigma^2 -
  \left<\hat{n}\right>}{\left<\hat{n}\right>^2}, 
\label{eqPart3_Nonclass_Nonclass6}
\end{eqnarray}
where $\sigma^2$ is the variance,
i.e., $\sigma$ is the standard deviation.

From \eqref{eqPart3_Nonclass_Nonclass6} it follows that $G^{(2)}$ can
be either greater or less than 1, depending on whether $\sigma^2$ or
$\left<\hat{n}\right>$ is larger. In the classical case $G^{(2)} \ge 1$, so
a criterion of nonclassicality can be taken as 
\begin{equation}
G^{(2)} < 1.
\label{eqPart3_Nonclass_Nonclass7}
\end{equation}

From (\ref{eqPart3_Nonclass_Nonclass6},
\ref{eqPart3_Nonclass_Nonclass7}) it follows that the condition for nonclassicality
is 
\(
\sigma^2 - \left<\hat{n}\right> < 0,
\) 
i.e., the variance must be less than the mean photon number. 
It is known that a photon flux for which
\(
\sigma^2 - \left<\hat{n}\right> = 0,
\)
has Poissonian statistics (an example is the coherent state). \rindex{Poisson distribution}
The case 
\(
\sigma^2 - \left<\hat{n}\right> < 0
\)
corresponds to a more regular flux, called sub-Poissonian. This is the case of photon antibunching. An example
of a state for which this holds is, for instance, a state resulting from parametric scattering,
which we will consider later. The case
\(
\sigma^2 - \left<\hat{n}\right> > 0
\)
corresponds to a less regular process where photons bunch
(taken as the thermal excitation of light). Such a state
is called super-Poissonian. All three cases are shown in
\autoref{figPart3Nonclass1}, where the timing of photodetection events is illustrated.

\input ./part3/nonclass/fignonclass1.tex

We conclude that nonclassical states correspond to the
sub-Poissonian threshold. The number $G^{(2)}$ is simply related to other
parameters characterizing photon number fluctuations: the variance and
the Fano factor, which frequently appear in the literature.
\begin{eqnarray}
\Phi  = \frac{\sigma^2}{\left<\hat{n}\right>},
\nonumber \\
G^{(2)} - 1 = \frac{\Phi - 1}{\left<\hat{n}\right>}
\nonumber
\end{eqnarray}
The statistical nature of the experiment can be established from an experiment
in which the second-order correlation function is measured
(the Brown-Twiss experiment). The experimental setup is shown in
\autoref{figPart3Nonclass2}.

\input ./part3/nonclass/fignonclass2.tex

The adjustable delay $\tau$ is implemented by changing the length of one of the
arms.

\input ./part3/nonclass/fignonclass3.tex

\autoref{figPart3Nonclass3} shows three curves obtained from the experiment. Curve (b) corresponds
to the coherent state. Curve (a) corresponds to photon bunching and
super-Poissonian statistics. Curve (c) corresponds
to antibunching and sub-Poissonian statistics. Such a curve is characteristic
of nonclassical light. The explanation of the behavior of the curves is simple. If
$\tau$ is large, photodetection events are random and independent of each other.
Therefore, the behavior of all three curves is determined by random
coincidences and will be the same for all three. For small $\tau$
the behavior of the curves differs. In the case of photon bunching, there is a
maximum at $\tau = 0$, and for antibunching – a minimum.

Another criterion for nonclassicality is the absence of positive
definiteness of the quasiprobability distribution $P\left(\alpha\right)$ when
using the coherent state representation.

In the classical case, the function $P\left(\alpha\right)$ must be everywhere
positive definite, i.e., $P\left(\alpha\right) \ge 0$ for all
values of $\alpha$. Let us consider this issue in more detail. The condition
of nonclassicality 
\(
G^{(2)} < 1,
\)
using \eqref{eqPart3_Nonclass_Nonclass1}, can be represented in the form
\begin{equation}
\left<\hat{a}^{\dag}\hat{a}^{\dag}\hat{a}\hat{a}\right> -
\left<\hat{a}^{\dag}\hat{a}\right>^2 < 0.
\label{eqPart3_Nonclass_Nonclass9}
\end{equation}
Consider the expression
\begin{equation}
\int P\left(\alpha\right)\left(\left|\alpha\right|^2 -
\left<\hat{n}\right>\right)^2 d^2\alpha < 0
\label{eqPart3_Nonclass_Nonclass10}
\end{equation}
\begin{eqnarray}
\int P\left(\alpha\right)\left(\left|\alpha\right|^2 -
\left<\hat{n}\right>\right)^2 d^2\alpha = 
\int P\left(\alpha\right)\left(\left|\alpha\right|^4 -
2\left|\alpha\right|^2 \left<\hat{n}\right> +
\left<\hat{n}\right>^2\right) d^2\alpha = 
\nonumber \\
=
\left<\hat{a}^{\dag}\hat{a}^{\dag}\hat{a}\hat{a}\right> -
\left<\hat{a}^{\dag}\hat{a}\right>^2 < 0,
\nonumber
\end{eqnarray}
i.e., expressions \eqref{eqPart3_Nonclass_Nonclass9} and
\eqref{eqPart3_Nonclass_Nonclass10} are equivalent. This follows from
the following relations
\begin{eqnarray}
  \left<\hat{a}^{\dag}\hat{a}^{\dag}\hat{a}\hat{a}\right> = \mathrm{Tr} \left(
  \hat{a}^{\dag}\hat{a}^{\dag}\hat{a}\hat{a} \rho \right)=
  \nonumber \\
  =
  \sum_n \bra{n}\left(\hat{a}^{\dag}\hat{a}^{\dag}\hat{a}\hat{a}
  \int P\left(\alpha\right)
  \left|\alpha\right>\left<\alpha\right| d^2\alpha
  \right)
  \ket{n} =
  \nonumber \\
  =
  \int P\left(\alpha\right)
  \left<\alpha\right|\hat{a}^{\dag}\hat{a}^{\dag}\hat{a}\hat{a}\left|\alpha\right>
  d^2\alpha = 
\int P\left(\alpha\right)\left|\alpha\right|^4 d^2\alpha,
\nonumber \\
\int P\left(\alpha\right)\left|\alpha\right|^2 d^2\alpha = 
\left<\hat{a}^{\dag}\hat{a}\right> = \left<\hat{n}\right>.
\nonumber
\end{eqnarray}
We have found that expression \eqref{eqPart3_Nonclass_Nonclass10} must
be negative. Since 
\(
\left(\left|\alpha\right|^2 -
\left<\hat{n}\right>\right)^2
\)
is always positive, a negative result will be obtained if 
$P\left(\alpha\right)$ is negative in at least part of the domain of $\alpha$.
Thus, for nonclassical light,
the quasiprobability $P\left(\alpha\right)$ is not a positive
definite function.