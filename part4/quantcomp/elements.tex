%% -*- coding:utf-8 -*- 
\section{Quantum Logic Elements}
How can an element be constructed that performs the transformation $\hat{U}_f$ \eqref{eqQuantCompQuantComp}? There is a set of elements from which an element performing the necessary transformation $\hat{U}_f$ can be constructed with a given accuracy. Such sets are called universal.

\subsection{Universal Set of Quantum Gates}

\subsubsection{Identity Transformation}

\[
\hat{\sigma}_0 = \begin{pmatrix}
1 & 0 \\
0 & 1
\end{pmatrix}
\]

\subsubsection{Negation}

\[
\hat{\sigma}_1 = \begin{pmatrix}
0 & 1 \\
1 & 0
\end{pmatrix}
\]

\subsubsection{Phase Shift}

\[
\hat{\sigma}_2 = \begin{pmatrix}
1 & 0 \\
0 & -1
\end{pmatrix}
\]

\subsubsection{Hadamard Transformation}
\rindex{Hadamard Transformation!definition}
One of the basic quantum logic elements is the Hadamard transformation (see \autoref{figQuantCompHadamar1}), which is defined by the following relations
\begin{eqnarray}
\hat{H} \ket{0} = \ket{+} =  
\frac{\ket{0} + \ket{1} }{\sqrt{2}},
\nonumber \\
\hat{H} \ket{1} = \ket{-} = 
\frac{\ket{0} - \ket{1} }{\sqrt{2}},
\nonumber
\end{eqnarray}

In matrix form this transformation can be written as
\begin{equation}
\hat{H} = \frac{1}{\sqrt{2}}
\begin{pmatrix}
1 & 1 \\
1 & -1
\end{pmatrix},
\label{eq:quantcomp:hadamar}
\end{equation}
where the basis vectors are chosen as
\[
\ket{0} = \begin{pmatrix}
1 \\ 0 
\end{pmatrix}
\]
and
\[
\ket{1} = \begin{pmatrix}
0 \\ 1 
\end{pmatrix}.
\]

\input ./part4/quantcomp/fighadamar1.tex

From \eqref{eq:quantcomp:hadamar}, we can obtain the following property of the operator $\hat{H}$:
\begin{equation}
\hat{H} \hat{H} = \frac{1}{2}
\begin{pmatrix}
1 & 1 \\
1 & -1
\end{pmatrix}
\begin{pmatrix}
1 & 1 \\
1 & -1
\end{pmatrix} = 
\begin{pmatrix}
1 & 0 \\
0 & 1
\end{pmatrix}.
\label{eq:quantcomp:hadamar_prop}
\end{equation}

This transformation is used to obtain a superposition of states containing all possible values of the argument of the computed function (see \autoref{figQuantCompHadamar2}).

\input ./part4/quantcomp/fighadamar2.tex

\subsubsection{CNOT}

The transformation matrix is as follows
\[
CNOT=\begin{pmatrix}
1 & 0 & 0 & 0 \\
0 & 1 & 0 & 0 \\
0 & 0 & 0 & 1 \\
0 & 0 & 1 & 0 
\end{pmatrix}
\]

This gate (see \autoref{figQuantCompCNOT}) is used for two qubits and inverts the state of the second qubit only if the first qubit is equal to one.

\input ./part4/quantcomp/figcnot.tex

Thus, if our initial state of two qubits was
\[
\ket{\psi_i} = a \ket{00} + b \ket{01} + c \ket{10} + d \ket{11}
\]
it transforms to
\[
CNOT \ket{\psi_i} = \ket{\psi_f} = 
a \ket{00} + b \ket{01} + c \ket{11} + d \ket{10}
\]

\subsubsection{Universal Set}

\begin{definition}[Universal Set of Quantum Gates]
A set of quantum gates is called universal if any unitary transformation can be approximated to a given accuracy by a finite sequence of gates from this set.
\end{definition}

\begin{theorem}[Kitaev]
The set $\hat{\sigma}_0, \hat{\sigma}_1,
\hat{\sigma}_2, \hat{H}, CNOT$ is universal.
\begin{proof}
TBD
\end{proof}
\end{theorem}

\subsection{Control Elements}

\input ./part4/quantcomp/figcelem.tex

\input ./part4/quantcomp/figphase.tex