%% -*- coding:utf-8 -*- 
\subsection{Finding the period of functions using quantum
  Fourier transform}

To determine the period of the function \eqref{eqPart4QuantCompShorClassPart},
the circuit shown in
\autoref{figQuantCompQuantPeriodFinding} is used.

\input part4/quantcomp/figquantperiodfinding.tex

The first element is the Hadamard transform on $n$ qubits,
\rindex{Hadamard transform} 
which
prepares the initial state as:
\begin{equation}
\ket{in} = \frac{1}{\sqrt{2^n}}\sum_{x=0}^{2^n - 1} \ket{x}
\otimes\ket{0}.
\nonumber
\end{equation}

After the element computing the function $\hat{U}_f$, the state becomes
\begin{equation}
\hat{U}_f\ket{in} = \frac{1}{\sqrt{2^n}}\sum_{x=0}^{2^n - 1} \ket{x}
\otimes\left|f\left(x\right)\right>.
\nonumber
\end{equation} 

\input part4/quantcomp/figshorquant.tex

After measuring the value of the function, only those elements for which the function value equals the measured value remain in the coordinate list. As a result, the input to the Fourier transform measurement element is a state of the form 
\begin{equation}
\ket{in'} = \sum_{x'} \ket{x'},
\nonumber
\end{equation} 
where all nonzero elements have the same amplitude and follow with a period equal to the period of the investigated function. The initial value will have a shift which depends on the experiment (different experiments will have different shifts). According to lemma
\ref{lemmaAddFourierDiscretFourierShiftTime}, the Fourier image will be
the same for different function measurements.

Furthermore, due to lemma \ref{lemmaAddFourierDiscretFourierPeriod}
(on periodicity) (see also comment
\ref{rem:dsp:fourier:periodprop}), it follows that the most
probable measurements (maxima 
of probability) occur with a period related to the original period
of the function. Thus, as a result of several experiments, the period
of the desired function can be found with the required probability level
(see \autoref{picPart4QuantCompShorQuantPart}).

\begin{example}
\emph{Finding the period of the function $f\left(x\right) = 2^x \mod 21$}
\label{exPart4QuantCompShorQuantPeriodFinding}
As an example, consider the task of finding the period of the function 
$f\left(x, a\right) = a^x \mod{N}$ with $a=2$, $N = 21$ see 
\autoref{picPart4QuantCompShorQuantPart}

The number of samples $M$ must be a power of two. In our example we
choose $M = 2^6 = 64$ as the number of samples. Thus,
6 qubits are required for our example.

The initial state after the Hadamard transform is:
\begin{equation}
\ket{in} = \frac{1}{8}\sum_{x = 0}^{63}\ket{x} \otimes \ket{0},
\nonumber
\end{equation}
where $\ket{x}$ is the tensor product 
\rindex{Tensor product}
of 6 qubits
which encode the binary representation of the argument of the investigated
function. For example, for $x=5_{10}=000101_2$ we have
\[
\ket{x} = \ket{0}\otimes \ket{0}\otimes
\ket{0}\otimes 
\ket{1}\otimes \ket{0}\otimes \ket{1}
\]

After calculating the function, the state is (see the upper graph
in \autoref{picPart4QuantCompShorQuantPart})
\begin{eqnarray}
\hat{U}_f\ket{in} = \frac{1}{8}\sum_{x = 0}^{63}\ket{x}
\otimes \left|f\left(x\right)\right> = 
\nonumber \\
=
\frac{1}{8}
\left(
\ket{0}\otimes\ket{2} + 
\ket{1}\otimes\ket{4} + 
\ket{2}\otimes\ket{8} + \dots +
\right.
\nonumber \\
\left.
+
\ket{62}\otimes\ket{8} +
\ket{63}\otimes\ket{16}
\right).
\label{eqPart4QuantCompShorPFExample1}
\end{eqnarray}

If the measurement result of the function was equal to $1$, then from the sum
\eqref{eqPart4QuantCompShorPFExample1} only those terms remain for which
the function value is equal to $1$ (see the middle graph
in \autoref{picPart4QuantCompShorQuantPart}):
\begin{equation}
\ket{in'} = \frac{1}{\sqrt{10}}\left( 
\ket{5}\otimes\ket{1} +
\ket{11}\otimes\ket{1} +
\ket{17}\otimes\ket{1} +
\dots +
\ket{60}\otimes\ket{1}
\right).
\label{eqPart4QuantCompShorPFExample2}
\end{equation} 
Expression \eqref{eqPart4QuantCompShorPFExample2} contains 10 terms
of equal amplitude, so the normalization factor is
$\frac{1}{\sqrt{10}}$.

The Fourier transform of the sequence
\eqref{eqPart4QuantCompShorPFExample2} is shown in the lower graph
in \autoref{picPart4QuantCompShorQuantPart}. The most probable
values of the Fourier transform measurement result will be those corresponding to local maxima which repeat with a period 
$\frac{M}{r}\approx10.67$ from which the period of the desired function
$r=6$ can be found. 

\end{example}