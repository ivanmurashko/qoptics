\section{Quantum Fourier Transform and Discrete Logarithms}

The discrete logarithm (see \autoref{AddDiscretLog}) is the basis for many modern cryptographic algorithms (see \autoref{sec:add:dm:elgamal}, \autoref{sec:add:dm:dh}). Meanwhile, Shor's method for factoring integers can also be applied to compute discrete logarithms, making it possible to break the corresponding cryptographic algorithms.

Let's set the task as follows: there is an expression 
\[
b = a^x \mod p,
\]
where the numbers $a, b,$ and $p$ are given, and the number $x$ is unknown, which needs to be determined. Analogous to the application of the quantum Fourier transform for factoring numbers (see \autoref{sec:part4:algoshor:periodfind}), we must construct some periodic function whose period will allow us to determine the unknown number $x$. We choose the function to be analyzed as
\begin{equation}
f\left(x_1, x_2\right) = b^{x_1}a^{x_2} = a^{x \cdot x_1} a^{x_2} \mod p
\label{eq:part4:quantcomp:discretlogfunc}
\end{equation}

As an example, we will consider the quantum analog of solving the problem from example \ref{ex:dm:discretlog}:
\begin{example}
\emph{($ind_3{14} \mod{17}$)}
\input ./part4/quantcomp/figdiscretlog0.tex
In our example $p = 17$, $b=14$, and $a=3$. The function \eqref{eq:part4:quantcomp:discretlogfunc} is
\[
f\left(x_1, x_2\right) = b^{x_1}a^{x_2} = 14^{x_1}3^{x_2}.
\]
and is depicted in \autoref{fig:part4:quantcomp:dl0}.

Both $b=14$ and $a=3$ are generators of $\left(\mathbb{Z}/17\mathbb{Z}\right)^\times$. Moreover, $3 \equiv 14^9 \mod 17$. The periods of the function shown, as seen from \autoref{fig:part4:quantcomp:dl0}, will be the following numbers 
\begin{eqnarray}
t_1 \equiv 1 \mod 16,
\nonumber \\
t_2 \equiv 9 \mod 16
\end{eqnarray} 
\label{ex:part4:quantcomp:discretlog:periodfinding0}
\end{example}

In analogy with the factorization problem, you measure this function. Suppose the measurement yields a number $c \in \left(\mathbb{Z}/p\mathbb{Z}\right)^\times$. Given that $a$ is a generating element (see definition \ref{def:add:algebra:cyclic_group}) of the multiplicative group $\left(\mathbb{Z}/p\mathbb{Z}\right)^\times$ (see definition \ref{def:add:algebra:mult_group}) $\exists x_0: c = a^{x_0}$. Then, considering \myref{addDiscretSmallFerma}{Fermat's Little Theorem} $a^{p-1} \equiv 1 \mod p$, it follows that
\[
x_0 \equiv x x_1 + x_2 \mod q,
\] 
where $q = p - 1$. From this expression, it follows that
\[
x_2 \equiv x_0 - x x_1 \mod q.
\]
This means if the function is periodic in the first argument:
\[
f(x_1 + t_1, x_2) = f(x_1,x_2),
\]
then it will also be periodic in the second argument
\[
f(x_1, x_2 + t_2) = f(x_1,x_2),
\]
where 
\begin{equation}
t_2 \equiv x t_1 \mod q.
\label{eq:part4:quantcomp:discretlogeq}
\end{equation}

\subsection{Two-Dimensional Fourier Transform and Period of the Function $f(x_1, x_2)$}
Our function of interest will be:
\begin{equation}
\label{eq:part4:quantcomp:shordiscretlog:fprime}
f'\left(x_1, x_2\right) = 
\begin{cases}
1, x x_1 + x_2 \equiv x_0 \mod q \\
0, x x_1 + x_2 \not\equiv x_0 \mod q \\
\end{cases}
\end{equation}
\begin{example}
\emph{($ind_3{14} \mod{17}$)}
\input ./part4/quantcomp/figdiscretlog1.tex
Continuing example \ref{ex:part4:quantcomp:discretlog:periodfinding0}, suppose the measurement of function $f$ yields $f = 3$. That is, $f = a^{x_0} = 3^{x_0} \equiv 3 \mod 17$. The only values for $x_1, x_2$ that remain are those corresponding to the observed function value (see \autoref{fig:part4:quantcomp:dl1}), i.e., $x x_1 + x_2 = x_0 \equiv 1 \mod 16$. With fixed values of $x, x_1$, and the number of counts $M = q = 16$. 
\label{ex:part4:quantcomp:discretlog:periodfinding1}
\end{example}

For the Fourier transform $\tilde{f'}$, we have 
\begin{eqnarray}
\tilde{f'}\left(j_1, j_2\right) = 
\frac{1}{M}\sum_{x_1 = 0}^{M-1}\sum_{x_2 = 0}^{M-1} 
f'\left(x_1, x_2\right)e^{-i \omega\left(x_1 j_1 + x_2j_2\right)},
\label{eq:part4:quantcomp:discretlog:ftq16_pre}
\end{eqnarray}
where $\omega = \frac{2 \pi}{M}$, $M$ is the number of samples. 

First, consider the case when $M = q$. There are two options for $x_2$:  
\begin{enumerate}
\item $x_2 = x_0 - x x_1$, if $x_0 \ge x x_1$
\item $x_2 = x_0 + q - x x_1$, if $x_0 < x x_1$
\end{enumerate}
Thus,
\begin{eqnarray}
e^{-i \omega x_2j_2} = e^{-i \omega\left(x_0 - x x_1 + q\right) j_2} = 
\nonumber \\
= e^{-i \omega\left(x_0 - x x_1\right) j_2 - i \omega q j_2} = 
e^{-i \omega\left(x_0 - x x_1\right) j_2},
\nonumber
\end{eqnarray}
i.e., both options coincide and can be reduced to the first:
$x_2 = x_0 - x x_1$.

Continuing from \eqref{eq:part4:quantcomp:discretlog:ftq16_pre}, we have
\begin{eqnarray}
\tilde{f'}\left(j_1, j_2\right) = 
\frac{1}{M}\sum_{x_1 = 0}^{M-1}\sum_{x_2 = 0}^{M-1} 
f'\left(x_1, x_2\right)e^{-i \omega\left(x_1 j_1 + x_2j_2\right)} =
\nonumber \\
=
 \frac{1}{M}\sum_{x_1 = 0}^{M-1}
e^{-i \omega\left(x_1 j_1 + (x_0 - x
   x_1) j_2\right)} = 
\nonumber \\
= e^{-i \omega x_0 j_2}\frac{1}{M}\sum_{x_1 = 0}^{M-1}
e^{-i  \omega x_1 \left(j_1 - x j_2\right)} =
\frac{1}{M} e^{-i \omega x_0 j_2} 
\sum_{x_1 = 0}^{M-1} e^{-i  \omega x_1 \left(j_1 - x j_2\right)}.
\label{eq:part4:quantcomp:discretlog:ftq16}
\end{eqnarray}
In expression \eqref{eq:part4:quantcomp:discretlog:ftq16}, $\tilde{f'}(j_1, j_2) = e^{-i \omega x_0 j_2} \ne 0$, if 
\begin{equation}
j_1 \equiv x j_2 \mod M.
\label{eq:part4:quantcomp:discretlog:j1xj2}
\end{equation} 
Otherwise, using the geometric progression formula:
\begin{eqnarray}
\tilde{f'}\left(j_1 \ne x j_2, j_2\right) = 
e^{-i \omega x_0 j_2}\frac{1}{M}
\sum_{x_1 = 0}^{M-1}e^{-i
  \omega x_1 \left(j_1 - x j_2\right)} = 
\nonumber \\
=
\frac{e^{-i \omega x_0 j_2}}{M} \frac{e^{-i
  \omega M \left(j_1 - x j_2\right)} - 1}{e^{-i
  \omega \left(j_1 - x j_2\right)} - 1} = 
\nonumber \\
=
 \frac{e^{-i \omega x_0 j_2}}{M} 
\frac{e^{-i \frac{2 \pi}{M} M \left(j_1 - x j_2\right)} - 1}{e^{-i
  \omega \left(j_1 - x j_2\right)} - 1} = 0.
\nonumber
\end{eqnarray} 

Thus, to determine the period, we need to find the coordinates $(j_1, j_2)$ of a maximum of the Fourier transform and use the expression 
\begin{equation}
x \equiv j_1 j_2^{-1} \mod M,
\label{eq:part4:quantcomp:discretlog:periodfourier}
\end{equation}
which follows from \eqref{eq:part4:quantcomp:discretlog:j1xj2}.

\begin{remark}[On Zero Divisors in $\mathbb{Z}/M\mathbb{Z}$]
If there exists such a number $y$ that 
\[
j_2 y \equiv 0 \mod M,
\]
then $j_2$ is called a zero divisor. 
It is obvious that 
\[
\gcd\left(j_2, M\right) \ne 1,
\]
thus, from \autoref{sec:add:discretmath:mod:equationsolve}, it follows that $j_2^{-1}$ does not exist. Therefore, for such $j_2$, expression \eqref{eq:part4:quantcomp:discretlog:periodfourier} is undefined. In that case, different coordinates $(j_1, j_2)$ must be used. 
\end{remark}

\begin{example}
\emph{($ind_3{14} \mod{17}$)}
\input ./part4/quantcomp/figdiscretlog2.tex

The Fourier transform of the function from \autoref{fig:part4:quantcomp:dl1} is shown in \autoref{fig:part4:quantcomp:dl2}, from which it is clear that the counts most likely to be registered follow at intervals of $T_{j_1} = 9$ along the $j_1$ coordinate and at intervals of $T_{j_2} = 1$ along the $j_2$ coordinate. Considering the number of counts $M=16$, the coordinates of the Fourier transform maximum can be obtained as $j_1 = 9$ and $j_2 = 1$. The solution to the equation $3^x \equiv 14 \mod 17$ is, according to \eqref{eq:part4:quantcomp:discretlog:periodfourier}, $x = 9 \cdot 1^{-1} = 9$, which corresponds to the result of example \ref{ex:dm:discretlog}.

A similar result can be obtained if the coordinates $j_1 = 11, j_2 = 3$ are considered. Given $3 \cdot 11 = 33 \equiv 1 \mod 16$, it follows that $j_2^{-1} \equiv 11 \mod 16$, i.e., $x \equiv 11 \cdot 11 \equiv 121 \equiv 9 \mod 16$, which again corresponds to the result of example \ref{ex:dm:discretlog}.

It is worth noting that points lying on the diagonal, for example, $j_1 = 6, j_2 = 6$, will not yield a correct result because $\gcd\left(6, 16\right) = 2 \ne 1$.

\label{ex:part4:quantcomp:discretlog:periodfinding}
\end{example}

It is worth noting that the obtained result \eqref{eq:part4:quantcomp:discretlog:periodfourier} is in direct correspondence with lemma \ref{lemmaAddFourierDiscretFourierPeriod} for the one-dimensional Fourier transform. Furthermore, an analogue of comment \ref{rem:dsp:fourier:periodprop} holds, stating that when the number of Fourier transform counts is not equal to $q$: $M \ne q$, but $M \approx q$, we can still approximately consider expression \eqref{eq:part4:quantcomp:discretlog:periodfourier} to be valid \cite{Proos:2003:SDL:2011528.2011531}.  

\begin{example}
\emph{($ind_2{14} \mod{59}$)}

As an example, consider $p = 59$ where the number of counts $M = 64 \approx q = p - 1 = 58$. The generator of the group $\mathbb{F}_{59}$ (see \autoref{sec:add:diskretmath:mod:fp}) is $g = 2$, i.e., $\mathbb{F}_{59} = \left<2\right>$. This means that the equation $2^x \equiv b \mod 59$ will have a solution for any $b$, in particular, $x = 19$ is a solution to the equation
\[
2^x \equiv 14 \mod 59.
\] 

The function under study is
\[
f(x_1, x_2) = 14^{x_1} 2^{x_2} \mod 59,
\]
Suppose $x_0 = 50$, i.e., the registered value of the function is $f(x_1, x_2) = 2^{x_0} = 2^{50} \equiv 3 \mod 59$.

\input ./part4/quantcomp/figdiscretlog4.tex

As mentioned earlier, the number of Fourier transform counts is $M=64$. Note that $q = p - 1 = 58 \not\equiv 0 \mod 64$.

The Fourier transform of the function 
\[
f'(x_1, x_2) = 
\begin{cases}
1, \mbox{ if } 14^{x_1} 2^{x_2} \equiv 3 \mod 59 \\
0, \mbox{ if } 14^{x_1} 2^{x_2} \not\equiv 3 \mod 59 
\end{cases}
\]
is shown in \autoref{fig:part4:quantcomp:dl4}. 
The three lower maxima have coordinates 
\[
(j_1, j_2) \approx (20,1), (41,2.2), (62,3), 
\]
which give the following estimates for $x$: $x \approx 20, 18.6, 20.6$, close to the actual value $x = 19$.
\label{ex:part4:quantcomp:discretlog:periodfinding3}
\end{example}

\begin{example}
\emph{($ind_3{14} \mod{19}$)}

As an example, consider the task of determining $x$ such that 
\[
3^x \equiv 14 \mod 19.
\]

The function under study is
\[
f(x_1, x_2) = 14^{x_1} 3^{x_2} \mod 19,
\]
Suppose $x_0 = 1$, i.e., the registered value of the function is $f(x_1, x_2) = 3$.

\input ./part4/quantcomp/figdiscretlog3.tex

The number of Fourier transform counts is chosen as $M=64$. Note that $18 \not\equiv 0 \mod 64$.

The Fourier transform of the function 
\[
f'(x_1, x_2) = 
\begin{cases}
1, \mbox{ if } 14^{x_1} 3^{x_2} \equiv 3 \mod 19 \\
0, \mbox{ if } 14^{x_1} 3^{x_2} \not\equiv 3 \mod 19 
\end{cases}
\]
is shown in \autoref{fig:part4:quantcomp:dl3}. The lowest maximum
has coordinates $j_1 = 46, j_2 = 3.5$, from which we get an estimate 
\[
x \approx \frac{46}{3.5} \approx 13.14.
\]
It is noteworthy that the solution to the desired equation $x = 13$ is close to the approximate solution found.
\label{ex:part4:quantcomp:discretlog:periodfinding2}
\end{example}

\subsection{Two-Dimensional Quantum Fourier Transform}

\input part4/quantcomp/figquantfourier2d.tex

To determine the periods of a function with two arguments, we can use a two-dimensional Fourier transform, which can be constructed from blocks implementing the one-dimensional Fourier transform, as depicted in \autoref{figQuantCompQuantFourier2d}. For analyzing this circuit, consider the trivial case where the input is (see also \eqref{eqPart4QuantCompShorQuantFourierSeries})
\begin{eqnarray}
\ket{x} = \ket{x}_1 \otimes \ket{x}_2,
\nonumber \\
\ket{x}_{1,2} = \sum_{k_{1,2} = 0}^{M-1} x_{k_{1,2}}^{(1,2)} \ket{k_{1,2}}.
\nonumber
\end{eqnarray}
Given that the output is
\[
\ket{\tilde{X}} = \ket{\tilde{X}_1} \otimes \ket{\tilde{X}_2},
\]
where
\[
\ket{\tilde{X}_{1,2}} = \sum_{j_{1,2} = 0}^{M-1} \tilde{X}_{j_{1,2}}^{(1,2)} \ket{j_{1,2}}
\]
and in accordance with \eqref{eqPart4QuantCompShorQuantFourierEnd}
\[
\tilde{X}_{j_{1,2}}^{(1,2)} = \frac{1}{\sqrt{M}}\sum_{k_{1,2} = 0}^{M - 1}e^{-i \omega_{1,2} k_{1,2} j_{1,2}} x_{k_{1,2}}^{(1,2)}.
\]
we obtain
\begin{eqnarray}
\ket{\tilde{X}} = \ket{\tilde{X}_1} \otimes \ket{\tilde{X}_2} = 
\nonumber \\
= \sum_{j_1 = 0}^{M-1}\sum_{j_2 = 0}^{M-1}
\tilde{X}_{j_{1}}^{(1)} \tilde{X}_{j_{2}}^{(2)} \ket{j_1} \otimes \ket{j_2} =
\nonumber \\
= \sum_{j_1 = 0}^{M-1}\sum_{j_2 = 0}^{M-1}
\tilde{X}_{j_{1},j_{2}} \ket{j_1} \otimes \ket{j_2}, 
\nonumber
\end{eqnarray}
where
\begin{eqnarray}
\tilde{X}_{j_{1},j_{2}} = \frac{1}{\left( \sqrt{M} \right)^2} 
\sum_{k_{1} = 0}^{M - 1}\sum_{k_{2} = 0}^{M - 1}
e^{-i \omega \left( k_{1} j_{1} + k_{2} j_{2}\right)}
x_{k_1}^{(1)}x_{k_2}^{(2)} =
\nonumber \\
= \frac{1}{\left( \sqrt{M} \right)^2}
\sum_{k_{1} = 0}^{M - 1}\sum_{k_{2} = 0}^{M - 1}
e^{-i \omega \left( k_{1} j_{1} + k_{2} j_{2}\right)}
x_{k_1, k_2}
\nonumber
\end{eqnarray}
which, in accordance with definition \ref{def:add:dsp:fourier2d}, is a two-dimensional Fourier transform of the initial two-dimensional signal
\[
\ket{x} = 
\sum_{k_1 = 0}^{M-1}\sum_{k_2 = 0}^{M-1}
x_{k_1}^{(1)}x_{k_2}^{(2)} \ket{k_1} \otimes \ket{k_2} =
\sum_{k_1 = 0}^{M-1}\sum_{k_2 = 0}^{M-1}
x_{k_1,k_2} \ket{k_1} \otimes \ket{k_2}.  
\]

\input part4/quantcomp/figquantperiodfinding2.tex

Therefore, using the scheme shown in \autoref{figQuantCompQuantPeriodFinding2}, the coordinates of the maxima of the two-dimensional Fourier transform $j_1, j_2$ can be determined and subsequently used in \eqref{eq:part4:quantcomp:discretlog:periodfourier} to find the desired $x$.