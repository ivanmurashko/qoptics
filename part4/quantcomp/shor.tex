%% -*- coding:utf-8 -*- 
\section{Shor's Algorithm}
\label{Part4QuantCompShor}
One of the most popular RSA encryption algorithms (see 
\autoref{AddRSA}) \rindex{RSA algorithm}
is based on the assumption of the complexity of factoring
(decomposing into prime factors) large numbers. Accordingly,
algorithms that enable decomposition into prime factors
are of particular interest. Below is a description of such an
algorithm proposed by Shor \cite{bShor94}.

\subsection{Factoring Numbers and Finding the Period of Functions}
\label{sec:part4:algoshor:periodfind}
The task of factoring a number $N$ is closely related to finding the period
of functions. Consider the following function, which is called the exponentiation
function modulo
\begin{equation}
f\left(x, a\right) = a^x \mod N.
\label{eqPart4QuantCompShorClassPart}
\end{equation}
Function \eqref{eqPart4QuantCompShorClassPart} depends on
the analyzed number $N$ and two arguments $x$ and $a$. The argument $a$
is chosen under the following conditions
\begin{eqnarray}
0 < a < N,
\nonumber \\
\mbox{GCD}\left(N, a\right) = 1.
\label{eqPart4QuantCompShorAConditions}
\end{eqnarray}
A typical graph of the function \eqref{eqPart4QuantCompShorClassPart} is shown in
\autoref{picPart4QuantCompShorClassPart}.

% maxima a = 2 N = 21
%draw2d(yrange=[0, 18], points_joined = true, point_type = 6,
%point_size = 2, points(makelist([x, power_mod(2, x, 21)], x, 0, 30)),
%user_preamble="set output 'picshorclass.tex'; set terminal latex; set
%xlabel '$x$'; set ylabel '$f(x)$'");

\input ./part4/quantcomp/figshorclass.tex

The conditions for choosing coefficient $a$
\eqref{eqPart4QuantCompShorAConditions} are such that $a$ and $N$ do not have
common divisors. If such divisors exist, they are
the desired solution for the factoring problem and can be easily found
using the Euclidean algorithm (see \autoref{AddEuclidean}).

Function \eqref{eqPart4QuantCompShorClassPart} is periodic,
i.e., there exists a number $r$ such that $f\left(x + r, a\right) = 
f\left(x, a\right)$. The smallest possible number $r$ is called
the period of the function \eqref{eqPart4QuantCompShorClassPart}. 

To prove periodicity, note that $f\left(x, a\right)$ cannot
be zero. Indeed, if the condition
$f\left(x, a\right) = 0$ is fulfilled, then 
\[
\exists x \in \left\{0, 1, \dots\right\}:
a^x = k \cdot N,
\]
where $k$ is an integer, which is not possible due to
the coprimality of $a$ and $N$ \eqref{eqPart4QuantCompShorAConditions}
\footnote{It is assumed here, obviously, that $N > 1$}.

Thus, the range of values of function
\eqref{eqPart4QuantCompShorClassPart} is limited to the set 
\begin{equation}
f\left(x,
a\right) \in \left\{1, \dots, N - 1\right\},
\nonumber
\end{equation}
from which 
\begin{equation}
\exists k,j: k > j, k,j \in \left\{0, 1, \dots, N\right\},
f\left(k,a\right) = f\left(j,a\right),
\nonumber
\end{equation}
which proves the periodicity of the function \eqref{eqPart4QuantCompShorClassPart}.

Let $k = j + r$, then
\[
a^k \mod{N} = a^{j + r} \mod{N} = a^j a^r \mod{N}= a^j \mod{N},
\]
since $a$ and $N$ are coprime, we can write
\begin{equation}
a^r \equiv 1 \mod{N}.
\label{eqPart4QuantCompShorPeriodConditions}
\end{equation}

The period of the function \eqref{eqPart4QuantCompShorClassPart} can be either
even or odd. In Shor's algorithm, we are interested in the first case:
period - even number. Otherwise, a new number $a$ is chosen and
the period is found again. Thus, considering $r= 2\cdot l$ we
can rewrite \eqref{eqPart4QuantCompShorPeriodConditions} as
\begin{equation}
a^{2 \cdot l} \equiv 1 \mod{N},
\nonumber
\end{equation}
and since $r$ is the smallest number satisfying the periodicity condition,
\[
a^{l} \not\equiv 1 \mod{N}.
\]
If, in addition, a number $a$ is selected such that 
\[
a^{l} \not\equiv -1 \mod{N},
\]
then we have 
\begin{equation}
\left(a^l - 1\right)\left(a^l + 1\right) = k \cdot N,
\label{eqPart4QuantCompShorPeriodFinal}
\end{equation}
where $k$ is some integer. From
\eqref{eqPart4QuantCompShorPeriodFinal} we find that $a^l \pm 1$ have
nontrivial common divisors (other than 1) with $N$.

\begin{example}
\emph{Finding divisors of the number $N=21$}
\label{exPart4QuantCompShorGCD}
As an example, consider the task of finding the divisors of the number $N =
21$. Choosing $a=2$, we get the period of
function \eqref{eqPart4QuantCompShorClassPart} $r = 6$ (see
\autoref{picPart4QuantCompShorClassPart}). 
Obviously, 
\[
2^3 \equiv 8 \mod{21} \not\equiv -1 \mod{21}.
\]
Thus, finding the corresponding greatest common divisors solves
the task
\begin{eqnarray}
\mbox{GCD}\left( 2^3 - 1, 21 \right) = \mbox{GCD}\left( 7, 21 \right)
= 7,
\nonumber \\
\mbox{GCD}\left( 2^3 + 1, 21 \right) = \mbox{GCD}\left( 9, 21 \right)
= 3,
\nonumber \\
21 = 7 \cdot 3.
\nonumber
\end{eqnarray}
\end{example}

Thus, the task of factoring the number $N$ can be reduced to
the task of finding the period of a certain function using the following
algorithm:
\rindex{Shor's algorithm}
\input part4/quantcomp/shoralgo.tex