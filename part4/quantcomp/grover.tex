%% -*- coding:utf-8 -*- 
\section{Grover's Algorithm}
\rindex{Grover's algorithm}
Consider the following problem. Suppose there is a large data set
consisting of $N$ elements in which it is necessary to find an element
satisfying certain conditions (see \autoref{figQuantCompSearch}).
If the data are sorted, then using algorithms of the ``divide and conquer'' type,
the desired element can be found in time on the order of $O\left(\log N\right)$
(see \autoref{addDivideAndConquer}). In some cases, the original data set
cannot be prepared for fast search; in this case,
a classical search takes time on the order of $O\left(N\right)$.

\input ./part4/quantcomp/figsearch.tex

One example is symmetric encryption algorithms, 
where the task is to determine the key given known ciphertext
and its corresponding original text. In this case,
preprocessing the data seems impossible, and the straightforward
solution to the problem is a simple brute force search over all possible values.

Grover's algorithm \cite{Grover96afast} solves the
unstructured search problem in time on the order of $O\left(\sqrt{N}\right)$.

\subsection{Algorithm Description}

Suppose we have a quantum circuit that computes the value
of a function $f\left(x\right)$ which can take only two values:
$0$ and $1$. Here, the value $1$ is true only for the desired
element:
\begin{eqnarray}
\left.f\left(x\right)\right|_{x = x^{\ast}} = 1,
\nonumber \\
\left.f\left(x\right)\right|_{x \ne x^{\ast}} = 0.
\label{eqQuantCompGroverF}
\end{eqnarray}
\autoref{figQuantCompGrover1} shows a scheme for computing
the desired function. At the output we have a state of the form
\begin{equation}
\ket{out} = \frac{1}{\sqrt{N}}\left(
 \sum_{x \ne x^{\ast}} \ket{x}\otimes\ket{0}
+ \left|x^{\ast}\right>\otimes\ket{1}
\right),
\label{eqQuantCompGroverFake}
\end{equation}
where $N$ is the total number of elements in the sequence in which
the search is performed.

\input ./part4/quantcomp/figgroverfake.tex

Looking at expression \eqref{eqQuantCompGroverFake}, one can
note that the proposed scheme, although it computes the function at the desired point,
does not allow selecting the sought element, because all elements of the resulting sequence
are equally probable, i.e., each element can be selected (as a result of measurement) with equal probability: $\frac{1}{N}$.

Grover proposed an algorithm that would increase
the probability of detecting the desired element in the resulting
superposition \eqref{eqQuantCompGroverFake}.

\input ./part4/quantcomp/figgrover.tex

\input ./part4/quantcomp/figgroverbase.tex

The scheme implementing Grover's algorithm is a certain block
described by the operator $\hat{U}_G$, which is repeated a certain number
of times (see \autoref{figQuantCompGrover}). At each step
of iteration, the probability of detecting the desired element increases.

The basic element $\hat{U}_G$ represents a sequential application
of two operators (see \autoref{figQuantCompGroverBase}):
\begin{equation}
\hat{U}_G=\hat{U}_s\hat{U}_{x^{\ast}},
\nonumber
\end{equation}
where $\hat{U}_{x^{\ast}}$ is the phase inversion operator, and $\hat{U}_s$
is the inversion about the mean operator.

\input ./part4/quantcomp/figgroverinv.tex

The action of the operator $\hat{U}_{x^{\ast}}$ is described by the following relation
(see \autoref{figQuantCompGroverInv}):
\begin{equation}
\hat{U}_{x^{\ast}}\left(\sum_x \alpha_x \ket{x}\right) = 
\sum_x \alpha_x \left(-1\right)^{f\left(x\right)}\ket{x}.
\label{eqQuantCompGroverUxast}
\end{equation} 
The operator $\hat{U}_{x^{\ast}}$ can be rewritten in the form
\begin{equation}
\hat{U}_{x^{\ast}} = \hat{I} - 2 \left|x^{\ast}\right>\left<x^{\ast}\right|.
\nonumber
\end{equation} 
Indeed,
\begin{eqnarray}
\left(\hat{I} - 2 \left|x^{\ast}\right>\left<x^{\ast}\right|\right)
\left(\sum_x \alpha_x \ket{x}\right) =
\nonumber \\
= \sum_x \alpha_x \ket{x} - 2 \alpha_{x^{\ast}}
\left|x^{\ast}\right> = 
\sum_{x\ne x^{\ast}} \alpha_x \ket{x} -  \alpha_{x^{\ast}}
\left|x^{\ast}\right> =
\nonumber \\
=
\sum_x \alpha_x \left(-1\right)^{f\left(x\right)}\ket{x},
\nonumber
\end{eqnarray}
which coincides with \eqref{eqQuantCompGroverUxast}.

\input ./part4/quantcomp/figgroverinvmiddle.tex

The action of the operator $\hat{U}_s$ is described by the following relation
(see \autoref{figQuantCompGroverInvMiddle}):
\begin{equation}
\hat{U}_s\left(\sum_x \alpha_x \ket{x}\right) = 
\sum_x \left(2 \mathcal{M} - \alpha_x \right)\ket{x},
\label{eqQuantCompGroverUs}
\end{equation} 
where $\mathcal{M} = \sum_x \frac{\alpha_x}{N}$.

The operator $\hat{U}_s$ can be rewritten in the following form
\begin{equation}
\hat{U}_s = 
2 \ket{s}\bra{s} - \hat{I},
\nonumber
\end{equation}
where $\ket{s}=\frac{1}{\sqrt{N}}\sum_x \ket{x}$ -
the initial state in Grover's algorithm.
Indeed,
\begin{eqnarray}
\left(2 \ket{s}\bra{s} - \hat{I}\right)
\left(\sum_x \alpha_x \ket{x}\right) =
\nonumber \\
=  2 \sum_x \alpha_x \bra{s}\ket{x} \ket{s} 
- \sum_x \alpha_x \ket{x} = 
\nonumber \\
=
\frac{2}{N} \sum_x \alpha_x \sum_x \ket{x} -
\sum_x \alpha_x \ket{x} = 
\nonumber \\
= \sum_x \left( 2 \mathcal{M} -\alpha_x \right) \ket{x},
\nonumber
\end{eqnarray}
which coincides with \eqref{eqQuantCompGroverUs}.

\input part4/quantcomp/groveranalyze.tex

\input part4/quantcomp/groverrealize.tex