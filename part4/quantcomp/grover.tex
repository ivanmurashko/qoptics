%% -*- coding:utf-8 -*- 
\section{Grover's Algorithm}
\rindex{Grover's algorithm}
Consider the following problem. Suppose we have a large dataset
consisting of $N$ elements in which it is necessary to find an element
satisfying certain conditions (see \autoref{figQuantCompSearch}).
If the data is sorted, the desired element can be
found using "divide and conquer" type algorithms in time $O\left(log N\right)$
(see \autoref{addDivideAndConquer}). In some cases, the original dataset
cannot be prepared for fast search, and in this case
classical search is performed in time $O\left(N\right)$.

\input ./part4/quantcomp/figsearch.tex

One example is symmetric encryption algorithms
where the task is to determine the key from the known encrypted
text and its corresponding original text. In this case,
preprocessing the data is impossible, and the "brute force" solution
is simply to try all possible values.

Grover's algorithm \cite{Grover96afast} solves the problem
of unstructured search in time $O\left(\sqrt{N}\right)$.

\subsection{Description of the Algorithm}

Suppose we have a quantum circuit that computes the value
of the function $f\left(x\right)$ which can take only two values:
$0$ and $1$. The value $1$ is true only for the desired
element: 
\begin{eqnarray}
\left.f\left(x\right)\right|_{x = x^{\ast}} = 1,
\nonumber \\
\left.f\left(x\right)\right|_{x \ne x^{\ast}} = 0.
\label{eqQuantCompGroverF}
\end{eqnarray}
In \autoref{figQuantCompGrover1} a scheme for calculating
the desired function is shown. At the output, we have a state of the form
\begin{equation}
\ket{out} = \frac{1}{\sqrt{N}}\left(
 \sum_{x \ne x^{\ast}} \ket{x}\otimes\ket{0}
+ \left|x^{\ast}\right>\otimes\ket{1}
\right),
\label{eqQuantCompGroverFake}
\end{equation}
where $N$ is the total number of elements in the sequence in which
the search is conducted.

\input ./part4/quantcomp/figgroverfake.tex

If you look at expression \eqref{eqQuantCompGroverFake}, you can
note that although the scheme calculates
the function at the desired point, it does not allow selecting the desired
element, because all elements of the resulting sequence are equally probable, i.e., each element can be selected (as a result
of measurement) with equal probability: $\frac{1}{N}$.

Grover proposed an algorithm that increases the
probability of detecting the desired element in the resulting
superposition \eqref{eqQuantCompGroverFake}.

\input ./part4/quantcomp/figgrover.tex

\input ./part4/quantcomp/figgroverbase.tex

The scheme implementing Grover's algorithm represents a block
described by the operator $\hat{U}_G$, which is repeated a certain number
of times (see \autoref{figQuantCompGrover}). With each
iteration step, the probability of detecting the desired element increases.

The basic element $\hat{U}_G$ is a sequential action
of two operators (see \autoref{figQuantCompGroverBase}):
\begin{equation}
\hat{U}_G=\hat{U}_s\hat{U}_{x^{\ast}},
\nonumber
\end{equation}
where $\hat{U}_{x^{\ast}}$ is the phase inversion operator, $\hat{U}_s$
is the inversion about the mean operator.

\input ./part4/quantcomp/figgroverinv.tex

The action of the operator $\hat{U}_{x^{\ast}}$ is described by the following relation
(see \autoref{figQuantCompGroverInv}):
\begin{equation}
\hat{U}_{x^{\ast}}\left(\sum_x \alpha_x \ket{x}\right) = 
\sum_x \alpha_x \left(-1\right)^{f\left(x\right)}\ket{x}.
\label{eqQuantCompGroverUxast}
\end{equation} 
The operator $\hat{U}_{x^{\ast}}$ can be rewritten as
\begin{equation}
\hat{U}_{x^{\ast}} = \hat{I} - 2 \left|x^{\ast}\right>\left<x^{\ast}\right|.
\nonumber
\end{equation} 
Indeed,
\begin{eqnarray}
\left(\hat{I} - 2 \left|x^{\ast}\right>\left<x^{\ast}\right|\right)
\left(\sum_x \alpha_x \ket{x}\right) =
\nonumber \\
= \sum_x \alpha_x \ket{x} - 2 \alpha_{x^{\ast}}
\left|x^{\ast}\right> = 
\sum_{x\ne x^{\ast}} \alpha_x \ket{x} -  \alpha_{x^{\ast}}
\left|x^{\ast}\right> =
\nonumber \\
=
\sum_x \alpha_x \left(-1\right)^{f\left(x\right)}\ket{x},
\nonumber
\end{eqnarray}
which coincides with \eqref{eqQuantCompGroverUxast}.

\input ./part4/quantcomp/figgroverinvmiddle.tex

The action of the operator $\hat{U}_s$ is described by the following relation
(see \autoref{figQuantCompGroverInvMiddle}):
\begin{equation}
\hat{U}_s\left(\sum_x \alpha_x \ket{x}\right) = 
\sum_x \left(2 \mathcal{M} - \alpha_x \right)\ket{x},
\label{eqQuantCompGroverUs}
\end{equation} 
where $\mathcal{M} = \sum_x \frac{\alpha_x}{N}$.

The operator $\hat{U}_s$ can be rewritten as follows
\begin{equation}
\hat{U}_s = 
2 \ket{s}\bra{s} - \hat{I},
\nonumber
\end{equation}
where $\ket{s}=\frac{1}{\sqrt{N}}\sum_x \ket{x}$ is the initial state in Grover's algorithm.
Indeed,
\begin{eqnarray}
\left(2 \ket{s}\bra{s} - \hat{I}\right)
\left(\sum_x \alpha_x \ket{x}\right) =
\nonumber \\
=  2 \sum_x \alpha_x \bra{s}\ket{x} \ket{s} 
- \sum_x \alpha_x \ket{x} = 
\nonumber \\
=
\frac{2}{N} \sum_x \alpha_x \sum_x \ket{x} -
\sum_x \alpha_x \ket{x} = 
\nonumber \\
= \sum_x \left( 2 \mathcal{M} -\alpha_x \right) \ket{x},
\nonumber
\end{eqnarray}
which coincides with \eqref{eqQuantCompGroverUs}.

\input part4/quantcomp/groveranalyze.tex

\input part4/quantcomp/groverrealize.tex