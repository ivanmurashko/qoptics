%% -*- coding:utf-8 -*- 
From the moment the importance of information was realized, means of its protection began to appear. 
%% Among the first to apply information protection methods were the ancient Greeks. For these purposes, they used
%% the first encryption machine - the scytale. The scytale represented a
%% cone-shaped baton around which a strip of leather was wound. Then a message was written on the leather. When the strip was removed,
%% the records on it represented a certain permutation
%% of the symbols of the original message - the ciphertext. 

%% Since then, the science of information protection, called
%% cryptography, has made significant progress. 
New encryption methods were invented, such as the Caesar cipher, where each letter
of the alphabet was replaced by another (for example, the one three positions after it
in the alphabet). Along with new encryption methods, techniques for breaking these ciphers appeared,
for example, for the Caesar cipher, one can use the statistical properties of the language in which
the original message was written.

Very often, the security of a cipher was ensured by keeping the encryption algorithm secret, as was the case
with the Caesar cipher mentioned above. In modern classical
cryptography, algorithms are usually published and available for study by anyone. Security
is ensured by mixing the message itself with a secret key
according to some public algorithm. 

Suppose we need to transmit a message from Alice to Bob via
some secure communication channel. The message must be
represented in some digital form.
The protocol describing such
transmission consists of several stages. In the first, Alice and Bob must
obtain a common random sequence of numbers, which will be called the key. This procedure is called key distribution. 

At the next stage, Alice should use
some algorithm $E$ to obtain from the original message $P$ and key $K$
the encrypted message $C$. This procedure can be described by the following relation: 
\begin{equation}
E_{K}\left(P\right) = C.
\label{eqPart3CryptoEncryptClass}
\end{equation}

At the third stage, the resulting encrypted message must be
transmitted to Bob.

In the last stage, Bob, using the known algorithm $D$ and the key $K$ obtained in
the first stage, should recover the original message $P$ from
the received encrypted message $C$. This procedure can be described
by the following relation
\begin{equation}
D_{K}\left(C\right) = P.
\label{eqPart3CryptoDeEncryptClass}
\end{equation}

Analyzing this protocol raises the following questions. How
to securely implement key distribution. Second - does
an absolutely secure algorithm exist. And finally - is
it possible to securely transmit an encrypted message without it being
eavesdropped on or tampered with. 

Classical cryptography gives a definitive answer only to the second
question. An absolutely secure algorithm exists - it is called
the one-time pad. A detailed description of this algorithm is given below.
