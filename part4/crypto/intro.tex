%% -*- coding:utf-8 -*- 
From the moment the importance of information was recognized, means to protect it began to emerge. 
%% Some of the first to apply methods
%% for protecting information were the ancient Greeks. For these purposes, they used
%% the first cipher machine - the scytale. The scytale consisted of
%% a cone-shaped baton around which a strip of leather was wrapped. Subsequently, a message was written on the leather. When the strip of leather
%% was removed, the inscriptions on it represented some permutation
%% of the original message's characters - the ciphertext. 

%% Since then, the science of information protection, called
%% cryptography, has come a long way.
New encryption methods were invented, such as the Caesar cipher, in which each letter
of the alphabet was replaced by another (for example, the one three positions ahead in
the alphabet). Along with new encryption methods, methods of breaking these ciphers also appeared, for example, for the Caesar cipher, one could
use the statistical properties of the language in which
the original message was written.

Very often the security of a cipher was ensured by keeping the encryption algorithm secret, as in the
Caesar cipher mentioned above. In modern classical
cryptography, algorithms are
often published and available for everyone to study. Secrecy
is ensured by mixing the message with a secret key
according to some open algorithm. 

Suppose we need to send a message from Alice to Bob via
a secure communication channel. The message must be
presented in a digital form.
The protocol describing such
a transfer consists of several stages. In the first stage, Alice and Bob must
obtain a shared random sequence of numbers, called the key. This procedure is called key distribution. 

In the next stage, Alice must use
an algorithm $E$ to obtain an encrypted message $C$ from the original message $P$ and the key $K$.
This procedure can be described by the following
relationship: 
\begin{equation}
E_{K}\left(P\right) = C.
\label{eqPart3CryptoEncryptClass}
\end{equation}

At the third stage, the obtained encrypted message must be
transmitted to Bob.

In the final stage, Bob uses the known algorithm $D$ and the key $K$ obtained in
the first stage to recover the original message $P$ from
the received encrypted $C$. This procedure can be described
by the following relationship
\begin{equation}
D_{K}\left(C\right) = P.
\label{eqPart3CryptoDeEncryptClass}
\end{equation}

Analyzing this protocol raises the following questions. How
to implement secure key distribution? Second - does an
absolutely secure algorithm exist? And finally, the last - is it
possible to securely transmit an encrypted message so that it cannot be
intercepted or altered? 

Classical cryptography provides a definitive answer only to the second
question. An absolutely unbreakable algorithm exists - it is called the
one-time pad. Below is a detailed description of this algorithm.