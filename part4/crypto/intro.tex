%% -*- coding:utf-8 -*- 
From the moment the importance of information was realized, means of its protection began to appear. 
%% Among the first to apply methods of information protection were the ancient Greeks. For this purpose, they used the first encryption machine - the scytale. The scytale was a cone-shaped baton around which a strip of leather was wound. Subsequently, a message was written on the leather. When the strip of leather was removed, the inscriptions on it represented a permutation of the characters of the original message - ciphertext.

%% Since then, the science of information protection, known as cryptography, has come a long way. 
New encryption methods were invented, such as the Caesar cipher, in which each letter of the alphabet was replaced with another one (for example, the one three positions further in the alphabet). Alongside new encryption methods, ways to crack these ciphers appeared, such as using statistical properties of the language in which the original message was written to crack the Caesar cipher.

Very often, the security of a cipher was ensured by keeping the encryption algorithm secret, as in the case of the Caesar cipher discussed above. In modern classical cryptography, algorithms are usually published and available for anyone to study. Secrecy is maintained by mixing the message with a secret key according to a certain open algorithm.

Suppose we need to transmit a message from Alice to Bob over a secure communication channel. The message must be presented in some digital form. The protocol describing this transmission involves several stages. Initially, Alice and Bob must obtain a common random sequence of numbers, which will be called a key. This process is called key distribution.

In the next stage, Alice must use a certain algorithm $E$ to obtain an encrypted message $C$ from the original message $P$ and key $K$. This process can be described by the following equation:
\begin{equation}
E_{K}\left(P\right) = C.
\label{eqPart3CryptoEncryptClass}
\end{equation}

In the third stage, the obtained encrypted message must be transmitted to Bob.

In the final stage, Bob, using a known algorithm $D$ and the key $K$ obtained in the first stage, must recover the original message $P$ from the received encrypted message $C$. This process can be described by the following equation:
\begin{equation}
D_{K}\left(C\right) = P.
\label{eqPart3CryptoDeEncryptClass}
\end{equation}

When analyzing this protocol, the following questions arise. How to implement secure key distribution? Secondly, is there an absolutely secure algorithm? And finally, is it possible to safely transmit an encrypted message such that it cannot be intercepted or altered?

Classical cryptography provides a clear answer only to the second question. An absolutely secure algorithm exists—it is called the one-time pad. A detailed description of this algorithm is presented below.