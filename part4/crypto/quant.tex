%% -*- coding:utf-8 -*- 
\section{Quantum Cryptography}
\label{subsecPart3QuantInfoQuantCrypto}
The randomness inherent in quantum objects suggests the idea
of using them for key distribution. There exist many different
secure key distribution schemes based on the use of
quantum objects. We will explore a scheme based on the verification
of Bell inequalities (see \ref{pPart3EntangleBell}). The considered scheme
is shown in \autoref{figPart3QuantInfoCryptoBell}.

\input ./part4/crypto/figquantcrypto.tex

A source of entangled photons \rindex{photon!entangled state} $S$ generates pairs of photons,
one of which is sent to Alice and the other to Bob, who perform
measurements of the Stokes parameters of their photons.

Alice randomly measures either $\hat{A} = \hat{S}_1^{(1)}$
or $\hat{A}' = \hat{S}_2^{(1)}$. Bob randomly measures the following quantities 
\begin{eqnarray}
\hat{B} = \frac{1}{\sqrt{2}}\left(\hat{S}_1^{(2)} +
  \hat{S}_2^{(2)}\right), 
\nonumber \\
\hat{B}' = \frac{1}{\sqrt{2}}\left(\hat{S}_1^{(2)} - \hat{S}_2^{(2)}\right),
\nonumber \\
\hat{C} = \hat{S}_1^{(2)},
\nonumber \\
\hat{C}' = \hat{S}_2^{(2)}.
\nonumber
\end{eqnarray}
Thus, as a result of the experiment, we obtain 8 pairs of values,
which can be grouped into three sets.

The first set contains
correlated combinations of operators $\hat{A}$, $\hat{A}'$, $\hat{C}$, and
$\hat{C}'$: $\left(a, c\right)$ and $\left(a', c'\right)$. For these
combinations, we can say that if one of the numbers $a$ or $a'$
equals $\pm 1$, then the other ($c$ or $c'$) equals $\mp 1$. Thus,
these numbers can be used to obtain a random
sequence of numbers, which will subsequently be used as the key.

The second group contains 4 pairs of values, which will
be used to verify Bell inequalities: $\left(a, b\right)$,
$\left(a', b\right)$, $\left(a, b'\right)$, and $\left(a', b'\right)$.

The last group contains pairs of values $\left(a, c'\right)$ and
$\left(a', c\right)$. This group of values is discarded in the future.

At the initial stage, Bob and Alice independently perform measurements in a random order.
Upon completion of the measurements, they communicate to each other (over a conventional open communication channel) which quantities they measured in each particular trial, without revealing the measurement results themselves. Subsequently, those trials in which either failed to register a photon are discarded, as well as results related to the third group of measurements. The results of the second group of measurements are openly published and the average value $\left<F\right>$
\eqref{eqEntangFmain} is calculated.

If the obtained value in magnitude is close to the predictions of quantum
mechanics \eqref{eqEntangQuant}
\[
\left<F\right>_{quant} = - \sqrt{2},
\]
then the results of the first group of measurements can be
interpreted as the key.

If an eavesdropper Eve attempts to learn the results of the first
group of measurements, which constitute the distributed key, one way 
to do this is to replace the entangled photon pair source herself and send Alice and Bob pairs of photons
with predetermined polarization properties, so that the outcomes of Alice’s and Bob’s experiments are predetermined in advance. However, in this case,
$\left<F\right>$, according to \eqref{eqEntangClass}, will lie in the interval:
\[
-1 \le \left<F\right>_{class} \le 1.
\]
Thus, Alice and Bob can determine the presence of Eve’s intervention by checking Bell inequalities,
and consider the obtained key compromised if the test detects the presence of Eve.

% FIXME add substitution of Alice by Eve for Bob and vice versa